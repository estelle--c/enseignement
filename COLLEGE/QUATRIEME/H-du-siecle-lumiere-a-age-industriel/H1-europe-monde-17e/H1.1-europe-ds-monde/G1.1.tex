 \documentclass{beamer}
  %\usepackage[utf8]{inputenc}
  \usepackage{fontspec}  %pour xelatex
 \usepackage{xunicode}  %pour xelatex
  %\usetheme{Montpellier}
   \usepackage{color}
   \usepackage{xcolor}
   \usepackage{graphicx}
   \usepackage{ulem}
%   \usepackage{xkeyval}
   \usepackage{pst-tree}
   \usepackage{tabularx}
   \usepackage[french]{babel}
 %  \usepackage{pstcol,pst-fill,pst-grad}
  \setbeamercolor{normal text}{fg=black}
 \setbeamercolor{section in head/foot}{fg=black}
  \setbeamercolor{subsection in head/foot}{fg=blue}
\beamerboxesdeclarecolorscheme{blocbleu}{black!60!white}{black!20!white}
\beamerboxesdeclarecolorscheme{blocimage}{black!60!white}{black!10!white}
\setbeamercolor{section in toc}{fg=black}
\setbeamercolor{subsection in toc}{fg=blue}
\date{}
%\usepackage{titlesec}
\usepackage{soul}


\AtBeginSection[]
{
 \begin{frame}
  \tableofcontents[currentsection,hideallsubsections]
 \end{frame}
}

\AtBeginSubsection[]
{
  \begin{frame}
  \tableofcontents[currentsection,currentsubsection]
  \end{frame}
}
  \title{{\textcolor{red}{Partie 1 - L'Europe et le monde au XVIIIe siècle \\ Chapitre 1 - L'Europe DANS le monde au XVIIIe siècle}}}



\begin{document}

\newcommand{\df}[2]{\textcolor{red}{\underline{#1}: #2}}

\newcommand{\doc}[1]{
\begin{flushright}
\fbox{Documents : #1}
\end{flushright}
}

\newcommand{\con}[1]{\textcolor{blue}{\underline{Consigne}: #1}}

\newcommand{\rep}[1]{\textcolor{green}{\underline{Réponse}: #1}}

\newcommand{\ntn}[1]{\textcolor{black}{\underline{Notion}: #1}}

\begin{frame}{Thème général de l'année}
\begin{center}
{\Huge \textcolor{red}{Du siècle des Lumières à l'âge industriel}}
\end{center}
\end{frame}

\begin{frame}
 \titlepage %{CHAPITRE 2 - LES IDENTIT�S MULTIPLES DE LA PERSONNE}
 \end{frame}

\begin{frame}
\underline{Fil directeur du chapitre} : Etudier le développement de la 1ère mondialisation par les pays d'Europe occidental (--> liaison avec la mer tjs)
\end{frame}

\begin{frame}
Important qu'ils comprennent que bcp d'échanges à cette époque la (ex. le lin en Bretagne)
\end{frame}

\section{I/ Les grandes puissances européennes et leurs domaines coloniaux au début du XVIIIe siècle}

\subsection{Qu'est-ce qu'une " puissance " ?}

\begin{frame}{}

\ntn{Puissance} \\

\df{Puissance}{Pays ayant un poids mondial assez fort (dû à son importance politique, culturelle, économique, démographique...) pour influencer les décisions d'autres pays dans le monde.}

\begin{itemize}
\item 
\end{itemize}

\end{frame}

\subsection{Quelles sont les grandes puissances coloniales et maritimes en ce début du XVIIIe siècle ?}

\begin{frame}
\begin{figure}
%\center 
\caption{Puissances et colonies}  
\includegraphics[width=12cm]{doc1.jpg}
\end{figure}
\end{frame}

\begin{frame}{}

\doc{4 p. 15}

\rep{France-Angleterre au plus haut. Angleterre : 1ère puissance en Europe. Portugal, Espagne en baisse mais restent présente, notammant en Amérique.}
\end{frame}

\begin{frame}
\begin{figure}
%\center 
\caption{Louis XIV, roi de France jusqu'en 1715}  
\includegraphics[scale=0.9]{Louis_14.jpg}

\end{figure}
\end{frame}

\subsection{Où sont les colonies ?}

\begin{frame}{}

\doc{4 p. 15}

\con{tableau : Angleterre, France, Espagne, Portugal, PU. Liste les différentes colonies (selon les continents)}

\vfill

\rep{Attention : bien cerner le côté littoral de la colonie}

\vfill

\rep{Bien cerner la différence entre la colonie et le comptoir}


\end{frame}

\subsection{Quelques grandes puissances continentales à se souvenir}

\begin{frame}
\begin{figure}
%\center 
\caption{Les puissances politiques au XVIIIe siècle}  
\includegraphics[width=12cm]{doc2.jpg}

\end{figure}
\end{frame}

\begin{frame}{}

\rep{Voir la Russie  : puissance continentale qui a fait le choix de se dvper dans les terres et n'a que peu de rapport avec la mer.}

\vfill

\rep{Voir la Chine et le Japon qui a l'époque ont fait le choix de ne pas s'ouvrir au commerce extérieur.}

\end{frame}

\begin{frame}
\begin{figure}
%\center 
\caption{Pierre le Grand, tsar de Russie}  
\includegraphics[scale=0.9]{pierre-le-grand.jpg}

\end{figure}
\end{frame}


\section{II/ Les grandes routes de l'échange mondial}

\subsection{Situation de ces grandes routes}

\begin{frame}

\rep{Voir que deux principales routes : celle qui fait un triangle (Afrique-Amérique-Europe) + celle qui fait le tour de l'Afrique directement les Indes.}

\end{frame}

\begin{frame}
\begin{figure}
%\center 
\caption{Les grandes routes commerciales au début du XVIIIe siècle}  
\includegraphics[width=10cm]{doc1.jpg}
\end{figure}
\end{frame}

\subsection{Le bateau, mode de transport utilisés dans les échanges}

\begin{frame}

\begin{figure}
%\center 
\caption{Vue des magasins de la Compagnie des Indes à Pondichéry, avant la destruction de la ville par les Anglais en 1761. (XVIIIe siècle, Lorient, musée de la Compagnie des Indes).}  
\includegraphics[scale=0.9]{Magasins_de_la_Compagnie_des_Indes_à_Pondichéry.jpg}

\end{figure}

\end{frame}

\begin{frame}
\begin{figure}
%\center 
\caption{Localisation de Pondichéry}  
\includegraphics[width=12cm]{doc1.jpg}
\end{figure}
\end{frame}


\begin{frame}
Rappeler que : 
- 17 : Hollande et France avait été puissances maritimes les plus importtes. Au 18e, arrivée des Anglais avec leurs comptoirs des Indes.

Bateaux marchands transportent les denrées depuis la France jusqu'aux colonies et vice-versa.

Mais c'est une GUERRE économique liée aux guerres politiques --> des corsaires sillones les mers pour détruire les navires commerciaux du pays ennemi en temps de guerre. Les corsaires ne sont pas des pirates : ce sont des marins qui font la guerre pas dans un but militaire (ce ne sont pas des soldats) mais dans un but commercial --> détruire le plus de bateau commercial du pays ennemi pour qu'il perde de l'argent.
\end{frame}


\subsection{Le port, interface majeure au XVIIIe siècle entre les colonies et l'Europe}

\begin{frame}

\begin{figure}
%\center 
\caption{Joseph Vernet, \textit{Vue du port de La Rochelle}, 1762}  
\includegraphics[width=8cm]{port-la-rochelle.eps}

\end{figure}
\end{frame}

\subsection{Le port, interface majeure au XVIIIe siècle entre les colonies et l'Europe}

\begin{frame}

\begin{figure}
%\center 
\caption{Joseph Vernet, \textit{Vue du port de La Rochelle}, 1762}  
\includegraphics[width=8cm]{port-la-rochelle.eps}

\end{figure}
\end{frame}

\begin{frame}
\begin{figure}
%\center 
\caption{Joseph Vernet}  
\includegraphics[width=6cm]{vernet.jpg}

\end{figure}
\end{frame}

\begin{frame}{Joseph Vernet histoire des arts}
Joseph Vernet (), Vue du Port de la Robechelle, prise de la petite Rive, 1762
165*253
Musée national de la Marine, Paris

Tableau faisant partie du style de peinture de marine (apparu au XVIIe siècle, surtt en Italie et aux PB).Vernet en est 1 des + célèbres représentants.

Pourquoi ? 18e : politique maritime de France profondément remise en cause. Angleterre domine les océans -> pvr royal pd cse de nécessiter d'augmenter sa flotte.

biographie : né à Avignon. Va 20 ans à Rome. Commande du roi des ports de France. Périple de 10 ans pour Vernet et sa famille.

Tableaux exécutés entre 1754 et 1768. Connaît immense succès dès leur 1ère exposition.
\end{frame}


\begin{frame}
Tableau de propagande : il ment. Vieux port de La Rochelle souffre d'envasements réguliers et ne peux accueillir autant de gros navires.  A époque de Vernet, port abandonné / bâtiments les + importants, obligés de mouiller en rade. Ms cette série des Ports de France doit proposer une image idéalisée de la France maritime -> choisi de peindre activité intense d'un port de commerce tourné vers les colonies.



\end{frame}

\subsection{Quels sont les produits échangés au XVIIIe siècle}

\begin{frame}
\begin{figure}
%\center 
\caption{Les principaux produits arrivant des colonies}  
\includegraphics[width=8cm]{doc3.jpg}

\end{figure}
\end{frame}



  \end{document}