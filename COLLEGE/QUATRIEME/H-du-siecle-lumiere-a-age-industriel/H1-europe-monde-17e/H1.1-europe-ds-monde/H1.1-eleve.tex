 \documentclass{beamer}
  %\usepackage[utf8]{inputenc}
  \usepackage{fontspec}  %pour xelatex
 \usepackage{xunicode}  %pour xelatex
  %\usetheme{Montpellier}
   \usepackage{color}
   \usepackage{xcolor}
   \usepackage{graphicx}
   \usepackage{ulem}
%   \usepackage{xkeyval}
   \usepackage{pst-tree}
   \usepackage{tabularx}
   \usepackage[french]{babel}
 %  \usepackage{pstcol,pst-fill,pst-grad}
  \setbeamercolor{normal text}{fg=black}
 \setbeamercolor{section in head/foot}{fg=black}
  \setbeamercolor{subsection in head/foot}{fg=blue}
\beamerboxesdeclarecolorscheme{blocbleu}{black!60!white}{black!20!white}
\beamerboxesdeclarecolorscheme{blocimage}{black!60!white}{black!10!white}
\setbeamercolor{section in toc}{fg=black}
\setbeamercolor{subsection in toc}{fg=blue}
\date{}
%\usepackage{titlesec}
\usepackage{soul}


\AtBeginSection[]
{
 \begin{frame}
  \tableofcontents[currentsection,hideallsubsections]
 \end{frame}
}

\AtBeginSubsection[]
{
  \begin{frame}
  \tableofcontents[currentsection,currentsubsection]
  \end{frame}
}
  %\title{{\textcolor{red}{Partie 1 - L'Europe et le monde au XVIIIe siècle \\ Chapitre 1 - L'Europe DANS le monde au XVIIIe siècle}}}



\begin{document}

\newcommand{\df}[2]{\textcolor{red}{\underline{#1}: #2}}

\newcommand{\doc}[1]{
\begin{flushright}
\fbox{Documents : #1}
\end{flushright}
}

\newcommand{\con}[1]{\textcolor{blue}{\underline{Consigne}: #1}}

\newcommand{\rep}[1]{\textcolor{green}{\underline{Réponse}: #1}}

\newcommand{\ntn}[1]{\textcolor{black}{\underline{Notion}: #1}}






\begin{frame}
\begin{figure}
%\center 
\caption{Puissances et colonies}  
\includegraphics[width=12cm]{doc1.jpg}
\end{figure}
\end{frame}

\begin{frame}
\begin{figure}
%\center 
\caption{Louis XIV, roi de France jusqu'en 1715}  
\includegraphics[scale=0.9]{Louis_14.jpg}

\end{figure}
\end{frame}



\begin{frame}
\begin{figure}
%\center 
\caption{Les puissances politiques au XVIIIe siècle}  
\includegraphics[width=12cm]{doc2.jpg}

\end{figure}
\end{frame}



\begin{frame}
\begin{figure}
%\center 
\caption{Pierre le Grand, tsar de Russie}  
\includegraphics[scale=0.9]{pierre-le-grand.jpg}

\end{figure}
\end{frame}




\begin{frame}
\begin{figure}
%\center 
\caption{Les grandes routes commerciales au début du XVIIIe siècle}  
\includegraphics[width=10cm]{doc1.jpg}
\end{figure}
\end{frame}


\begin{frame}

\begin{figure}
%\center 
\caption{Vue des magasins de la Compagnie des Indes à Pondichéry, avant la destruction de la ville par les Anglais en 1761. (XVIIIe siècle, Lorient, musée de la Compagnie des Indes).}  
\includegraphics[width=11cm]{Magasins_de_la_Compagnie_des_Indes_à_Pondichéry.jpg}

\end{figure}

\end{frame}

\begin{frame}
\begin{figure}
%\center 
\caption{Localisation de Pondichéry}  
\includegraphics[width=12cm]{doc1.jpg}
\end{figure}
\end{frame}




\begin{frame}

\begin{figure}
%\center 
\caption{Joseph Vernet, \textit{Vue du port de La Rochelle}, 1762}  
\includegraphics[width=11cm]{port-la-rochelle.eps}

\end{figure}
\end{frame}

\begin{frame}
\begin{figure}
%\center 
\caption{Joseph Vernet}  
\includegraphics[width=6cm]{vernet.jpg}

\end{figure}
\end{frame}


\begin{frame}
\begin{figure}
%\center 
\caption{Les principaux produits arrivant des colonies}  
\includegraphics[width=8cm]{doc3.jpg}

\end{figure}
\end{frame}



  \end{document}