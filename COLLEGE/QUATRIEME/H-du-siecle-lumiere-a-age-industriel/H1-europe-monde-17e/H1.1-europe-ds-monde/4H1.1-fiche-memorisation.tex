
\documentclass[12pt,a4paper,landscape,twocolumn]{article}
%\usepackage[utf8]{inputenc}
\usepackage[francais]{babel}
%\usepackage[T1]{fontenc}
\usepackage{graphicx}
\usepackage[left=0.5cm,right=0.5cm,top=0.5cm,bottom=0.5cm]{geometry}
 \usepackage{fontspec}  %pour xelatex
 \usepackage{xunicode}  %pour xelatex
 \usepackage{xltxtra}
  \setmainfont[Mapping=tex-text]{Century Schoolbook L}

\begin{document}


\fbox{
\begin{minipage}{13.5cm}
TITRE : 
\vspace{0.7cm}
\end{minipage}
}

\vspace{0.2cm}

\fbox{
\begin{minipage}{13.5cm}
PERSONNAGES


\vspace{3cm}

\end{minipage}
}

\vspace{0.2cm}

\fbox{
\begin{minipage}{13.5cm}
DÉFINITIONS
\vspace{5cm}
\end{minipage}
}

\vspace{0.2cm}

\fbox{
\begin{minipage}{7cm}
CAPACITÉS : \\
Je connais et sait utiliser les repères suivants :  \\
- Les grandes puissances politiques en Europe sur une carte de l’Europe au début du XVIIIe siècle \\
- Leurs empires coloniaux sur une carte du monde au début du XVIIIe siècle \\
- Quelques grandes routes maritimes \\


\vspace{2cm}
\end{minipage}
}
%\hspace{0.2}
\fbox{
\begin{minipage}{19.8cm}
ORGANIGRAMME DE RÉSUMÉ
\vspace{8cm}
\end{minipage}
}

\newpage
\fbox{
\begin{minipage}{13cm}
\vspace{1cm}  %dates importantes, périodes...
%\vspace{3cm}
\includegraphics[width=12cm]{trait-de-chrono-vierge.eps}

\vspace{1cm}

\end{minipage}
}

\vspace{0.2cm}

\fbox{
\begin{minipage}{13cm}

\includegraphics[width=11.8cm]{carte-monde.eps}

\vspace{1cm}
\end{minipage}
}

\end{document}