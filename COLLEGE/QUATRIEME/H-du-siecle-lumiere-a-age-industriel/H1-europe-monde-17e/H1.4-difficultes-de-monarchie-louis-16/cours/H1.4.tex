  \documentclass{beamer}
  %\usepackage[utf8]{inputenc}
  \usepackage{fontspec}  %pour xelatex
 \usepackage{xunicode}  %pour xelatex
  \usetheme{default}
   \usepackage{color}
   \usepackage{graphicx}
   \usepackage{ulem}
%   \usepackage{xkeyval}
   \usepackage{pst-tree}
   \usepackage{tabularx}
   \usepackage[french]{babel}
 %  \usepackage{pstcol,pst-fill,pst-grad}
  \setbeamercolor{normal text}{fg=blue}
 \setbeamercolor{section in head/foot}{fg=black}
  \setbeamercolor{subsection in head/foot}{fg=blue}
\beamerboxesdeclarecolorscheme{blocbleu}{black!60!white}{black!20!white}
\beamerboxesdeclarecolorscheme{blocimage}{black!60!white}{black!10!white}
\setbeamercolor{section in toc}{fg=black}
\setbeamercolor{subsection in toc}{fg=blue}



\AtBeginSection[]
{
 \begin{frame}
  \tableofcontents[currentsection,hideallsubsections]
 \end{frame}
}

\AtBeginSubsection[]
{
  \begin{frame}
  \tableofcontents[currentsection,currentsubsection]
  \end{frame}
}
 
      \title{{\textcolor{red}{Chapitre 4 - Les difficultés de la monarchie sous Louis XVI }}}

\begin{document}
 
\begin{frame}
\titlepage %{CHAPITRE 2 - LES IDENTITÉS MULTIPLES DE LA PERSONNE}
\end{frame}

\begin{frame}
\tableofcontents
\end{frame}

\section{I/ Une érosion générale de l'autorité royale}

\begin{frame}
\begin{beamerboxesrounded}[scheme=blocimage]{Doc 1. Marie-Antoinette en famille}
\begin{columns}[c]
\begin{column}{7 cm}
\includegraphics[scale=0.40]{marie-antoinette.jpg}
\end{column}

\begin{column}{3 cm}
\textcolor{black}{\small \textit{Huile sur toile d'Elisabeth Louise Vigée-Le Brun, 1789, musée	national du Château de Versailles}}
\end{column}
\end{columns}
\end{beamerboxesrounded}
\end{frame}

\begin{frame}
\begin{beamerboxesrounded}[scheme=blocimage]{Doc 2. Pamphlet contre Marie-Antoinette}
\textcolor{black}{L'AUTRICHIENNE EN GOGUETTES OU L'ORGIE ROYALE\\
" La Reine, élève de feu Sacchini et protectrice de tout ce qui est compositeur ultramondain, a la ferme persuasion qu'elle est bonne musicienne, parce qu'elle estropie quelques sonnaies sur son clavecin, et qu'elle chante faux dans les concerts qu'elle donne, et où elle a soin de ne laisser entrer que de vils admirateurs. Quant à Louis XVI, on peut se faire une idée de son goût pour l'harmonie, en apprenant que les sons discordants et insupportables de deux flambeaux d'argents frottés avec force sur une table de marbre, ont des attraits pour son oreille anti-musicale." }\\
\begin{flushright}
\textcolor{black}{\small \textit{Extrait de la 1ère page d'un pamphlet attribué à un acteur, François-Marie Mayeur, 1789}}
\end{flushright}
\end{beamerboxesrounded}
\end{frame}

\begin{frame}
\begin{flushright}
{\tiny \textcolor{orange}{Compétences utilisées : \\
C5 : je sais tirer des informations d'une peinture, \\d'une source historique (ici un pamphlet).\\}}
\end{flushright}
\underline{Exercice}
\begin{itemize}
\item Présente les deux documents (date, auteur, nature).
\item doc 1. Décrit la peinture. 
\item doc 2. De quel sujet se moque l'auteur ?
\item doc 2. Comment l'autorité royale est-elle contestée dans la seconde moitié du XVIIIe siècle ?
\end{itemize}
\end{frame}

\begin{frame}
\begin{itemize}
\item Présente les deux documents(date, auteur, nature).\\
\pause \textcolor{black!70!green}{Le premier document est une huile sur toile, faite par Elisabeth Louise Vigée-Le Brun en 1789. \\ Le second document est un extrait d'un pamphlet, écrit par François-Marie mMayeur en 1789.}
\vfill
\item doc 1. Décrit la peinture. \\
\pause \textcolor{black!70!green}{Cette peinture est un portrait de la reine Marie-Antoinette avec trois de ses enfants. Son but est de montrer une scène d'une vie "quotidienne" et d'une mère aimante avec ses enfants.}.
\end{itemize}
\end{frame}

\begin{frame}
\begin{itemize}
\item doc 2. De quel sujet se moque l'auteur ? \\
\pause \textcolor{black!70!green}{L'auteur se moque des goûts musicaux du roi et de la reine.}
\vfill
\item doc 2. Comment l'autorité royale est-elle contestée dans la seconde moitié du XVIIIe siècle ? \\
\pause \textcolor{black!70!green}{L'autorité royale est dénigrée. On chercher à montrer que le roi et la reine ont beaucoup moins de grâce que ce qu'on aurait pensé.}.
\end{itemize}
\end{frame}

\begin{frame}
\setlength{\parindent}{1cm} Louis XVI a été roi de France de 1774 à 1791.
\vfill
\setlength{\parindent}{1cm}Dans un contexte de déchristianisation et de progression de l'alphabétisation, les publications de \textcolor{red}{pamphlets} et d'écrits philosophiques contre la monarchie explosent. Le roi est monarque absolu de droit divin. Cependant, il n'arrive pas à censurer tous les écrits qui conteste son autorité. Il est tourné en ridicule par une partie de la société qui lui reprochent les frasques de son épouse.
\vfill
\textcolor{red}{\underline{Pamphlet}: texte court et virulent qui remet en cause la famille royale.}
\end{frame}

\section{II/ Quelles revendications et aspirations se développent ?}

\subsection{Une remise en cause de la société d'ordre.}

\begin{frame}
\begin{beamerboxesrounded}[scheme=blocimage]{Doc 1. Caricature des trois ordres}
\begin{columns}[c]
\begin{column}{7 cm}
\includegraphics[scale=0.40]{caricature-trois-ordres.jpg}
\end{column}

\begin{column}{3 cm}
\textcolor{black}{\small \textit{Eau-forte coloriée anonyme, publiée au printemps 1789, BNF, Paris \\
Texte porté sur la gravure : dans la poche de l'abbé (ses titres, les griefs), sur l'épée du noble "rougie de sang", dans la poche du paysan "sel et tabac, taille, corvée, dîmes, milice".}}
\end{column}
\end{columns}
\end{beamerboxesrounded}
\end{frame}

\begin{frame}{Méthode : la caricature}
\begin{itemize}
\item \textcolor{red}{\underline{caricature} : gravure imprimée ou dessinée qui charge certains traits de caractère souvent drôles ou ridicules dans la représentation d'un sujet.}
\item Au XVIIIe, critique de la monarchie française. Elle doit être analysée avec esprit critique. Elle n'est pas la description objectif d'un fait historique, mais une prise de position.
\item Elle véhicule ici un message politique accessible à tous.
\end{itemize}
\end{frame}



\begin{frame}
\begin{flushright}
{\tiny \textcolor{orange}{Compétences utilisées : \\
C5 : je sais tirer des informations d'une caricature}}
\end{flushright}
\underline{Exercice}
\begin{itemize}
\item Présente le document (date, auteur, nature).
\item A quoi reconnaît-on le noble ? le clerc ? le paysan ? 
\item Qu'est-ce qui oppose le clerc et le noble au paysan.
\item Que veut signifier la caricature ?
\end{itemize}
\end{frame}

\begin{frame}
\begin{itemize}
\item Présente le document (date, auteur, nature).
\pause\textcolor{black!70!green}{C'est une caricature (un dessin grossisant les traits des personnages), écrite par un auteur anonyme et publiée au printemps 1789.}
\item A quoi reconnaît-on le noble ? le clerc ? le paysan ? 
\pause\textcolor{black!70!green}{Le noble est bien habillé, une épée sur le côté. Le clerc a une croix dans la main. Le paysan a une chemise trouée et tient une pioche.}
\item Qu'est-ce qui oppose le clerc et le noble au paysan.
\pause\textcolor{black!70!green}{Le noble et le clerc sont sur le dos du paysan.}
\item Que veut signifier la caricature ?
\pause\textcolor{black!70!green}{Cette caricature montre l'inégalité de la société de l'époque : le Tiers-Etat payant tous les impôts, le clergé et la noblesse vivant à ses dépends.}
\end{itemize}
\end{frame}

\begin{frame}
\setlength{\parindent}{1cm} A la veille de la révolution française, différentes contestations apparaissent et remettent en cause la politique du roi. Cette contestation du modèle de la société des trois ordres est relayée par des articles de journaux et des \textcolor{red}{caricatures.}
\end{frame}

\subsection{La Révolution américaine, terreau de nouvelles idées. }

\begin{frame}
\begin{beamerboxesrounded}[scheme=blocimage]{Doc 1. La Déclaration d'indépendance de 1776}
\textcolor{black}{
"Les hommes sont doués par leur Créateur de droits inaliénables; parmi ces droits se trouve la liberté. Nous, les représentants des Etats-Unis d'Amérique, déclarons solennellement au nom du bon peuple de ces colonies, que ces colonies unies ont droit d'être et sont des Etats libres et indépendants; que tout lien politique entre elles et l'Etat de Grande-Bretagne est et doit être entièrement dissous."}
\end{beamerboxesrounded}
\end{frame}

\begin{frame}
\begin{flushright}
{\tiny \textcolor{orange}{Compétences utilisées : \\
C5 : je sais tirer des informations d'un texte.}}
\end{flushright}
\underline{Exercice}
\begin{itemize}
\item Présente le document (date, auteur, nature).
\item Quel droit les Américains mettent comme droit fondamental de la personne ?
\item Contre qui se battent les Américains ?
\end{itemize}
\end{frame}

\begin{frame}
\setlength{\parindent}{1cm} Les officiers français envoyés par Louis XVI pour aider les colons lors de la guerre d'Indépendance américaine (1776-1783), reviennent au pays avec de nouvelles idées basées sur la liberté et l'égalité entre les hommes.
\end{frame}

\section{III/ De la crise générale aux Etats Généraux}
\begin{frame}
\begin{beamerboxesrounded}[scheme=blocimage]{Doc 1. Les finances de l'Etat royal en 1788}
\includegraphics[scale=0.28]{finance.jpg}
\begin{flushright}
\textcolor{black}{\small \textit{Belin, 2010, p.48}}
\end{flushright}
\end{beamerboxesrounded}
\end{frame}

\begin{frame}
\begin{beamerboxesrounded}[scheme=blocimage]{Doc 2.La crise agricole des années 1788-1789}
\includegraphics[scale=0.25]{agri.jpg}
\begin{flushright}
\textcolor{black}{\small \textit{Belin, 2010, p.48}}
\end{flushright}
\end{beamerboxesrounded}
\end{frame}

\begin{frame}
\begin{flushright}
{\tiny \textcolor{orange}{Compétences utilisées : \\
C5 : je sais tirer des informations d'un graphique.}}
\end{flushright}
\underline{Exercice}
\begin{itemize}
\item doc 1. Quelle est la situation du royaume en 1788 ?
\item doc 2 Quelles sont les conséquences des mauvaises récoltes ?.
\end{itemize}
\end{frame}

\begin{frame}
\begin{beamerboxesrounded}[scheme=blocimage]{Doc 3. Un cahier de doléance.} 
\textcolor{black}{4e - Les députés solliciteront l' abolition entière de tous les privilèges des nobles et des ecclésiastiques.\\
5e - l'abolition de la gabelle, ce désastreux impôt, ainsi que des tailles, capitations, vingtièmes, des aides.\\
6e - Pour remplacer ces impôts, qu'il soit établie par les Etats généraux une capitation personnelle, une taxe foncière et une taxe d'exploitation, qui frapperont indistinctement et sans aucun privilège tous les citoyens des trois ordres.\\
9e - Les députés solliciteront aussi l'abolition totale des justices et polices seigneuriales, des droits de chasse et de pêche exclusifs, des banalités.\\
15e - Que les emplois civils, militaires, ecclésiastiques soient accessibles indistinctement de manière que la noblesse n'ait pas de préférence et que le tiers état n'ait plus d'exclusion.}
\begin{flushright}
\textcolor{black}{Extraits du \textit{cahier de doléances} de la Chapelle-Craonnaise (Mayenne), 1789}
\end{flushright}
\end{beamerboxesrounded}
\end{frame}

\begin{frame}
\begin{flushright}
{\tiny \textcolor{orange}{Compétences utilisées : \\
C5 : je sais tirer des informations d'un texte.}}
\end{flushright}
\underline{Exercice}
\begin{itemize}
\item Classe les doléances des habitants de ce village : doléances fiscales, sociales ou judiciaire.
\end{itemize}
\end{frame}

\begin{frame}
\setlength{\parindent}{1cm} En 1788 (veille de la Révolution française), l'Etat est en proie à une crise financière (plus de dépense, notamment liée aux différentes guerres, que de recettes), une crise agricole (sécheresse, troubles agraires) et une crise industrielle. Ces crises alimentent le mécontentement général de la population.
\setlength{\parindent}{1cm} Dans ce contexte, le roi décide de demander la réunion des \textcolor{red}{Etats-Généraux} en juillet 1788. Grâce aux cahiers de doléances, les habitants du royaume font remonter leurs problèmes à la monarchie.
\end{frame}
  \end{document}