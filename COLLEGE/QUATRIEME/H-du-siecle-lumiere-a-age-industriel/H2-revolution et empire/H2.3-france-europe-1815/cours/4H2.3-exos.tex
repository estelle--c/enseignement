\documentclass[12pt]{article}
\usepackage{fontspec}
\usepackage{xltxtra}
\setmainfont[Mapping=tex-text]{Century Schoolbook L}
 \usepackage[francais]{babel}
 
 \usepackage{geometry}
 \geometry{ hmargin=0.5cm, vmargin=0.5cm } 

\makeatletter
\renewcommand\section{\@startsection
{section}{1}{0mm}    
{\baselineskip}
{0.5\baselineskip}
{\normalfont\normalsize\textbf}}
\makeatother


\begin{document}

\underline{\textbf{Exercice 1 - Le congrès de Vienne.}}\\

\begin{minipage}{9cm}
Dans le texte, surligne le passage montrant que les idéaux de la Révolution française se sont diffusés dans tous les pays d'Europe.\\

Que propose Metternich aux souverains d'Europe pour faire face aux idées révolutionnaires ?
\vspace{4cm}
\end{minipage}
\fbox{
\begin{minipage}{9cm}
\textbf{ Doc 1 - La volonté d'un retour à l'ordre ancien.}\\
 << Il n'existe en Europe qu'une seule affaire, c'est la Révolution. [...] L'intérieur de tous les pays européens, sans en excepter aucun, est travaillé par une fièvre ardente. C'est la guerre à mort entre les anciens et les nouveaux principes, entre l'ancien et le nouvel ordre social. [...]\\
 Si, dans cette crise effrayante, les principaux souverains de l'Europe étaient désunis, nous serions tous emportés dans un petit nombre d'années.>>
 \begin{flushright}
 D'après le prince de Metternich, Chancelier d'Autriche, texte écrit en 1815.
 \end{flushright}
\end{minipage}}

\vspace{1cm}

\underline{\textbf{Exercice 2 - Les protagonistes du Congrès de Vienne (1815)}}\\

\begin{minipage}{3cm}
Place sur la frise chronologique le Congrès de Vienne.
\vspace{2cm}
\end{minipage}
\begin{minipage}{15cm}
\includegraphics[scale=1]{chrono1.eps}
\end{minipage}

\textbf{5 p. 95.} \\
Décrit la peinture.\\ %\textcolor{black!70!green}{Réunion des rois d'Europe après la chute de Napoléon.}

Quel est le but du Congrès ? %\textcolor{black!70!green}{Se partager l'Europe napoléonienne.}

\vfill

\underline{\textbf{Exercice 3 - Constat : L'Europe en 1815.}}\\

 \begin{minipage}{8cm}
\textbf{1 p 92 / 2 p. 93.} Regarde les changements entre 1811 et 1815 \\
 Remplis la carte ci-jointe (Nomme les empires et royaumes indiqués de A à E et leurs capitales pointées de 1 à 5).
 
 \vspace{0.5cm}
 
 \textbf{3 p. 15 / 2 p. 93} Retrouve-t-on l'ancien partage de l'Europe ?
 \vspace{2cm}
  %\textcolor{black!70!green}{       }\\
 \end{minipage}
 \begin{minipage}{8cm}
\includegraphics[scale=0.18]{doc4.jpg}
 \end{minipage}

\newpage

\underline{\textbf{Exercice 1 - L'émergence du sentiment national espagnol.}}\\

\begin{minipage}{5cm}
\begin{enumerate}
 \item Lit l'introduction p. 90.
 \item Inscrit le soulèvement espagnol contre Napoléon Ier dans la chronologie.
 \item Répond à l'activité << histoire des arts >> p. 91
 \end{enumerate}
\end{minipage}
\begin{minipage}{10cm}
\includegraphics[scale=1]{chrono3.eps}
\end{minipage}

\vfill

\underline{\textbf{Exercice 1 - L'émergence du sentiment national espagnol.}}\\

\begin{minipage}{5cm}
\begin{enumerate}
 \item Lit l'introduction p. 90.
 \item Inscrit le soulèvement espagnol contre Napoléon Ier dans la chronologie.
 \item Répond à l'activité << histoire des arts >> p. 91
 \end{enumerate}
\end{minipage}
\begin{minipage}{10cm}
\includegraphics[scale=1]{chrono3.eps}
\end{minipage}

\vfill
\underline{\textbf{Exercice 1 - L'émergence du sentiment national espagnol.}}\\

\begin{minipage}{5cm}
\begin{enumerate}
 \item Lit l'introduction p. 90.
 \item Inscrit le soulèvement espagnol contre Napoléon Ier dans la chronologie.
 \item Répond à l'activité << histoire des arts >> p. 91
 \end{enumerate}
\end{minipage}
\begin{minipage}{10cm}
\includegraphics[scale=1]{chrono3.eps}
\end{minipage}

\end{document}