 \documentclass{beamer}
  %\usepackage[utf8]{inputenc}
  \usepackage{fontspec}  %pour xelatex
 \usepackage{xunicode}  %pour xelatex
  \usetheme{Montpellier}
   \usepackage{color}
   \usepackage{xcolor}
   \usepackage{graphicx}
   \usepackage{ulem}
%   \usepackage{xkeyval}
   \usepackage{pst-tree}
   \usepackage{tabularx}
   \usepackage[french]{babel}
 %  \usepackage{pstcol,pst-fill,pst-grad}
  %\setbeamercolor{normal text}{fg=black}
 \setbeamercolor{section in head/foot}{fg=black}
  \setbeamercolor{subsection in head/foot}{fg=blue}
\beamerboxesdeclarecolorscheme{blocbleu}{black!60!white}{black!20!white}
\beamerboxesdeclarecolorscheme{blocimage}{black!60!white}{black!10!white}
\setbeamercolor{section in toc}{fg=black}
\setbeamercolor{subsection in toc}{fg=blue}



\AtBeginSection[]
{
 \begin{frame}
  \tableofcontents[currentsection,hideallsubsections]
 \end{frame}
}

\AtBeginSubsection[]
{
  \begin{frame}
  \tableofcontents[currentsection,currentsubsection]
  \end{frame}
}
  \title{{\textcolor{red}{Chapitre 3 - La France et l'Europe en 1815.}}}

\begin{document}
\begin{frame}
 \titlepage %{CHAPITRE 2 - LES IDENTIT�S MULTIPLES DE LA PERSONNE}
 \end{frame}
 
 \begin{frame}
 \tableofcontents
 \end{frame}
 
\begin{frame} Introduction \\
\underline{But du cours :} Comprendre le remaniement de l'Europe au début du XIXe siècle, à la chute de l'Empire.
\end{frame}
 
\section{ I/ La France et l'Europe à l'heure du congrès de Vienne.}
 
 \begin{frame} \underline{\textbf{Exercice 1 - Le congrès de Vienne.}}\\
\textbf{ Doc 1 - La volonté d'un retour à l'ordre ancien.}\\
\fbox{
\begin{minipage}{11cm}
 << Il n'existe en Europe qu'une seule affaire, c'est la Révolution. [...] L'intérieur de tous les pays européens, sans en excepter aucun, est travaillé par une fièvre ardente. C'est la guerre à mort entre les anciens et les nouveaux principes, entre l'ancien et le nouvel ordre social. [...]\\
 Si, dans cette crise effrayante, les principaux souverains de l'Europe étaient désunis, nous serions tous emportés dans un petit nombre d'années.>>
 \begin{flushright}
 D'après le prince de Metternich, Chancelier d'Autriche, texte écrit en 1815.
 \end{flushright}
\end{minipage}}
\end{frame}

 \begin{frame}
 Dans le texte, surligne le passage montrant que les idéaux de la Révolution française se sont diffusés dans tous les pays d'Europe.
 
\fbox{
\begin{minipage}{11cm}
%<< Il n'existe en Europe qu'une seule affaire, c'est la Révolution. [...] \underline{L'intérieur de tous les pays européens, sans en excepter aucun, est} \\ \underline{travaillé par une fièvre ardente. C'est la guerre à mort entre les} \\ \underline{anciens et les nouveaux principes, entre l'ancien et le nouvel ordre \\ %\underline{social.} [...]\\
Si, dans cette crise effrayante, les principaux souverains de l'Europe étaient désunis, nous serions tous emportés dans un petit nombre d'années.>>
\begin{flushright}
D'après le prince de Metternich, Chancelier d'Autriche, texte écrit en 1815.
\end{flushright}
\end{minipage}}
\end{frame}

\begin{frame}

 Que propose Metternich aux souverains d'Europe pour faire face aux idées révolutionnaires ? \pause \textcolor{black!70!green}{de s'unir contre les idées révolutionnaires}
 \end{frame}
 
 \begin{frame} \underline{\textbf{Exercice 2 - Les protagonistes du Congrès de Vienne (1815)}}\\
Place sur la frise chronologique le Congrès de Vienne.\\
\includegraphics[scale=0.80]{chrono1.eps}

\end{frame}

\begin{frame} Correction
\includegraphics[scale=0.80]{chrono2.eps}

\end{frame}
\begin{frame}
\textbf{5 p. 95.}\\
\includegraphics[scale=0.10]{doc1.jpg}\\
Décrit la peinture. %\textcolor{black!70!green}{Réunion des rois d'Europe après la chute de Napoléon.}
\vfill
Quel est le but du Congrès ? %\textcolor{black!70!green}{Se partager l'Europe napoléonienne.}

 \end{frame}
 
 \begin{frame} \underline{\textbf{Exercice 3 - Constat : L'Europe en 1815.}}\\
 \begin{minipage}{5cm}
\textbf{1 p 92 / 2 p. 93.} Regarde les changements entre 1811 et 1815 \\
 Remplis la carte ci-jointe (Nomme les empires et royaumes indiqués de A à E et leurs capitales pointées de 1 à 5).\\
  %\textcolor{black!70!green}{       }\\
 \end{minipage}
 \begin{minipage}{5cm}
\includegraphics[scale=0.10]{doc4.jpg}
 \end{minipage}
 \end{frame}
 
 \begin{frame} Correction de la carte.
 % A faire avec gimp sur doc4-correction.jpg.
 \end{frame}
 
\begin{frame}
\textbf{3 p. 15 / 2 p. 93} Retrouve-t-on l'ancien partage de l'Europe ? \\ 
%\textcolor{black!70!green}{ Disparition du Saint Empire, grains Taux des grands vainqueurs (Prusse, Russie), domination de Autriche en Italie}
 \end{frame}
 
 \begin{frame}
 \textcolor{blue}{ %\underline{Sur le cahier : }
  En 1815, Napoléon est vaincu par la coalition européenne à Waterloo. Sa chute provoque un bouleversement dans l'équilibre européen. Les rois veulent, au congrès de Vienne, restaurer l'ordre politique ancien et liquider l'héritage de la Révolution. Cependant, le partage qui est fait est loin de la restauration de l'état ancien de l'Europe.}
 \end{frame}

 
 \section{La diffusion de l'idée de nation et du sentiment national pendant et après l'époque Napoléonienne.}
 
 \begin{frame} \underline{\textbf{Exercice 1 - L'émergence du sentiment national espagnol.}}
 \begin{enumerate}
 \item Lit l'introduction p. 90.
 \item Inscrit le soulèvement espagnol contre Napoléon Ier dans la chronologie.
 \item Répond à l'activité << histoire des arts >> p. 91
 \end{enumerate}
 
 \end{frame}
 
 \begin{frame} Correction chronologie.
\includegraphics[scale=0.90]{chrono4.eps}
 \end{frame}
 
 \begin{frame}Correction des questions p. 91

 \end{frame}
  \end{document}