\documentclass[a4paper,12pt]{exam}

%printanswers % Pour ne pas imprimer les réponses (énoncé)
\addpoints % Pour compter les points
\pointsinrightmargin % Pour avoir les points dans la marge à droite
%\bracketedpoints % Pour avoir les points entre crochets
%\nobracketedpoints % Pour ne pas avoir les points entre crochets
\pointformat{.../\textbf{\themarginpoints}}
% \noaddpoints % pour ne pas compter les points
%\qformat{\textbf{Question\thequestion}\quad(\thepoints)\hfill} % Pour définir le style des questions (facultatif)
%\qformat{\thequestiontitle \dotfill \thepoints}


\usepackage{fontspec}
%\usepackage{xunicode}
 \usepackage{graphicx}
\usepackage{amsmath}
\usepackage{amssymb}
\usepackage{wasysym}
\usepackage{marvosym}
\usepackage{cwpuzzle}
 \usepackage{geometry}
 \geometry{hmargin=1.7cm, vmargin=1cm } 
\defaultfontfeatures{Mapping=tex-text}
%\setmainfont{Linux Libertine}
\setmainfont{Century Schoolbook L}
%\usepackage[margin=1cm]{geometry}
    \usepackage[francais]{babel}
    \title{DS - Histoire - L'Orient ancien}
   
     
    % Si on imprime les réponses
    \ifprintanswers
    \newcommand{\rep}[1]{}
    \newcommand{\chariot}{}
    \else
    \newcommand{\rep}[1]{\fillwithdottedlines{#1}}
    \newcommand{\chariot}{\newpage}
    \fi


\makeatletter
\renewcommand\section{\@startsection
{section}{1}{0mm}    
{\baselineskip}
{0.5\baselineskip}
{\normalfont\normalsize\textbf}}
\makeatother
%\usepackage{titling}
%\renewcommand{\maketitlehooka}

 
\begin{document}

\begin{minipage}{4cm}
  Nom :
  
  Prénom :
  
  Classe : 
  
  Date : 
\end{minipage}
\hfill
\begin{minipage}{3.5cm}

{\small \begin{questions} \question[1] Orthographe et expression
\question[1] Présentation \end{questions}
}
\end{minipage}


\vspace{1cm}

\begin{center}

{\Large DS - Histoire - Révolution et Empire}

\vspace{0.5cm}
  \end{center}
Appréciation : \hfill {\large …/\numpoints\ } %\quad\quad …/\textbf{20}

\section*{Repères}

\begin{questions} % Début de l'examen. Débute la numérotation des questions
\question[4] Remplis le tableau en notant soit la date, soit l’événement relié.
\end{questions}

\begin{tabular}{|p{7cm}|p{7cm}|}
\hline 14 juillet 1789 & \vspace{1cm} \\ 
\hline \vspace{1cm}  & proclamation de la République \\ 
\hline 1789-1799 & \vspace{1cm}  \\ 
\hline \vspace{1cm}  & Napoléon Ier, empereur des Français \\ 
\hline 
\end{tabular} 

\section*{Développement écrit}
\begin{questions}
\question[6]  Raconte un des événements suivants. {\tiny Ta réponse doit être rédigée et cohérente (quelques lignes pour introduire ton propos, le récit, quelques lignes pour conclure ton propos)}
\begin{itemize}
\item La prise de la Bastille
\item Le peuple dans la Révolution et dans l'Empire.
\end{itemize}
  % \begin{solution}
   % dispersion d'un peuple à travers le monde.
   % \end{solution}
   \rep{10cm}
\end{questions}

\section*{Connaissances}
\begin{questions}
\question[2] Cite quatre transformations majeures mise en place sous la révolution.

 \end{questions}
 % \begin{solution}
  %dispersion d'un peuple à travers le monde.
 %\end{solution}
  \rep{2cm}

\section*{Etude sur document. La Marseillaise}

\begin{minipage}{10cm}
{\small \textit{En 1792, Rouget de Lisle compose le Chant de guerre pour l'armée du Rhin. Chanté par les Marseillais lorsqu'ils arrivent à Paris fin juillet 1792 (à la suite de la déclaration de guerre de la France à l'Autriche). Il devient la Marseillaise.}}
\includegraphics[scale=0.20]{doc1.jpg}
\end{minipage}
\begin{minipage}{8cm}
\includegraphics[scale=0.20]{doc2.jpg}
\end{minipage}


\begin{questions}
\question[1] Par qui et quand ce chant a-t-il été composé ? 
 % \begin{solution}
  %dispersion d'un peuple à travers le monde.
 %\end{solution}
  \rep{2cm}
  
  \question[1] Dans quel contexte a-t-il été écrit (pendant quelle période historique) ?
   % \begin{solution}
    %dispersion d'un peuple à travers le monde.
   %\end{solution}
    \rep{2cm}
  
\question[1] A qui s'adresse ce chant ?
 % \begin{solution}
  %dispersion d'un peuple à travers le monde.
 %\end{solution}
  \rep{2cm}
  
\question[1] De quels principes hérités de la Révolution ce chant est-il porteur ?
 % \begin{solution}
  %dispersion d'un peuple à travers le monde.
 %\end{solution}
  \rep{2cm}
  
\question[1] Quels ennemis ce chant désigne-t-il ?
 % \begin{solution}
  %dispersion d'un peuple à travers le monde.
 %\end{solution}
  \rep{1cm}
  
\question[1] Souligne les termes guerriers dans le texte.
 % \begin{solution}
  %dispersion d'un peuple à travers le monde.
 %\end{solution}
  %\rep{2cm}
  
%\question[2] Montre en quelques lignes pourquoi la Marseillaise est un chant majeur de la révolution (idées révolutionnaires et guerrières)
 % \begin{solution}
  %dispersion d'un peuple à travers le monde.
 %\end{solution}
 % \rep{2cm}
  
 \end{questions}




\end{document}