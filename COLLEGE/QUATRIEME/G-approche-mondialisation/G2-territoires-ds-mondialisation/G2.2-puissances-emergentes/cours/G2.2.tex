  \documentclass{beamer}
  %\usepackage[utf8]{inputenc}
  \usepackage{fontspec}  %pour xelatex
 \usepackage{xunicode}  %pour xelatex
  \usetheme{default}
   \usepackage{color}
   \usepackage{graphicx}
   \usepackage{ulem}
%   \usepackage{xkeyval}
   \usepackage{pst-tree}
   \usepackage{tabularx}
   \usepackage[french]{babel}
 %  \usepackage{pstcol,pst-fill,pst-grad}
  \setbeamercolor{normal text}{fg=blue}
 \setbeamercolor{section in head/foot}{fg=black}
  \setbeamercolor{subsection in head/foot}{fg=blue}
\beamerboxesdeclarecolorscheme{blocbleu}{black!60!white}{black!20!white}
\beamerboxesdeclarecolorscheme{blocimage}{black!60!white}{black!10!white}
\setbeamercolor{section in toc}{fg=black}
\setbeamercolor{subsection in toc}{fg=blue}



\AtBeginSection[]
{
 \begin{frame}
  \tableofcontents[currentsection,hideallsubsections]
 \end{frame}
}

\AtBeginSubsection[]
{
  \begin{frame}
  \tableofcontents[currentsection,currentsubsection]
  \end{frame}
}
 
      \title{{\textcolor{red}{ %Partie 2 - Les territoires de la mondialisation.
      \\Chapitre 2 - Les puissances émergentes}}}

\begin{document}
 
\begin{frame}
\titlepage %{CHAPITRE 2 - LES IDENTITÉS MULTIPLES DE LA PERSONNE}
\end{frame}

\begin{frame}
\tableofcontents
\end{frame}

% rapport puissances émergentes- mondialisation : 
%- insertion dans la mondialisation du à leur montée en puissance
% - effets paradoxaux de cette insertion sur les T.
% fil directeur : tension entre ouverture au monde et creusement des inégalités.

% définition des puissances émergentes : poids croissant ds éco mondiale, insertion rapide à la mondialisation (adoption de pol favorables à leur ouverture commerciale et financière).
% réduction de pauvreté globale ms creusement des inégalités sociales à ttes les échelles spatiales.
% renforcement de pois sur scène internationale : attribut de hard et soft power.

\section{I/ Etude de cas : les conséquences de la mondialisation sur l'organisation du territoire indien}

%Savoir : 
%- localiser et situer les gds pays émergents sur un planisphère
%- décrire et expliquer les caractéristiques essentielles d'un pays émergent.
% Localiser et situer au moins 3 métropoles 
% realiser un croquis rendant compte des gds traits de l'org du T

\subsection{Les territoires de l'ouverture : métropole et régions motrices.}
% métrololes et régions motrice : transformations spectaculaires. Mumbai : capitale éco de l'Etat. ville mondiale (modernisation de gde ampleur, menée dep peu de temps). Neaux centres des métropoles et leurs audaces architecturales, neaux ports et aéroports ou extensions : ex de insertion dans la mondialisation.
% des gdes métropoles s'ancrent sur puissantes régions éco qu'elles entraînent, en situation de façade maritime et bien reliée au monde / flux.
\begin{frame}
\begin{beamerboxesrounded}[scheme=blocimage]{Doc 2. L'Inde: richesse et dynamiques.} 
\includegraphics[scale=0.30]{inde.jpg}
\end{beamerboxesrounded}
\end{frame}

\begin{frame}
Situe l'Inde et la ville de Mumbai.
\end{frame}

\begin{frame}
\begin{beamerboxesrounded}[scheme=blocimage]{Doc 1. Un territoire en mutation} 
Mumbay présente au début du XXIe siècle tous les éléments pour devenir une ville mondiale jouant un rôle essentiel entre l'Inde et le monde. Créée à la fin des années 1990, la région métropolitaine de "Greater Mumbai" a vu la construction en moins de 20 ans d'un port, de la zone franche de Santacruz et de plusieurs autres en cours, d'un ensemble industriel, d'une ville nouvelle et d'espaces résidentiels. Cette extension justifie la volonté de développer, dans la partie centrale de la nouvelle agglomération, la zone industrielle à Andheri et d'édifier un centre d'affaires au sud de l’aéroport, avec une zone résidentielle empiétant sur le bidonville de Dharavi, l'un des plus grands d'Asie.
\begin{flushright}
P. Cadène, \textit{Atlas de l'Inde}, 2008.
\end{flushright}
\end{beamerboxesrounded}
\end{frame}

\begin{frame}
Liste les différents aménagements qui permettent à Mumbai de devenir une ville mondiale. 
Dans un croquis, quel figuré utilise-tu pour montrer une ville mondiale ?
\end{frame}

\begin{frame}
\begin{beamerboxesrounded}[scheme=blocimage]{Doc 2. L'Inde: richesse et dynamiques.} 
\includegraphics[scale=0.30]{inde.jpg}
\end{beamerboxesrounded}
\end{frame}

\begin{frame}
Cite les autres villes qui ont un rôle mondial.
Que remarques-tu par rapport à leur positionnement géographique ?
Par quel figuré la carte montre-t-elle leurs liens avec le monde ?
Quel lien fais-tu entre région motrice et métropole ?
Où sont les régions qui ne bénéficient pas de la mondialisation ?

\end{frame}

\begin{frame}
L'Inde est un pays émergent. Ses métropoles et ses régions motrices subissent des transformations spectaculaire. Mumbai est la capitale économique du pays. C'est une ville mondiale qui a subi une modernisation de grande ampleur depuis peu de temps. Les nouveaux centres de ces métropoles indiennes (Mumbai, Calcutta, Delhi) et leurs audaces architecturales, les nouveaux ports (Mumbai, Chennai) et aéroports sont des exemples de l'insertion de l'Inde dans la mondialisation.
Ces grandes métropoles s'ancrent sur de puissantes régions économiques qu'elles entraînent, qui se situent en façade maritime et bien reliée au monde par les flux.

% métrololes et régions motrice : transformations spectaculaires. Mumbai : capitale éco de l'Etat. ville mondiale (modernisation de gde ampleur, menée dep peu de temps). Neaux centres des métropoles et leurs audaces architecturales, neaux ports et aéroports ou extensions : ex de insertion dans la mondialisation.
% des gdes métropoles s'ancrent sur puissantes régions éco qu'elles entraînent, en situation de façade maritime et bien reliée au monde / flux.
\end{frame}

\subsection{Une intégration sélective des territoires à la mondialisation}

% accélération du creusement des inégalités socio-spatiales. Inde et surtt Chine st passé d'une situation assez égalitaire fin 80's, à une situation d'inélaité très marquée fin 90's. ces puissances comptent parmi les sociétés les + inégalitaires du monde.
%- entre \textbf{régions + riches}, métropoles et régions maritimes / région de intérieur, plus rurales et enclavées. Nuance : dynamiques de diffusion de croissance et de ouvertue vers intérieur.
% ex : Inde : 
%* contraste nets entre métropoles mondialisées (Mumbay, Bangalore), régions riches / pauvres
% * ligne SO/NE (quadrilatère de pauvreté au NE, Etats du Madhyar Pradesh au Bihar et Orissa.)

% - échelle intra urb : nlle classe d'entreprineurs et cadres. acteurs mondialisés, vivant ds quartiers huppé et barricadés côtoient perdants de mond (migrants ruraux exclus, proportions d'urbains précarisés).

% revers de ouverture : pression accrue et dégradation de environnement.

\begin{frame}
doc 3 p.274
\end{frame}

\begin{frame}
La mondialisation touche-t-elle autant les urbains que les ruraux ?
\end{frame}

\begin{frame}
doc 4 p.275
\end{frame}

\begin{frame}
Dans un premier paragraphe, tu racontes les progrès qui ont permis à la population indienne de mieux vivre actuellement. Dans un second paragraphe, tu montres que ces progrès ne concernent cependant pas toute la population indienne.
\end{frame}

\subsection{Croquis : Les conséquences de la mondialisation sur l'organisation du territoire indien.}
\begin{frame}
Classe les informations que tu as rassemblé en deux partie. Donne un titre à chaque partie de la légende.

Légende : 

I/ Les contrastes de développement
Inde dynamiques : littoraux et la vallée du Gange
Inde intermédiaire (agriculture prédominante)
Espaces en marges


II/ Une ouverture inégale sur le monde.
Littoral ouvert sur l'espace mondial
Métropoles
grandes villes
pôles industriels majeurs en relations avec le reste du monde
puissance économique proche.



\end{frame}

\section{Quels sont les critères pour définir les "pays émergents" ? }

Carte p. 281.
Exercice : 
Liste les pays émergents selon chaques continents.
Qui sont les BRIC ?

Ce que je doit savoir : %- localiser et situer les gds pays émergents sur un planisphère

Exercice : définir une notion. p. 287


%- décrire et expliquer les caractéristiques essentielles d'un pays émergent.


% Mise en perspective : planisphère voir voir critères (pop, PIB/hab, part du PIB mondial, IDH...)
% identification d'un groupe de pays émergents, hétérogène et à géométrie variable : noyau dur d'Etat, en mesure de contester l'ordre de la puissance mondiale.
% constat : monde polycentrique.


 \begin{frame}
Le groupe des pays émergent est hétérogène. Les pays n'ont pas tous les mêmes caractéristiques. Cependant, tous sont en mesure de contester l'ordre de puissance mondiale établi.

Organigramme.
 
  
 \end{itemize}
\end{frame}
  \end{document}