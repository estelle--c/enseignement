\documentclass{beamer}
  %\usepackage[utf8]{inputenc}
  \usepackage{fontspec}  %pour xelatex
 \usepackage{xunicode}  %pour xelatex
  \usetheme{default}
   \usepackage{color}
   \usepackage{graphicx}
   \usepackage{ulem}
%   \usepackage{xkeyval}
   \usepackage{pst-tree}
   \usepackage{tabularx}
   \usepackage[french]{babel}
 %  \usepackage{pstcol,pst-fill,pst-grad}
  \setbeamercolor{normal text}{fg=blue}
 \setbeamercolor{section in head/foot}{fg=black}
  \setbeamercolor{subsection in head/foot}{fg=blue}
\beamerboxesdeclarecolorscheme{blocbleu}{black!60!white}{black!20!white}
\beamerboxesdeclarecolorscheme{blocimage}{black!60!white}{black!10!white}
\setbeamercolor{section in toc}{fg=black}
\setbeamercolor{subsection in toc}{fg=blue}



\AtBeginSection[]
{
 \begin{frame}
  \tableofcontents[currentsection,hideallsubsections]
 \end{frame}
}

\AtBeginSubsection[]
{
  \begin{frame}
  \tableofcontents[currentsection,currentsubsection]
  \end{frame}
}
 
      \title{{\textcolor{red}{ %Partie 2 - Les territoires de la mondialisation.
      \\Chapitre 2 - Les pays pauvres}}}

\begin{document}
 
 % localiser et situer l'état étudié et sa capitale
 % qq PMA
 
 % Décrire et expliquer les caractéristiques essentielles d'1 PMA à partir de l'exemple de l'état étudié.
 
 \section{I/ Qu'est-ce qu'un PMA ? Exemple du Mali}

Questions : 
1) Carte p. 290. Situe le Mali et Bamako.

Exercice : les caractéristiques d'un PMA : Le Mali

\textcolor{red}{\underline{PMA} : Pays les moins avancés. Voir leurs caractéristiques dans le tableau.}

\section{II/ Où sont les PMA ?}
Carte p. 296-297 (à revoir pour le devoir)

Questions : où se trouvent la plupart des PMA ? Quels sont les deux pays sortis de cette catégorie.


  \end{document}