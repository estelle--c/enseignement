  \documentclass{beamer}
  %\usepackage[utf8]{inputenc}
  \usepackage{fontspec}  %pour xelatex
 \usepackage{xunicode}  %pour xelatex
  \usetheme{default}
   \usepackage{color}
   \usepackage{graphicx}
   \usepackage{ulem}
%   \usepackage{xkeyval}
   \usepackage{pst-tree}
   \usepackage{tabularx}
   \usepackage[french]{babel}
 %  \usepackage{pstcol,pst-fill,pst-grad}
  \setbeamercolor{normal text}{fg=blue}
 \setbeamercolor{section in head/foot}{fg=black}
  \setbeamercolor{subsection in head/foot}{fg=blue}
\beamerboxesdeclarecolorscheme{blocbleu}{black!60!white}{black!20!white}
\beamerboxesdeclarecolorscheme{blocimage}{black!60!white}{black!10!white}
\setbeamercolor{section in toc}{fg=black}
\setbeamercolor{subsection in toc}{fg=blue}



\AtBeginSection[]
{
 \begin{frame}
  \tableofcontents[currentsection,hideallsubsections]
 \end{frame}
}

\AtBeginSubsection[]
{
  \begin{frame}
  \tableofcontents[currentsection,currentsubsection]
  \end{frame}
}
 
      \title{{\textcolor{red}{Partie 2 - Les territoires de la mondialisation.
      \\Chapitre 1 - Les États-Unis }}}

\begin{document}
 
\begin{frame}
\titlepage %{CHAPITRE 2 - LES IDENTITÉS MULTIPLES DE LA PERSONNE}
\end{frame}

\begin{frame}
\tableofcontents
\end{frame}

\section{I/ Les Etats-Unis dans la mondialisation : première puissance mondiale}
\subsection{Le hard power}

\begin{frame}
Activité p. 248
Exercice 1. Doc 1 à 6 : reproduisez le tableau suivant et complétez-le.
\end{frame}

\begin{frame}

hard power : capacité d'influencer le comportement d'autre pays pays la contrainte, la coercition, voire par la violence.



Les Etats-Unis sont présents sur tous les continents et dans la quasi-totalité des activités. Première puissance militaire au monde (41 \% des dépenses mondiales en 2009), ils disposent de FTN parmi les plus puissantes. Elles investissement sur ts les continent et les USA sont la première destination des investissement étrangers. Ils sont aussi le premier exportateur mondial de produits agricoles. Mais ils sont concurrencés par des pays émergents comme la Chine.

Les Etats-unis sont qualifiée de "\textcolor{red}{superpuissance}". Son \textcolor{red}{\textit{hard power}} lui confèrent tous les éléments de la puissance : un territoire avec de fortes capacités, une population, une puissance économique, politique et militaire.

Les Etats-Unis sont le modèle dominant de la mondialisation, un moteur.

\textcolor{red}{\underline{Puissance} : capacité d'un Etat à influence les autres pays du monde.}

\textcolor{red}{\underline{Superpuissance} : nation dont le rayonnement économique, culturel politique et militaire est prééminent dans le monde. Elle est capable d'influencer des évènements à l'échelle mondiale.}


\end{frame}

--> étude de planisphères et exemples spatialisés pr dvper qq aspect de puissance us dans monde.
- insister sur spécificités, son aptitude à configuer hard et soft power : justifie qu'elle soit qualifiée de superpuissance.
hard power : éléments trad de puissance : T, pop, puissance éco, pol et militaire.
soft power : "puissance douce", dévpée / Joseph Nye en 90 : "capacité d'arriver à ses fins par un pouvoir de séduction et d'attirance plutôt que par la menace et le marchandage". prédominance scientifique (ac dvpt technologies de l'info et de Internet, permet diffusion rapide et planétaire de information, de culture - surtt musicale et ciné - du mode de vie à l'am.)


\section{II/ Les conséquences de la mondialisation sur l'organisation du territoire.}

  \end{document}