\documentclass[12pt,a4paper]{exam}
\printanswers % Pour ne pas imprimer les réponses (énoncé)
\addpoints % Pour compter les points
\pointsinrightmargin % Pour avoir les points dans la marge à droite
%\bracketedpoints % Pour avoir les points entre crochets
%\nobracketedpoints % Pour ne pas avoir les points entre crochets
\pointformat{.../\textbf{\themarginpoints}}
% \noaddpoints % pour ne pas compter les points
%\qformat{\textbf{Question\thequestion}\quad(\thepoints)\hfill} % Pour définir le style des questions (facultatif)
%\qformat{\thequestiontitle \dotfill \thepoints}


\usepackage{fontspec}
\usepackage{amsmath}
\usepackage{amssymb}
\usepackage{wasysym}
\usepackage{marvosym}
\usepackage{cwpuzzle}
  \usepackage{graphicx}
\defaultfontfeatures{Mapping=tex-text}
%\setmainfont{Linux Libertine}
\setmainfont{Century Schoolbook L}
%\usepackage[margin=1cm]{geometry}
    \usepackage[francais]{babel}
    \title{DS - Histoire - Les débuts de l'Islam}
   \usepackage[left=0.5cm,right=2cm,top=0.5cm,bottom=0.5cm]{geometry}
     
    % Si on imprime les réponses
    \ifprintanswers
    \newcommand{\rep}[1]{}
    \newcommand{\chariot}{}
    \else
    \newcommand{\rep}[1]{\fillwithdottedlines{#1}}
    \newcommand{\chariot}{\newpage}
    \fi



\makeatletter
\renewcommand\section{\@startsection
{section}{1}{0mm}    
{\baselineskip}
{0.5\baselineskip}
{\normalfont\normalsize\textbf}}
\makeatother
%\usepackage{titling}
%\renewcommand{\maketitlehooka}
\begin{document}

\begin{minipage}{19cm}
  Nom : \hspace{3cm}  Prénom : \hspace{3cm}   Classe : 4e2 \hspace{3cm}  Date : 
\end{minipage}
\hfill
%\begin{minipage}{3.5cm}

%{\small \begin{questions} \question[1] Orthographe et expression
%\question[1] Présentation \end{questions}
%}
%\end{minipage}


\vspace{1cm}

\begin{center}

{\Large Contrôle de connaissances}

\vspace{0.5cm}
  \end{center}

 \hfill {\large …/\numpoints\ } %\quad\quad …/\textbf{20}
 
 \begin{questions}
 \question[2] Explique la notion de "monarchie absolue de droit divin" \\
 
 
 \fillwithdottedlines{3cm}
 
 \begin{solution}
 \vspace{3cm}
 \end{solution}
 
 \question[1] Cite deux des trois catégories de population vivant au XVIIIe siècle
  \fillwithdottedlines{2cm}
   
   \begin{solution}
   \vspace{2cm}
   \end{solution}
  
   \question[2] Explique pourquoi la société du XVIIIe siècle est une société inégalitaire.
   \fillwithdottedlines{3cm}
      
      \begin{solution}
      \vspace{3cm}
      \end{solution}
 \end{questions}


 
 
\end{document}