 \documentclass{beamer}
  %\usepackage[utf8]{inputenc}
  \usepackage{fontspec}  %pour xelatex
 \usepackage{xunicode}  %pour xelatex
  \usetheme{Montpellier}
   \usepackage{color}
   \usepackage{xcolor}
   \usepackage{graphicx}
   \usepackage{ulem}
%   \usepackage{xkeyval}
   \usepackage{pst-tree}
   \usepackage{tabularx}
   \usepackage[french]{babel}
 %  \usepackage{pstcol,pst-fill,pst-grad}
  \setbeamercolor{normal text}{fg=black}
 \setbeamercolor{section in head/foot}{fg=black}
  \setbeamercolor{subsection in head/foot}{fg=blue}
\beamerboxesdeclarecolorscheme{blocbleu}{black!60!white}{black!20!white}
\beamerboxesdeclarecolorscheme{blocimage}{black!60!white}{black!10!white}
\setbeamercolor{section in toc}{fg=black}
\setbeamercolor{subsection in toc}{fg=blue}
\date{}
%\usepackage{titlesec}
\usepackage{soul}


\AtBeginSection[]
{
 \begin{frame}
  \tableofcontents[currentsection,hideallsubsections]
 \end{frame}
}

\AtBeginSubsection[]
{
  \begin{frame}
  \tableofcontents[currentsection,currentsubsection]
  \end{frame}
}
  \title{{\textcolor{red}{Chapitre 2 - L'Europe des Lumières}}}

\usepackage{ifthen}
\makeatletter
\renewcommand\section{\@startsection
{section}{1}{0mm}
{\baselineskip}
{0.5\baselineskip}
{\normalfont\normalsize\textbf}}
\makeatother


\begin{document}

\newboolean{Professeur}
\setboolean{Professeur}{true} % « true» (vrai) si le document est le document du professeur (sans trous). « Professeur » a la valeur « false » par défaut. Il faut donc décommenter la ligne pour mettre « Professeur » à « true »
\newcommand{\Trouer}[1]{
\ifthenelse{\boolean{Professeur}} % si « Professeur » est vrai,
{\textbf{#1}} %les mots cachés sont en gras
{\underline{\phantom{#1.5}}} % (else) sinon les mots sont remplacés par une ligne sur laquelle l'élève peut écrire.
}

\newcommand{\df}[2]{\textcolor{red}{\underline{#1}: #2}}

\newcommand{\doc}[1]{
\begin{flushright}
\fbox{Documents : #1}
\end{flushright}
}

\newcommand{\con}[1]{\textcolor{blue}{\underline{Consigne}: #1}}

\newcommand{\rep}[1]{\textcolor{green}{\underline{Réponse}: #1}}

\begin{frame}
 \titlepage %{CHAPITRE 2 - LES IDENTIT�S MULTIPLES DE LA PERSONNE}
 \end{frame}
 \begin{frame}
 Fil directeur : Comprendre les idées des Lumières et leur revendication en suivant la vie d'un personnage-clé du XVIIIe : Jean le Rond d'Alembert (1717-1783).
 \end{frame}

\begin{frame}
\begin{figure}

\includegraphics[width=5cm]{Alembert.jpg}

\caption{Portrait de d'Alembert d'après Quentin de La Tour.
(vers 1753)}
\end{figure}



\end{frame}

\section{I/ Je né et je grandis sous une monarchie absolue}

\begin{frame}
Brain storming : la monarchie absolue / la société des trois ordres.

\rep{Voir que le régime de Louis XV est un régime de \textcolor{red}{Monarchie absolue de droit divin.}} \\
\rep{Voir que la société du XVIIIe est toujours basée sur une division en trois ordres (noblesse, clergé, TE). La noblesse et le clergé ayant tous les pvrs (politiques, ne paient pas d'impôt.)} \\
\rep{Prob qui se pose : la société a changé. nlle catégorie sociales apparaissent (les bourgeois, liés aux marchannds. S'enrichissent. tendent à vouloir des postes auxquel leur situation ne les tend pas. tendent a vouloir plus de pouvoir.) Pcpalement des personnes de ces familles de robes (pas ancienns noblesses ms liés au monde de la justice, notaires, parlements). Ont nlles idées qui ont du mal à s'imposer.} \\
\end{frame}

\begin{frame}{Louis XV, roi de France}
\begin{figure}
\includegraphics[width=5cm]{Louis-15.jpg}
\caption{Portrait par Louis-Michel van Loo}
\end{figure}
\end{frame}

\begin{frame}{Naître de parents inconnus.}


\begin{figure}
\fbox{
\begin{minipage}{11cm}
Fruit d'un amour illégitime entre la célèbre femme de lettres Claudine Guérin de Tencin et le chevalier Louis-Camus Destouches [...], Jean le Rond D'Alembert naît le 16 novembre 1717 à Paris. Le lendemain, il est abandonné par sa mère qui le fait porter sur les escaliers de la chapelle Saint-Jean-le-Rond, attenant à la tour nord de Notre-Dame-de-Paris. Comme le veut la coutume, il est nommé du nom du saint de la chapelle et devient Jean Le Rond. Il est d'abord placé à l'hospice des Enfants-Trouvés, mais son père le retrouve rapidement et le place dans une famille d'adoption.

\begin{flushright}
www.wikipedia.org
\end{flushright}

\end{minipage}
}

\caption{Naissance de Jean le Rond d'Alembert}
\end{figure}
\end{frame}

\begin{frame}
\con{
de quel milieu social provient D'Alembert ? Dans quel milieu social a-t-il été élevé ?
D'où vient son prénom ?
}

\rep{Voir qu'à la base, milieu social bourgeois --> élevé. Famille de lettrés. Mais Abandonné -> aurait pu rester à orphelinat. Adopté par une famille modeste ms son père lui donne une rente afin qu'il puisse faire des études.}

\end{frame}


\section{II/ Avec mes amis du courant des Lumières, je réfléchis à changer la société.}



\begin{frame}{Un intellectuel de son temps.}
\begin{figure}
\fbox{
\begin{minipage}{11cm}
A 21 ans, il présente à l'Académie des Sciences, son premier travail en mathématique [...]. Dès 1742, à 24 ans, il est nommé adjoint de la section d’Astronomie de l’Académie des sciences. \\
Ami de Voltaire, il est aussi un homme de lettres. Il dirige la rédaction de l'Encyclopédie avec Diderot.En 1754, D’Alembert est élu membre de l’Académie française, dont il deviendra le secrétaire perpétuel le 9 avril 1772. 
\begin{flushright}
www.wikipedia.org
\end{flushright}

\end{minipage}
}

\caption{Un homme cultivé de son temps.}
\end{figure}
\end{frame}

\begin{frame}
D'Alembert a plusieurs métiers : \Trouer{mathématicien}, \Trouer{philosophe}, \Trouer{lettrés...}. Il participe au cercle de ces \Trouer{bourgeois urbains} qui réfléchissent à la \Trouer{société}. Il est ami avec tous les grands penseurs influents de son temps dont \Trouer{Voltaire} et \Trouer{Diderot}.
\end{frame}



\begin{frame}{Les nouvelles idées des Lumières }
%L'idée de tolérance pour tous, malgré leurs religions. Idée d'égalité et de justice pour tous. Idée que tt le monde doit payer des impôts.

\doc{3 p. 27 + p. 29}


Au XVIIIe siècle, les élites commencent à dénoncer quelques travers de la monarchie absolue de droit divin. Ils font des critiques : 

Tableau -> explique contre quoi s'insurge chaque intellectuels

\end{frame}

\begin{frame}
\begin{figure}
\includegraphics[width=10cm]{doc1.jpg}

\caption{Certaines idées des Lumières}
\end{figure}
\end{frame}

\begin{frame}
\begin{figure}
\includegraphics[width=10cm]{doc2.jpg}

\caption{voltaire et l'affaire Calas}
\end{figure}
\end{frame}


\begin{frame}
Méthode de présentation de document. \\
-> Voir les différentes natures \\
-> comprendre que un document peux ne pas être contemporain des éléments qu'il montre.
\end{frame}

\section{III/ Les lieux de diffusion de ces nouvelles idées}

\begin{frame}{Les salons et les académies, vecteurs des nouvelles idées}
\doc{2 p. 27}

\con{Présente ce document. Décrit ce tableau.}

\end{frame}

\begin{frame}
\begin{figure}
\includegraphics[width=10cm]{doc3.jpg}

\caption{Le salon littéraire et philosophique de Mme Geoffrin}
\end{figure}
\end{frame}


\begin{frame}
Au XVIIIe siècle, les \Trouer{intellectuels et savants} discutent des idées nouvelles dans de nombreux lieux de rencontres : des \Trouer{salons} comme celui de Mme Geoffrin, des \Trouer{académies}... Des savants de toute l'\Trouer{Europe} participent à ses réunion. Cela permet de diffuser les nouvelles idées.

\vfill

\df{Salons}{L'élite cultivé d'une ville se réunit dans le salon d'une personne afin d'échanger leurs idées et se distraire.}

\vfill

\df{Académie}{Assemblée de gens de lettres, savants et artistes. Ils publient des livres.}

\end{frame}

\begin{frame}{L'Encyclopédie permet de "populariser" les idées des Lumières.}{La naissance de l'Encyclopédie.}
\fbox{
\begin{minipage}{11cm}
"Le but d'une encyclopédie est de rassembler les connaissances éparses sur la surface de la terre, d'en exposer le système général aux hommes avec qui nous vivons et de le transmettre aux hommes qui viendront après nous. J'ai dit qu'il n'appartenais qu'à un siècle de philosophe de tenter une encyclopédie et je le dis parce que cet ouvrage demande partout plus de hardiesse dans l'esprit qu'on en a communément. Il faut tout examiner, tout remuer sans exception et sans ménagement. Il faut fouler aux pieds toutes ces vieilles puérilités, renverser les barrières que la raison n'aura point posées, rendre aux sciences et aux arts une liberté précieuse."

\begin{flushright}
Diderot et d'Alembert, prospectus de l'Encylcopédie, 1750
\end{flushright}
\end{minipage}
}
\end{frame}

\begin{frame}
\begin{itemize}
\item D'Alembert est chargé en 1745 de traduire une encyclopédie anglaise. En fait, va en écrire une nlle, avec Diderot.
\item D'Alembert va écrire la plupart des articles sur les maths, l'astronomie et la physique -> rédige près de 1700 articles. Il y critique sévèrement l'inquisition et se moque des éclesiatiques.
\end{itemize}

\end{frame}

\begin{frame}
Parmi les livres qui permettent de diffuser les idées nouvelles, l'\Trouer{Encyclopédie} est le plus important. Ecrit par \Trouer{Diderot} et \Trouer{d'Alembert} en langue française, il est vendu en nombre restreint, mais dans toute l'Europe. Il finit par être \Trouer{censuré} par le roi de France.

\vfill

\df{Encylopédie}{Ouvrages en une 30aine de tomes (des tomes de définitions et des tomes de dessins) réalisé au milieu du XVIIIe siècle. Il regroupe toutes les nouvelles recherches et les nouvelles idées de ce temps.}

\vfill

\df{Censure}{Interdiction de vendre et d'avoir certains ouvrages jugés en désaccords avec la politique royale.}
\end{frame}



  \end{document}