\documentclass[12pt]{article}
\usepackage{fontspec}
\usepackage{xltxtra}
\setmainfont[Mapping=tex-text]{Century Schoolbook L}
\usepackage[francais]{babel}
\usepackage{geometry}
\geometry{ hmargin=0.5cm, vmargin=0.5cm }
\usepackage{ifthen}
\makeatletter
\renewcommand\section{\@startsection
{section}{1}{0mm}
{\baselineskip}
{0.5\baselineskip}
{\normalfont\normalsize\textbf}}
\makeatother
\begin{document}
\begin{large}
\newboolean{Professeur}
%\setboolean{Professeur}{true} % « true» (vrai) si le document est le document du professeur (sans trous). « Professeur » a la valeur « false » par défaut. Il faut donc décommenter la ligne pour mettre « Professeur » à « true »
\newcommand{\Trouer}[1]{
\ifthenelse{\boolean{Professeur}} % si « Professeur » est vrai,
{\textbf{#1}} %les mots cachés sont en gras
{\underline{\phantom{#1.5}}} % (else) sinon les mots sont remplacés par une ligne sur laquelle l'élève peut écrire.
}

D'Alembert a plusieurs métiers : \Trouer{mathématicien}, \Trouer{philosophe}, \Trouer{lettrés...}. Il participe au cercle de ces \Trouer{bourgeois urbains} qui réfléchissent à la \Trouer{société}. Il est ami avec tous les grands penseurs influents de son temps dont \Trouer{Voltaire} et \Trouer{Diderot}.

\vfill

Au XVIIIe siècle, les \Trouer{intellectuels et savants} discutent des idées nouvelles dans de nombreux lieux de rencontres : des \Trouer{salons} comme celui de Mme Geoffrin, des \Trouer{académies}... Des savants de toute l'\Trouer{Europe} participent à ses réunion. Cela permet de diffuser les nouvelles idées.

\vfill

Parmi les livres qui permettent de diffuser les idées nouvelles, l'\Trouer{Encyclopédie} est le plus important. Ecrit par \Trouer{Diderot} et \Trouer{d'Alembert} en langue française, il est vendu en nombre restreint, mais dans toute l'Europe. Il finit par être \Trouer{censuré} par le roi de France.


\vfill

D'Alembert a plusieurs métiers : \Trouer{mathématicien}, \Trouer{philosophe}, \Trouer{lettrés...}. Il participe au cercle de ces \Trouer{bourgeois urbains} qui réfléchissent à la \Trouer{société}. Il est ami avec tous les grands penseurs influents de son temps dont \Trouer{Voltaire} et \Trouer{Diderot}.

\vfill

Au XVIIIe siècle, les \Trouer{intellectuels et savants} discutent des idées nouvelles dans de nombreux lieux de rencontres : des \Trouer{salons} comme celui de Mme Geoffrin, des \Trouer{académies}... Des savants de toute l'\Trouer{Europe} participent à ses réunion. Cela permet de diffuser les nouvelles idées.

\vfill

Parmi les livres qui permettent de diffuser les idées nouvelles, l'\Trouer{Encyclopédie} est le plus important. Ecrit par \Trouer{Diderot} et \Trouer{d'Alembert} en langue française, il est vendu en nombre restreint, mais dans toute l'Europe. Il finit par être \Trouer{censuré} par le roi de France.


\end{large}
\end{document}