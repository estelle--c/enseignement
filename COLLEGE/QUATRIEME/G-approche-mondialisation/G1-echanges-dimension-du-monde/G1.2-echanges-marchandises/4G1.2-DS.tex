\documentclass[a4paper,12pt]{exam}

\printanswers % Pour ne pas imprimer les réponses (énoncé)
\addpoints % Pour compter les points
\pointsinrightmargin % Pour avoir les points dans la marge à droite
%\bracketedpoints % Pour avoir les points entre crochets
%\nobracketedpoints % Pour ne pas avoir les points entre crochets
\pointformat{.../\textbf{\themarginpoints}}
% \noaddpoints % pour ne pas compter les points
%\qformat{\textbf{Question\thequestion}\quad(\thepoints)\hfill} % Pour définir le style des questions (facultatif)
%\qformat{\thequestiontitle \dotfill \thepoints}


\usepackage{fontspec}
\usepackage{amsmath}
\usepackage{amssymb}
\usepackage{wasysym}
\usepackage{marvosym}
\usepackage{cwpuzzle}
  \usepackage{graphicx}
\defaultfontfeatures{Mapping=tex-text}
%\setmainfont{Linux Libertine}
\setmainfont{Century Schoolbook L}
%\usepackage[margin=1cm]{geometry}
    \usepackage[francais]{babel}
    \title{DS - Histoire - Les débuts de l'Islam}
   \usepackage[left=0.5cm,right=2cm,top=0.5cm,bottom=0.5cm]{geometry}
     
    % Si on imprime les réponses
    \ifprintanswers
    \newcommand{\rep}[1]{}
    \newcommand{\chariot}{}
    \else
    \newcommand{\rep}[1]{\fillwithdottedlines{#1}}
    \newcommand{\chariot}{\newpage}
    \fi


\makeatletter
\renewcommand\section{\@startsection
{section}{1}{0mm}    
{\baselineskip}
{0.5\baselineskip}
{\normalfont\normalsize\textbf}}
\makeatother
%\usepackage{titling}
%\renewcommand{\maketitlehooka}

 
\begin{document}

\begin{minipage}{4cm}
  Nom :
  
  Prénom :
  
  Classe : 
  
  Date : 
\end{minipage}
\hfill
\begin{minipage}{3.5cm}

{\small \begin{questions} \question[1] Orthographe et expression
\question[1] Présentation \end{questions}
}
\end{minipage}


\vspace{1cm}

\begin{center}

{\Large DS - Histoire - Les Lumières}

\vspace{0.5cm}
  \end{center}

 \hfill {\large …/\numpoints\ } %\quad\quad …/\textbf{20}

Attention, tu dois rédiger des phrases pour répondre aux questions. Tu n'auras pas les points si ce n'est pas le cas.



\begin{questions}
\section{Je sais rédiger un texte bref, cohérent (compétence C1)}
 
 \question[4] Raconte un épisode marquant de la vie de Jean le Rond d'Alembert. Explique en quoi il est révélateur du siècle des Lumières.
 
 \begin{solution}
 \vspace{3cm}
 \end{solution}
 
 \section{Je sais présenter un document.}
 
 \includegraphics[width=19cm]{doc3.jpg}
 
 \fbox{
 \begin{minipage}{19cm}

 \end{minipage}
 }
 
 \question[4] Présente le document.
  
  \begin{solution}
  \vspace{3cm}
  \end{solution}

 \question[3] Contexte de l'oeuvre représentée : Explique ce qu'est un "salon" et son rôle dans la diffusion des nouvelles idées.
  
  \begin{solution}
  \vspace{3cm}
  \end{solution}
 
 
 \question[2] Contexte de la peinture : Explique que ce tableau ne montre pas une réalité vécue mais une image de la bonne société du XVIIIe siècle.
  
  \begin{solution}
  \vspace{3cm}
  \end{solution}
 
 
 
 \end{questions}
 
\end{document}