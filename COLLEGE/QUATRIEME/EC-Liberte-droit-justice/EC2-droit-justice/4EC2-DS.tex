\documentclass[a4paper,12pt]{exam}

%printanswers % Pour ne pas imprimer les réponses (énoncé)
\addpoints % Pour compter les points
\pointsinrightmargin % Pour avoir les points dans la marge à droite
%\bracketedpoints % Pour avoir les points entre crochets
%\nobracketedpoints % Pour ne pas avoir les points entre crochets
\pointformat{.../\textbf{\themarginpoints}}
% \noaddpoints % pour ne pas compter les points
%\qformat{\textbf{Question\thequestion}\quad(\thepoints)\hfill} % Pour définir le style des questions (facultatif)
%\qformat{\thequestiontitle \dotfill \thepoints}


\usepackage{fontspec}
\usepackage{amsmath}
\usepackage{amssymb}
\usepackage{wasysym}
\usepackage{marvosym}
\usepackage{cwpuzzle}
  \usepackage{graphicx}
\defaultfontfeatures{Mapping=tex-text}
%\setmainfont{Linux Libertine}
\setmainfont{Century Schoolbook L}
%\usepackage[margin=1cm]{geometry}
    \usepackage[francais]{babel}
    \title{DS - Histoire - L'Orient ancien}
   
     
    % Si on imprime les réponses
    \ifprintanswers
    \newcommand{\rep}[1]{}
    \newcommand{\chariot}{}
    \else
    \newcommand{\rep}[1]{\fillwithdottedlines{#1}}
    \newcommand{\chariot}{\newpage}
    \fi


\makeatletter
\renewcommand\section{\@startsection
{section}{1}{0mm}    
{\baselineskip}
{0.5\baselineskip}
{\normalfont\normalsize\textbf}}
\makeatother
%\usepackage{titling}
%\renewcommand{\maketitlehooka}

 
\begin{document}

\begin{minipage}{4cm}
  Nom :
  
  Prénom :
  
  Classe : 
  
  Date : 
\end{minipage}
\hfill
\begin{minipage}{3.5cm}

{\small \begin{questions} \question[1] Orthographe et expression
\question[1] Présentation \end{questions}
}
\end{minipage}


\vspace{1cm}

\begin{center}

{\Large DS - Education civique - Droit et justice en France}

\vspace{0.5cm}
  \end{center}
Appréciation : \hfill {\large …/\numpoints\ } %\quad\quad …/\textbf{20}



\section*{Connaissances}

Parmi les affirmations suivantes, lesquelles sont exactes ? \\


\begin{questions}
\question[1] L'égalité.\\
  \begin{oneparchoices}
 \choice Devant la justice, les mineurs ont moins de droits que les majeurs.\\
 \choice Devant la justice, la loi est la même pour tous. \\
  %\CorrectChoice 70
  \end{oneparchoices}
  \vspace{0.5cm}
\question[1] La présomption d'innocence.\\
  \begin{oneparchoices}
 \choice Tant qu'une personne n'a pas été reconnue coupable par le jugement d'un tribunal, elle est présumée innocente. \\
 \choice Placer une personne en détention, dans l'attente de son procès, signifie qu'elle est reconnue coupable.\\
  %\CorrectChoice 70
  \end{oneparchoices}
  \vspace{0.5cm}
\question[1] Les voies de recours\\
  \begin{oneparchoices}
 \choice Un jugement prononcé par le tribunal correctionnel est toujours définitif.
 \choice Un justiciable qui n'a pas obtenu satisfaction devant les juridictions nationales peut saisir la Cour européenne des droits de l'homme de Strasbourg. \\
  %\CorrectChoice 70
  \end{oneparchoices}

\question[1] Qu'est-ce que le débat contradictoire ?
 % \begin{solution}
      % dispersion d'un peuple à travers le monde.
      % \end{solution}
      \rep{2cm}


\section*{Etude de documents}

\begin{minipage}{8cm}
\includegraphics[scale=0.20]{ds1.jpg}

\end{minipage}
\fbox{
\begin{minipage}{8cm}
\textbf{doc 2. Protéger et punir} \\
12 personnes impliquées dans le trafic de faux tableaux étaient poursuivies par le tribunal correctionnel de Créteil pour << imitation de signature en vue de tromper l'acheteur >>, << escroquerie en bande organisée >> et << faux et usage de faux >>. Une vingtaine de victimes s'étaient portée partie civile. Interpellée un an plus tard, un artiste peintre résidant dans le Val-de-Marne aurait reconnu être l'auteur des faux tableaux.\\
Le tribunal a condamné les douze prévenus : 5 ans de prison dont 30 mois avec sursis pour le galeriste qui dirigeait le trafic; 3 ans de prison dont 2 avec sursis pour le peintre.
\begin{flushright}
D'après www.france2.fr, 17 juillet 2010
\end{flushright}
\end{minipage}
}

\textbf{Document 1.} \\
\question[1] Qui sont les parties en conflit ?
 % \begin{solution}
      % dispersion d'un peuple à travers le monde.
      % \end{solution}
      \rep{2cm}
\question[1] Quel tribunal a arbitré ce conflit ? 
 % \begin{solution}
      % dispersion d'un peuple à travers le monde.
      % \end{solution}
      \rep{1cm}
\question[2] Explique en quelques lignes le but de ce tribunal.
 % \begin{solution}
      % dispersion d'un peuple à travers le monde.
      % \end{solution}
      \rep{2cm}
\question[1] Quel jugement a-t-il prononcé.
 % \begin{solution}
      % dispersion d'un peuple à travers le monde.
      % \end{solution}
      \rep{1cm}

\textbf{Document 2.}
\question[1] Qui sont les justiciables opposés dans cette affaire de faux tableaux ?
 % \begin{solution}
      % dispersion d'un peuple à travers le monde.
      % \end{solution}
      \rep{2cm}
\question[1] Devant quel tribunal l'affaire est-elle portée ?
 % \begin{solution}
      % dispersion d'un peuple à travers le monde.
      % \end{solution}
      \rep{1cm}
\question[2] Explique en quelques lignes le but de ce tribunal.
 % \begin{solution}
      % dispersion d'un peuple à travers le monde.
      % \end{solution}
      \rep{2cm}
\question[1] Pourquoi différentes peines sont-elles prononcées ?
 % \begin{solution}
      % dispersion d'un peuple à travers le monde.
      % \end{solution}
      \rep{2cm}

 \section*{Développement écrit}
 
 \question[5] Explique en quoi le droit permet à la société de fonctionner (utilise des exemples du cours ou de la vie courante, ainsi que la définition du droit)
% \begin{solution}
      % dispersion d'un peuple à travers le monde.
      % \end{solution}
      \rep{10cm}
\end{questions}

\end{document}