\documentclass[12pt]{article}
\usepackage{fontspec}
\usepackage{xltxtra}
\setmainfont[Mapping=tex-text]{Century Schoolbook L}
 \usepackage[francais]{babel}
 
 \usepackage{geometry}
 \geometry{ hmargin=0.5cm, vmargin=0.5cm } 

\makeatletter
\renewcommand\section{\@startsection
{section}{1}{0mm}    
{\baselineskip}
{0.5\baselineskip}
{\normalfont\normalsize\textbf}}
\makeatother


\begin{document}

\fbox{\textbf{Les symboles de la justice.}} \\

\begin{minipage}{8cm}{Doc 1. Allégorie de la Justice.}
 \includegraphics[scale=0.40]{doc1.jpg}\\
\tiny Allégorie : forme de représentation indirecte (chose, personne) comme signe d'une autre chose.)
\end{minipage}
\begin{minipage}{10cm}
\includegraphics[scale=0.70]{justice.eps}
\end{minipage}

\vspace{0.3cm}

\begin{minipage}{19cm}
\begin{enumerate}
\item  Situe sur la photographie les attributs de la justices suivants (qui représentent la justice) : le glaive, la balance, les yeux bandés, le genoux dénudé.
\item Explique ce qu'est une << Allégorie >> \\

\item Rattache chaque symbole à son interprétation. \\

\end{enumerate}
\end{minipage}

\vfill

\fbox{\textbf{Le centre équestre de Villebois contre Mme Gagnère.}}

\begin{enumerate}
\item L'audience
\begin{enumerate}
\item  Qu'a fait Mme Gagnère ? \\

\item Qu'a fait le président du club après l'accident ? \\

\item  Y-a-t'il lors du procès une procédure contradictoire ? (Les deux partis peuvent-t'il faire entendre leurs arguments ?)\\

\item  Y-a-t'il des avocats lors des sessions des prud'hommes ?\\
\end{enumerate}

\item La conciliation. 
\begin{enumerate}
\item Cela se passe-t-il dans la salle du tribunal ? \\

\item Quel est le but de cette réunion ? Résoudre certaines problèmes sans passer par la salle d'audience. \\
 
\end{enumerate}

\item Le délibéré.
\begin{enumerate}
\item Cela se passe-t-il dans la salle du tribunal ? \\
 
\item Que décident les juges ? \\
\end{enumerate}

\end{enumerate}

\fbox{
\begin{minipage}{8cm}{Que dit le droit ?}
Art. L511-1. Les conseils de prud'hommes, juridiction élective{\tiny 1} et paritaire{\tiny 2}, règlent par voie de conciliation les différends qui peuvent s'élever à l'occasion de tout contrat de travail entre les employeurs [...] et les salariés [...]. Ils jugent les différends à l'égard desquels la conciliation n'a pas abouti. \\

{\tiny 1 et 2. Les conseillers prud'homaux sont élus pour 5 ans par les employeurs et les salariés. Le bureau de jugement est composé de 2 conseillers représentant les salariés et de 2 conseillers représentant les employeurs.}

\begin{flushright}
Code du travail.
\end{flushright}
\end{minipage}
}
\begin{minipage}{10cm}
\begin{enumerate}
\item Quel texte de loi est la base lors des audiences des Prud'hommes ? \\

\item Entre qui et qui se passe les prud'hommes ? \\

\item Qui sont les conseillers prud'homaux ? \\

\end{enumerate}
\end{minipage}
 %Code du travail
 %Art. 511-1 : Les conseils de prud'hommes [...] règlent par voie de conciliation les différends qui peuvent s'élever [...] entre les employeurs [...] et les salariés [...]. Ils jugent les litiges lorsque la conciliation n'a pas abouti. 
\vspace{0.5cm}

\fbox{Titre} \\
Situation : coupure de presse. Jugement et résultat.

\fbox{
\begin{minipage}{8cm}
Art. 8. La loi ne doit établir que des peines strictement et évidemment nécessaires, et nul ne peut être puni qu'en vertu d'une loi [...] légalement appliquée. \\
\begin{flushright}
Déclaration des droits de l'homme et du citoyen, 1789
\end{flushright}
\end{minipage}
}

\newpage

\fbox{Innocent ou coupable ?} \\

\fbox{
\begin{minipage}{10cm}
Art. 8. La loi ne doit établir que des peines strictement et évidemment nécessaires, et nul ne peut être puni qu'en vertu d'une loi établi et promulguée antérieurement au délit, et légalement appliquée. \\
\begin{flushright}
Déclaration des droits de l'homme et du citoyen, 1789. \\
\end{flushright}

Art. 66. Nul ne peux être arbitrairement (=illégalement) détenu. L'autorité judiciaire, gardienne de la liberté individuelle, assure le respect de ce principe dans des conditions prévues par la loi ? \\
\begin{flushright}
Constitution de la Ve République, 1958. \\
\end{flushright}

Toute personne accusée d'un acte délictueux est présumée innocente jusqu'à ce que sa culpabilité ait été légalement établie au cours d'un procès public où toutes les garanties nécessaires à sa défense lui auront été assurées.\\
\begin{flushright}
Art. 11 de la Déclaration universelle des droits de l'homme, 1948.
\end{flushright}
\end{minipage}
}
\begin{minipage}{8cm}
\begin{enumerate}
\item Lorsqu'il passe en procès, une personne est-elle coupable ou innocente ? \\

\item Souligne en rouge ce qui montre qu'on ne être accusé sans preuve.
\item Si une loi passe en 2014, peut-on être accusé d'un délit qu'on aurait commis en 2010 ?
\vspace{7cm}
\end{enumerate}
\end{minipage}

\newpage
\fbox{
\begin{minipage}{18cm}{Le jury populaire, une garantie démocratique}\\
<<Art. 255 : Peuvent seuls remplir les fonctions de juré, les citoyens de l'un ou de l'autre sexe, âgés de plus de vingt-trois ans, sachant lire et écrire en français, jouissant des droits politiques, civils et de famille, et ne se trouvant dans aucun cas d'incapacité ou d'incompatibilité énumérés par les deux articles suivants [personnes privées de ce droit à cause d'une procédure judiciaire, personnes exerçant certaines fonctions politiques]. >>
\begin{flushright}
Code de procédure pénale.
\end{flushright}
\end{minipage}
}



\fbox{
\begin{minipage}{19cm}{Peut-on refuser d'être juré à une cour d'assises ?}\\
<< Les jurés à une cour d'assises sont tirés au sort parmi les personnes inscrites sur les listes électorales et convoqués pour une sessions d'assises. L'absence d'un juré le jour de l'audience sans motif légitime est passible d'une amende de 3750 euros. L'employeur est dans l'obligation de libérer le salarié de ses obligations professionnelles. >>
\begin{flushright}
http://vosdroits.service-public.fr/F1044.html
\end{flushright}
\end{minipage}
}

\begin{enumerate}
\item Tout le monde peut-il être juré ? \\

\item Quel âge faut-il avoir pour pouvoir être juré ? \\

\item Souligne le passage qui montre qu'on ne peut pas refuse, ni nous, ni l'employeur, d'être juré.
\end{enumerate}






\end{document}