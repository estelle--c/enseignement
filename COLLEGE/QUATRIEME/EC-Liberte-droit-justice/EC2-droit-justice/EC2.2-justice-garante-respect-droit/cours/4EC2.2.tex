 \documentclass{beamer}
  %\usepackage[utf8]{inputenc}
  \usepackage{fontspec}  %pour xelatex
 \usepackage{xunicode}  %pour xelatex
  \usetheme{Montpellier}
   \usepackage{color}
   \usepackage{graphicx}
   \usepackage{ulem}
%   \usepackage{xkeyval}
   \usepackage{pst-tree}
   \usepackage{tabularx}
   \usepackage[french]{babel}
 %  \usepackage{pstcol,pst-fill,pst-grad}
  \setbeamercolor{normal text}{fg=black}
 \setbeamercolor{section in head/foot}{fg=black}
  \setbeamercolor{subsection in head/foot}{fg=blue}
\beamerboxesdeclarecolorscheme{blocbleu}{black!60!white}{black!20!white}
\beamerboxesdeclarecolorscheme{blocimage}{black!60!white}{black!10!white}
\setbeamercolor{section in toc}{fg=black}
\setbeamercolor{subsection in toc}{fg=blue}



\AtBeginSection[]
{
 \begin{frame}
  \tableofcontents[currentsection,hideallsubsections]
 \end{frame}
}

\AtBeginSubsection[]
{
  \begin{frame}
  \tableofcontents[currentsection,currentsubsection]
  \end{frame}
}
  \title{{\textcolor{red}{Chapitre 2 - La justice est garante du respect du droit}}}

\begin{document}
\begin{frame}
 \titlepage %{CHAPITRE 2 - LES IDENTIT�S MULTIPLES DE LA PERSONNE}
 \end{frame}

 
 \begin{frame}
 \tableofcontents
 \end{frame}
 
 \begin{frame}{Exercice. L'allégorie de la Justice.}
\includegraphics[scale=0.60]{doc1.jpg}
 \end{frame}
 
 \begin{frame}
 \underline{But du chapitre} : comprendre quelles sont les missions et les principes de la Justice à travers trois chambres de justices différentes.
 \end{frame}
 
 \section{I/ Le conseil des prud'hommes.}
 
 \begin{frame}{Exercice. Le centre équestre de Villebois contre Mme Gagnère.}
 
 \end{frame}
 
 %Vidéo. 1) J-> 3'30
 %2) conciliation  3'30
  %justimemo.justice.gouv.fr/JustMemo.php?id=85
 
 \begin{frame}
 Le conseil des prud'hommes arbitre les conflits du travail. Il est original pour 2 raisons : 
 \begin{itemize}
 \item la présence de juges élus
 \item les procédures de conciliation : essayer de trouver une solution sans passer par la salle du tribunal.
 \end{itemize}
 \end{frame}
% I/ Le conseil des prod'hummes arbitre les conflits du travail.
 %2 aspects ft qu'il est original : présence de juges élus + procédure de conciliation.
 
 
 %Prezi sur conseil des prudhommes. règle litige entre employés et employeurs (salaires, )
 %juridiction paritaire (juge issu du monde travail). non professionnel élus par les salariés et les employeurs.
 %5 sections (cadres et salariés, industrie, commerce, agriculture, activités diverses).
 %essait de négocier entre les 2 parties. Sinon, audience.
 %applique règle du code du Travail.
 

%Allégorie --> missions de la Justice (protéger, punir, arbitrer). Symboles (glaive et balance) + extraits de textes constitutionnels co doc ref (art. de DDHC)

%Justice agit en suivant des pcpes : procédure contradictoire, droits de la défense, non rétroactivité des lois et présomption d'innocence.

%Attache tps à présomption d'innocence. Débats particuliers car régulièrement bafouée. 

%Pcpes et missions doivent service de fil rouge à étude des trois juridictions proposées.

\section{II/ Le tribunal correctionnel.}
\begin{frame}{La situation}

\end{frame}

\begin{frame}{Etre innocent.}

\end{frame}

\begin{frame}
Le tribunal correctionnel juge les délits. Les infractions sont graves (pouvant encourir la prison). Les procès sont fréquents et régulièrement couverts dans la presse locale. \\
La justice interprète le Droit. Lecture en prenant en compte la situation de la personne en accusation, l'intervention de la défense, les droits des victimes, la pression sociale...
\end{frame}

\section{III/ La cours d'assise}

\begin{frame}{La situation}

\end{frame}

\begin{frame}{Exercice : être juré lors d'un procès}

\end{frame}

\begin{frame}
La cours d'assises juge les crimes. La présence d'un jury populaire en fait une exception dans les cours de justice.
\end{frame}

%III/ La cours d'assises juge les crimes. présence d'1 jury populaire (magistrats + tirés au sort parmi listes électorales.)

%Y intégrer dans une la lecture d'un jugement ou de son résultat. Mise en relation entre le Droit et le jugemen rendu.



 \begin{frame}{conclusion}
 La Justice a trois mission (protéger, punir, arbitrer) qu'elle pratique dans différentes institutions. Elle n'est pas exemplaire mais essai d'aider aux mieux les citoyens.
 
 \end{frame}

  \end{document}