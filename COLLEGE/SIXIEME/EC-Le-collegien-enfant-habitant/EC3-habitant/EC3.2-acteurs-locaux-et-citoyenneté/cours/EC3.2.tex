  \documentclass{beamer}
  %\usepackage[utf8]{inputenc}
  \usepackage{fontspec}  %pour xelatex
 \usepackage{xunicode}  %pour xelatex
  \usetheme{default}
   \usepackage{color}
   \usepackage{graphicx}
   \usepackage{ulem}
%   \usepackage{xkeyval}
   \usepackage{pst-tree}
   \usepackage{tabularx}
   \usepackage[french]{babel}
 %  \usepackage{pstcol,pst-fill,pst-grad}
  \setbeamercolor{normal text}{fg=blue}
 \setbeamercolor{section in head/foot}{fg=black}
  \setbeamercolor{subsection in head/foot}{fg=blue}
\beamerboxesdeclarecolorscheme{blocbleu}{black!60!white}{black!20!white}
\beamerboxesdeclarecolorscheme{blocimage}{black!60!white}{black!10!white}
\setbeamercolor{section in toc}{fg=black}
\setbeamercolor{subsection in toc}{fg=blue}



\AtBeginSection[]
{
 \begin{frame}
  \tableofcontents[currentsection,hideallsubsections]
 \end{frame}
}

\AtBeginSubsection[]
{
  \begin{frame}
  \tableofcontents[currentsection,currentsubsection]
  \end{frame}
}
 
      \title{{\textcolor{red}{Partie 2 - L'habitant.
      \\Chapitre 2 - Les acteurs locaux et la citoyenneté }}}

\begin{document}
 
\begin{frame}
\titlepage %{CHAPITRE 2 - LES IDENTITÉS MULTIPLES DE LA PERSONNE}
\end{frame}

\begin{frame}
\tableofcontents
\end{frame}

\section{I/ Participer à la vie de la commune par le biais d'associations}
\subsection{L'association x}

\subsection{Le fonctionnement d'une association}
\begin{frame}
\begin{beamerboxesrounded}[scheme=blocimage]{Doc 1. Le fonctionnement d'une association} 
Une association est organisée par la loi du 1er juillet 1901. Elle se compose de \underline{membres adhérents} qui payent une cotisation annuelle. Une \textit{assemblée générale} (ou AG) se réunit au moins une fois par an et comprend tous les membres de l'association. Les décisions de l'AG sont prises à la majorité des membres présents. La gestion de l'association est assurée par un bureau élu, composé par un \underline{président} (qui représente l'association), un \underline{secrétaire} qui administre l'association et un \underline{trésorier} (qui gère les comptes). Ces personnes doivent être majeures.
\end{beamerboxesrounded}
\end{frame}

\begin{frame}
Quelle loi organise le fonctionnement d'une association (doc 1)?

En utilisant les mots soulignés dans le doc 1, complète le schéma ci-contre.

\end{frame}

\section{II/ Contribuer au respect de l'environnement et du cadre de vie : l'agenda 21}

  \end{document}