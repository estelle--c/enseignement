  \documentclass{beamer}
  %\usepackage[utf8]{inputenc}
  \usepackage{fontspec}  %pour xelatex
 \usepackage{xunicode}  %pour xelatex
  \usetheme{default}
   \usepackage{color}
   \usepackage{graphicx}
   \usepackage{ulem}
%   \usepackage{xkeyval}
   \usepackage{pst-tree}
   \usepackage{tabularx}
   \usepackage[french]{babel}
 %  \usepackage{pstcol,pst-fill,pst-grad}
  \setbeamercolor{normal text}{fg=blue}
 \setbeamercolor{section in head/foot}{fg=black}
  \setbeamercolor{subsection in head/foot}{fg=blue}
\beamerboxesdeclarecolorscheme{blocbleu}{black!60!white}{black!20!white}
\beamerboxesdeclarecolorscheme{blocimage}{black!60!white}{black!10!white}
\setbeamercolor{section in toc}{fg=black}
\setbeamercolor{subsection in toc}{fg=blue}



\AtBeginSection[]
{
 \begin{frame}
  \tableofcontents[currentsection,hideallsubsections]
 \end{frame}
}

\AtBeginSubsection[]
{
  \begin{frame}
  \tableofcontents[currentsection,currentsubsection]
  \end{frame}
}
 
      \title{{\textcolor{red}{Partie 2 - L'habitant.
      \\Chapitre 2 - Les acteurs locaux et la citoyenneté }}}

\begin{document}
 
\begin{frame}
\titlepage %{CHAPITRE 2 - LES IDENTITÉS MULTIPLES DE LA PERSONNE}
\end{frame}

\begin{frame}
\tableofcontents
\end{frame}

\section{I/ Participer à la vie de la commune par le biais d'associations}
\subsection{L'association x}

\subsection{Le fonctionnement d'une association}
\begin{frame}
\begin{beamerboxesrounded}[scheme=blocimage]{Doc 1. Le fonctionnement d'une association} 
Une association est organisée par la loi du 1er juillet 1901. Elle se compose de \underline{membres adhérents} qui payent une cotisation annuelle. Une \textit{assemblée générale} (ou AG) se réunit au moins une fois par an et comprend tous les membres de l'association. Les décisions de l'AG sont prises à la majorité des membres présents. La gestion de l'association est assurée par un bureau élu, composé par un \underline{président} (qui représente l'association), un \underline{secrétaire} qui administre l'association et un \underline{trésorier} (qui gère les comptes). Ces personnes doivent être majeures.
\end{beamerboxesrounded}
\end{frame}

\begin{frame}
Quelle loi organise le fonctionnement d'une association (doc 1)?

En utilisant les mots soulignés dans le doc 1, complète le schéma ci-contre.

\end{frame}

\section{II/ Contribuer au respect de l'environnement et du cadre de vie : le Plan Climat Energie Territorial et l'agenda 21}

\begin{frame}
\begin{beamerboxesrounded}[scheme=blocimage]{Doc 1. Le Plan Climat Energie Territorial (adopté en septembre 2013)} 
"Le climat change, nous devons tous agir ! Engagé dans l’élaboration d’un Plan Climat Energie Territorial depuis 2010, le Grésivaudan prend part aux engagements nationaux et européens dans les domaines de l’énergie et du climat. Ce projet ambitieux vise à réduire les émissions de gaz à effet de serre et les consommations d’énergie sur notre territoire, et préparer à l’adaptation du Grésivaudan aux impacts du changement climatique. Le Plan Climat s’inscrit naturellement dans la démarche Agenda 21 mise en œuvre par Le Grésivaudan, dont il constitue le volet énergie-climat."

Plusieurs actions sont pensées : 
\begin{itemize}
\item Construction et rénovation performante
\item Isolation et choix de matériaux
\item Choix d'énergies
\item Energies renouvelables
\item Chauffage et eau chaude
\item Consommation responsable
\item Aides financières
\end{itemize}

\begin{flushright}
Source : www.le-gresivaudan.fr
\end{flushright}
\end{beamerboxesrounded}
\end{frame}

\begin{frame}
\begin{beamerboxesrounded}[scheme=blocimage]{Doc 2. Qu'est-ce qu'un agenda 21 ?} 
Ratifié au sommet de la Terre à Rio en 1992, l'Agenda 21 est l'exemple le plus important de mobilisation citoyenne; il constat les dérèglements qu'entraînent nos mode de vie et propose à tous de se mobiliser pour construire un monde plus responsable et juste.
\end{beamerboxesrounded}
\end{frame}

\begin{frame}
Les habitants des communes, doivent, en temps que citoyen, participer au respect de leur cadre de vie. Chaque commune ou communauté de commune (celle du Grésivaudan pour Pontcharra) doit mettre en place son "agenda 21". C'est un plan de route afin de réduire à l'environnement. La CC du Grésivaudan a en train de mettre en place l'Agenda 21. Il a crée le "Plan Climat Energie Territorial", qui compose un volet de l'agenda 21. Son but est de faire baisser les émissions de gaz à effet de serre et les consommations d'énergie dans la CC. 
Chacun doit participer à la protection de l'environnement à travers ce Plan ou par des moyens très simples (ex: le tri sélectif des déchets)
\end{frame}


  \end{document}