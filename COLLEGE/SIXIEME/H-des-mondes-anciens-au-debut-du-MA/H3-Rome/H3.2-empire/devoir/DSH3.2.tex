\documentclass[a4paper,12pt]{exam}

%printanswers % Pour ne pas imprimer les réponses (énoncé)
\addpoints % Pour compter les points
\pointsinrightmargin % Pour avoir les points dans la marge à droite
%\bracketedpoints % Pour avoir les points entre crochets
%\nobracketedpoints % Pour ne pas avoir les points entre crochets
\pointformat{.../\textbf{\themarginpoints}}
% \noaddpoints % pour ne pas compter les points
%\qformat{\textbf{Question\thequestion}\quad(\thepoints)\hfill} % Pour définir le style des questions (facultatif)
%\qformat{\thequestiontitle \dotfill \thepoints}

 %\usepackage[utf8]{inputenc} %pour pdflatex
      \usepackage{fontspec}  %pour xelatex
     % \usepackage{xunicode}  %pour xelatex
      \usepackage{graphicx}
\usepackage{amsmath}
\usepackage{amssymb}
\usepackage{wasysym}
\usepackage{marvosym}
\usepackage{cwpuzzle}
\defaultfontfeatures{Mapping=tex-text}
%\setmainfont{Linux Libertine}
\setmainfont{Century Schoolbook L}
%\usepackage[margin=1cm]{geometry}
    \usepackage[francais]{babel}
    \title{DS - Histoire - L'Orient ancien}
    
 
   
     
    % Si on imprime les réponses
    \ifprintanswers
    \newcommand{\rep}[1]{}
    \newcommand{\chariot}{}
    \else
    %\newcommand{\rep}[1]{\makeemptybox{#1}}
    \newcommand{\rep}[1]{\fillwithdottedlines{#1}}
    \newcommand{\chariot}{\newpage}
    \fi

 \usepackage{geometry}
 \geometry{ hmargin=1.8cm, vmargin=0.5cm } 

\makeatletter
\renewcommand\section{\@startsection
{section}{1}{0mm}    
{\baselineskip}
{0.5\baselineskip}
{\normalfont\normalsize\textbf}}
\makeatother
%\usepackage{titling}
%\renewcommand{\maketitlehooka}

 
\begin{document}

\begin{minipage}{4cm}
  Nom :
  
  Prénom :
  
  Classe : 
  
  Date : 
\end{minipage}
\hfill
\begin{minipage}{3.5cm}

{\small \begin{questions} \question[1] Orthographe et expression
\question[1] Présentation \end{questions}
}
\end{minipage}


\vspace{1cm}

\begin{center}

{\Large DS - Histoire - Rome - L'Empire}

\vspace{0.5cm}
  \end{center}
Appréciation : \hfill {\large …/\numpoints\ } %\quad\quad …/\textbf{20}

%Compétence à valider : \\
%\begin{tabular}{|c|c|c|c|}
%\hline  & Validé & En cours d'acquisition & Non validée \\ 
%\hline C5 : Avoir des outils \\pour comprendre l'unité \\et la complexité du monde &  &  &  \\ 
%\hline 
%\end{tabular} 

\section{Connaissances}
\begin{questions} % Début de l'examen. Débute la numérotation des questions
\question [1] Qui a fondé l'Empire ?\\
 \begin{oneparchoices}
\choice Auguste
 \choice Caracalla
 \choice Trajan
 %\CorrectChoice Auguste
 \end{oneparchoices}
 \vspace{0.5cm}

\question [1] Où est Rome ? \\
  \begin{oneparchoices}
 \choice En Grèce
  \choice En Italie
  \choice En Gaule
  %\CorrectChoice En Italie
  \end{oneparchoices}
  \vspace{0.5cm}
  
  \question [1] Quelle est la <<capitale>> des Trois Gaules ?\\
   \begin{oneparchoices}
  \choice Paris
   \choice Lyon
   \choice Marseille
   %\CorrectChoice Lyon
   \end{oneparchoices}
   \vspace{0.5cm}
 
   \question [1] En quelle année tous les habitants de l'Empire sont-ils devenus citoyens romains ?\\
    \begin{oneparchoices}
   \choice 78 av J.-C.
    \choice 212
    \choice 318
    %\CorrectChoice 212
    \end{oneparchoices}
    \vspace{0.5cm}

  \question [1] Comment les frontières de l'Empire sont-elles protégées ?\\
   \begin{oneparchoices}
  \choice Par une muraille
   \choice Par des barbelés
   %\CorrectChoice Par une muraille
   \end{oneparchoices}
   \vspace{0.5cm}
 
  \question[3] Cite les trois objectifs de la romanisation.
     % \begin{solution}
     % Solution
     % \end{solution}
     \rep{2cm}
 
\end{questions}
\newpage
\section{Décris une ville gallo-romaine : Arles }
 
 Compétence à valider : \\
 
<<<<<<< HEAD
=======
 ~ 
  
>>>>>>> 9f3685e4c48331fd3b113526b0400e868d25355d
  \begin{tabular}{|p{7cm}|c|c|c|}
  \hline  & Validée & En cours d'acquisition & Non validée \\ 
  \hline C5 : Je sais tirer des informations de différents documents : \par carte, plan, photo. & &  &  \\ 
  \hline 
  \end{tabular} \\
 
 
 \includegraphics[scale=0.20]{DS1.jpg}
 \includegraphics[scale=0.20]{DS2.jpg}\\
 \includegraphics[scale=0.20]{DS3.jpg}

 \begin{questions} 
 \question[1]  Doc 1. Trouves dans quelle province de la Gaule romaine Arles a-t-elle été fondée.
 % \begin{solution}
 % Solution
 % \end{solution}
 \rep{2cm}
 
 \question[1] Doc 1. Indique pourquoi Arles est bien située pour faire du commerce.
    % \begin{solution}
    % Solution
    % \end{solution}
    \rep{2cm}
   
 \question[1] Doc 3. Cite les deux grandes voies qui se croisent dans la ville.
  % \begin{solution}
  % Solution
  % \end{solution}
  \rep{3cm}
  
  \question[1] Doc 3. Liste les bâtiments de loisirs que tu voie dans le plan.
   % \begin{solution}
   % Solution
   % \end{solution}
   \rep{2cm}
 
 
 \question[2] Doc 2 et 3. Trouve les deux éléments de la ville antique indiqués sur le plan qui sont visibles sur la photographie. 
    % \begin{solution}
    % Solution
    % \end{solution}
    \rep{2cm}
    
\question[4] Décrit la ville gallo-romaine d'Arles.
   % \begin{solution}
   % Solution
   % \end{solution}
   \rep{7cm}
  \end{questions}
      
\end{document}