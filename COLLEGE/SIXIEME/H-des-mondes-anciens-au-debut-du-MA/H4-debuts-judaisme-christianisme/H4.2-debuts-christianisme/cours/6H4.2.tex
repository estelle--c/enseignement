 \documentclass{beamer}
  %\usepackage[utf8]{inputenc}
  \usepackage{fontspec}  %pour xelatex
 \usepackage{xunicode}  %pour xelatex
  \usetheme{Montpellier}
   \usepackage{color}
   \usepackage{graphicx}
   \usepackage{ulem}
   \usepackage{xcolor}
%   \usepackage{xkeyval}
   \usepackage{pst-tree}
   \usepackage{tabularx}
   \usepackage[french]{babel}
 %  \usepackage{pstcol,pst-fill,pst-grad}
  \setbeamercolor{normal text}{fg=black}
 \setbeamercolor{section in head/foot}{fg=black}
  \setbeamercolor{subsection in head/foot}{fg=blue}
\beamerboxesdeclarecolorscheme{blocbleu}{black!60!white}{black!20!white}
\beamerboxesdeclarecolorscheme{blocimage}{black!60!white}{black!10!white}
\setbeamercolor{section in toc}{fg=black}
\setbeamercolor{subsection in toc}{fg=blue}



\AtBeginSection[]
{
 \begin{frame}
  \tableofcontents[currentsection,hideallsubsections]
 \end{frame}
}

\AtBeginSubsection[]
{
  \begin{frame}
  \tableofcontents[currentsection,currentsubsection]
  \end{frame}
}
  \title{{\textcolor{red}{Partie 4 - Les débuts du judaïsme et du christianisme. \\ Chapitre 2 - Les débuts du Christianisme}}}

\begin{document}
\begin{frame}
 \titlepage %{CHAPITRE 2 - LES IDENTIT�S MULTIPLES DE LA PERSONNE}
 \end{frame}
 
 \begin{frame} Introduction.
 But du chapitre : 
 \end{frame}
 
 \begin{frame}
 \tableofcontents
 \end{frame}
     
     
    \section{ I/ L'élaboration de la nouvelle religion.}
  
  \subsection{A/ Où, et quand est apparu le christianisme ?}
  
  \begin{frame}
  \fbox{
  \begin{minipage}{11cm}
 {\small  << Comme les juifs provoquaient constamment des troubles à l'instigation de Chrestus (Christ), l'empereur les chassa de Rome >>.} \begin{flushright}
  {\small Suétone (auteur romain), v. 49}
  \end{flushright}
 {\small  
  << L'empereur Néron trouva des coupables tout indiqués qu'il soumit à des tortures exemplaires. Le peuple les appelait chrétiens. Ce nom leur venait de Christ, supplicié sous l'empereur Tibère par le gouverneur Ponce-Pilate. Leur funeste superstition avait été réprimée immédiatement mais elle avait refaisait surface, non seulement en Judée [à Jérusalem] mais aussi à Rome."}
  \begin{flushright}
 {\small  Tacite (auteur romain), Annales, Livre 44, vers 116}
  \end{flushright}
  
{\small   << Vous êtes tous fils de Dieu par la foi dans Jésus-Christ. Oui, vous tous qui avez été baptisés. Il n'y a plus ni Juif ni Grec; il n'y a plus ni esclave, ni homme libre; il n'y a plus l'homme et la femme, car tous vous être un dans Jésus-Christ.>>}
  \begin{flushright}
 {\small  Paul (auteur chrétien), \textit{Epître aux Galates}, 3, 26-28, milieu du Ier siècle.}
  \end{flushright}
  \end{minipage}
  }
  \end{frame}
  
  
  \begin{frame}
 \textcolor{blue}{Tu es un enquêteur. Tu dois prouver la naissance et l'existence des chrétiens. Pour y arriver, remplis le tableau ci-dessous (en classant les sources par ordre chronologique)}
 
 \begin{tabular}{|p{3cm}|p{3cm}|p{3cm}|}
 \hline Source & Lieu où auraient résidé les chrétiens & Date à laquelle ils y auraient résidé \\ 
 \hline \vspace{3cm} &  &  \\ 
 \hline \vspace{3cm} &  &  \\ 
 \hline \vspace{3cm} &  &  \\ 
 \hline 
 \end{tabular} 
  \end{frame}
  
  \begin{frame} Sur la carte : 
  \begin{enumerate}
  \item Grâce à la carte 4 p. 139, inscrit Jérusalem, Rome, la Palestine.
 % \item Colorie en rouge les zones des premières communautés chrétienne sur la carte.
  \end{enumerate}
  
%  \includegraphics[scale=0.20]{6H4.2-carte.eps}
  
  \end{frame}
  
  \begin{frame} Sur la carte : 
   \begin{enumerate}
  % \item Grâce à la carte 4 p. 139, inscrit Jérusalem, Rome, la Palestine.
   \item Colorie en rouge les zones des premières communautés chrétienne sur la carte.
   \end{enumerate}
   
 %  \includegraphics[scale=0.20]{6H4.2-carte.eps}
   
   \end{frame}
 
 \begin{frame}
 La religion chrétienne (christianisme) est issu du judaïsme. Elle se développe dans l'empire romain.
 \end{frame}
 
 
 \subsection{TD2. Jésus et les Evangiles}
 
 \begin{frame}\textbf{ Exercice 1. Qui est Jésus-Christ ?}
 \begin{enumerate}
 \item Repère et surligne sur la frise la vie de Jésus. 
% \item Vers quelle année est-il mort ?
 \end{enumerate}
 
 \includegraphics[scale=0.13]{doc1.jpg}
 \end{frame}
 
  \begin{frame} Correction
  \begin{enumerate}
  \item Repère et surligne sur la frise la vie de Jésus. 
 % \item Vers quelle année est-il mort ?
  \end{enumerate}
  
  \includegraphics[scale=0.13]{doc1-1.jpg}
  \end{frame}
  
  \begin{frame} Correction
    \begin{enumerate}
    \item %Repère et surligne sur la frise la vie de Jésus. 
   \item Vers quelle année est-il mort ? %\textcolor{black!70!green}{Vers 30.}
    \end{enumerate}
    
    \includegraphics[scale=0.13]{doc1-1.jpg}
    \end{frame}
 
    
 \begin{frame} \textbf{Exercice 2. Les évangiles, sources sur la vie de Jésus}
 \includegraphics[scale=0.60]{jesus11.eps} \\
 \end{frame}
   
   
  \begin{frame}
  Replace les termes donnés dans les cases : \\
   \begin{minipage}{5cm}
   \begin{itemize}
   \item Revenu à la vie
   \item Livre écrit par un disciple de Jésus
   \item Dieu
   \item Les autorités romaines
   \end{itemize}
   \end{minipage}
   \begin{minipage}{5cm}
   \begin{itemize}
   \item lieu de naissance de Jésus
   \item Selon la religion, sauveur qui viendra aider le peuple juif.
   \item les juifs de Jérusalem \\
   \end{itemize}
   \end{minipage}
   \end{frame}
   
    \begin{frame}
    \includegraphics[scale=0.60]{jesus10.eps} \\
    \end{frame}
   
   \begin{frame}
   Grâce aux explications, raconte la mort et la résurrection de Jésus.
   \end{frame}
   
  
 
     %Issue du judaïsme, s'en éloigne en fct° des circonstances (chute du Temple, reconstruction du judaïsme autour de la synagogue et du rabbinat qui sépare définitivement chemins christianisme et judaïsme.). Se dvpe ds contexte culturel et politique du monde gréco-romain (cités et empire). Peut s'y étende (christianisation de empire) car s'y adapte selon 1 "pcpe d'accomodement" (romanisation et impérialisation du christianisme)
    
    %TD1 - Pourquoi, où, et quand est apparu le christianisme. ?
     % \underline{\textbf{ Exercice 1. Paul et la naissance d'une nouvelle religion.}}
   
     
    \section{II/ De la persécution au succès de la religion chrétienne.}
     % : empire romain tolérant ac religion sauf si menace relation étroite religion des Romains - Cités (rites qui maintiennent pax deorum), confusion prêtrise-magistrature, culte de Rome et Empereur.
     %Or pratique des chrétien (foi exclusive) contredit cet équilibre. chrétien manquant à piétas qui assure pérennité de Cité et prospérité. Ds dimension officielle et institutionnelle, brutales flambée de persécution naissent de cet écart. Ms séparée / longues "paix de église".
     %deb du 4e : christianisme encore minoritaire dans l'empire, accès, au hasard de guerres civiles, au statut de religion privilégiée / pvr et religion officielle. empereur devient chrétien, empire devient chrétien.
     
    \subsection{ Exercice 1 - Je repère les trois phases d'acceptation du christianisme.}
     
     \begin{frame}
     \includegraphics[scale=0.13]{doc1.jpg}
          \begin{enumerate}
          \item Entoure en vert les trois phases d'évolution d'acceptation de la nouvelle religion.
          \end{enumerate}
     \end{frame}
     
     \begin{frame}
          \includegraphics[scale=0.13]{doc1-2.jpg}
               \begin{enumerate}
               \item Entoure en vert les trois phases d'évolution d'acceptation de la nouvelle religion.
               \end{enumerate}
          \end{frame}
     
    \subsection{ Exercice 2 - Raconte un épisode de persécution.}
 
 \begin{frame}
 \textbf{2 p. 136.} Cite dans le texte, le reproche majeur que Saturninus fait à Spératus. \textcolor{black!70!green}{}\\
      
      3 p. 136. Décrit la mosaïque. \\
      \textcolor{black!70!green}{}\\
 \end{frame}    
         
    \subsection{ Exercice 3 - Le christianisme religion officielle}
     
     \begin{frame}
      \fbox{
          \begin{minipage}{11cm}
   \small{ << Dans la  guerre civile qui s'était allumée, les forces de Maxence$ ^{1} $ étaient plus grandes que celles de l'empereur Constantin. Celui-ci s'approcha de Rome et campa au pont de Milvius. Dans son sommeil, \underline{il est averti de faire peindre sur les boucliers de ses soldats le signe de la} \underline{croix}. Il obéit. [...] Ses troupes, fortifiées par cette protection céleste, partent à l'assaut. Dieu favorisait Constantin : ses ennemis s'effraient. Maxence veut se sauver, mais emporter par la multitude des fuyards, il est précipité dans le Tibre. Après une si importante victoire, Constantin est reçu dans Rome avec l'applaudissement du Sénat et du peuple.>>} \\
          \begin{flushright}
          D'après Lactance, auteur chrétien, \textit{De la mort des persécuteurs}, début du IVe siècle.
          \end{flushright}
          
          $ ^{1 : concurrent de l'empereur Constantin.} $
          \end{minipage}
          }
          \begin{enumerate}
          \item Explique la phrase soulignée. \\
          
         % \item Surligne dans le texte ce qui rappelle l'empire romain vu dans les chapitre précédents.\\
          \end{enumerate}
     \end{frame}
    
     \begin{frame}
          \fbox{
              \begin{minipage}{11cm}
       \small{ << Dans la  guerre civile qui s'était allumée, les forces de Maxence$ ^{1} $ étaient plus grandes que celles de l'empereur Constantin. Celui-ci s'approcha de Rome et campa au pont de Milvius. Dans son sommeil, \underline{il est averti de faire peindre sur les boucliers de ses soldats le signe de la} \underline{croix}. Il obéit. [...] Ses troupes, fortifiées par cette protection céleste, partent à l'assaut. Dieu favorisait Constantin : ses ennemis s'effraient. Maxence veut se sauver, mais emporter par la multitude des fuyards, il est précipité dans le Tibre. Après une si importante victoire, Constantin est reçu dans Rome avec l'applaudissement du Sénat et du peuple.>>} \\
              \begin{flushright}
              D'après Lactance, auteur chrétien, \textit{De la mort des persécuteurs}, début du IVe siècle.
              \end{flushright}
              
              \tiny{1 : concurrent de l'empereur Constantin.}
              \end{minipage}
              }
              \begin{enumerate}
              \item %Explique la phrase soulignée. \\
              
              \item Surligne dans le texte ce qui rappelle l'empire romain vu dans les chapitres précédents.\\
              \end{enumerate}
         \end{frame}
     
       \begin{frame}
               \fbox{
                   \begin{minipage}{11cm}
            \small{ << Dans la  guerre civile qui s'était allumée, les forces de Maxence$ ^{1} $ étaient plus grandes que celles de l'\colorbox{yellow}{empereur} Constantin. Celui-ci s'approcha de Rome et campa au pont de Milvius. Dans son sommeil, \underline{il est averti de faire peindre sur les boucliers de ses soldats le signe de la} \underline{croix}. Il obéit. [...] Ses troupes, fortifiées par cette protection céleste, partent à l'assaut. Dieu favorisait Constantin : ses ennemis s'effraient. Maxence veut se sauver, mais emporter par la multitude des fuyards, il est précipité dans le Tibre. Après une si importante victoire, Constantin est reçu dans Rome avec \colorbox{yellow}{l'applaudissement du Sénat et du peuple}.>>} \\
                   \begin{flushright}
                   D'après Lactance, auteur chrétien, \textit{De la mort des persécuteurs}, début du IVe siècle.
                   \end{flushright}
                   
                   $ ^{1 : concurrent de l'empereur Constantin.} $
                   \end{minipage}
                   }
                   \begin{enumerate}
                   \item %Explique la phrase soulignée. \\
                   
                   \item Surligne dans le texte ce qui rappelle l'empire romain vu dans les chapitres précédents.\\
                   \end{enumerate}
              \end{frame}
 
 \begin{frame}
   \fbox{
      \begin{minipage}{11cm}
      << Nous, Constantin et Licinius$ ^{1} $ avons décidés d'accorder aux chrétiens et à tous les autres la liberté de pratiquer la religion qu'ils préfèrent [...]. >> \\
      
      \tiny{1 : l'empire romain est divisé en deux partie : l'empire d'occident dirigé par Constantin et celui d'Orient dirigé par Licinius}
      
      \begin{flushright}
      Extrait de l'édit de Milan, 313.
      \end{flushright}
      \end{minipage}
      }
      \begin{enumerate}
      \item Souligne en rouge dans la chronologie l'édit de Milan.
      \item Raconte en quelques ligne l'importance de l'empereur Constantin.
      \end{enumerate}
 \end{frame}    
 
 \begin{frame}
           \includegraphics[scale=0.13]{doc1-2.jpg}
               \begin{enumerate}
                    \item Entoure en rouge dans la chronologie l'édit de Milan.
                    %\item Raconte en quelques ligne l'importance de l'empereur Constantin.
                    \end{enumerate}
           \end{frame}
           
           \begin{frame}
                      \includegraphics[scale=0.13]{doc1-3.jpg}
                          \begin{enumerate}
                               \item Entoure en rouge dans la chronologie l'édit de Milan.
                               %\item Raconte en quelques ligne l'importance de l'empereur Constantin.
                               \end{enumerate}
                      \end{frame}
 
 \begin{frame}
          \fbox{
                \begin{minipage}{11cm}
                << Nous, Constantin et Licinius$ ^{1} $ avons décidés d'accorder aux chrétiens et à tous les autres la liberté de pratiquer la religion qu'ils préfèrent [...]. >> \\
                
                \tiny{1 : l'empire romain est divisé en deux partie : l'empire d'occident dirigé par Constantin et celui d'Orient dirigé par Licinius}
                
                \begin{flushright}
                Extrait de l'édit de Milan, 313.
                \end{flushright}
                \end{minipage}
                }
                \begin{enumerate}
                     \item %Entoure en rouge dans la chronologie l'édit de Milan.
                     \item Raconte en quelques ligne l'importance de l'empereur Constantin.
                     \end{enumerate}
            \end{frame}
 
 \begin{frame} \underline{\textcolor{blue}{Sur le cahier:}} \\
\textcolor{blue}{L'empire romain est tolérant avec toutes les religions, à la condition qu'elles ne menacent pas la religion officielle publique qui maintient la \textit{pax deorum}. La pratique religieuse des chrétiens contredit cet équilibre. Les persécutions en sont la conséquence. Au début du IVe siècle, le christianisme accède au au statut de religion officielle : l'empereur devient chrétien, l'empire devient chrétien.}
 \end{frame}     
   
     \section{III/ De l'Eglise des communautés à l'Eglise institutionnelle.}
     
    \subsection{ Exercice 1 - L'évolution du bâtiment "basilique", lieu de culte.}
     
     \begin{frame}
     \fbox{
     \begin{minipage}{11cm}
     \textbf{Doc 1. La basilique.} \\
     << A Rome, la basilique civile est située au coeur du forum. Elle a une double fonction : marché couvert et tribunal. [...]\\
     Au IVe siècle, les chrétiens rejettent les temples et prennent pour modèle la basilique civile qui permet d'accueillir tous les << fidèles >>, et est un symbole d'autorité. Enfin, en la choisissant pour modèle, les chrétiens placent l'édifice, désormais chrétien, au coeur de la vie civile, de la << cité de Dieu >>. >>
     \begin{flushright}
     D'après \textit{Art roman : de la Rome païenne à la Rome paléochrétienne}
     \end{flushright}
     \end{minipage}
     }
      \begin{minipage}{7cm}
     Choisi la bonne réponse: \\
     A quoi servait la basilique sous les Romains ? \\
     \begin{minipage}{5cm}
     O Un lieu de culte.
     \end{minipage}
     \begin{minipage}{5cm}
     O Un lieu de marché.
     \end{minipage}
     \begin{minipage}{5cm}
     O Un lieu de justice. \\
     \end{minipage}
     \end{minipage}
     \end{frame}
     
     \begin{frame}
          \fbox{
          \begin{minipage}{11cm}
          \textbf{Doc 1. La basilique.} \\
          << A Rome, la basilique civile est située au coeur du forum. Elle a une double fonction : marché couvert et tribunal. [...]\\
          Au IVe siècle, les chrétiens rejettent les temples et prennent pour modèle la basilique civile qui permet d'accueillir tous les << fidèles >>, et est un symbole d'autorité. Enfin, en la choisissant pour modèle, les chrétiens placent l'édifice, désormais chrétien, au coeur de la vie civile, de la << cité de Dieu >>. >>
          \begin{flushright}
          D'après \textit{Art roman : de la Rome païenne à la Rome paléochrétienne}
          \end{flushright}
          \end{minipage}
          }
           \begin{minipage}{7cm}
          Choisi la bonne réponse: \\
          A quoi servait la basilique sous les Romains ? \\
          \begin{minipage}{5cm}
          O Un lieu de culte.
          \end{minipage}
          \begin{minipage}{5cm}
          O \textbf{Un lieu de marché.}
          \end{minipage}
          \begin{minipage}{5cm}
          O Un lieu de justice. \\
          \end{minipage}
          \end{minipage}
          \end{frame}
         
               \begin{frame}
                    \fbox{
                    \begin{minipage}{11cm}
                    \textbf{Doc 1. La basilique.} \\
                    << A Rome, la basilique civile est située au coeur du forum. Elle a une double fonction : marché couvert et tribunal. [...]\\
                    Au IVe siècle, les chrétiens rejettent les temples et prennent pour modèle la basilique civile qui permet d'accueillir tous les << fidèles >>, et est un symbole d'autorité. Enfin, en la choisissant pour modèle, les chrétiens placent l'édifice, désormais chrétien, au coeur de la vie civile, de la << cité de Dieu >>. >>
                    \begin{flushright}
                    D'après \textit{Art roman : de la Rome païenne à la Rome paléochrétienne}
                    \end{flushright}
                    \end{minipage}
                    }
                     \begin{minipage}{7cm}
                    Choisi la bonne réponse: \\
                    A quoi servait la basilique sous les Romains ? \\
                    \begin{minipage}{5cm}
                    O Un lieu de culte.
                    \end{minipage}
                    \begin{minipage}{5cm}
                    O Un lieu de marché.
                    \end{minipage}
                    \begin{minipage}{5cm}
                    O \textbf{Un lieu de justice.} \\
                    \end{minipage}
                    \end{minipage}
                    \end{frame}

     \begin{frame}
      \fbox{
          \begin{minipage}{11cm}
          \textbf{Doc 1. La basilique.} \\
          << A Rome, la basilique civile est située au coeur du forum. Elle a une double fonction : marché couvert et tribunal. [...]\\
          Au IVe siècle, les chrétiens rejettent les temples et prennent pour modèle la basilique civile qui permet d'accueillir tous les << fidèles >>, et est un symbole d'autorité. Enfin, en la choisissant pour modèle, les chrétiens placent l'édifice, désormais chrétien, au coeur de la vie civile, de la << cité de Dieu >>. >>
          \begin{flushright}
          D'après \textit{Art roman : de la Rome païenne à la Rome paléochrétienne}
          \end{flushright}
          \end{minipage}
          }
    
     A quoi sert la basilique sous les chrétiens ? \\
     \begin{minipage}{5cm}
     O Un lieu de culte.
     \end{minipage}
     \begin{minipage}{5cm}
     O Un lieu de marché.
     \end{minipage}
     \begin{minipage}{5cm}
     O Un lieu de justice.
     \end{minipage}
  \end{frame}
  
   \begin{frame}
        \fbox{
            \begin{minipage}{11cm}
            \textbf{Doc 1. La basilique.} \\
            << A Rome, la basilique civile est située au coeur du forum. Elle a une double fonction : marché couvert et tribunal. [...]\\
            Au IVe siècle, les chrétiens rejettent les temples et prennent pour modèle la basilique civile qui permet d'accueillir tous les << fidèles >>, et est un symbole d'autorité. Enfin, en la choisissant pour modèle, les chrétiens placent l'édifice, désormais chrétien, au coeur de la vie civile, de la << cité de Dieu >>. >>
            \begin{flushright}
            D'après \textit{Art roman : de la Rome païenne à la Rome paléochrétienne}
            \end{flushright}
            \end{minipage}
            }
      
       A quoi sert la basilique sous les chrétiens ? \\
       \begin{minipage}{5cm}
       O \textbf{Un lieu de culte.}
       \end{minipage}
       \begin{minipage}{5cm}
       O Un lieu de marché.
       \end{minipage}
       \begin{minipage}{5cm}
       O Un lieu de justice.
       \end{minipage}
    \end{frame}
      
      \begin{frame} \textbf{Doc 2. Plan d'une basilique.}\\
      \begin{minipage}{7cm}
          \includegraphics[scale=0.40]{basilique.eps}
          \end{minipage}
          \begin{minipage}{3cm}
          \includegraphics[scale=0.08]{doc3.jpg}
          \end{minipage}
          
          En utilisant le schéma de droite, complète le croquis de la basilique avec les définitions suivantes : 
          \begin{itemize}
          \item Lieu où prient les fidèles
          \item Endroit où l'on se fait baptiser
          \item Endroit où le chef de l'Eglise dirige la messe
          \item Endroit où le prêtre officie
          \end{itemize}
      \end{frame}
      
      \begin{frame} Correction \\
           
                \includegraphics[scale=0.60]{basilique1.eps}
               
            \end{frame}
    
    
     
     
     \begin{frame}
     Au début, les fidèles sont surtout urbains. Ils se réunissent chez eux par petits groupes.  Avec l'empire romain devenu chrétien, l'Eglise devient plus institutionnelle : construction de basiliques chrétiennes. Ils s'organisent et définissent une hiérarchie pour assurer l'unité (l'évêque est la personne la plus importante).
     \end{frame}
     
    
  \end{document}