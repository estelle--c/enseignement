\documentclass[a4paper,12pt]{exam}

%printanswers % Pour ne pas imprimer les réponses (énoncé)
\addpoints % Pour compter les points
\pointsinrightmargin % Pour avoir les points dans la marge à droite
%\bracketedpoints % Pour avoir les points entre crochets
%\nobracketedpoints % Pour ne pas avoir les points entre crochets
\pointformat{.../\textbf{\themarginpoints}}
% \noaddpoints % pour ne pas compter les points
%\qformat{\textbf{Question\thequestion}\quad(\thepoints)\hfill} % Pour définir le style des questions (facultatif)
%\qformat{\thequestiontitle \dotfill \thepoints}


\usepackage{fontspec}
\usepackage{amsmath}
\usepackage{amssymb}
\usepackage{wasysym}
\usepackage{marvosym}
\usepackage{cwpuzzle}
  \usepackage{graphicx}
\defaultfontfeatures{Mapping=tex-text}
%\setmainfont{Linux Libertine}
\setmainfont{Century Schoolbook L}
%\usepackage[margin=1cm]{geometry}
    \usepackage[francais]{babel}
    \title{DS - Histoire - L'Orient ancien}
   
     
    % Si on imprime les réponses
    \ifprintanswers
    \newcommand{\rep}[1]{}
    \newcommand{\chariot}{}
    \else
    \newcommand{\rep}[1]{\fillwithdottedlines{#1}}
    \newcommand{\chariot}{\newpage}
    \fi


\makeatletter
\renewcommand\section{\@startsection
{section}{1}{0mm}    
{\baselineskip}
{0.5\baselineskip}
{\normalfont\normalsize\textbf}}
\makeatother
%\usepackage{titling}
%\renewcommand{\maketitlehooka}

 
\begin{document}

\begin{minipage}{4cm}
  Nom :
  
  Prénom :
  
  Classe : 
  
  Date : 
\end{minipage}
\hfill
\begin{minipage}{3.5cm}

{\small \begin{questions} \question[1] Orthographe et expression
\question[1] Présentation \end{questions}
}
\end{minipage}


\vspace{1cm}

\begin{center}

{\Large DS - Histoire - Les débuts du Christianisme}

\vspace{0.5cm}
  \end{center}
Appréciation : \hfill {\large …/\numpoints\ } %\quad\quad …/\textbf{20}

 \section*{Connaissances}
 
 \begin{questions} % Début de l'examen. Débute la numérotation des questions
 
 \question[3] Complète la chronologie ci-dessous \\
 
\includegraphics[scale=0.40]{ds1.jpg}

 Explique ce qu'est / qui est:  
  \question[1] une basilique chrétienne.
   % \begin{solution}
     % dispersion d'un peuple à travers le monde.
     % \end{solution}
     \rep{2cm}
       
    \question[1] Jésus
     % \begin{solution}
       % dispersion d'un peuple à travers le monde.
       % \end{solution}
       \rep{2cm}
\end{questions}

\newpage

\section*{Je sais me repérer sur une carte.}
\begin{questions}
 \question[2] Sur cette carte du monde romain au IVe siècle, replace la Palestine et Jérusalem.
\end{questions}
\includegraphics[scale=0.90]{carte.eps}

\section*{Développement écrit}
\begin{questions}
 \question[3] Raconte l'épisode de l'autorisation du christianisme par Constantin.

  % \begin{solution}
       % dispersion d'un peuple à travers le monde.
       % \end{solution}
       \rep{5cm}
\end{questions}

\newpage

\section*{Etude de document : la persécution des chrétiens accusés de l'incendie de Rome en 64}

\textit{En 64 après Jésus-Christ, un incendie ravage une bonne partie de la ville de Rome. L'empereur Néron accuse les chrétiens d'avoir allumés l'incendie. Il les persécute pour les punir.}

\fbox{
\begin{minipage}{17cm}
\textbf{Doc 1. La persécution des chrétiens par Néron }\\
<< L'empereur Néron infligea des tourments raffinés à ceux que leurs abominations faisaient détester et que la foule appelait chrétiens. \\
On commença donc par se saisir de ceux qui confessaient leur foi, puis sur leurs révélations, d'une multitude d'autres. On ne se contenta pas de les faire périr : on se fit le jeu de les revêtir de peaux de bêtes pour qu'ils fussent déchirés par les dents des chiens; ou bien ils étaient attachés à des croix, enduits de matières inflammables, et quand le jour avait fuit, ils éclairaient les ténèbres comme des torches. >>
\begin{flushright}
Tacite (consul romain), Annales, Livre XV, rédigé vers 117.
\end{flushright}
\end{minipage}


}

\begin{questions}
 \question[1] L'auteur de ce texte est-il chrétien ou paien ?
   % \begin{solution}
        % dispersion d'un peuple à travers le monde.
        % \end{solution}
        \rep{1cm}
 
 \question[1] Qu'est-ce qu'une persécution ?
   % \begin{solution}
        % dispersion d'un peuple à travers le monde.
        % \end{solution}
        \rep{2cm}
  \question[2] Raconte cette persécution.
    % \begin{solution}
         % dispersion d'un peuple à travers le monde.
         % \end{solution}
         \rep{2cm}
   \question[2] En quelle année a-t-elle eu lieu ? Replace cette persécution sur la chronologie ci-dessus.
     % \begin{solution}
          % dispersion d'un peuple à travers le monde.
          % \end{solution}
          \rep{1cm}
    \question[1] Replace Rome sur la carte donnée.
    \question[1] Raconte l'évènement déclencheur de cette persécution.
      % \begin{solution}
           % dispersion d'un peuple à travers le monde.
           % \end{solution}
           \rep{2cm}
     
     
     


 \end{questions}

\end{document}