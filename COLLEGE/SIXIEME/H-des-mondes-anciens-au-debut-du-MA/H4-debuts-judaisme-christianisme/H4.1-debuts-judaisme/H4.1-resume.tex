\documentclass[12pt]{article}
\usepackage{fontspec}
\usepackage{xltxtra}
\setmainfont[Mapping=tex-text]{Century Schoolbook L}
 \usepackage[francais]{babel}
 
 \usepackage{geometry}
 \geometry{ hmargin=0.5cm, vmargin=0.5cm } 

\makeatletter
\renewcommand\section{\@startsection
{section}{1}{0mm}    
{\baselineskip}
{0.5\baselineskip}
{\normalfont\normalsize\textbf}}
\makeatother


\begin{document}
\begin{center}
\textbf{{\large PARTIE 4 - Les débuts du judaïsme et du christianisme }\\ {\large Chapitre 1 - Les débuts du judaïsme. }}
\end{center}

\textbf{I/ Des royaumes hébreux au peuple Juifs.}

\vspace{0.5cm}
\setlength{\parindent}{1cm} Au VIIIe siècle av JC, deux royaumes hébreux coexistent au Proche-Orient : Israël et Juda. Ils ont une langue et une culture commune, se réclament d'un dieu national mais connaissent aussi le \underline{polythéisme}.\\

\setlength{\parindent}{1cm}Au VIIe siècle av JC, Josias, roi de Juda entreprend une réforme religieuse. Il centralise le culte d'un seul dieu à un seul lieu : le Temple de Jérusalem.\\

\setlength{\parindent}{1cm}Au VIe siècle, Le royaume de Juba et le \underline{Temple de Jérusalem} sont détruits. Les hébreux passent sous la domination de la civilisation perse. Les élites sont envoyés en \underline{exil} à Babylone. Après quelques temps, le chef Perse, Cyrus, autorise la population à revenir en Israël. Les juifs reconstruisent le temple. La province a perdu son indépendance. La région passe aux mains des Grecs puis des Romains. Après deux révoltes, les Romains détruisent le seconde Temple de Jérusalem (70 ap JC). Après cet épisode, les juifs s'éparpillent dans le monde méditerranéen : c'est la \underline{diaspora}.

\vspace{0.5cm}

\underline{Polythéisme} (Croire en plusieurs dieux) / \underline{monothéisme} (croire en un seul dieu).\\
\underline{Temple de Jérusalem} : bâtiment religieux majeurs dans la religion juive.\\
\underline{Exil} : être poussé à partir d'un endroit pour un autre, sans possibilité de revenir chez soi.\\
\underline{Diaspora} : dispersion d'un peuple à travers le monde.

\vfill

\begin{center}
\textbf{{\large PARTIE 4 - Les débuts du judaïsme et du christianisme }\\ {\large Chapitre 1 - Les débuts du judaïsme. }}
\end{center}

\textbf{I/ Des royaumes hébreux au peuple Juifs.}

\vspace{0.5cm}
\setlength{\parindent}{1cm} Au VIIIe siècle av JC, deux royaumes hébreux coexistent au Proche-Orient : Israël et Juda. Ils ont une langue et une culture commune, se réclament d'un dieu national mais connaissent aussi le \underline{polythéisme}.\\

\setlength{\parindent}{1cm}Au VIIe siècle av JC, Josias, roi de Juda entreprend une réforme religieuse. Il centralise le culte d'un seul dieu à un seul lieu : le Temple de Jérusalem.\\

\setlength{\parindent}{1cm}Au VIe siècle, Le royaume de Juba et le \underline{Temple de Jérusalem} sont détruits. Les hébreux passent sous la domination de la civilisation perse. Les élites sont envoyés en \underline{exil} à Babylone. Après quelques temps, le chef Perse, Cyrus, autorise la population à revenir en Israël. Les juifs reconstruisent le temple. La province a perdu son indépendance. La région passe aux mains des Grecs puis des Romains. Après deux révoltes, les Romains détruisent le seconde Temple de Jérusalem (70 ap JC). Après cet épisode, les juifs s'éparpillent dans le monde méditerranéen : c'est la \underline{diaspora}.

\vspace{0.5cm}

\underline{Polythéisme} (Croire en plusieurs dieux) / \underline{monothéisme} (croire en un seul dieu).\\
\underline{Temple de Jérusalem} : bâtiment religieux majeurs dans la religion juive.\\
\underline{Exil} : être poussé à partir d'un endroit pour un autre, sans possibilité de revenir chez soi.\\
\underline{Diaspora} : dispersion d'un peuple à travers le monde.

\newpage
{\large Exercice 1. Etude de cartes.}\textbf{ doc 2 p. 116 / 3 p. 123}\\
\underline{But de l'exercice }: en comparant ces deux cartes, montre l'évolution géopolitique de la région d’Israël entre le VIIIe siècle av JC et le IIe siècle ap JC.\\
\textit{{\small Compétence du programme : savoir localiser la Palestine, Jérusalem sur une carte de Empire romain.}}

\vfill

{\large Exercice 1. Etude de cartes.}\textbf{ doc 2 p. 116 / 3 p. 123}\\
\underline{But de l'exercice }: en comparant ces deux cartes, montre l'évolution géopolitique de la région d’Israël entre le VIIIe siècle av JC et le IIe siècle ap JC.\\
\textit{{\small Compétence du programme : savoir localiser la Palestine, Jérusalem sur une carte de Empire romain.}}

\vfill

{\large Exercice 1. Etude de cartes.}\textbf{ doc 2 p. 116 / 3 p. 123}\\
\underline{But de l'exercice }: en comparant ces deux cartes, montre l'évolution géopolitique de la région d’Israël entre le VIIIe siècle av JC et le IIe siècle ap JC.\\
\textit{{\small Compétence du programme : savoir localiser la Palestine, Jérusalem sur une carte de Empire romain.}}

\vfill

{\large Exercice 1. Etude de cartes.}\textbf{ doc 2 p. 116 / 3 p. 123}\\
\underline{But de l'exercice }: en comparant ces deux cartes, montre l'évolution géopolitique de la région d’Israël entre le VIIIe siècle av JC et le IIe siècle ap JC.\\
\textit{{\small Compétence du programme : savoir localiser la Palestine, Jérusalem sur une carte de Empire romain.}}

\vfill

{\large Exercice 1. Etude de cartes.}\textbf{ doc 2 p. 116 / 3 p. 123}\\
\underline{But de l'exercice }: en comparant ces deux cartes, montre l'évolution géopolitique de la région d’Israël entre le VIIIe siècle av JC et le IIe siècle ap JC.\\
\textit{{\small Compétence du programme : savoir localiser la Palestine, Jérusalem sur une carte de Empire romain.}}

\vfill

{\large Exercice 1. Etude de cartes.}\textbf{ doc 2 p. 116 / 3 p. 123}\\
\underline{But de l'exercice }: en comparant ces deux cartes, montre l'évolution géopolitique de la région d’Israël entre le VIIIe siècle av JC et le IIe siècle ap JC.\\
\textit{{\small Compétence du programme : savoir localiser la Palestine, Jérusalem sur une carte de Empire romain.}}

\vfill

\newpage
{\large Exercice 2. Etude de documents (texte, plan) : la destruction du second Temple (70 ap JC)}. \textbf{doc 3 p. 125}.\\
questions du livre (sauf Récit).\\
\textit{{\small Compétence : connaître la destruction du second temple en 70 ap JC.}}

\vfill

{\large Exercice 2. Etude de documents (texte, plan) : la destruction du second Temple (70 ap JC)}. \textbf{doc 3 p. 125}.\\
questions du livre (sauf Récit).\\
\textit{{\small Compétence : connaître la destruction du second temple en 70 ap JC.}}

\vfill

{\large Exercice 2. Etude de documents (texte, plan) : la destruction du second Temple (70 ap JC)}. \textbf{doc 3 p. 125}.\\
questions du livre (sauf Récit).\\
\textit{{\small Compétence : connaître la destruction du second temple en 70 ap JC.}}

\vfill

{\large Exercice 2. Etude de documents (texte, plan) : la destruction du second Temple (70 ap JC)}. \textbf{doc 3 p. 125}.\\
questions du livre (sauf Récit).\\
\textit{{\small Compétence : connaître la destruction du second temple en 70 ap JC.}}

\vfill

{\large Exercice 2. Etude de documents (texte, plan) : la destruction du second Temple (70 ap JC)}. \textbf{doc 3 p. 125}.\\
questions du livre (sauf Récit).\\
\textit{{\small Compétence : connaître la destruction du second temple en 70 ap JC.}}

\vfill

{\large Exercice 2. Etude de documents (texte, plan) : la destruction du second Temple (70 ap JC)}. \textbf{doc 3 p. 125}.\\
questions du livre (sauf Récit).\\
\textit{{\small Compétence : connaître la destruction du second temple en 70 ap JC.}}

\vfill

{\large Exercice 2. Etude de documents (texte, plan) : la destruction du second Temple (70 ap JC)}. \textbf{doc 3 p. 125}.\\
questions du livre (sauf Récit).\\
\textit{{\small Compétence : connaître la destruction du second temple en 70 ap JC.}}

\vfill

\newpage
{\large Exercice 3. Etude de carte.} \textbf{Carte 3 p. 123} \\
But de l'exercice : utilise le doc 3 p. 125 et la carte et sa légende pour décrire et expliquer la diaspora juive.
{\small \textit{Compétence : \textbf{décrire} et \textbf{expliquer} la diaspora.}}

\vfill

{\large Exercice 3. Etude de carte.} \textbf{Carte 3 p. 123} \\
But de l'exercice : utilise le doc 3 p. 125 et la carte et sa légende pour décrire et expliquer la diaspora juive.
{\small \textit{Compétence : \textbf{décrire} et \textbf{expliquer} la diaspora.}}

\vfill

{\large Exercice 3. Etude de carte.} \textbf{Carte 3 p. 123} \\
But de l'exercice : utilise le doc 3 p. 125 et la carte et sa légende pour décrire et expliquer la diaspora juive.
{\small \textit{Compétence : \textbf{décrire} et \textbf{expliquer} la diaspora.}}

\vfill

{\large Exercice 3. Etude de carte.} \textbf{Carte 3 p. 123} \\
But de l'exercice : utilise le doc 3 p. 125 et la carte et sa légende pour décrire et expliquer la diaspora juive.
{\small \textit{Compétence : \textbf{décrire} et \textbf{expliquer} la diaspora.}}

\vfill

{\large Exercice 3. Etude de carte.} \textbf{Carte 3 p. 123} \\
But de l'exercice : utilise le doc 3 p. 125 et la carte et sa légende pour décrire et expliquer la diaspora juive.
{\small \textit{Compétence : \textbf{décrire} et \textbf{expliquer} la diaspora.}}

\vfill

{\large Exercice 3. Etude de carte.} \textbf{Carte 3 p. 123} \\
But de l'exercice : utilise le doc 3 p. 125 et la carte et sa légende pour décrire et expliquer la diaspora juive.
{\small \textit{Compétence : \textbf{décrire} et \textbf{expliquer} la diaspora.}}

\vfill

{\large Exercice 3. Etude de carte.} \textbf{Carte 3 p. 123} \\
But de l'exercice : utilise le doc 3 p. 125 et la carte et sa légende pour décrire et expliquer la diaspora juive.
{\small \textit{Compétence : \textbf{décrire} et \textbf{expliquer} la diaspora.}}

\vfill

{\large Exercice 3. Etude de carte.} \textbf{Carte 3 p. 123} \\
But de l'exercice : utilise le doc 3 p. 125 et la carte et sa légende pour décrire et expliquer la diaspora juive.
{\small \textit{Compétence : \textbf{décrire} et \textbf{expliquer} la diaspora.}}

\vfill

\newpage
\textbf{II/ La Bible, entre Histoire et croyances.}

\vspace{0.5cm}

\setlength{\parindent}{1cm}La \textbf{Bible} est la principale source écrite pour connaître l'histoire des juifs. Cet ouvrage n'a pas de visée historique mais chercher à raconter une histoire du monde depuis la Création jusqu'à la relation d'Israël avec le dieu unique : \textbf{Yahvé}. Cette histoire est centrée sur le peuple hébreu et est faite d'emprunt aux traditions et récits des peuples voisins.
Le temps de l'\textbf{Exil} à Babylone a été fondamental dans l'élaboration du livre. Israël et Juda n'existent plus en tant qu'Etat souverain. Le 1er temple est détruit. A Babylone, les prêtres rassemblent et modifient les traditions orales et les premiers textes écrits afin de conserver leur religion. Ils donnent naissances aux premiers "livres bibliques".\\
ATTENTION : en histoire, les récits religieux comme la Bible doivent être confrontés à d'autres sources historiques (archéologiques, textes, ...) afin d'arriver à faire la différence entre des événements réellement passés (Exil, destruction des deux temples) et les \underline{croyances}.

\vspace{0.5cm}

\underline{Croyance} : fait de croire en quelque chose.

\vfill

\textbf{II/ La Bible, entre Histoire et croyances.}

\vspace{0.5cm}

\setlength{\parindent}{1cm}La \textbf{Bible} est la principale source écrite pour connaître l'histoire des juifs. Cet ouvrage n'a pas de visée historique mais chercher à raconter une histoire du monde depuis la Création jusqu'à la relation d'Israël avec le dieu unique : \textbf{Yahvé}. Cette histoire est centrée sur le peuple hébreu et est faite d'emprunt aux traditions et récits des peuples voisins.
Le temps de l'\textbf{Exil} à Babylone a été fondamental dans l'élaboration du livre. Israël et Juda n'existent plus en tant qu'Etat souverain. Le 1er temple est détruit. A Babylone, les prêtres rassemblent et modifient les traditions orales et les premiers textes écrits afin de conserver leur religion. Ils donnent naissances aux premiers "livres bibliques".\\
ATTENTION : en histoire, les récits religieux comme la Bible doivent être confrontés à d'autres sources historiques (archéologiques, textes, ...) afin d'arriver à faire la différence entre des événements réellement passés (Exil, destruction des deux temples) et les \underline{croyances}.

\vspace{0.5cm}

\underline{Croyance} : fait de croire en quelque chose.

\vfill

\textbf{II/ La Bible, entre Histoire et croyances.}

\vspace{0.5cm}

\setlength{\parindent}{1cm}La \textbf{Bible} est la principale source écrite pour connaître l'histoire des juifs. Cet ouvrage n'a pas de visée historique mais chercher à raconter une histoire du monde depuis la Création jusqu'à la relation d'Israël avec le dieu unique : \textbf{Yahvé}. Cette histoire est centrée sur le peuple hébreu et est faite d'emprunt aux traditions et récits des peuples voisins.
Le temps de l'\textbf{Exil} à Babylone a été fondamental dans l'élaboration du livre. Israël et Juda n'existent plus en tant qu'Etat souverain. Le 1er temple est détruit. A Babylone, les prêtres rassemblent et modifient les traditions orales et les premiers textes écrits afin de conserver leur religion. Ils donnent naissances aux premiers "livres bibliques".\\
ATTENTION : en histoire, les récits religieux comme la Bible doivent être confrontés à d'autres sources historiques (archéologiques, textes, ...) afin d'arriver à faire la différence entre des événements réellement passés (Exil, destruction des deux temples) et les \underline{croyances}.

\vspace{0.5cm}

\underline{Croyance} : fait de croire en quelque chose.

\newpage

Exercice : complète le tableau.

\begin{tabular}{|p{3cm}|p{4cm}|p{4cm}|p{4cm}}
\hline  & Doc 1 p. 118 & Doc 2 p. 118 & Doc 4 p. 119 \\ 
\hline Quel épisode est relaté ? & \hspace{4cm} &  &  \\ 
\hline Décrit en quelques lignes l'épisode & \hspace{4cm}  &  &  \\ 
\hline 
\end{tabular} 

\vspace{1cm}

Compétence : raconter et expliquer quelques uns 1 des grands récits de la Bible significatifs des croyances.

\vfill

Exercice : complète le tableau.

\begin{tabular}{|p{3cm}|p{4cm}|p{4cm}|p{4cm}}
\hline  & Doc 1 p. 118 & Doc 2 p. 118 & Doc 4 p. 119 \\ 
\hline Quel épisode est relaté ? & \hspace{4cm} &  &  \\ 
\hline Décrit en quelques lignes l'épisode & \hspace{10cm}  &  &  \\ 
\hline 
\end{tabular} 

\vspace{1cm}

Compétence : raconter et expliquer quelques uns 1 des grands récits de la Bible significatifs des croyances.

\vfill

Exercice : complète le tableau.

\begin{tabular}{|p{3cm}|p{4cm}|p{4cm}|p{4cm}}
\hline  & Doc 1 p. 118 & Doc 2 p. 118 & Doc 4 p. 119 \\ 
\hline Quel épisode est relaté ? & \hspace{4cm} &  &  \\ 
\hline Décrit en quelques lignes l'épisode & \hspace{10cm}  &  &  \\ 
\hline 
\end{tabular} 

\vspace{1cm}

Compétence : raconter et expliquer quelques uns 1 des grands récits de la Bible significatifs des croyances.


\end{document}