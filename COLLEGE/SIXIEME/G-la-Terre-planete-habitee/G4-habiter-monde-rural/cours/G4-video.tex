 \documentclass{beamer}
  %\usepackage[utf8]{inputenc}
  \usepackage{fontspec}  %pour xelatex
 \usepackage{xunicode}  %pour xelatex
  \usetheme{Montpellier}
   \usepackage{color}
   \usepackage{graphicx}
   \usepackage{ulem}
%   \usepackage{xkeyval}
   \usepackage{pst-tree}
   \usepackage{tabularx}
   \usepackage[french]{babel}
 %  \usepackage{pstcol,pst-fill,pst-grad}
 % \setbeamercolor{normal text}{fg=blue}
 \setbeamercolor{section in head/foot}{fg=black}
  \setbeamercolor{subsection in head/foot}{fg=blue}
\beamerboxesdeclarecolorscheme{blocbleu}{black!60!white}{black!20!white}
\beamerboxesdeclarecolorscheme{blocimage}{black!60!white}{black!10!white}
\setbeamercolor{section in toc}{fg=black}
\setbeamercolor{subsection in toc}{fg=blue}



\AtBeginSection[]
{
 \begin{frame}
  \tableofcontents[currentsection,hideallsubsections]
 \end{frame}
}

\AtBeginSubsection[]
{
  \begin{frame}
  \tableofcontents[currentsection,currentsubsection]
  \end{frame}
}
  \title{{\textcolor{red}{Partie 4 - Habiter le monde rural}}}

\begin{document}
\begin{frame}
 \titlepage %{CHAPITRE 2 - LES IDENTIT�S MULTIPLES DE LA PERSONNE}
 \end{frame}

\begin{frame}
\underline{But du cours} : comprendre les différentes facettes du monde rural.
\end{frame}
 
 \begin{frame}
 \tableofcontents
 \end{frame}
 

 
 \section{I/ UNE campagne ou DES campagnes ?}
 
 \begin{frame}
\textbf{Exercice 1. Description de photographies.}\\

\includegraphics[scale=0.06]{doc1.jpg}
\includegraphics[scale=0.09]{doc7.jpg}

  \begin{itemize}
  \item Note sur les photographies les éléments que tu repères. \\
  Se retrouvent-t-il sur les deux photographies ?
  \end{itemize}
\end{frame}


 
\begin{frame}
\textbf{Exercice 2} : Pour toi, existe-t-il UN monde rural ou PLUSIEURS mondes ruraux ? Pourquoi ?
\textcolor{black!70!green}{}
\end{frame}
 
 \begin{frame} \underline{ Sur le cahier : } 
 
 \textcolor{blue}{}
 
%\begin{itemize}
%\item La moitié de la population mondiale vit à la campagne.
 %\item Monde rural = campagne.
 %\item Monde rural opposé à monde urbain.
 %\item Dans une campagne, on trouve 
 %\begin{itemize}
 %\item
 %\item
 %\item
 %\item
 
 %\end{itemize}
%\end{itemize}
 
 \end{frame}

 
 \section{II/ Le monde rural et l'agriculture.}
 
 \textbf{Exercice 1. Qu'est-ce qu'une agriculture de subsistance ?} \\
 
 \begin{minipage}{10cm}
  \begin{itemize}
   \item 2 p. 262. 
   
  
  \begin{enumerate}
  \item Situe sur la carte le pays et le continent où se trouve la photographie.
  \item Décrit la photographie.\\
  
  \item Que plantent les paysans du texte ? \\
  
  \item Souligne dans le texte ce qui montre que ce travail n'est pas moderne.
  \item Grâce au texte, explique ce qu'est une << agriculture de subsistance >>.
  \vspace{3cm}
  \end{enumerate}
   \end{itemize}
 \end{minipage}
  \fbox{
  \begin{minipage}{9cm}
 << Les paysans et les paysannes d'un village tchadien travaillent collectivement un champ à la houe avant d'y planter du mil. Ils sarclent en cadence, encouragés par les chants et la musique que scande l'homme [...]. Les outils agricoles, rudimentaires sont néanmoins adaptés à un travail du sol à la main. Bien que la culture attelée se développe rapidement la principale énergie utilisée est le travail humain. Une part importante des récoltes est obtenue dans le cadre d'une agriculture familiale, pluviale (pas d'irrigation) à faibles productions (500 kg de mil à l'hectare). >>
 \begin{flushright}
 D'après Sylvie Brunel, \textit{Les Tiers-Monde}, Documentation photographique, n°7014, décembre 1992.
 \end{flushright}
  \end{minipage}
  }
  
   \textbf{Exercice 2. Qu'est-ce qu'une agriculture d'exportation ?} \\
 
 \begin{frame}
  \fbox{
 \begin{minipage}{6cm}
 {\small Dans l'Etat de Washington (nord-ouest des Etats-Unis) est pratiquée une agriculture puissamment motomécanisée, qui concerne seulement 30 millions d'agriculteurs dans le monde. Ici, une équipe de trois moissonneuses-batteuses récolte du blé. Chacune d'elles peut moissonner un hectare à l'heure sur des exploitations qui couvrent de cinq cents à mille hectares. Dans ces conditions, une seule personne peut produire de 1000 à 2000 tonnes de grains. On trouve principalement ce type d'agriculture dans les pays riches, mais aussi dans des pays comme l'Argentine ou le Brésil.}
 \begin{flushright}
{\tiny  D'après JP Charvet, << L'agriculture mondialisée >>, \textit{La Doc photo}, n°8059, sept-oct 2007}
 \end{flushright}
 \end{minipage}
 }
 \begin{minipage}{4cm}
 \includegraphics[scale=0.10]{doc9.jpg}
 \end{minipage}
 \end{frame}
 
 \begin{frame}
 \begin{enumerate}
  \item Situe le pays et le continent de ces documents sur la carte. \textcolor{black!70!green}{}
 
 %\includegraphics{name} 
  
 \item Explique ce qu'est une agriculture << productive >>. \textcolor{black!70!green}{}  \\
 
 \item Pourquoi ce genre d'agriculture est dite << d'exportation >> \textcolor{black!70!green}{} \\
 \end{enumerate}
 \end{frame}

 \begin{frame} \underline{Sur le cahier : }
 % TE : Deux campagnes : 
  %- dans les pays pauvres : composée de "paysans". Utilisent encore des outils agricoles à l'ancienne. But de l'agriculture : subsistance. Ms commence un peu l'agriculture d'exportation.
  
  
  %- dans les pays riches : composés de quelques agriculteurs + de populations qui n'ont rien à voir avec l'agriculture (qui travaillent en ville). Agriculture d'exportation. Plus d'agriculture de subsistance.
 \end{frame}

 
 
\section{III/ Les autres activités du monde rural.}
 
 \begin{frame}
 \begin{minipage}{4cm}
\includegraphics[scale=0.10]{doc6.jpg}
 \end{minipage}
 \begin{minipage}{6cm}
 1 p. 246. Ont-ils du temps pour autre chose que le travail agricole ? \textcolor{black!70!green}{}\\
 \end{minipage}
  
 \end{frame}

\begin{frame}
%\begin{minipage}{4cm}
 5 p. 251. Surligne dans le texte les autres activités que l'on peut faire à Saint-Nectaire
 %\end{minipage}
    \fbox{
    \begin{minipage}{11cm}
     {\small << Saint-Nectaire est un authentique village auvergnat doublé d'une ville thermale construite au XIXe siècle. Le hameau touristique du << haut >> surplombé par sa superbe église romane veille sur les grands hôtels et les restaurants du << bas >> où l'office du tourisme a investi l'ancien bâtiment des grands thermes.\\
     Le village possède aussi un vaste ensemble de plusieurs espaces consacrés aux plaisirs de l'eau et du corps, l'espace aquadétente. Parcours d'orientation, escalade, randonnées pédestres, circuit VTT et même survol du territoire en montgolfière, les activités sportives ne manquent pas. L'hiver, le ski de piste et de fond, la luge et les randonnées en raquette viennent compléter le planning des plus sportifs ! A Saint-Nectaire et dans ses alentours, on peut visiter un site de production et une cave d'affinage témoin. >>}
     \begin{flushright}
     D'après la brochure publicitaire, \textit{Guides des visites et des activités}, publiée par l'office du tourisme de Saint-Nectaire, 2008
     \end{flushright}
  \end{minipage}
     }
\end{frame}    
 
 \begin{frame} \underline{Sur le cahier : }
% TE : 
 %- ds les pays pauvres : autres activités des personnes pr améliorer le quotidien (tissage de la soie, vente de soupe, distillation d'alcool.)
 
 %- ds pays riches : but de diversifié les revenus : tourisme, agriculture plus durable...
 \end{frame}   
   
  \end{document}