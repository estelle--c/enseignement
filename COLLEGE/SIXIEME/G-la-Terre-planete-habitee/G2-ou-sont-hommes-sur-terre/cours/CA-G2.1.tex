  \documentclass{beamer}
  %\usepackage[utf8]{inputenc}
  \usepackage{fontspec}  %pour xelatex
 \usepackage{xunicode}  %pour xelatex
  \usetheme{default}
   \usepackage{color}
   \usepackage{graphicx}
   \usepackage{ulem}
%   \usepackage{xkeyval}
   \usepackage{pst-tree}
   \usepackage{tabularx}
   \usepackage[french]{babel}
 %  \usepackage{pstcol,pst-fill,pst-grad}
  \setbeamercolor{normal text}{fg=blue}
 \setbeamercolor{section in head/foot}{fg=black}
  \setbeamercolor{subsection in head/foot}{fg=blue}
\beamerboxesdeclarecolorscheme{blocbleu}{black!60!white}{black!20!white}
\beamerboxesdeclarecolorscheme{blocimage}{black!60!white}{black!10!white}
\setbeamercolor{section in toc}{fg=black}
\setbeamercolor{subsection in toc}{fg=blue}



\AtBeginSection[]
{
 \begin{frame}
  \tableofcontents[currentsection,hideallsubsections]
 \end{frame}
}

\AtBeginSubsection[]
{
  \begin{frame}
  \tableofcontents[currentsection,currentsubsection]
  \end{frame}
}
 
      \title{{\textcolor{red}{Chapitre 2 - Où sont les hommes sur la Terre ?}}}

\begin{document}
 
\begin{frame}
\titlepage %{CHAPITRE 2 - LES IDENTITÉS MULTIPLES DE LA PERSONNE}
\end{frame}

\begin{frame}
\tableofcontents
\end{frame}

\section{I/ L'inégale distribution de la population sur terre.}
\subsection{Sur la planète.}
\begin{frame}
\begin{beamerboxesrounded}[scheme=blocimage]{Doc 1. La répartition de la population sur Terre.} 
\includegraphics[scale=0.45]{densite-pop.eps}
\end{beamerboxesrounded}
\end{frame}

\begin{frame}
\begin{flushright}
{\tiny \textcolor{orange}{Compétences utilisées : \\
C5 : je sais tirer des informations d'une carte.\\}}
\end{flushright}
\underline{Exercice}
\begin{itemize}
\item Quelles sont les grandes zones de peuplement ?
\item Quels sont les espaces vides ?
\item les populations sont-elles plus situées sur les littoraux ou à l'intérieur des continents ?
\end{itemize}
\end{frame}

\begin{frame}
\begin{itemize}
\item Quelles sont les grandes zones de peuplement ?
\textcolor{black!70!green}{Les grandes zones de peuplement sont : l'Europe, le sous-continent indien, l'Asie de l'Est et du Sud-Est, Afrique de l'ouest, centreale, les littoraux d'Amérique du Sud et l'Amérique centrale.}
\vfill
\pause \item Quels sont les espaces vides ?
\pause \textcolor{black!70!green}{Une grande partie de l'Asie centrale, l'Australie, l'Afrique saharienne, le centre de l'Amérique du Sud et la majorité de l'Amérique du Nord.}
\vfill
\pause
\item les populations sont-elles plus situées sur les littoraux ou à l'intérieur des continents ?
\pause \textcolor{black!70!green}{Les populations sont très souvent situées sur les littoraux. L'intérieur des continents est vide.}
\end{itemize}
\end{frame}


\begin{frame}
\setlength{\parindent}{1cm} En 2013, nous sommes plus de 7 milliards de personnes sur la planète. Cette population est inégalement répartie : 
\begin{itemize}
\item 2/3 habitent sur 1/10e de la surface
\item 60 \% de la population habitent dans les trois foyers majeurs (Europe, Asie de l'Est-du Sud-Est, monde indien)
\end{itemize} 
Une grande majorité habitent sur les \textcolor{red}{littoraux}. Toutes les grandes villes actuelles se situent dans des espaces très fortement peuplés. L'indicateur géographique qui nous permet d'étudier la population est la \textcolor{red}{Densité de population}.
\vfill\textcolor{red}{\underline{Littoral} : Espace situé au bord de la mer.}
\vfill\textcolor{red}{\underline{Densité de population} : Nombre d'habitants par km².}
\end{frame}

\subsection{Exemple : l'Asie de l'Est}

\begin{frame}
\underline{Activité p.203}
\begin{enumerate}
\item Doc 4, 5, 6. Comparez les trois photographies et attribuez à chacune les expressions suivantes : ville, campagne, faible densité, forte densité.

\pause \textcolor{black!70!green}{doc 4 : ville / forte densité \\ doc 5 : campagne / forte densité \\doc 6 : campagne / faible densité}

\item Doc 2. Quelle est la part de l'Asie de l'Est dans la population mondiale ?\\

\pause \textcolor{black!70!green}{32\% de la population totale du monde habite en Asie de l'Est.}

\item Doc 1 et 3. Les densités sont-elles égales partout ? Relevez un État aux très faibles densités et un autre aux fortes densités\\

\pause \textcolor{black!70!green}{Non certains espaces sont fortement peuplés. D'autres sont faiblement peuplés. Exemple : la Mongolie est faiblement peuplée. Le Japon est fortement peuplé.}
\end{enumerate}
\end{frame}

\begin{frame}
\setlength{\parindent}{1cm} L'Asie de l'Est est le premier \textcolor{red}{foyer de peuplement au monde}. En 2013, il regroupe 1 milliard et demi d'habitants. Quatre pays sont concernés : Chine, les deux Corées, Japon.\\
\setlength{\parindent}{1cm} Les densités de peuplement ne sont pas égales partout. Ainsi, le Japon est un pays très densément peuplé, tandis que la Mongolie n'est pas très faiblement peuplée.
\end{frame}


\section{II/ Pourquoi cette inégale répartition de la population ?}
\subsection{Les facteurs naturels}

\begin{frame}
\underline{doc 1 p.210, 4 et 5 p. 211.}
\begin{itemize}
\item Mettez en relation ces trois planisphères pour expliquer le rôle des contraintes naturelles dans la répartition de la population.\\

\vfill\textcolor{red}{\underline{Contrainte naturelle} : Espace naturel qui empêche l'installation des hommes}

\pause \textcolor{black!70!green}{Les contraintes naturelles jouent un rôle dans la répartition de la population. Par exemple : dans les espaces arides comme les déserts, peu de population y vivent (ex: Sahara)}
\end{itemize}
\end{frame}

\begin{frame}
\setlength{\parindent}{1cm}Différentes \textcolor{red}{contraintes naturelles} limitent la population sur terre. Les espace de grand froid, d'aridité (désert) ou de haute altitude, sont souvent des espaces faiblement peuplés.
\vfill
\setlength{\parindent}{1cm}Cependant, il faut nuancer. Les montagnes, par exemple, ne sont pas toujours une zone répulsive. L'Himalaya (Asie) est une des zones les plus peuplées du monde. c'est aussi la plus haute montagne du monde.
\vfill
\setlength{\parindent}{1cm} Chaque société regarde son territoire en terme de ressource ou de contrainte. Chez certains, la montagne est une contrainte (sociétés européenne), chez d'autres c'est une ressource.

\vfill\textcolor{red}{\underline{Contrainte naturelle} : Espace naturel qui empêche l'installation des hommes}
\vfill\textcolor{red}{\underline{Ressource} : Zone où les hommes s'installent car ils peuvent en tirer de quoi survivre.}
\end{frame}

\subsection{L'histoire du peuplement.}

\begin{frame}
\begin{columns}
\begin{column}[b]{5cm}
\begin{beamerboxesrounded}[scheme=blocimage]{Doc 1. La population en 200 ap J-C.} 
\includegraphics[scale=0.13]{pop-deb-ere-che.jpg}
\end{beamerboxesrounded}
\vfill
\underline{Questions :} 
Que remarques-tu de l'évolution de la population ? Quels sont les foyers permanent à travers le temps ? \\
%\textcolor{black!70!green}{L'Europe et l'Asie sont les deux foyers qui se retrouve tout au long de l'histoire du peuplement de la Terre.}\\
\end{column}
\begin{column}[b]{5cm}
\begin{beamerboxesrounded}[scheme=blocimage]{Doc 2. La répartition de la population} 
\includegraphics[scale=0.15]{pop.jpg}
\end{beamerboxesrounded}
\end{column}
\end{columns}
\end{frame}

\begin{frame}
\begin{columns}
\begin{column}[b]{5cm}
\begin{beamerboxesrounded}[scheme=blocimage]{Doc 1. La population en 200 ap J-C.} 
\includegraphics[scale=0.13]{pop-deb-ere-che.jpg}
\end{beamerboxesrounded}
\vfill
\underline{Questions :} 
Que remarques-tu de l'évolution de la population ? Quels sont les foyers permanent à travers le temps ? \\
\textcolor{black!70!green}{L'Europe, l'Asie et le monde indien sont les trois foyers qui se retrouvent tout au long de l'histoire du peuplement de la Terre.}\\
\end{column}
\begin{column}[b]{5cm}
\begin{beamerboxesrounded}[scheme=blocimage]{Doc 2. La répartition de la population} 
\includegraphics[scale=0.15]{pop.jpg}
\end{beamerboxesrounded}
\end{column}
\end{columns}
\end{frame}

\begin{frame}
\begin{beamerboxesrounded}[scheme=blocimage]{Doc 3. Le rôle de l'agriculture dans le peuplement} 
\textcolor{black}{"L'Asie abrite les plus grandes paysanneries du monde. Tant d'hommes et depuis si longtemps : la Chine et le monde indien comptaient déjà chacun au début de l'ère chrétienne de 20 à 25\% de la population mondiale. En commun dans ces pays, le contrôle traditionnel de l'eau, technique permettant l'intensification des cultures [...]. C'est en Asie que l'on trouve les plus grandes superficies irriguées du monde. La riziculture inondée a imposé de fortes densités de population, nécessaires à la maîtrise de l'eau et les a rendus possibles, en dégageant des surplus agricoles du fait de l'efficacité des systèmes de cultures.}
\begin{flushright}
\textcolor{black}{\tiny D'après Michel Foucher, \textit{Asies Nouvelles}, 2002}
\end{flushright}
\end{beamerboxesrounded}
\end{frame}

\begin{frame}
\underline{Questions :} 
Pourquoi la population est-elle allée se concentrée dans certains espaces d'Asie de l'Est ?\\
\textcolor{black!70!green}{Ils sont allés se concentrer en Asie de l'Est car l'agriculture qui y était produite, la riziculture, était très demandeuse de main-d'oeuvre. Elle leur a permit de survivre.}\\
\end{frame}

\begin{frame}
\setlength{\parindent}{1cm}Les trois principaux foyers (Asie orientale, monde indien et Europe) sont déjà lisibles au début de l'ère chrétienne. Sur la longue durée, il existe une stabilité des foyers de peuplement. 
\vfill
\setlength{\parindent}{1cm} L'Asie de l'Est connaît une profonde et ancienne humanisation des campagnes. Une agriculture très demandeuse de main-d'oeuvre (notamment la riziculture) a permit de canaliser les populations sur les mêmes surfaces depuis des siècles.
\end{frame}



  \end{document}