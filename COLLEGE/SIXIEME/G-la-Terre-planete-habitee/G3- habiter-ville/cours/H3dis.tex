\documentclass[12pt]{article}
\usepackage{fontspec}
\usepackage{xltxtra}
\setmainfont[Mapping=tex-text]{Century Schoolbook L}
 \usepackage[francais]{babel}
 
 \usepackage{geometry}
 \geometry{ hmargin=0.5cm, vmargin=0.5cm } 
 
\usepackage{ifthen}

\makeatletter
\renewcommand\section{\@startsection
{section}{1}{0mm}    
{\baselineskip}
{0.5\baselineskip}
{\normalfont\normalsize\textbf}}
\makeatother


\begin{document}
\begin{large}
\newboolean{Professeur}
%\setboolean{Professeur}{true} % « true» (vrai) si le document est le document du professeur (sans trous). « Professeur » a la valeur « false » par défaut. Il faut donc décommenter la ligne pour mettre « Professeur » à « true »

\newcommand{\Trouer}[1]{
\ifthenelse{\boolean{Professeur}} % si « Professeur » est vrai,
{\textbf{#1}} %les mots cachés sont en gras
{\underline{\phantom{#1.5}}} % (else) sinon les mots sont remplacés par une ligne sur laquelle l'élève peut écrire.
}

Toutes les villes sont \Trouer{différentes}. Chaque paysage urbain, chaque ville est liée à l'aire culturelle, à l'\Trouer{histoire} de son pays. Le centre historique des villes européennes, les immenses banlieues pavillonnaires d'Amérique font intervenir des codes culturels.

Aujourd'hui, la \Trouer{moitié de la population} de la planète habite en ville. La planète connaît un \Trouer{phénomène d'urbanisation accélérée}. Le nombre de ville augmente. Elles sont aussi de plus en plus grandes. Beaucoup se situent en \Trouer{Asie}. Le monde devient quasi intégralement urbain.

\end{large}
\end{document}