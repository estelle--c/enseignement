\documentclass[12pt,a4paper]{article}
%\usepackage[utf8]{inputenc}
\usepackage[francais]{babel}
%\usepackage[T1]{fontenc}
\usepackage{graphicx}
\usepackage[left=0.5cm,right=0.5cm,top=0.5cm,bottom=0.5cm]{geometry}
 \usepackage{fontspec}  %pour xelatex
 \usepackage{xunicode}  %pour xelatex
 \usepackage{xltxtra}
  \setmainfont[Mapping=tex-text]{Century Schoolbook L}

\usepackage{ifthen}
\makeatletter
\renewcommand\section{\@startsection
{section}{1}{0mm}
{\baselineskip}
{0.5\baselineskip}
{\normalfont\normalsize\textbf}}
\makeatother


\begin{document}


\includegraphics[width=16cm]{piliers-DD.eps}

\begin{large}
\newboolean{Professeur}
%\setboolean{Professeur}{true} % « true» (vrai) si le document est le document du professeur (sans trous). « Professeur » a la valeur « false » par défaut. Il faut donc décommenter la ligne pour mettre « Professeur » à « true »
\newcommand{\Trouer}[1]{
\ifthenelse{\boolean{Professeur}} % si « Professeur » est vrai,
{\textbf{#1}} %les mots cachés sont en gras
{\underline{\phantom{#1.5}}} % (else) sinon les mots sont remplacés par une ligne sur laquelle l'élève peut écrire.
}

\vfill

Tous les \Trouer{dix ans}, les dirigeants \Trouer{mondiaux} se réunissent lors de sommets de la Terre. Ils essaient de stimuler les politiques de mise en place du développement durable grâce à des \Trouer{agenda 21}. L'exemple de Bordeaux nous a montré que même si on réfléchi aux politiques au niveau mondial, les actions, elles, se passent au niveau \Trouer{local} (ville, département, région, pays).

\vfill

Le mode de développement durable pose certains \Trouer{problèmes}. Le développement a permis, en France, de \Trouer{mieux vivre} grâce à des innovations, pour la plupart \\ \Trouer{polluantes} (industries, mode de consommation : on jette un produit quand il est cassé au lieu de le réparer...). Les \Trouer{pays pauvres}, eux, ont besoins de ces innovations pour se développer. Ils s’intéressent peu au côté environnemental du développement durable.
\end{large}

\end{document}