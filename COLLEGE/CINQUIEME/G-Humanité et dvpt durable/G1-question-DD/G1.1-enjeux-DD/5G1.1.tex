 \documentclass{beamer}
  %\usepackage[utf8]{inputenc}
  \usepackage{fontspec}  %pour xelatex
 \usepackage{xunicode}  %pour xelatex
  \usetheme{Montpellier}
   \usepackage{color}
   \usepackage{xcolor}
   \usepackage{graphicx}
   \usepackage{ulem}
%   \usepackage{xkeyval}
   \usepackage{pst-tree}
   \usepackage{tabularx}
   \usepackage[french]{babel}
 %  \usepackage{pstcol,pst-fill,pst-grad}
  \setbeamercolor{normal text}{fg=black}
 \setbeamercolor{section in head/foot}{fg=black}
  \setbeamercolor{subsection in head/foot}{fg=blue}
\beamerboxesdeclarecolorscheme{blocbleu}{black!60!white}{black!20!white}
\beamerboxesdeclarecolorscheme{blocimage}{black!60!white}{black!10!white}
\setbeamercolor{section in toc}{fg=black}
\setbeamercolor{subsection in toc}{fg=blue}
\date{}
%\usepackage{titlesec}
\usepackage{soul}


\AtBeginSection[]
{
 \begin{frame}
  \tableofcontents[currentsection,hideallsubsections]
 \end{frame}
}

\AtBeginSubsection[]
{
  \begin{frame}
  \tableofcontents[currentsection,currentsubsection]
  \end{frame}
}
  \title{{\textcolor{red}{Partie 1 - La question du développement durable \\ Chapitre 1 - Les enjeux du développement durable}}}

\usepackage{ifthen}
\makeatletter
\renewcommand\section{\@startsection
{section}{1}{0mm}
{\baselineskip}
{0.5\baselineskip}
{\normalfont\normalsize\textbf}}
\makeatother


\begin{document}

\newboolean{Professeur}
\setboolean{Professeur}{true} % « true» (vrai) si le document est le document du professeur (sans trous). « Professeur » a la valeur « false » par défaut. Il faut donc décommenter la ligne pour mettre « Professeur » à « true »
\newcommand{\Trouer}[1]{
\ifthenelse{\boolean{Professeur}} % si « Professeur » est vrai,
{\textbf{#1}} %les mots cachés sont en gras
{\underline{\phantom{#1.5}}} % (else) sinon les mots sont remplacés par une ligne sur laquelle l'élève peut écrire.
}

\newcommand{\df}[2]{\textcolor{red}{\underline{#1}: #2}}

\newcommand{\doc}[1]{
\begin{flushright}
\fbox{Documents : #1}
\end{flushright}
}

\newcommand{\con}[1]{\textcolor{blue}{\underline{Consigne}: #1}}

\newcommand{\rep}[1]{\textcolor{green}{\underline{Réponse}: #1}}

\begin{frame}{Thème général de l'année}
\begin{center}
{\Huge \textcolor{red}{Humanité et développement durable}}
\end{center}
\end{frame}

\begin{frame}
\df{Développement}{Arriver, grâce à un enrichissement général de l'humanité, à améliorer le bien-être des individus (à satisfaire ses besoins). Tous les pays du monde ne sont pas encore arrivé à un niveau correct de dvpt}

\vfill

\df{Développement durable}{Mode de développement qui répond aux besoins du présent, sans compromettre la capacité des générations futures de répondre aux leurs. Rapport Brundtland, 1987}
\end{frame}

\begin{frame}
 \titlepage %{CHAPITRE 2 - LES IDENTIT�S MULTIPLES DE LA PERSONNE}
 \end{frame}

\section{I/ Etude de cas : Se déplacer à Bordeaux}

\begin{frame}{Les transports : un besoin pour une population en mouvement}

\con{Enumères les différents moyens de se déplacer. Explique en les causes. Dit s'il faut être riche ou non pour l'utiliser -> fait trois case : causes, constat, csq.} \\
\con{Pourquoi dvpt des transports en ville ?}

\end{frame}

\begin{frame}{Se déplacer à Bordeaux}

{\small -> comprendre l'intérêt de la voiture dans les villes françaises : gain de temps, gain d'argent... \\
-> comprendre ces nouveaux modes de déplacements qui apparaissent dans les villes. \\
-> voir la méthode de la carte : Je lit le titre, je lit la légende, je fait figurer la légende avec la carte. Ensuite seulement, je répond aux questions. \\
-> pas faire la 5e. Mais à la place : explique pourquoi ces nouveaux modes de transports permettront aux générations futurs à Bordeaux de continuer à se déplacer. \\
 
OU
Deux exercices : 
\begin{itemize}
\item Tu es un enfant de la fin du XXIe siècle, décris ta vision du transport à Bordeaux, après un siècle de transports plus doux pour l'environnement.\\
\item Tu es un enfant de la fin du XXIe siècle, décris ta vision du transport à Bordeaux, alors que la ville n'a pas acceptée les transports plus doux et est resté aux transports polluants.
\end{itemize}}
\end{frame}

\section{II/ Quelles sont les idées du développement durable aujourd'hui ?}

\begin{frame}{Les 3 piliers du DD}

1 p. 180

Remplir les trois piliers


\end{frame}

\begin{frame}{Le leitmotiv du DD : penser global, agir local}

Tous les \Trouer{dix ans}, les dirigeants \Trouer{mondiaux} se réunissent lors de sommets de la Terre. Ils essaient de stimuler les politiques de mise en place du développement durable grâce à des \Trouer{agenda 21}. L'exemple de Bordeaux nous a montré que même si on réfléchi aux politiques au niveau mondial, les actions, elles, se passent au niveau \Trouer{local} (ville, département, région, pays).

\vfill

\df{Agenda 21}{plan d'action pour mettre en place un mode de vie basé sur le développement durable.}

\end{frame}

\begin{frame}{Les problèmes du DD}

\doc{6 p. 183}

\con{Explique pourquoi, pour l'auteur, le DD est un retour en arrière par rapport au dvpt de la France (utilise des exemples tirés du texte).}

\rep{But du dvpt : d'arriver à libérer l'Homme des contraintes matérielles (passer du temps à faire la vaisselle, laver le linge...). Si on revient à un mode plus durables, peur de revenir aussi sur ses nouveautés.}
\end{frame}

\begin{frame}
Le mode de développement durable pose certains \Trouer{problèmes}. Le développement a permis, en France, de \Trouer{mieux vivre} grâce à des innovations, pour la plupart \Trouer{polluantes} (industries, mode de consommation : on jette un produit quand il est cassé au lieu de le réparer...). Les \Trouer{pays pauvres}, eux, ont besoins de ces innovations pour se développer. Ils s’intéressent peu au côté environnemental du développement durable.
\end{frame}

  \end{document}