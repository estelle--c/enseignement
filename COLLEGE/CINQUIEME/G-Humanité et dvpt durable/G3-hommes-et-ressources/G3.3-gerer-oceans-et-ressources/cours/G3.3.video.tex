 \documentclass{beamer}
  %\usepackage[utf8]{inputenc}
  \usepackage{fontspec}  %pour xelatex
 \usepackage{xunicode}  %pour xelatex
  \usetheme{Montpellier}
   \usepackage{color}
   \usepackage{xcolor}
   \usepackage{graphicx}
   \usepackage{ulem}
%   \usepackage{xkeyval}
   \usepackage{pst-tree}
   \usepackage{tabularx}
   \usepackage[french]{babel}
 %  \usepackage{pstcol,pst-fill,pst-grad}
  \setbeamercolor{normal text}{fg=blue}
 \setbeamercolor{section in head/foot}{fg=black}
  \setbeamercolor{subsection in head/foot}{fg=blue}
\beamerboxesdeclarecolorscheme{blocbleu}{black!60!white}{black!20!white}
\beamerboxesdeclarecolorscheme{blocimage}{black!60!white}{black!10!white}
\setbeamercolor{section in toc}{fg=black}
\setbeamercolor{subsection in toc}{fg=blue}



\AtBeginSection[]
{
 \begin{frame}
  \tableofcontents[currentsection,hideallsubsections]
 \end{frame}
}

\AtBeginSubsection[]
{
  \begin{frame}
  \tableofcontents[currentsection,currentsubsection]
  \end{frame}
}
  \title{{\textcolor{red}{Partie 3 - Des hommes et des ressources \\ Chapitre 1 - Gérer les océans et leurs ressources.}}}

\begin{document}
\begin{frame}
 \titlepage %{CHAPITRE 2 - LES IDENTIT�S MULTIPLES DE LA PERSONNE}
 \end{frame}
 
 \begin{frame}
 \tableofcontents
 \end{frame}
 
 \begin{frame} Introduction.
 %La planète est composée a plus de 70 \% d'eau. Il est donc nécessaire d'avoir une bonne gestion des océans. Les poissons sont une denrée essentielle dans l'alimentation humaine.
 
\underline{ But du cours :} comprendre la situation de la pêche actuellement. Peux-t-on arriver à une pêche durable ?
 \end{frame}
 
 \section{I/ Etude de cas : la pêche en Galice.}

\begin{frame} Vidéo : la pêche en Galice.
TD1. Questionnaire sur la vidéo.
\end{frame}
 
 \begin{frame}
  \begin{itemize}
  \item Où se situe la Galice ? %\textcolor{black!70!green}{Nord de Espagne}
  \item Quel type de pêche est pratiqué à Vigo ?%\textcolor{black!70!green}{Pêche industrielle}
  \item Quel type de pêche est pratiqué dans les petits villages de Galice ? %\textcolor{black!70!green}{Pêche artisanale}
   \item Que demande l'eurodéputée Rosa Miguelez à l'UE ? %\textcolor{black!70!green}{Plus de moyens pour les pêcheurs}
  \item Liste les différents métiers de la mer. %\textcolor{black!70!green}{pêcheurs de poissons (morue, merlu...), pêcheurs de coquillages, vendeurs de poissons à la criée, tisseuses de filets, scientifiques}
  \end{itemize}
 \end{frame}
 
 \begin{frame}
\begin{itemize}
  \item Pourquoi la pêche est-elle une activité importante pour la population ? %\textcolor{black!70!green}{nourriture + travail pour la population.}
  \item de quoi est responsable le centre océanographique de Vigo? %\textcolor{black!70!green}{du nombre de quotas de poisson}
  \item quel est le rôle de l'Agence communautaire de contrôle de la pêche ? %\textcolor{black!70!green}{????}
  \item Quel est le rôle des quotas selon l'eurodéputé Raul Romeva ? %\textcolor{black!70!green}{????}
  \item Quelles sont les deux principales menaces qui pèsent sur les ressources halieutiques selon les associations environnementales. %\textcolor{black!70!green}{????}
  \end{itemize}
 \end{frame}


 \section{II/ Les ressources halieutiques au niveau mondial.}
%\subsection{Les ressources aujourd'hui: constat, lieu.}

\begin{frame}
\textbf{chrono p. 279.} Quelle est l'évolution de la quantité de poisson capturé depuis le début du XXe siècle ? \\
%\textcolor{black!70!green}{Correction : pratiquement multiplié par 100.}\\
\end{frame}


\begin{frame}
p. 284-285.

\begin{minipage}{3cm}
 1) Inscrit sur la carte les deux principales zones de pêches (bleu majuscule)\\
 2) les pays principaux producteurs (noir majuscule)
 \end{minipage}
 \begin{minipage}{6cm}
 \includegraphics[scale=0.40]{schema-monde.eps}
 \end{minipage}

\end{frame}

\begin{frame}
p. 284-285.

 \begin{minipage}{3cm}
 1) Inscrit sur la carte les deux principales zones de pêches (bleu majuscule)\\
 2) les pays principaux producteurs (noir majuscule)
 \end{minipage}
 \begin{minipage}{6cm}
 \includegraphics[scale=0.40]{schema-monde-2.eps}
 \end{minipage}

\end{frame}



%\subsection{Les problèmes : pollution, surpêche.}

\begin{frame} 
\begin{beamerboxesrounded}[scheme=blocimage]{Doc 1. L'océan, victime de la pollution terrestre.} 
La pollution est transportée par les cours d'eau ou l'air. Elle est aussi bien due à l'agriculture qui utilise pesticides et engrais, qu'à l'industrie et aux déchets domestiques. Chaque année, on observe des zones océaniques impropres à la vie, elles sont situées au large des côtes où se concentrent les activités industrielles.
\begin{flushright}
D'après \textit{l'Atlas de l'océan mondial, 2007}
\end{flushright}
\end{beamerboxesrounded}
\vfill
\begin{beamerboxesrounded}[scheme=blocimage]{Doc p. 279 + 2 p. 286} 

\end{beamerboxesrounded}
\end{frame}

\begin{frame}
Questions : 
Explique pourquoi l'océan n'est pas une ressource durable actuellement.\\
%\textcolor{black!70!green}{pollution + pêche au chalut --> surpêche.}

\end{frame}

\begin{frame}
\underline{Sur le cahier :} \\
L'océan est une ressource vitale pour la planète (la quantité de poissons prélevée est essentielle pour l'alimentation). Le nombre de poisson pêché à fortement augmenté depuis 1 siècle, dû à de nouvelles techniques de pêche plus invasives. Cependant, ces pêches sont en contradiction avec l'idée d'un développement durable : surpêche (non-renouvellement des bans de poissons, utilisation du chalut), pollution sont à regretter.
\end{frame}
 
 \section{III/ Vers une pêche durable ?}
 %\subsection{Une définition.}
 Choisit parmi les éléments suivants ceux qui correspondent à des méthodes de pêche durable. Rédige ensuite une définition de la pêche durable.
 \begin{itemize}
 \item Satisfaire les besoins alimentaires humains actuels
 \item Réduire le nombre d'emplois liés à la pêche
 \item Sauvegarder les ressources halieutiques
 \item Priver les générations futures de consommer du poisson
 \item Réduire la quantité de poisson dans l'alimentation.
 \item Permettre aux pêcheurs de vivre de leur métier.
 \item Permettre aux générations futures de consommer du poisson.
 \item Interdire toute forme de pêche pour préserver les espèces marines.
 \item Limiter l'activité de pêche aux pays les plus riches.
 \end{itemize}
 
 \begin{frame} Correction
 \begin{itemize}
  \item \pause[1]\colorbox{yellow}{Satisfaire les besoins alimentaires humains actuels}
  \item Réduire le nombre d'emplois liés à la pêche
  \item \pause[2]\colorbox{yellow}{Sauvegarder les ressources halieutiques}
  \item Priver les générations futures de consommer du poisson
  \item Réduire la quantité de poisson dans l'alimentation.
  \item \pause[3]\colorbox{yellow}{Permettre aux pêcheurs de vivre de leur métier.}
  \item \pause[4]\colorbox{yellow}{Permettre aux générations futures de consommer du poisson.}
  \item Interdire toute forme de pêche pour préserver les espèces marines.
  \item Limiter l'activité de pêche aux pays les plus riches.
  \end{itemize}
 \end{frame}
 
 \begin{frame}
 Définition : %\textcolor{black!70!green}{}
 \end{frame}
 
%\subsection{aquaculture, quotas et lois mondiales}
\begin{frame} 
\begin{beamerboxesrounded}[scheme=blocimage]{Doc 2 p. 288. Le saumon d'élevage}
<< La croissance de l'élevage du saumon résulte de l'apport d'aliments industriels à base de farine (il faut, pour 1 kg de saumon, 3 kg de poisson). L'aquaculture est agressive pour le milieu. Des poissons d'élevage transgénique s'échappent parfois des enclos et se reproduisent en mer. De plus, les animaux qui vivent dans des bassins où ils sont nombreux, nécessitent des soins préventifs ou curatifs importants et l'emploi de médicaments, vaccins, antibiotiques, dont une partie se disperse dans le milieu. L'aquaculture de transformation produit enfin de nombreux déchets responsables de pollutions.>>
\begin{flushright}
D'après Y. Veyret \textit{Dictionnaire de l'environnement}, 2007
\end{flushright}
\end{beamerboxesrounded}
\end{frame}

\begin{frame}
\begin{beamerboxesrounded}[scheme=blocimage]{Le droit de la mer} 
<< En 1982, la Convention sur le droit de la mer est adoptée. Aujourd'hui respectée par 157 pays, elle définit les limites territoriales et maritimes et pose des règles pour l'exploitation des ressources maritimes. Le tribunal international du droit de la mer est chargé de régler les conflits.
En 1995, la FAO adopte un texte international, un <<code de bonne conduite>> pour une exploitation durable des ressources océaniques. Il insiste sur la nécessité de protéger les espèces, par des quotas mais aussi la protection des poissons jeunes. Ce texte propose aussi de limiter certaines techniques de pêche qui ont des effets destructeurs parce qu'elles ne sélectionnent pas le bon poisson et capturent des espèces non désirées >>.\\
\begin{flushright}
D'après \textit{l'Atlas de l'océan mondial}, 2007.
\end{flushright}

\end{beamerboxesrounded}
\end{frame}

\begin{frame}
A-t-on réellement trouvé, actuellement, comment faire une pêche durable ? %\textcolor{black!70!green}{}
\end{frame}


 
  \end{document}