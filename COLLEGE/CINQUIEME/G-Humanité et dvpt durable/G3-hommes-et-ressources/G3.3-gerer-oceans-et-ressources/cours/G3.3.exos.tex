\documentclass[12pt]{article}
\usepackage{fontspec}
\usepackage{xltxtra}
\setmainfont[Mapping=tex-text]{Century Schoolbook L}
 \usepackage[francais]{babel}
 
 \usepackage{geometry}
 \geometry{ hmargin=0.5cm, vmargin=0.5cm } 

\makeatletter
\renewcommand\section{\@startsection
{section}{1}{0mm}    
{\baselineskip}
{0.5\baselineskip}
{\normalfont\normalsize\textbf}}
\makeatother


\begin{document}

\fbox{\textbf{TD1 - Vidéo - La pêche en Galice}}
\begin{enumerate}
\item Où se situe la Galice ?
\item Quel type de pêche est pratiqué à Vigo ?
\item Quel type de pêche est pratiqué dans les petits villages de Galice ?
 \item Que demande l'eurodéputée Rosa Miguelez à l'UE ?
\item Liste les différents métiers de la mer.
\item Pourquoi la pêche est-elle une activité importante pour la population ?
\item de quoi est responsable le centre océanographique de Vigo?
\item quel est le rôle de l'Agence communautaire de contrôle de la pêche ?
\item Quel est le rôle des quotas  selon l'eurodéputé Raul Romeva ?
\item Quelles sont les deux principales menaces qui pèsent sur les ressources halieutiques selon les associations environnementales.
\end{enumerate}

\vfill

\fbox{\textbf{TD1 - Vidéo - La pêche en Galice}}
\begin{enumerate}
\item Où se situe la Galice ?
\item Quel type de pêche est pratiqué à Vigo ?
\item Quel type de pêche est pratiqué dans les petits villages de Galice ?
 \item Que demande l'eurodéputée Rosa Miguelez à l'UE ?
\item Liste les différents métiers de la mer.
\item Pourquoi la pêche est-elle une activité importante pour la population ?
\item de quoi est responsable le centre océanographique de Vigo?
\item quel est le rôle de l'Agence communautaire de contrôle de la pêche ?
\item Quel est le rôle des quotas  selon l'eurodéputé Raul Romeva ?
\item Quelles sont les deux principales menaces qui pèsent sur les ressources halieutiques selon les associations environnementales.
\end{enumerate}

\vfill

\fbox{\textbf{TD1 - Vidéo - La pêche en Galice}}
\begin{enumerate}
\item Où se situe la Galice ?
\item Quel type de pêche est pratiqué à Vigo ?
\item Quel type de pêche est pratiqué dans les petits villages de Galice ?
 \item Que demande l'eurodéputée Rosa Miguelez à l'UE ?
\item Liste les différents métiers de la mer.
\item Pourquoi la pêche est-elle une activité importante pour la population ?
\item de quoi est responsable le centre océanographique de Vigo?
\item quel est le rôle de l'Agence communautaire de contrôle de la pêche ?
\item Quel est le rôle des quotas  selon l'eurodéputé Raul Romeva ?
\item Quelles sont les deux principales menaces qui pèsent sur les ressources halieutiques selon les associations environnementales.
\end{enumerate}

\newpage
\fbox{\textbf{TD2 - Les ressources halieutiques aujourd'hui.}}\\

\textbf{Exercice 1 - Constat, lieu.}\\

\begin{enumerate}
\item \textbf{chrono p. 279}. Quelle est l'évolution de la quantité de poisson capturé depuis le début du XXe siècle ?\\
 \vspace{1cm}

\begin{minipage}{3cm}
 \item \textbf{p. 284-285.} Inscrit sur la carte les deux principales zones de pêches (bleu majuscule)\\
\item \textbf{p. 284-285.} les pays principaux producteurs (noir majuscule)
 \end{minipage}
 \begin{minipage}{10cm}
 \includegraphics[scale=0.80]{schema-monde.eps}
 \end{minipage}


%\newpage
%\fbox{\textbf{TD3 - Les problèmes de la pêche actuelle.}} \\
\vspace{1cm}
\textbf{Exercice 2 - Les problèmes de la pêche actuelle}\\
\begin{minipage}{12cm}
\item \textbf{doc 1 ci-joint / p. 279 / 2 p. 286}. Explique pourquoi l'océan n'est pas une ressource durable actuellement.
\vspace{6cm}
\end{minipage}
\fbox{
\begin{minipage}{7cm}
\textbf{Doc 1. L'océan, victime de la pollution terrestre.}\\
La pollution est transportée par les cours d'eau ou l'air. Elle est aussi bien due à l'agriculture qui utilise pesticides et engrais, qu'à l'industrie et aux déchets domestiques. Chaque année, on observe des zones océaniques impropres à la vie, elles sont situées au large des côtes où se concentrent les activités industrielles.
\begin{flushright}
D'après \textit{l'Atlas de l'océan mondial, 2007}
\end{flushright}
\end{minipage}
}
\end{enumerate}

\newpage
\fbox{\textbf{TD3 - Vers une pêche durable ?}}\\

\textbf{Exercice 1. Une définition de la pêche durable}\\
Surligne parmi les éléments suivants ceux qui correspondent à des méthodes de pêche durable. Rédige ensuite une définition de la pêche durable.\\

\fbox{
\begin{minipage}{10cm}
\begin{itemize}
 \item Satisfaire les besoins alimentaires humains actuels
 \item Réduire le nombre d'emplois liés à la pêche
 \item Sauvegarder les ressources halieutiques
 \item Priver les générations futures de consommer du poisson
 \item Réduire la quantité de poisson dans l'alimentation.
 \item Permettre aux pêcheurs de vivre de leur métier.
 \item Permettre aux générations futures de consommer du poisson.
 \item Interdire toute forme de pêche pour préserver les espèces marines.
 \item Limiter l'activité de pêche aux pays les plus riches.
 \end{itemize}
\end{minipage}
}
 \begin{minipage}{8cm}
Définition : 
\vspace{7cm}
 \end{minipage}

\vspace{1cm}

\textbf{Exercice 2 - Aquaculture, quotas et lois mondiales}\\

\begin{minipage}{10cm}
\textbf{Doc 2 p. 288 + doc ci-joint}. A-t-on réellement trouvé, actuellement, comment faire une pêche durable ?
\vspace{7cm}
\end{minipage}
\fbox{
\begin{minipage}{9cm}
En 1982, la Convention sur le droit de la mer est adoptée. Aujourd'hui respectée par 157 pays, elle définit les limites territoriales et maritimes et pose des règles pour l'exploitation des ressources maritimes. Le tribunal international du droit de la mer est chargé de régler les conflits.
En 1995, la FAO adopte un texte international, un <<code de bonne conduite>> pour une exploitation durable des ressources océaniques. Il insiste sur la nécessité de protéger les espèces, par des quotas mais aussi la protection des poissons jeunes. Ce texte propose aussi de limiter certaines techniques de pêche qui ont des effets destructeurs parce qu'elles ne sélectionnent pas le bon poisson et capturent des espèces non désirées>>.
D'après \textit{l'Atlas de l'océan mondial}, 2007.
\end{minipage}

}

\end{document}