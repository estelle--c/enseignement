 \documentclass{beamer}
  %\usepackage[utf8]{inputenc}
  \usepackage{fontspec}  %pour xelatex
 \usepackage{xunicode}  %pour xelatex
  \usetheme{Montpellier}
   \usepackage{color}
   \usepackage{xcolor}
   \usepackage{graphicx}
   \usepackage{ulem}
%   \usepackage{xkeyval}
   \usepackage{pst-tree}
   \usepackage{tabularx}
   \usepackage[french]{babel}
   \usepackage[absolute,showboxes,overlay]{textpos}     % déclaration du package
   %\textblockorigin{x}{y}                               % origine des positions
  % \TPshowboxestrue                                     % affiche le contour des textblock
   \TPshowboxesfalse 
   \setlength{\TPHorizModule}{30mm}
   \setlength{\TPVertModule}{\TPHorizModule}
   \date{}
 %  \usepackage{pstcol,pst-fill,pst-grad}
  \setbeamercolor{normal text}{fg=black}
 \setbeamercolor{section in head/foot}{fg=black}
  \setbeamercolor{subsection in head/foot}{fg=blue}
\beamerboxesdeclarecolorscheme{blocbleu}{black!60!white}{black!20!white}
\beamerboxesdeclarecolorscheme{blocimage}{black!60!white}{black!10!white}
\setbeamercolor{section in toc}{fg=black}
\setbeamercolor{subsection in toc}{fg=blue}
%\setbeamertemplate{background canvas}{\includegraphics
%   [width=\paperwidth,height=\paperheight]{Coran-7s.jpg}}


\AtBeginSection[]
{
 \begin{frame}
  \tableofcontents[currentsection,hideallsubsections]
 \end{frame}
}

\AtBeginSubsection[]
{
  \begin{frame}
  \tableofcontents[currentsection,currentsubsection]
  \end{frame}
}
  \title{{\textcolor{red}{Chapitre 1 - Les débuts de l'Islam}}}




\begin{document}

\newcommand{\df}[2]{\textcolor{red}{\underline{#1}: #2}}

\newcommand{\doc}[1]{
\begin{flushright}
\fbox{Documents : #1}
\end{flushright}
}

\newcommand{\con}[1]{\textcolor{blue}{\underline{Consigne}: #1}}

\newcommand{\perso}[2]{\textcolor{green}{\underline{#1}: #2}}

\begin{frame}
 \titlepage %{CHAPITRE 2 - LES IDENTIT�S MULTIPLES DE LA PERSONNE}
 \end{frame}

\section{Introduction}

\begin{frame}{Où est le monde musulman ? Quand a-t-il rayonné ?}{Capacité : je sais lire une chronologie et d'une carte}

\doc{Carte et chrono p. 11}

\con{Situe chronologiquement et spatialement le monde musulman. Ne fait pas de phrase.}


% BUT : mettre un alinéa au début du parag

%Prof : but de l'exercice : arriver en 2 ou 3 phrase à situer le contexte de ce que l'on va étudier.

\end{frame}

\begin{frame}
\underline{Fil directeur :} comment l'Islam devient, au Moyen Age, l'un des empires les plus important de l'Europe ?
\end{frame}

\section{I/ Le monde musulman : un nouvel empire qui apparaît et conquiert des territoires}

\begin{frame}{Un exemple de conquête musulmane : la Syrie-Palestine}{Capacité : je fais le récit d'un évènement historique}

\con{Vous êtes un journaliste et vous devez écrire un article sur la conquête pour votre journal. Lisez les documents. A l'aide de la méthode, collectez les informations demandées et notez-les dans la méthode.} \\

\con{Grâce aux infos collectées, remplissez le texte de l'article}

\end{frame}

\begin{frame}

\doc{3 p. 17}

\begin{itemize}
\item Entre le VIIe siècle et IXe siècle, les Arabes conquièrent toutes les côtes du sud et de l'est de la Méditerranée. Sortent d'Arabie peu après la mort du prophète Mohammed. Vont jusqu'à l'Espagne. 
\item Cause : djihad pour protéger l'Islam du christianisme, recherche de butin (razzias), islamisation.
\item Opposant : les anciens empires perses et byzantins. St déjà très affaiblis par des luttes internes. 
\item Lutte pr les T avec les byzantins pendant tout le 8e siècle. Méd devient rapidement une frontière entre les deux.

\end{itemize}



\end{frame}

\section{II/ Ce que nous apprennent les textes qui font l'Islam}

\begin{frame}{Que faut-il comprendre des principaux textes religieux musulmans ?}

\begin{textblock}{3}(3.4,0)
\includegraphics[width=1.7cm]{Coran-1s-Hegire.jpg}
\end{textblock}

Souligne dans le tableau : 
\begin{itemize}
\item en vert, les textes qui parlent de religion
\item en rouge, les textes qui parlent de la vie de Mohammed
\item en noir, les textes qui parlent des règles de la société.
\end{itemize}
\end{frame}


\begin{frame}
\begin{itemize}
\item 3 sortes de textes st importants.
\item Toute la vie des musulmans s'y retrouvent : la religion, l'histoire du prophète et les règles de vie en société.
\end{itemize}

\end{frame}


\begin{frame}{Qui est Mohammed ?}

\doc{carte de Arabie au temps de Mahomet}

\begin{itemize}
\item Arabie : terre de nomades
\item Mohammed serait né à La Mecque vers 570. Vient d'une famille de caravaniers -> Arabie sur les routes pour les échanges avec le monde byzantin et avec la Mer med (voir photo caravane).
\item Religion de Mahomet et de ces tribus : polythéistes
\item passer doc suivant. v. 610, il aurait reçu la "révélation" (env. 30 ans) : il aurait reçu de l'ange Gabriel des paroles de Dieu (voir doc). Il serait le messager pour passer le message d'Allah (prendrait la suite des messager précédent : Moïse et Jésus)
\item Il essait de répandre sa parole à La Mecque mais n'est pas cru. Il est chassé : l'Hégire (622) avec ses disciples. Il se réfugie à Yathrib qu'il rebaptise Médine (voir sur la carte).
\end{itemize}
\end{frame}

\begin{frame}

\begin{itemize}

\item comme un chef de clan de caravane, Mahomet fait des raids contre les intérêts éco de ses ennemis de La Mecque. Il finit par conquérir une grosse partie de l'Arabie (voir sur la carte). Il finit par conquérir La Mecque.
\end{itemize}

\vfill

\perso{Mohammed ou Mahomet (v. 570-632)}{Prophète de l'Islam. Il serait devenu le messager de Dieu après une révélation de Gabriel. En 622, il est exilé de sa ville natale, La Mecque vers Médine : c'est l'Hégire.}

\vfill

\df{Polythéisme}{Croire en plusieurs dieux.}

\df{Prophète}{personne qui interprète et transmet la parole divine.}

%\begin{itemize}
%\item Né à La Mecque
%\item aurait reçu la révélation : Dieu se serait adressé à lui
%\item a essayé de convertir son peuple. Fonctionne pas. Exilé de La Mecque. Se rend à Médine (L'hégire, 622)
%\end{itemize}

\end{frame}

\begin{frame}{Toute la vie sociale des musulmans est basée sur les textes sacrés.}{Exercice en autonomie, sur le cahier}

\doc{2 p. 16}

\con{Lit le texte pour toi. En 2-3 lignes, explique ce qui, dans le texte, montre que la vie des musulmans est dictée par les textes sacrés.}

\end{frame}


\section{III/ Vision de l'islam médiéval}

\begin{frame}{La ville de Bagdad, symbole de la culture musulmane}
% localisation, qui l'a construite... : Al-Mansur, 762, apogée au Xe. Dynastie des 

\doc{p. 20-21}

\end{frame}

\begin{frame}{-}{Exercice : étudier un plan de ville}

\doc{1 p. 20}

Noter sur la carte : Bagdad + fondation de Bagdad \\

\con{Dans ce plan, trouve des lieux appartenant à chacune des catégories suivantes :
\begin{itemize}
\item deux lieux religieux
\item deux lieux commerciaux
\item deux lieux politiques
\item deux infrastructures permettant le transport (eau, marchandise...)
\item une infrastructure permettant la protection de la ville.
\end{itemize}}
\end{frame}

\begin{frame}
\begin{itemize}
\item Bagdad a été fondée / calife Al-Mansour en 762. Capitale de l'Empire musulman. Ville majeure de l'empire. On y retrouve tous les lieux caractéristiques d'une grande ville : 

\end{itemize}
\end{frame}

\begin{frame}{La mosquée, un des lieux les plus important de la ville musulmane}



\begin{itemize}
\item Combien de mosquée à Bagdad.
\item Voir qe c'est un lieu de culte, ms aussi de culture (bibliothèque et savants y résident).


\doc{2 p. 18}

\item but pr le calife : construire la plus belle mosquée afin d'y attirer gens du monde entier.
\end{itemize}



\end{frame}

\section{Conclusion}

\begin{frame}{L'explosion du monde musulman}

\doc{3 p. 17}

\con{L'empire musulman est-il resté unifié ? Qui sont les voisins de l'empire musulman ?}
\end{frame}

%\begin{frame}{Exercice}{Je sais me repérer sur une frise chronologique}

%\df{Ici}{ici}

%\doc{1p.20}

%\con{En quelle date se sont passés les évènements suivants ?}

%\end{frame}

%\begin{frame}{Définitions}
%\df{Sumer}{ville de Mésopotamie} 

%\df{Egypte}{pays d'Afrique}

%\end{frame}



%\section{II/ }
%\section{III/ }

  \end{document}