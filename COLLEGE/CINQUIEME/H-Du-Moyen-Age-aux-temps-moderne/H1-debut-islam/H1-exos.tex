\documentclass[12pt,a4paper]{article}
%\usepackage[utf8]{inputenc}
\usepackage[francais]{babel}
%\usepackage[T1]{fontenc}
\usepackage{graphicx}
\usepackage[left=0.5cm,right=0.5cm,top=0.5cm,bottom=0.5cm]{geometry}
 \usepackage{fontspec}  %pour xelatex
 \usepackage{xunicode}  %pour xelatex
 \usepackage{xltxtra}
  \setmainfont[Mapping=tex-text]{Century Schoolbook L}

\begin{document}
\begin{center}
{\Huge Méthode 1 : Je sais raconter un événement historique : la conquête de la Syrie-Palestine}
\end{center}

\begin{enumerate}
\item Je dois collecter différentes informations dans les documents.
\begin{enumerate}
\item Début de l'évènement
\begin{itemize}
\item date (chrono) : 
\item lieu (carte) : 
\item acteurs (1 et 3 p. 12): 
\item causes (2 p. 12):
\vspace{1cm}
\end{itemize}

\item Déroulement de l'évènement
\begin{itemize}
\item Qu'est-ce qui se déroule ? (3 lignes env) (1 et 3 p. 12, 4 p. 13)
\end{itemize}
\vspace{2cm}

\item Conséquences
\begin{itemize}
\item comment il se finit (5 p. 13): 
\item pourquoi cet événement est important pour les populations (5 p. 13): 
\end{itemize} 
\end{enumerate}

\vspace{2cm}

\item J'écris un petit texte en me servant des informations que j'ai recueilli.
\end{enumerate}

\vfill

\begin{tabular}{|p{6cm}|p{6cm}|p{6cm}|}
\hline Nom du livre & Date probable & A quoi sert-il ? \\ 
\hline Le Coran & Version définitive du texte a été rédigée au IXe siècle & Il regroupe les message que Mohammed aurait reçu de Dieu afin de les dire aux hommes. Ce livre est la base des croyances et devoirs des musulmans \\ 
\hline Les hadiths & Textes regroupés entre le VII et le IXe siècle. Ils forment la Sunna. & Récits et autres paroles que Mohammed aurait dites. Ils servent d'exemples et de modèles aux musulmans. Ils servent à rendre la justice. \\ 
\hline La sira & Le texte le plus ancien date du IXe siècle & Connaître la vie de Mohammed (différentes sources) \\ 
\hline 
\end{tabular} 




\end{document}