\documentclass[12pt]{article}
\usepackage{fontspec}
\usepackage{xltxtra}
\setmainfont[Mapping=tex-text]{Century Schoolbook L}
 \usepackage[francais]{babel}
 
 \usepackage{geometry}
 \geometry{ hmargin=0.5cm, vmargin=0.5cm } 

\makeatletter
\renewcommand\section{\@startsection
{section}{1}{0mm}    
{\baselineskip}
{0.5\baselineskip}
{\normalfont\normalsize\textbf}}
\makeatother



\begin{document}
%\renewcommand{\figurename}{Doc.}

\newpage

\section*{Les cabinets de curiosité.}

\tiny{Les cabinets de curiosité sont des lieux où les particuliers entreposaient et exposaient les objets qu'ils collectionnaient. Apparaissent à la Renaissance (milieu XVe-milieu XVIIe). Ancêtre des musées actuels.}\\

\begin{minipage}{8cm}
\includegraphics[scale=1]{cab1.jpg} \\
\tiny Frontispice de Musei Wormiani Historia montrant l'intérieur du cabinet de curiosités de Ole Worm.
\end{minipage}
\begin{minipage}{8cm}
\includegraphics[scale=0.40]{cab2.jpg} 
\tiny Cabinet d'un particulier, Frans II Francken, 1625, Kunsthistorisches \\
\end{minipage}

Légende : 
\vspace{3cm}


{\normalsize 
\begin{enumerate}
\item A quelle époque apparaissent les cabinets de curiosité ?
\item quels sont leurs rôles ?
\item Liste les différentes types d'objets présents dans ces deux cabinets.
\vspace{1cm}
\end{enumerate}}

\section*{Les nouvelles sciences de la navigation et de la géographie.}

\begin{minipage}{9cm}
\begin{minipage}{9cm}
\includegraphics[scale=0.50]{Cantino_planisphere_(1502).jpg} \\
Planisphère de Cantino, 1502
\end{minipage}
\begin{enumerate}
 \item \normalsize Sur les deux cartes, entoure les espaces découverts par les Européens entre la fin du XVe et le début du XVIIe siècle.
\end{enumerate}
\end{minipage}
\hfill
\begin{minipage}{9cm}
\begin{minipage}{9cm}
\includegraphics[scale=0.15]{Mercator-1594.jpg} \\
Planisphère de Mercator, 1594 \\
\end{minipage}
\end{minipage}
\begin{minipage}{10cm}
\includegraphics[scale=0.08]{caravelle.jpg}
\includegraphics[scale=0.20]{Sextant-2.jpg}
\end{minipage}
\begin{minipage}{8cm}
{\tiny L'apparition d'une nouvelle forme de navigation avec la caravelle (à gauche) permet d'aller plus loin plus facilement. Le sextant (à droite) remplace l'astrolabe pour calculer les trajectoires des navires : suivant les étoiles, il est plus fiable et plus précis. }
\end{minipage}

%\hfill
%\begin{figure}

%\end{figure}
%\end{minipage}


\begin{enumerate}
\item \normalsize Explique les nouveauté qui ont permis aux Européens de conquérir de nouveaux territoires.
\end{enumerate}

\newpage

\section*{Le tour du monde de Fernand de Magellan par Antonio Pigafetta}

\textit{\small{Au XVe siècle, le commerce entre l'Europe et l'Asie est dominé par les Arabes et les Vénitiens. Pour éviter ces intermédiaires, Portugais et Espagnols recherchent de nouvelles routes maritimes qui leur permettraient d'atteindre directement l'Asie. En 1517, le roi d'Espagne confie à Fernand de Magellan la mission d'atteindre les îles Moluques, îles des épices.}} \\

\subsection*{Exercice 1. La première << circumnavigation de l'histoire >>}
\fbox{
\begin{minipage}{6cm}
\textbf{\normalsize doc 1. Les raisons du voyage.} \\
\small{<< Pour en venir au commencement de mon voyage, ayant entendu dire qu'il y avait en la cité de Séville une petite armée de cinq navires prête pour faire ce long voyage, à savoir la découverte des îles Moluques d'où viennent les épices, armée dont le capitaine général était Magellan, gentilhomme portugais, je partis avec plusieurs lettres en ma faveur de Barcelone. >>}
\begin{flushright}
A. Pigafetta, \textit{Navigation et découvrement de l'Inde supérieure et Iles de Malucque où naissent les clous de girofle (1519-1522)}
\end{flushright}
\end{minipage}
}
\begin{minipage}{12cm}
\fbox{
\begin{minipage}{12cm}
\textbf{\normalsize Doc 2. Antonio Pigafetta décrit la vie à bord.} \\
\small{<< Nous navigâmes pendant trois mois et vingt jours sans goûter d'aucune nourriture fraîche. Le biscuit que nous mangions n'était plus du pain mais une poussière mêlée de vers et imprégnée d'urine de souris. L'eau que nous étions obligés de boire était putride (= contaminée et non potable). \\
Nous fûmes même contraints, pour ne pas mourir de faim, de manger des morceaux de cuir. \\
Notre plus grand malheur était de nous voir attaqués d'une espèce de maladie (le scorbut), par laquelle les gencives se gonflaient au point de sumonter les dents. Et ceux qui étaient attaqués ne pouvaient prendre aucune nourriture. Dix-neuf d'entre nous moururent. >>}
\begin{flushright}
A. Pigafetta, \textit{Navigation et découvrement de l'Inde supérieure et Iles de Malucque où naissent les clous de girofle (1519-1522)} \\
\end{flushright}
\end{minipage} 
}

\includegraphics[scale=0.70]{dessin.eps}
\end{minipage}

%\includegraphics[scale=0.20]{Antonio.png}
\fbox{
\includegraphics[scale=0.25]{Magellan-tour-monde.eps}
}

\normalsize{
\begin{enumerate}
\item Cite les buts de l'expédition de Magellan : \\

\item Surligne sur la carte les îles Moluques cher à Antonio.
\item Inscrit sur la frise chronologique le début du voyage, la mort de Magellan et la date de fin du voyage.
\item Explique pourquoi c'est un voyage autour du monde. \\

%\item Tu es Antonio Pigafetta, participant à l'expédition de Magellan. Raconte cette expédition.
\end{enumerate}
}
\newpage

\textbf{Exercice 2. La conquêtes des terres visitées.}

\fbox{
\begin{minipage}{8cm}
\normalsize << Nous entrâmes [...] le jour de Sainte-Lucie, 13e du mois de décembre [à proximité de] la terre du Brésil qui abonde en toute sortes de denrées [...]. Les hommes et les femmes sont bien bâtis et conformés comme nous. Ils mangent quelques fois de la chair humaine, mais seulement celles de leurs ennemis. >>
\end{minipage}
}
\begin{minipage}{10cm}
Comment Pigafetta décrit-il les hommes rencontrés au cours de l'expédition ?
\vspace{3cm}
\end{minipage}

%\vfill

%\section*{Exercice 3. L'élaboration d'un nouveau regard géographique et géopolitique sur le monde.}

%\newpage

\section*{Localisation de la Renaissance.}

\begin{minipage}{10cm}
Carte.
\end{minipage}
\begin{minipage}{8cm}
\normalsize Grâce à la carte 3 p. 129, localise sur ta carte plusieurs éléments : 
\begin{enumerate}
\item colorie en rouge le foyer premier de la Renaissance : L'Italie.
\item colorie en orange les foyers hors d'Italie.
\end{enumerate}
\end{minipage}

\section*{Qui est Léonard de Vinci ?}


\fbox{
\begin{minipage}{11cm}
\textbf{Doc. 1 Biographie de Léonard de Vinci} \\
1452 : né à Vinci (près de Florence, Italie) \\
1468 : apprenti dans un atelier d'art (peinture, sculpture) de Florence. \\
1481 : peintre, sculpteur, ingénieur et musicien au service de Ludovic Sforza, duc de Milan \\
1495-1498 : réalise \textit{La Cène} \\
1499 : ingénieur à Venise. \\
1500-1515 : séjours à Rome (au service du pape), à Florence (au service de Laurent de Médicis) et à Milan. Il acquiert de nouvelles compétences (géométrie, anatomie, mécanique, architecture, urbanisme). \\
1503-1519 : Réalise \textit{La Joconde} \\
1516 : arrive en France au service de François Ier.  \\
1519 : mort à Amboise (France)
\end{minipage}
}
\begin{minipage}{8cm}
\begin{enumerate}
\item Surligne dans le texte les différents lieux dans lequel est allé Léonard de Vinci.
\item Souligne ces lieux sur la carte précédente. Que remarques-tu ?
\vspace{3cm} \\
\end{enumerate}
\end{minipage}

\section*{Quelques œuvres de Léonard de Vinci. En quoi sont-elles caractéristiques de la Renaissance ?}

\textbf{Doc 1. Quelques caractéristiques de la Renaissance.} \\
\fbox{
\begin{minipage}{14cm}
[Le tableau de la Dame à l'hermine] est le << premier portrait moderne de l'histoire.>> [...] La dame, Cécilia Gallerani est représentée alors qu'elle se détourne en un brusque mouvement, comme distraite par la venue de quelqu'un [...] et même le petit animal nous est montré en alarme comme effrayé par un évènement tout proche.
\begin{flushright}
D'après Pietro Marani, Léonard de Vinci, une carrière de Peinture, 2003. \\
\end{flushright}
Léonard distingue trois sortes de perspectives complémentaires : dans la première [...] plus un objet s'éloigne de l'oeil, plus il diminue en taille. Léonard codifie mathématiquement cette pratique. [...] La seconde perspective, dite linénaire, consiste à définir << comment les objets doivent être achevé avec d'autant moins de minutie q'ils sont éloignés >>. [Et enfin, dans la troisième perspective], les couleurs tendent à s'uniformiser à mesure que les objets s'éloignent. 
[...]
Contrairement à ces prédécesseurs, Léonard est porté par une véritable ambition scientifique. Il complète sa connaissance superficielle de l'homme par l'analyse de la structure interne des corps. [...] Lorsqu'il analyse le bras, il dessine d'abord l'os, puis il ajoute les muscles, les veines, les nerfs jusqu'à la peau.
\begin{flushright}
D'après Jérémie Koering, \textit{Léonard de Vinci}, musée du Louvre, 2003.
\end{flushright}
\end{minipage}
}
\begin{minipage}{4cm}

\begin{enumerate}
\item Surligne les trois nouveautés en peinture à la Renaissance.
\item Cite les trois nouvelles idées dans la perspective.
\item Entoure en rouge dans le tableau ces trois nouveautés.
\item Entoure en vert deux indices qui te montrent que L. de Vinci s'inspire de l'Antiquité. \\
\end{enumerate}
\end{minipage}

\underline{Le mécenat de Laurent de Médicis.} \\
\fbox{
\begin{minipage}{8cm}
<< Il avait rempli ses jardins de belles sculptures antiques, les allées du parc et toutes les pièces étaient garnies d'admirables statues anciennes, de peintures et d'bojets. [...]
Laurent favorisa toujours les beaux génies. A ceux qui, trop pauvres, n'auraient pu se consacrer à l'étude du dessin, il assurait les moyens de vivre et de se vêtir. Il accordait d'immenses récompenses à ceux qui réalisaient les meilleurs travaux. >>
\begin{flushright}
D'après Giogio Vasari, \textit{Les vies des plus excellents peintres, sculpteurs et architectes italiens}, vers 1550.
\end{flushright}
\end{minipage}
}
\begin{minipage}{10cm}
\begin{enumerate}
\item Explique à quoi sert un mécène. \\

\item Avec la biographie de L. de Vinci, cite les différents mécènes qu'il a eu.
\vspace{4cm}
\end{enumerate}
\end{minipage}

\newpage

\section*{Vésale et le corps humain, Copernic et l'héliocentrisme}
\fbox{
\begin{minipage}{19cm}{Biographie de Vésale} \\
André Vésale est un médecin et un anatomiste belge du XVIe siècle. Ses travaux sur le corps humain ont permis de faire entrer l'anatomie dans la modernité et de faire progresser les connaissances en médecine. Il publie un des libres les plus novateurs sur l'anatomie humaine, \textit{De humani corporis fabrica (Sur le fonctionnement du corps humain)}. 
\end{minipage} 
}

\begin{minipage}{6cm}{Base du cerveau montrant le chiasma optique, le cervelet.... Livre de Vésale}
\includegraphics[scale=2]{cerveau.jpg}
\begin{enumerate}
\item Grâce à la biographie 1 p. 130, complète le schéma ci-dessous avec les termes << Terre >> et << Soleil >>.
\item 1 p. 130. Quel problème Copernic a-t-il eu avec l'Eglise ?
\item Explique en quoi, chacun à leur manière, Vésale et Copernic ont été pionniers dans la pensée humaniste (Avant, Dieu était au centre des recherches, maintenant l'homme est au centre)
\end{enumerate}
\end{minipage}
\begin{minipage}{10cm}
\fbox{
\includegraphics[scale=0.70]{Copernic.eps}}
\end{minipage}

\vfill
%\newpage
\section*{Luther et les 95 thèses}

\underline{La question du salut divise les chrétiens.}\\ 

{\small \textit{La question du salut est récurrente dans la pensée chrétienne au Moyen Age et Renaissance : où va-t-on aller après notre mort (Paradis ? Purgatoire ? Enfer ?). Cette question va entraîner une division majeure dans le christianisme et la naissance d'une nouvelle religion, le protestantisme (dont l'un des acteurs majeur est Luther).}}

%\begin{minipage}{10cm}
\fbox{
\begin{minipage}{8cm}
\textbf{Doc 1. Depuis le Moyen-Age, l'Eglise catholique a instauré le système des \textbf{indulgences}}.\\ Contre des actes de piétés (prière, pélerinage), les mauvaises actions du pêcheurs sont pardonnées sur cette terre et au Purgatoire. Au fil du temps, ce système est perverti. Au lieu d'actes de piété, on donne de l'argent. \\
En 1506, le pape Léon X accorde des indulgences à tous ceux qui accepteraient de l'aider à financer sa nouvelle église à Rome, la basilique Saint-Pierre. 
\end{minipage}
}
\fbox{
\begin{minipage}{11cm}{\textbf{Doc 2. 95 thèses de Luther contre le pape}}\\ 
\textit{\tiny Les 95 thèses affichées à la porte de l'église de Wittenberg condamnent la vente des indulgences pour la basilique Saint-Pierre.}\\
6. Le pape ne peut pardonner les péchés qu'au nom de Dieu.\\
32. Toux ceux qui pensent gagner le ciel moyennant les lettres de pardon délivrées par les hommes s'en iront en enfer avec ceux qui les endoctrinent ainsi.\\
43. On doit enseigner aux chrétiens que celui qui fait du bien aux pauvres est à préférer à celui qui achète des indulgences.\\
86. Pourquoi le pape, dont le sac est plus gros que celui des plus riches, n'édifie-t-il pas au moins cette basilique de Saint-Pierre avec ses propres deniers, plutôt qu'avec l'argent des pauvres fidèles?
\end{minipage}
}
%\end{minipage}

\begin{minipage}{8cm}{\textbf{Doc 3. Le pape vendant des indulgences.}} \\
\includegraphics[scale=0.40]{antichrist.png} \\
\tiny L'Antéchrist vu par Lucas Cranach l'Ancien
\end{minipage}

\begin{enumerate}
\item Surligne Wittenberg sur la carte ci-dessous
\item Raconte en quelques lignes la dénonciation des indulgences par Martin Luther (cause, déroulement, conséquences)
\end{enumerate}

\section*{L'état religieux de l'Europe en 1648.}
%\includegraphics[scale=0.20]{file}

\begin{itemize}
\item Rempli la carte ci-dessous.
\item colorie en rouge les pays catholiques
\item colorie en vert les pays protestants.
\item colorie en bleu les pays anglicans.
\end{itemize}


\end{document}