 \documentclass{beamer}
  %\usepackage[utf8]{inputenc}
  \usepackage{fontspec}  %pour xelatex
 \usepackage{xunicode}  %pour xelatex
  \usetheme{Montpellier}
   \usepackage{color}
   \usepackage{xcolor}
   \usepackage{graphicx}
   \usepackage{ulem}
%   \usepackage{xkeyval}
   \usepackage{pst-tree}
   \usepackage{tabularx}
   \usepackage[french]{babel}
 %  \usepackage{pstcol,pst-fill,pst-grad}
  \setbeamercolor{normal text}{fg=black}
 \setbeamercolor{section in head/foot}{fg=black}
  \setbeamercolor{subsection in head/foot}{fg=blue}
\beamerboxesdeclarecolorscheme{blocbleu}{black!60!white}{black!20!white}
\beamerboxesdeclarecolorscheme{blocimage}{black!60!white}{black!10!white}
\setbeamercolor{section in toc}{fg=black}
\setbeamercolor{subsection in toc}{fg=blue}



\AtBeginSection[]
{
 \begin{frame}
  \tableofcontents[currentsection,hideallsubsections]
 \end{frame}
}

\AtBeginSubsection[]
{
  \begin{frame}
  \tableofcontents[currentsection,currentsubsection]
  \end{frame}
}
  \title{{\textcolor{red}{Partie 4 - Vers la modernité (fin XVe-XVIIe siècle) \\ Chapitre 1 - Les bouleversements culturels et intellectuels.}}}

\begin{document}
\begin{frame}
 \titlepage %{CHAPITRE 2 - LES IDENTIT�S MULTIPLES DE LA PERSONNE}
 \end{frame}

\begin{frame}
%TD1 - Les cabinets de curiosité.
\includegraphics[scale=1.2]{cab1.jpg} \\
\small Frontispice de Musei Wormiani Historia montrant l'intérieur du cabinet de curiosités de Ole Worm.
\end{frame}

\begin{frame}
\includegraphics[scale=0.50]{cab2.jpg} \\
\small Cabinet d'un particulier, Frans II Francken, 1625, Kunsthistorisches \\
\end{frame}


\begin{frame} 
%A l'époque de la Renaissance (XVe-XVIe siècle) apparaissent chez les particuliers lettrés un nouveau type de pièce : les cabinets de curiosité. Ils y rangent toutes les nouveautés de leurs temps.

%But du chapitre : comprendre quelles sont ces nouveautés qui ont permis, entre le   et le   siècle, d'entrer dans une ère << moderne >>.
\end{frame}

\section{I/Bouleversement géographique : l'ouverture du monde aux Européens.} 

Pourquoi a-t-on commencé à vouloir explorer la planète ?

\begin{frame}
Entoure en rouge les objets de découverte présent dans les cabinets de curiosité.
Note le en-dessous.
\end{frame}


\subsection{Les progrès de la navigation et des cartes géographiques}

\begin{frame}
\begin{minipage}{5cm}{Caravelle}
\includegraphics[scale=0.10]{caravelle.jpg}
\end{minipage}
\begin{minipage}{5cm}{Sextant}
\includegraphics[scale=0.40]{Sextant-2.jpg}
\end{minipage}
\end{frame}

%\begin{frame}{\underline{Les progrès de la navigation}}
%\begin{minipage}{6cm}{Caravelle} \\
%\includegraphics[scale=0.10]{caravelle.jpg} \\
%\end{minipage}
%\begin{minipage}{5cm}{Sextant} \\
%\includegraphics[scale=0.30]{Sextant-2.jpg} \\
%\end{minipage}
%\end{frame}

%\begin{minipage}{9cm}
%\fbox{
%\begin{minipage}{9cm}
\begin{frame}

\includegraphics[scale=0.50]{Cantino_planisphere_(1502).jpg}\\
Planisphère de Cantino, 1502
\end{frame}

%\end{minipage}
%}

\begin{frame}
\includegraphics[scale=0.18]{Mercator-1594.jpg} \\
Planisphère de Mercator, 1594
\end{frame}

\begin{frame}
\begin{enumerate}
 \item Sur les deux cartes, entoure les espaces découverts par les Européens entre la fin du XVe et le début du XVIIe siècle.
\end{enumerate}
\end{frame}

\begin{frame}
Explique les nouveauté qui ont permis aux Européens de conquérir de nouveaux territoires.
\end{frame}

\subsection{Magellan fait le tour de la Terre.}

\begin{frame}{Exercice. Le tour du monde de Fernand de Magellan par Antonio Pigafetta}
\end{frame}

\begin{frame}
\fbox{
\begin{minipage}{11cm}
\textbf{doc 1. Les raisons du voyage.} \\
\small{<< Pour en venir au commencement de mon voyage, ayant entendu dire qu'il y avait en la cité de Séville une petite armée de cinq navires prête pour faire ce long voyage, à savoir la découverte des îles Moluques d'où viennent les épices, armée dont le capitaine général était Magellan, gentilhomme portugais, je partis avec plusieurs lettres en ma faveur de Barcelone. >>}
\begin{flushright}
A. Pigafetta, \textit{Navigation et découvrement de l'Inde supérieure et Iles de Malucque où naissent les clous de girofle (1519-1522)}
\end{flushright}
\end{minipage}
}
\end{frame}

\begin{frame}
\fbox{
\begin{minipage}{11cm}
\textbf{Doc 2. Antonio Pigafetta décrit la vie à bord.} \\
\small{<< Nous navigâmes pendant trois mois et vingt jours sans goûter d'aucune nourriture fraîche. Le biscuit que nous mangions n'était plus du pain mais une poussière mêlée de vers et imprégnée d'urine de souris. L'eau que nous étions obligés de boire était putride (= contaminée et non potable). \\
Nous fûmes même contraints, pour ne pas mourir de faim, de manger des morceaux de cuir. \\
Notre plus grand malheur était de nous voir attaqués d'une espèce de maladie (le scorbut), par laquelle les gencives se gonflaient au point de sumonter les dents. Et ceux qui étaient attaqués ne pouvaient prendre aucune nourriture. Dix-neuf d'entre nous moururent. >>}
\begin{flushright}
A. Pigafetta, \textit{Navigation et découvrement de l'Inde supérieure et Iles de Malucque où naissent les clous de girofle (1519-1522)} \\
\end{flushright}
\end{minipage} 
}
\end{frame}

\begin{frame}
\includegraphics[scale=0.60]{dessin.eps}
\end{frame}

\begin{frame}
\fbox{
\includegraphics[scale=0.18]{Magellan-tour-monde.eps}
}
\end{frame}

\begin{frame}

\end{frame}

%\begin{frame}
%\begin{enumerate}
%\item Quel est le but de l'expédition de Magellan ? \pause\textcolor{black!70!green}{Aller chercher des épices.}
%\item Surligne sur la carte les îles Moluques cher à Antonio.
%\item Inscrit sur la frise chronologique le début du voyage, la mort de Magellan et la date de fin du voyage.
%\item Souligne dans le doc 1. les termes guerriers. Pourquoi autant de termes guerriers dans une expédition de découverte ? \\

%\item Tu es Antonio Pigafetta, participant à l'expédition de Magellan. Raconte cette expédition.
%\end{enumerate}
%\end{frame}

%\begin{frame}
%\begin{enumerate}
%\item Quel est le but de l'expédition de Magellan ? \pause\textcolor{black!70!green}{Aller chercher des épices.}
%\item Surligne sur la carte les îles Moluques cher à Antonio.
%\item Inscrit sur la frise chronologique le début du voyage, la mort de Magellan et la date de fin du voyage.
%\item Souligne dans le doc 1. les termes guerriers. Pourquoi autant de termes guerriers dans une expédition de découverte ? \\

%\item Tu es Antonio Pigafetta, participant à l'expédition de Magellan. Raconte cette expédition.
%\end{enumerate}
%\end{frame}

%\begin{frame}
%\includegraphics[scale=0.17]{Magellan-tour-monde.eps}
%\end{frame}

%\begin{frame} Correction
%\includegraphics[scale=0.17]{Magellan-tour-monde-1.eps}
%\end{frame}

%\begin{frame}
%\begin{enumerate}
%\item Quel est le but de l'expédition de Magellan ? \pause\textcolor{black!70!green}{Aller chercher des épices.}
%\item Surligne sur la carte les îles Moluques cher à Antonio.
%\item Inscrit sur la frise chronologique le début du voyage, la mort de Magellan et la date de fin du voyage.
%\item Souligne dans le doc 1. les termes guerriers. Pourquoi autant de termes guerriers dans une expédition de découverte ? \\

%\item Tu es Antonio Pigafetta, participant à l'expédition de Magellan. Raconte cette expédition.
%\end{enumerate}
%\end{frame}

%\begin{frame}
%\includegraphics[scale=0.60]{dessin.eps}
%\end{frame}

%\begin{frame} Correction
%\includegraphics[scale=0.60]{chrono1.eps}
%\end{frame}

%\begin{frame} Correction
%\includegraphics[scale=0.60]{chrono3.eps}
%\end{frame}

%\begin{frame} Correction
%\includegraphics[scale=0.60]{chrono2.eps}
%\end{frame}

%\includegraphics[scale=0.20]{Antonio.png}

%\begin{frame}
%\begin{enumerate}
%\item %Quel est le but de l'expédition de Magellan ? \pause\textcolor{black!70!green}{Aller chercher des épices.}
%\item %Surligne sur la carte les îles Moluques cher à Antonio.
%\item %Inscrit sur la frise chronologique le début du voyage, la mort de Magellan et la date de fin du voyage.
%\item Souligne dans le doc 1. les termes guerriers. Pourquoi autant de termes guerriers dans une expédition de découverte ? \\

%\item Tu es Antonio Pigafetta, participant à l'expédition de Magellan. Raconte cette expédition.
%\end{enumerate}

%\fbox{
%\begin{minipage}{11cm}
%\textbf{doc 1. Les raisons du voyage.} \\
%\small{<< Pour en venir au commencement de mon voyage, ayant entendu dire qu'il y avait en la cité de Séville une petite armée de cinq navires prête pour faire ce long voyage, à savoir la découverte des îles Moluques d'où viennent les épices, armée dont le capitaine général était Magellan, gentilhomme portugais, je partis avec plusieurs lettres en ma faveur de Barcelone. >>}
%\begin{flushright}
%A. Pigafetta, \textit{Navigation et découvrement de l'Inde supérieure et Iles de Malucque où naissent les clous de girofle (1519-1522)}
%\end{flushright}
%\end{minipage}
%}

%\end{frame}

%\begin{frame}
%\begin{enumerate}
%\item %Quel est le but de l'expédition de Magellan ? \pause\textcolor{black!70!green}{Aller chercher des épices.}
%\item %Surligne sur la carte les îles Moluques cher à Antonio.
%\item %Inscrit sur la frise chronologique le début du voyage, la mort de Magellan et la date de fin du voyage.
%\item %Souligne dans le doc 1. les termes guerriers. Pourquoi autant de termes guerriers dans une expédition de découverte ? \\

%\item Tu es Antonio Pigafetta, participant à l'expédition de Magellan. Raconte cette expédition.
%\end{enumerate}
%\end{frame}

%\begin{frame} Correction
%\begin{enumerate}
%\item %Quel est le but de l'expédition de Magellan ? \pause\textcolor{black!70!green}{Aller chercher des épices.}
%\item %Surligne sur la carte les îles Moluques cher à Antonio.
%\item %Inscrit sur la frise chronologique le début du voyage, la mort de Magellan et la date de fin du voyage.
%\item %Souligne dans le doc 1. les termes guerriers. Pourquoi autant de termes guerriers dans une expédition de découverte ? \\

%\item Tu es Antonio Pigafetta, participant à l'expédition de Magellan. Raconte cette expédition.
%\end{enumerate}
%\end{frame}

%\begin{frame}
%carte des conquêtes et des découvertes.
%Liste les différentes espaces géographiques concernés par ces conquêtes.
%\end{frame}

%\begin{frame}
% Deux idées dans les voyages de découvertes : 
% \begin{itemize}
%\item L'extraordinaire aventure des grandes découvertes : \colorbox{red}{voyage de Magellan (1519-1522)} : 1er voyage à faire le tour du monde.
%\item violence de la conquête et destruction des civilisations locales.
%\end{itemize}

%Ces découvertes permettent aux européens d'avoir un regard géographique et géopolitique nouveau sur le monde

%Grâce à cela, les Européens prennent possessions de terres partout dans le monde.
%\end{frame}

\section{II/ Bouleversement culturel : le temps de la Renaissance artistique}

\begin{frame}
Entoure en vert sur les deux cabinets des curiosités tous les objets faisant partis de la Renaissance artistique. \\
Note les dans la légende.

\end{frame}


\subsection{Où se passe la Renaissance ?}

\begin{frame}{Exercice. Localisation de la Renaissance}
3 p. 129
\includegraphics[scale=0.20]{carte-ren.jpg}
\end{frame}

\subsection{Exemple d'un artiste de la Renaissance, Léonard de Vinci.}

\begin{frame}{Exercice. Qui est L. de Vinci ?}
\begin{beamerboxesrounded}[scheme=blocimage]{Doc 1. Biographie de Léonard de Vinci}
1452 : né à Vinci (près de Florence, Italie) \\
1468 : apprenti dans un atelier d'art (peinture, sculpture) de Florence. \\
1481 : peintre, sculpteur, ingénieur et musicien au service de Ludovic Sforza, duc de Milan \\
1495-1498 : réalise \textit{La Cène} \\
1499 : ingénieur à Venise. \\
1500-1515 : séjours à Rome (au service du pape), à Florence (au service de Laurent de Médicis) et à Milan. Il acquiert de nouvelles compétences (géométrie, anatomie, mécanique, architecture, urbanisme). \\
1503-1519 : Réalise \textit{La Joconde} \\
1516 : arrive en France au service de François Ier.  \\
1519 : mort à Amboise (France)
\end{beamerboxesrounded}
\end{frame}

\begin{frame}{Exercice. En quoi les oeuvres de L. de Vinci sont-elles caractéristiques de la Renaissance ? }
\begin{beamerboxesrounded}[scheme=blocimage]{Doc. Quelques caractéristiques de la Renaissance}
[Le tableau de la Dame à l'hermine] est le << premier portrait moderne de l'histoire.>> [...] La dame, Cécilia Gallerani est représentée alors qu'elle se détourne en un brusque mouvement, comme distraite par la venue de quelqu'un [...] et même le petit animal nous est montré en alarme comme effrayé par un évènement tout proche.
\begin{flushright}
D'après Pietro Marani, Léonard de Vinci, une carrière de Peinture, 2003. \\
\end{flushright}
\end{beamerboxesrounded}
\end{frame}

\begin{frame}
\begin{beamerboxesrounded}[scheme=blocimage]{}
Léonard distingue trois sortes de perspectives complémentaires : dans la première [...] plus un objet s'éloigne de l'oeil, plus il diminue en taille. Léonard codifie mathématiquement cette pratique. [...] La seconde perspective, dite linénaire, consiste à définir << comment les objets doivent être achevé avec d'autant moins de minutie q'ils sont éloignés >>. [Et enfin, dans la troisième perspective], les couleurs tendent à s'uniformiser à mesure que les objets s'éloignent. 
[...] \\
Contrairement à ces prédécesseurs, Léonard est porté par une véritable ambition scientifique. Il complète sa connaissance superficielle de l'homme par l'analyse de la structure interne des corps. [...] Lorsqu'il analyse le bras, il dessine d'abord l'os, puis il ajoute les muscles, les veines, les nerfs jusqu'à la peau.
\begin{flushright}
D'après Jérémie Koering, \textit{Léonard de Vinci}, musée du Louvre, 2003.
\end{flushright}
\end{beamerboxesrounded}
\end{frame}

\subsection{L'importance du mécène.}

\begin{frame}{Exercice. Le mécenat de Laurent de Médicis}
\begin{beamerboxesrounded}[scheme=blocimage]{Doc. Un mécène}
<< Il avait rempli ses jardins de belles sculptures antiques, les allées du parc et toutes les pièces étaient garnies d'admirables statues anciennes, de peintures et d'bojets. [...]
Laurent favorisa toujours les beaux génies. A ceux qui, trop pauvres, n'auraient pu se consacrer à l'étude du dessin, il assurait les moyens de vivre et de se vêtir. Il accordait d'immenses récompenses à ceux qui réalisaient les meilleurs travaux. >>
\begin{flushright}
D'après Giogio Vasari, \textit{Les vies des plus excellents peintres, sculpteurs et architectes italiens}, vers 1550.
\end{flushright}
\end{beamerboxesrounded}
\end{frame}

\section{III/ Bouleversement scientifique : l'homme au centre des recherches.}

\begin{frame}
Sur les photographies des cabinets de curiosité, entoure en bleu ce qui peut relever des recherches scientifiques.
\end{frame}

\subsection{La recherche d'une meilleur connaissance de l'homme...}

\begin{frame}{Exercice. Vésale et le corps humain, Copernic et l'héliocentrisme.}
\begin{beamerboxesrounded}[scheme=blocimage]{Portraits de André Vésale et Nicolas Copernic}
\includegraphics[scale=0.30]{vesale.jpg}
\hfill
\includegraphics[scale=0.50]{copernic.jpg}
\end{beamerboxesrounded}
\end{frame}

\begin{frame}{Exercice. Vésale et le corps humain, Copernic et l'héliocentrisme.} 
\begin{beamerboxesrounded}[scheme=blocimage]{Vésale}
\includegraphics[scale=1.5]{cerveau.jpg}
\begin{minipage}{7cm}{Biographie de Vésale} \\
André Vésale est un médecin et un anatomiste belge du XVIe siècle. Ses travaux sur le corps humain ont permis de faire entrer l'anatomie dans la modernité et de faire progresser les connaissances en médecine. Il publie un des libres les plus novateurs sur l'anatomie humaine, \textit{De humani corporis fabrica (Sur le fonctionnement du corps humain)}. 
\end{minipage} 
\end{beamerboxesrounded}
\end{frame}

\begin{frame}{Exercice. Vésale et le corps humain, Copernic et l'héliocentrisme.} 
\begin{beamerboxesrounded}[scheme=blocimage]{Biographie de Copernic.}
Nicolas Copernic est un astronome polonais. Il effectue ses études universitaires en Pologne et Italie. Ses recherches en astronomie le conduisent à tirer la conclusion que la Terre et les autres planètes tournent autour du Soleil. Il s'oppose ainsi aux théories du savant grec Ptolémée énoncées au IIe siècle et admises par l'Eglise selon lesquelles la Terre, et donc l'humanité, est le centre de l'Univers. Prudent, Copernic repousse la publication de son livre, \textit{De la révolution des orbes célestes}, aux derniers instants de sa vie.
\end{beamerboxesrounded}
\end{frame}

\begin{frame}

\end{frame}

\section{IV/ Bouleversements religieux.}

\subsection{La remise en cause de l'unité du christianisme occidental.}

\begin{frame}
\begin{itemize}
\item Pré-requis : souvenir que, au Moyen Age, toute l'Europe occidentale est chrétienne.
\item Le salut
\end{itemize}
\end{frame}

\begin{frame}{Exercice : Luther et les 95 thèses.}
\begin{beamerboxesrounded}[scheme=blocimage]{Doc 1. Depuis le Moyen-Age, l'Eglise catholique a instauré le système des \textbf{indulgences}}
 Contre des actes de piétés (prière, pélerinage), les mauvaises actions du pêcheurs sont pardonnées sur cette terre et au Purgatoire. Au fil du temps, ce système est perverti. Au lieu d'actes de piété, on donne de l'argent. \\
En 1506, le pape Léon X accorde des indulgences à tous ceux qui accepteraient de l'aider à financer sa nouvelle église à Rome, la basilique Saint-Pierre. 
\end{beamerboxesrounded}
\end{frame}

\begin{frame}{Exercice : Luther et les 95 thèses.}
\begin{beamerboxesrounded}[scheme=blocimage]{Doc 2. 95 thèses de Luther contre le pape}
\textit{\tiny Les 95 thèses affichées à la porte de l'église de Wittenberg condamnent la vente des indulgences pour la basilique Saint-Pierre.}\\
6. Le pape ne peut pardonner les péchés qu'au nom de Dieu.\\
32. Toux ceux qui pensent gagner le cier moyennant les lettres de pardon délivées par les hommes s'en iron en enfer avec ceux qui les endoctrinent ainsi.\\
43. On doit enseigner aux chrétiens que celui qui fait du bien aux pauvres est à préférer à celui qui achète des indulgences.\\
86. Pourquoi le pape, dont le sac est plus gros que celui des plus riches, n'édifie-t-il pas au moin cette basilique de Saint-Pierre avec ses propres deniers, plutôt qu'avec l'argent des pauvres fidèles?
\end{beamerboxesrounded}
\end{frame}


\begin{frame}{Exercice : Luther et les 95 thèses.}
\begin{beamerboxesrounded}[scheme=blocimage]{Doc 3. Le pape vendant des indulgences.}
\includegraphics[scale=0.50]{antichrist.png} \\
\tiny L'Antéchrist vu par Lucas Cranach l'Ancien
\end{beamerboxesrounded}
\end{frame}

\subsection{L'état de l'Europe en 1648.}

\begin{frame}
\includegraphics[scale=0.20]{religion.png}
\end{frame}


  \end{document}