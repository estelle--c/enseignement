\documentclass[a4paper,12pt]{exam}

%printanswers % Pour ne pas imprimer les réponses (énoncé)
\addpoints % Pour compter les points
\pointsinrightmargin % Pour avoir les points dans la marge à droite
%\bracketedpoints % Pour avoir les points entre crochets
%\nobracketedpoints % Pour ne pas avoir les points entre crochets
\pointformat{.../\textbf{\themarginpoints}}
% \noaddpoints % pour ne pas compter les points
%\qformat{\textbf{Question\thequestion}\quad(\thepoints)\hfill} % Pour définir le style des questions (facultatif)
%\qformat{\thequestiontitle \dotfill \thepoints}


\usepackage{fontspec}
\usepackage{amsmath}
\usepackage{amssymb}
\usepackage{wasysym}
\usepackage{marvosym}
\usepackage{cwpuzzle}
  \usepackage{graphicx}
\defaultfontfeatures{Mapping=tex-text}
%\setmainfont{Linux Libertine}
\setmainfont{Century Schoolbook L}
%\usepackage[margin=1cm]{geometry}
    \usepackage[francais]{babel}
    \title{DS - Histoire - L'Orient ancien}
   
     
    % Si on imprime les réponses
    \ifprintanswers
    \newcommand{\rep}[1]{}
    \newcommand{\chariot}{}
    \else
    \newcommand{\rep}[1]{\fillwithdottedlines{#1}}
    \newcommand{\chariot}{\newpage}
    \fi


\makeatletter
\renewcommand\section{\@startsection
{section}{1}{0mm}    
{\baselineskip}
{0.5\baselineskip}
{\normalfont\normalsize\textbf}}
\makeatother
%\usepackage{titling}
%\renewcommand{\maketitlehooka}

 
\begin{document}

\begin{minipage}{4cm}
  Nom :
  
  Prénom :
  
  Classe : 
  
  Date : 
\end{minipage}
\hfill
\begin{minipage}{3.5cm}

{\small \begin{questions} \question[1] Orthographe et expression
\question[1] Présentation \end{questions}
}
\end{minipage}


\vspace{1cm}

\begin{center}

{\Large DS - Histoire - Les bouleversements culturels et intellectuels}

\vspace{0.5cm}
  \end{center}
Appréciation : \hfill {\large …/\numpoints\ } %\quad\quad …/\textbf{20}



\section*{Connaissances}
 
 \begin{questions} % Début de l'examen. Débute la numérotation des questions

\question[2] Explique qui est Luther et l'importance qu'il a eu au XVIe siècle.
% \begin{solution}
% Solution
% \end{solution}
\rep{2cm}


\question[2] Comment se nomme la théorie développée par N. Copernic ? Que dit-elle ?
% \begin{solution}
% Solution
% \end{solution}
\rep{2cm}


\question[1] Dans quel pays démarre la Renaissance ? Cite une ville de ce pays, connue pour son influence artistique à cette époque.
% \begin{solution}
% Solution
% \end{solution}
\rep{2cm}


\section*{Développement écrit}
\question[6] Grâce à tes connaissance, raconte le voyage de Magellan (3 paragraphes : causes, déroulement, conséquences) 
% \begin{solution}
% Solution
% \end{solution}
\rep{6cm}

\newpage

\rep{4cm}

\section*{Etude de peinture}

\question[6] Entoure et cite le nom des trois nouveauté artistiques de la Renaissance. 

\includegraphics[scale=0.50]{p1.jpg}



 \end{questions}

\end{document}