 \documentclass{beamer}
  %\usepackage[utf8]{inputenc}
  \usepackage{fontspec}  %pour xelatex
 \usepackage{xunicode}  %pour xelatex
  \usetheme{default}
   \usepackage{color}
   \usepackage{graphicx}
   \usepackage{ulem}
%   \usepackage{xkeyval}
   \usepackage{pst-tree}
   \usepackage{tabularx}
   \usepackage[french]{babel}
 %  \usepackage{pstcol,pst-fill,pst-grad}
  \setbeamercolor{normal text}{fg=blue}
 \setbeamercolor{section in head/foot}{fg=black}
  \setbeamercolor{subsection in head/foot}{fg=blue}
\beamerboxesdeclarecolorscheme{blocbleu}{black!60!white}{black!20!white}
\beamerboxesdeclarecolorscheme{blocimage}{black!60!white}{black!10!white}
\setbeamercolor{section in toc}{fg=black}
\setbeamercolor{subsection in toc}{fg=blue}



\AtBeginSection[]
{
 \begin{frame}
  \tableofcontents[currentsection,hideallsubsections]
 \end{frame}
}

\AtBeginSubsection[]
{
  \begin{frame}
  \tableofcontents[currentsection,currentsubsection]
  \end{frame}
}
 

\begin{document}
\begin{frame}
 \titlepage %{CHAPITRE 2 - LES IDENTIT�S MULTIPLES DE LA PERSONNE}
 \end{frame}
 
 \begin{frame}
 \tableofcontents
 \end{frame}
      \title{{\textcolor{red}{Partie 3 - Regards sur l'Afrique.}}}
      
      % Connaissance 1. 1 civilisation de l'Afrique subsaharienne (au choix) + gds courants d'�changes des marchandises (saisis dans leurs permanences: sel or du Soudan, esclaves...) entre le 8e et le XVIe.
      % d�marchaes : �tude du temps longs : entre le VIIIe et le XVIe si�cle + ex au choix d'une civilisation de l'Afrique subsaharienne parmi les suivantes (Mali, Monomotapa)

mise en valeur des points suivants : 
- extension (importance du support carto)
- modalit�s du pvr
- richesse et participation aux grands circuits des �changes
- r�alisation artistiques et architecturales
- contact ac monde arabo-musulman

But : montrer que Af sub a donn� naissance, dans les si�cles correspondant au MA europ, � des civilisations brillantes et originales

Introduction : 
carte et chrono p. 93
Explication orale : Avant : Empire du Ghana. Ms à décliné pour différentes raisons (poussée des musulmans almoravides, + sécheresse du à exploitation intensive des ressources forestières).
Des petits rois (royaume de sosso, Mali) acquiert leur indépendance.

Rappel : 
- Islam et expansion musulmane : 7s siècle : Mahomet fonde la nlle religion. diffusion de islam + conquête Af du N entre le 7e et le 8e siècle.
- Au même moment s'est développé 1 empire : l'Empire du Ghana (VIII-12e)
- dvpt de la religion musulmane en Af subsaharienne et orientale.

Docs : 
chrono.
Carte de expansion musulmane en Af du N.
carte des Empires africains --> 
- voir Empire du Ghana.
- voir les 4 civilisations majeures

Résumé : Entre le VIII et le XVIe siècle, l'Afrique subsaharienne a donné naissance, à des civilisations brillantes et originales. Les quatre principales se sont développées en Afrique, trois à l'Ouest dans la région du Mali, une au Sud-Est. Ces civilisations se sont développées en contact avec le monde musulman et l'Occident médiéval. Elles se sont imprégnées pour certaines de l'Islam.

\section{I/ Une civilisation : l'Empire du Mali.}

\begin{frame}
\textbf{Questions : }

\underline{Un empire étendue et riche.}\\
1) Situe l'Empire du Mali au niveau géographique et chronologique.
2) Lit l'introduction p. 94.
doc 1 p. 94 : quelles sont les richesses du Mali ?

\underline{Un empire en lien avec la civilisation arabo-musulmane et occidentale.}\\
doc 5 p. 95. Quelles sont les deux religions de l'Empire du Mali ?
doc 4 p. 99. Sur la carte, par quels moyens s'échangent les produits entre le monde musulman et l'Afrique? Quels sont ces produits ?
doc 3 p. 99. Comment sont représentés les musulmans sur la carte ? D'où provient cette carte ? Qu'est-ce que cela prouve sur la connaissance des empires africains par les Occidentaux ?

\underline{Un développement artistique lié à sa richesse.}\\
Décrit les photographies de la mosquée de Tombouctou. En quoi est-ce un bâtiment impressionnant ?
\end{frame}


\begin{frame}
\hspace{1cm} L'Empire du Mali s'est constitué sur les ruines de l'Empire du Ghana, en Afrique de l'Ouest. Il s'est développé entre le XIII et le XIVe siècle.
 
Il s'enrichit sous le règne de Mansa Musa (XIVe siècle). Sa puissance se fonde sur le contrôle du commerce de l'or et de villes majeures dans les échanges transsahariens. La richesse de Mansa Musa est connue jusqu'en Occident (il est dessiné à l'égal des rois occidentaux sur une carte portugaise).
\end{frame}
\begin{frame}

Ces empires africains sont en contact avec le monde arabo-musulman :  
\begin{itemize}
\item Ils reçoivent des caravanes de marchands traversant le Sahara. A l'est, ils commercent grâce aux \textcolor{red}{comptoirs} arabo-musulmans. Ils vendent leurs richesses locales (or, sel, ivoire, esclaves) contre des tissus, armes, du blés.
\item Les empereurs maliens se sont convertis à l'Islam. Ils construisent de grandes mosquées afin de convertir le reste de la population.
\item Ils accueillent des savants, lettrés et juristes musulmans venus visiter (la mosquée de Tombouctou est construite par un architecte andalou).
\end{itemize} 

Tout cela permet une émulation artistique et architecturale (mosquée de Tombouctou).

 \textcolor{red}{\underline{Comptoir}: installation commerciale d'un Etat dans un pays étranger.}

\end{frame}



%empire du Mali (12-14e) a �t� un Etat r�put� jusqu'en Europe, surtt � son apog�e lors du r�gne du c�l�bre Kankan Moussa, repr�sent� � l'�gal d'un roi europ�en sur un portulan de 1375. 
%fondation l�gendaire, tjs chant� / les griots actuels. fonde puissance sur le contr�le du commerce de l'or et des grandes villes du n�goce transsaharien, domin� 1 gigantesque territoire (de Atlantique � est du Niger actuel), et dvp� une civilisation brillante attirant lettr�s, juristes, savants.

  % Capacit� : 
    % - connaitre la p�riode et la situation de la civilisation de l'Afrique subsaharienne choisie.
    % - qq aspect d'une civilisation de l'Af subsaharienne et de sa production artistique. 
    
\subsection{Routes, acteurs et victimes des traites}

  \begin{frame}
  \textcolor{red}{\underline{Traite}: commerce et transport des esclaves noirs africains.}
  \end{frame}

\subsection{La traite transsaharienne.}

 \underline{La traite transsaharienne}
 Questions p. 102-103 (Oral)
 doc 1 p. 102.
 1) Définit le terme "razzia" grâce au vocabulaire p. 100.
 2) Qui sont les deux acteurs qui enlèvent les esclaves ? De quelle origine sont les esclaves ?
 
 doc 3 p. 103. 
 1) Quelles sont les conditions de voyage à travers le Sahara ? Comment se nomment les convois ?
 2) Où s'arrêtent-ils pour se ravitailler ?
 
doc 5 p. 103.
1) Décrit la miniature.

Exercice écrit : décrit quelques aspects de la traite transsaharienne. 
Parag 1 : situe au niveau géographique et chronologique cette traite
Parag 2 : razzias, acteurs, vente
Parag 3 : quel besoin d'esclaves ?
 
 \textcolor{red}{\underline{razzia}: attaque rapide contre un territoire pour capturer ses habitants.}
 
 \subsection{Les traites africaines.}
 doc 3 p. 105 (écrit)
 Tableau : 

Correction.
 
 
 \end{itemize}
 
      % Connaissance 2. traites orientales, transahariennes et interne � l'Afrique noire : routes commerciales, acteurs et victimes du trafic.
      % d�marche : �tude de la naissance et du dvpt des traites n�gri�res � partir de exemple au choix d'une route ou d'un trafic des esclaves vers l'Af du Nord ou l'Orient --> d�bouche sur une carte des courants de la traite avt le XVIe si�cle.
      
% capacit� : 
% - conqu�te et expansion arabo-musulmane en Af du Nord et en Af orientale.
% - carte de l'Af et de ses �changes entre le 8e et le 16e.
% - d�crire qq aspects de la traite orientale ou de la traite transsaharienne.

  \end{document}