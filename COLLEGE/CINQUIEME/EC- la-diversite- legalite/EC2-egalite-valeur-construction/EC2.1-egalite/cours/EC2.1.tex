\documentclass{beamer}
  %\usepackage[utf8]{inputenc}
  \usepackage{fontspec} %pour xelatex
 \usepackage{xunicode} %pour xelatex
  \usetheme{default}
   \usepackage{color}
   \usepackage{graphicx}
   \usepackage{ulem}
% \usepackage{xkeyval}
   \usepackage{pst-tree}
   \usepackage{tabularx}
   \usepackage[french]{babel}
 % \usepackage{pstcol,pst-fill,pst-grad}
  \setbeamercolor{normal text}{fg=blue}
 \setbeamercolor{section in head/foot}{fg=black}
  \setbeamercolor{subsection in head/foot}{fg=blue}
\beamerboxesdeclarecolorscheme{blocbleu}{black!60!white}{black!20!white}
\beamerboxesdeclarecolorscheme{blocimage}{black!60!white}{black!10!white}
\setbeamercolor{section in toc}{fg=black}
\setbeamercolor{subsection in toc}{fg=blue}



\AtBeginSection[]
{
 \begin{frame}
  \tableofcontents[currentsection,hideallsubsections]
 \end{frame}
}

\AtBeginSubsection[]
{
  \begin{frame}
  \tableofcontents[currentsection,currentsubsection]
  \end{frame}
}
 

\begin{document}
\begin{frame}
 \titlepage %{CHAPITRE 2 - LES IDENTIT?S MULTIPLES DE LA PERSONNE}
 \end{frame}
 
 \begin{frame}
 \tableofcontents
 \end{frame}
      \title{{\textcolor{red}{PARTIE 2 - L'Egalit�, une valeur en construction \\ Chapitre 1 - L'�galit� : un principe r�publicain}}}


% 6h en tout (DS compris)

% Conn : pcpe fondamental de la Rep. r�sultat des conqu�tes historiques progressives et s'inscrit dans la loi.
% D�marche : �tude centr�e sur r�le de redistribution dans la r�duction des in�galit�.  Fct� de fiscalit� et de protection sociale explicit� � partir d'ex: progressivit� de l'imp�t sur le revenu, pcpes de la s�curit� sociale. pcpe de contribution est 1 aspect d�cisif de la responsabilit� individuelle.

% Egalit� = aucun individu ne saurait avoir des droits sup � 1 autre. Egalit� devant la loi, devant l'imp�t, devant le service public.
% 2 id�e :�galit� des droits ms aussi meilleur r�partition des biens et ressources 

\section{I/ La notion d'�galit�.}

\begin{frame}
\underline{Questions :}
Exercice �crit : Rempli l'organigramme.
\end{frame}



\section{II/ Une notion durement acquise.}
\begin{frame}
\begin{beamerboxesrounded}[scheme=blocimage]{Doc 1. Une chronologie de l'�galit�} 
\includegraphics[scale=0.10]{chrono-egalite.jpg}
\end{beamerboxesrounded}


\underline{Question :}
Depuis quand se bat-on pour l'�galit� ?
\end{frame}

\begin{frame}
L'�galit� entre tous les individus est un combat de tous les jours qui a commenc� au moment de la R�volution fran�aise. Depuis, diff�rentes batailles nous ont permis d'avoir l'�galit� que l'on poss�de aujourd'hui.
\end{frame}






