 \documentclass{beamer}
   %\usepackage[utf8]{inputenc}
   \usepackage{fontspec} %pour xelatex
  \usepackage{xunicode} %pour xelatex
   \usetheme{Montpellier}
    \usepackage{color}
    \usepackage{graphicx}
    \usepackage{ulem}
 % \usepackage{xkeyval}
    \usepackage{pst-tree}
    \usepackage{tabularx}
    \usepackage[french]{babel}
  % \usepackage{pstcol,pst-fill,pst-grad}
   \setbeamercolor{normal text}{fg=blue}
  \setbeamercolor{section in head/foot}{fg=black}
   \setbeamercolor{subsection in head/foot}{fg=blue}
 \beamerboxesdeclarecolorscheme{blocbleu}{black!60!white}{black!20!white}
 \beamerboxesdeclarecolorscheme{blocimage}{black!60!white}{black!10!white}
 \setbeamercolor{section in toc}{fg=black}
 \setbeamercolor{subsection in toc}{fg=blue}
 
 
 
 \AtBeginSection[]
 {
  \begin{frame}
   \tableofcontents[currentsection,hideallsubsections]
  \end{frame}
 }
 
 \AtBeginSubsection[]
 {
   \begin{frame}
   \tableofcontents[currentsection,currentsubsection]
   \end{frame}
 }
  
 \title{{\textcolor{red}{PARTIE 2 - L'Egalité, une valeur en construction \\ Chapitre 1 - L'égalité : un principe républicain}}}
 
 \begin{document}
 \begin{frame}
  \titlepage %{CHAPITRE 2 - LES IDENTIT�S MULTIPLES DE LA PERSONNE}
  \end{frame}
  
  \begin{frame}
  \tableofcontents
  \end{frame}
       
 
 
 % 6h en tout (DS compris)
 
 % Conn : pcpe fondamental de la Rep. résultat des conquêtes historiques progressives et s'inscrit dans la loi.
 % Démarche : étude centrée sur rôle de redistribution dans la réduction des inégalité.  Fct° de fiscalité et de protection sociale explicité à partir d'ex: progressivité de l'impôt sur le revenu, pcpes de la sécurité sociale. pcpe de contribution est 1 aspect décisif de la responsabilité individuelle.
 
 % Egalité = aucun individu ne saurait avoir des droits sup à 1 autre. Egalité devant la loi, devant l'impôt, devant le service public.
 % 2 idée :égalité des droits ms aussi meilleur répartition des biens et ressources 
 
 \section{I/ La notion d'égalité.}
 
 \begin{frame}
 \underline{Questions :}
 Exercice écrit : Rempli l'organigramme.
 \end{frame}
 
 
 
 \section{II/ Une notion durement acquise.}
 \begin{frame}
 \begin{beamerboxesrounded}[scheme=blocimage]{Doc 1. Une chronologie de l'égalité} 
 \includegraphics[scale=0.15]{chrono-egalite.jpg}
 \end{beamerboxesrounded}
 
 
 \underline{Question :}
 Depuis quand se bat-on pour l'égalité ?
 \end{frame}
 
 \begin{frame}
 L'égalité entre tous les individus est un combat de tous les jours qui a commencé au moment de la Révolution française. Depuis, différentes batailles nous ont permis d'avoir l'égalité que l'on possède aujourd'hui.
 \end{frame}
 
 
 \end{document}
 
 
 
 