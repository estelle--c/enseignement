\documentclass[12pt]{article}
\usepackage{fontspec}
\usepackage{xltxtra}
\setmainfont[Mapping=tex-text]{Century Schoolbook L}
 \usepackage[francais]{babel}
 
 \usepackage{geometry}
 \geometry{ hmargin=0.2cm, vmargin=0.2cm }

\makeatletter
\renewcommand\section{\@startsection
{section}{1}{0mm}
{\baselineskip}
{0.5\baselineskip}
{\normalfont\normalsize\textbf}}
\makeatother


\begin{document}

\fbox{
\begin{minipage}{19cm}
\textbf{Doc 1. La notion d'égalité}\\
Evidemment, nous ne sommes pas égaux naturellement : nous avon des tailles inégales, des poids inégaux, des forces physiques inégales. Nous ne pouvons pas tous être champion olympique ou prix Nobel. Une des merveilles de l'humanité réside dans les différences qui fon que nous reconnaissons chaque femme et chaque homme comme une personne différente de toutes les autres personnes. La République ne nie pas cette réalité, ni ne veut supprimer les différences entre chaque homme et chaque femme. Mais elle leur reconnaît la même dignité et veut organiser la société pour que chacun ait les mêmes droits. \\
L'égalité est un \textbf{ideal} et un \textbf{programme} : elle n'est jamais acquise. Elle signifie que la République doit toujours progresser dans le sens de l'égalité. Elle doit par exemple faire en sorte que les hommes ne dominent pas les femmes. \\
\begin{flushright}
Alain Etchegoyen, \textit{L'idée républicaine aujourd'hui}, 2004
\end{flushright}
\end{minipage}}

\vspace{0.2cm}

\fbox{
\begin{minipage}{19cm}
\textbf{Doc 2. Différents textes de loi.}\\
"Les hommes naissent et demeurent libres et égaux en droits." \\
\begin{flushright}
\textit{Déclaration des droits de l'Homme et du citoyen}, art. 1, 1789 \\
\end{flushright}

"3. La loi garantit à la femme, dans tous les domaines, des droits égaux à ceux de l'homme". \\
10. La Nation assure à l'individu et à la famille les conditions nécessaires à son développement. \\
11. Elle garantit à tous, notamment à l'enfant, à la mère et aux vieux travailleurs, la protection matériel, le repos et les loisirs. Tout être humains qui, en raison de son âge, de son état hysique, de sa situation économique, se trouve dans l'incapacité de travailler à le droit d'obtenir  de la collectivité de moyens convenables d'existence."
\begin{flushright}
\textit{Préambule de la Constitution de 1946}, alinéa 3, 10, 11
\end{flushright}
\end{minipage}}

\vspace{0.2cm}
\fbox{
\begin{minipage}{5cm}
\textbf{Doc 3. Photographie d'étudiants passant un examen}\\
\includegraphics[scale=0.50]{doc3.jpg}
\end{minipage}} \hfill
\fbox{
\begin{minipage}{14cm}
\textbf{Doc 4. L'impôt au service de l'égalité}\\
L'impôt sur le revenu modifie la répartition des richesses entre les individus qui composent une société : entre les riches et les pauvres, les familles et les célibataires, entre les générations. L'impôt est un moyen privilégié de redistribution pour réduire les inégalités. L'impôt sur le revenu est un pilier essentiel de notre République. Il incarne le principe selon lequel chaque citoyen a droit d'accès au même bien public, mais aussi le devoir de participer à la solidarité nationale. Il est un moyen de corriger les inégalités.\\
\begin{flushright}
D'après le site du CREG de l'Académie de Versailles, 2007
\end{flushright}
\end{minipage}}

\end{document}