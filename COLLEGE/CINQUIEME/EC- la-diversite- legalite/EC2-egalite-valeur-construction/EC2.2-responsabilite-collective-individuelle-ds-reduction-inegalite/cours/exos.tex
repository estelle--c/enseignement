\documentclass[12pt]{article}
\usepackage{fontspec}
\usepackage{xltxtra}
\setmainfont[Mapping=tex-text]{Century Schoolbook L}
 \usepackage[francais]{babel}
 
 \usepackage{geometry}
 \geometry{ hmargin=0.5cm, vmargin=0.5cm }

\makeatletter
\renewcommand\section{\@startsection
{section}{1}{0mm}
{\baselineskip}
{0.5\baselineskip}
{\normalfont\normalsize\textbf}}
\makeatother


\begin{document}

\fbox{
\begin{minipage}{19cm}
\textbf{Doc 1. L'action d'une association}\\
L'AFEV (Association de la Fondation étudiante pour la ville) regroupe des étudiants bénévoles qui, deux heures par semaine, aident des jeunes, principalement des collégiens, à surmonter leurs difficultés scolaires. Ils les accompagnent en lecture, les aident à devenir autonomes. Ils les informent sur leurs parcours d'orientation. Ils interviennent aussi auprès des enfants nouvellement arrivés en France.\\
\begin{flushright}
www.afev.fr
\end{flushright}
\end{minipage}}

\vspace{0.2cm}

\fbox{
\begin{minipage}{19cm}
\textbf{Doc 2. L'action de l'Etat}\\
Repérés par les chefs d'établissement, les enfants de l'internat d'excellence de Sourdun (Seine-et-Marne) ont été recrutés parmi les élèves méritants issus de milieux défavorisés. Tous sont boursier. [...] L'établissement accueillera des élèves de la 6e jusqu'aux classes préparatoires aux grandes écoles. L'établissement a déjà reçu 1,5 million d'euros de l'Etat et des collectivités territoriales. <<Avec ces moyens sans commune mesure avec un collège lambda, on peut réaliser énormément de projets>>, explique une enseignante.\\
Chaque classe a déjà son programme. Chaque soir, les heures d'étude sont au programme. \\
\begin{flushright}
D'après Marie-Estelle Pech, \textit{Le Figaro}, 31 août 2009.\\
\end{flushright}
\end{minipage}}

\vspace{0.2cm}
\fbox{
\begin{minipage}{7cm}\\
\textbf{Doc 3. Une association engagée pour l'emploi des handicapés}\\
\includegraphics[scale=0.15]{exo1.jpg}
\end{minipage}}\hfill
\fbox{
\begin{minipage}{12cm}
\textbf{Doc 4. Que dit la loi ?}\\
<<Afin de garantir le respect du principe d'égalité de traitement à l'égard des travailleurs handicapés, l'employeur prend, en fonction des besoins dans une situation concrète, les mesures appropriées pour leur permettre [...] d'accéder à un emploi ou de conserver un emploi correspondant à leur qualification [...] ou pour qu'une formation adaptée à leurs besoins leur soit dispensée.>>\\
\begin{flushright}
Article L5213-6, Code du travail.
\end{flushright}
\end{minipage}}
\hfill
\begin{enumerate}
\item doc 1 et 2. Montre que les actions de l'AFEC et de l'Etat ont pour objectif de favoriser <<l'égalité des chances>> à l'école
\item doc 3. A qui cette affiche est-elle destinée ? Que propose-t-elle ?
\item Doc 3 et 4. Comment l'association Agefiph aide-t-elle les employeurs à appliquer le texte de loi en faveur des travailleurs handicapés ?
\end{enumerate}.

\underline{Exercice écrit :} Qui agit contre les inégalités ? De quelles manières ?

\newpage

\end{document}