 \documentclass{beamer}
   %\usepackage[utf8]{inputenc}
   \usepackage{fontspec} %pour xelatex
  \usepackage{xunicode} %pour xelatex
   \usetheme{Montpellier}
    \usepackage{color}
    \usepackage{graphicx}
    \usepackage{ulem}
 % \usepackage{xkeyval}
    \usepackage{pst-tree}
    \usepackage{tabularx}
    \usepackage[french]{babel}
  % \usepackage{pstcol,pst-fill,pst-grad}
   \setbeamercolor{normal text}{fg=blue}
  \setbeamercolor{section in head/foot}{fg=black}
   \setbeamercolor{subsection in head/foot}{fg=blue}
 \beamerboxesdeclarecolorscheme{blocbleu}{black!60!white}{black!20!white}
 \beamerboxesdeclarecolorscheme{blocimage}{black!60!white}{black!10!white}
 \setbeamercolor{section in toc}{fg=black}
 \setbeamercolor{subsection in toc}{fg=blue}
 
 
 
 \AtBeginSection[]
 {
  \begin{frame}
   \tableofcontents[currentsection,hideallsubsections]
  \end{frame}
 }
 
 \AtBeginSubsection[]
 {
   \begin{frame}
   \tableofcontents[currentsection,currentsubsection]
   \end{frame}
 }
  
 \title{{\textcolor{red}{Chapitre 2 - Responsabilités collectives et individuelles dans la réduction des inégalités.}}}
 
 \begin{document}
 \begin{frame}
  \titlepage %{CHAPITRE 2 - LES IDENTIT�S MULTIPLES DE LA PERSONNE}
  \end{frame}
  
  \begin{frame}
  \tableofcontents
  \end{frame}
       
 
 
 % 6h en tout (DS compris)
 
 
 \begin{frame}

 \end{frame}

docs + exos.

Les inégalités et les discrimination sont combattue par actions qui engagent les citoyens individuellement et collectivement. Des politiques visant à lutter contre les inégalités et discriminations font l'objet de débats entre les citoyens, entre les mouvements politiques

% prob de égalité entre hommes et femmes.
 
 \end{document}
 
 
 
 