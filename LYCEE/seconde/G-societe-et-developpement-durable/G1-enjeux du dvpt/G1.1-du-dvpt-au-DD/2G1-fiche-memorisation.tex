
\documentclass[12pt,a4paper,landscape,twocolumn]{article}
%\usepackage[utf8]{inputenc}
\usepackage[francais]{babel}
%\usepackage[T1]{fontenc}
\usepackage{graphicx}
\usepackage[left=0.5cm,right=0.5cm,top=0.5cm,bottom=0.5cm]{geometry}
 \usepackage{fontspec}  %pour xelatex
 \usepackage{xunicode}  %pour xelatex
 \usepackage{xltxtra}
  \setmainfont[Mapping=tex-text]{Century Schoolbook L}

\begin{document}




\fbox{
\begin{minipage}{13.5cm}
TITRE
\vspace{0.7cm}
\end{minipage}
}

\vspace{0.2cm}

\fbox{
\begin{minipage}{13.5cm}
NOTIONS ET MOTS-CLÉS \\
%PIB, IDH, IPH
%taux de croissance
%Dvpt, DD, inégalités, jeux d'échelle, Nord et Sud, transition démographique, durabilité des modes de dvpt.


\vspace{8.2cm}
\end{minipage}
}

\vspace{0.2cm}

\fbox{
\begin{minipage}{7cm}
MÉTHODES ET CAPACITÉS \\
- prélever, hiérarchiser et confronter des informations dans une carte.

- Lire une carte et en exprimer par écrit les idées clés

- Rédiger un texte construit et argumenté en utilisant le vocabulaire géographique spécifique.
\vspace{4cm}
\end{minipage}
}
\hspace{0.2pt}
\fbox{
\begin{minipage}{19.8cm}
ORGANIGRAMME DE RÉSUMÉ
\vspace{8cm}
\end{minipage}
}

\newpage
\fbox{
\begin{minipage}{13cm}
CHIFFRES-CLÉS %chiffres importants à connaitre

\vspace{3cm}
\end{minipage}
}

\vspace{0.3cm}

\fbox{
\begin{minipage}{13cm}
CARTE DE REPÉRAGE

\includegraphics[width=11cm]{Carte_monde.eps}

%\vspace{6.2cm}
\end{minipage}
}


\end{document}