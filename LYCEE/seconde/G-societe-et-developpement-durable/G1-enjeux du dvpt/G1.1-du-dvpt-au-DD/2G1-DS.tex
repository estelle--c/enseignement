% !TeX encoding = UTF-8
\documentclass[a4paper,12pt]{exam}

\printanswers % Pour ne pas imprimer les réponses (énoncé)
\addpoints % Pour compter les points
\pointsinrightmargin % Pour avoir les points dans la marge à droite
%\bracketedpoints % Pour avoir les points entre crochets
%\nobracketedpoints % Pour ne pas avoir les points entre crochets
\pointformat{.../\textbf{\themarginpoints}}
% \noaddpoints % pour ne pas compter les points
%\qformat{\textbf{Question\thequestion}\quad(\thepoints)\hfill} % Pour définir le style des questions (facultatif)
%\qformat{\thequestiontitle \dotfill \thepoints}


\usepackage{fontspec}
\usepackage{amsmath}
\usepackage{amssymb}
\usepackage{wasysym}
\usepackage{marvosym}
\usepackage{cwpuzzle}
  \usepackage{graphicx}
\defaultfontfeatures{Mapping=tex-text}
%\setmainfont{Linux Libertine}
\setmainfont{Century Schoolbook L}
%\usepackage[margin=1cm]{geometry}
    \usepackage[francais]{babel}
    \title{DS - Histoire - Les débuts de l'Islam}
   \usepackage[left=0.5cm,right=2cm,top=0.5cm,bottom=0.5cm]{geometry}
     
    % Si on imprime les réponses
    \ifprintanswers
    \newcommand{\rep}[1]{}
    \newcommand{\chariot}{}
    \else
    \newcommand{\rep}[1]{\fillwithdottedlines{#1}}
    \newcommand{\chariot}{\newpage}
    \fi


\makeatletter
\renewcommand\section{\@startsection
{section}{1}{0mm}    
{\baselineskip}
{0.5\baselineskip}
{\normalfont\normalsize\textbf}}
\makeatother
%\usepackage{titling}
%\renewcommand{\maketitlehooka}

 
\begin{document}

\begin{minipage}{4cm}
  Nom :
  
  Prénom :
  
  Classe : 
  
  Date : 
\end{minipage}
\hfill
\begin{minipage}{3.5cm}

{\small \begin{questions} \question[1] Orthographe et expression
\question[1] Présentation \end{questions}
}
\end{minipage}


\vspace{1cm}

\begin{center}

{\Large DS - Géographie - Les enjeux du développement durable}

\vspace{0.5cm}
  \end{center}

 Appréciation \hfill {\large …/\numpoints\ } %\quad\quad …/\textbf{20}

%Attention, tu dois rédiger des phrases pour répondre aux questions. Tu n'auras pas les points si ce n'est pas le cas.


\section*{Etude de carte}



\includegraphics[width=10cm]{DS1.jpg}

\begin{questions}
\question[1] Quel indicateur est représenté sur cette carte ?

\begin{solution}
\vspace{1cm}
\end{solution}

\question[1] Explique pourquoi est-ce un indicateur utile dans la définition de la notion de développement.

\begin{solution}
\vspace{1cm}
\end{solution}

\question[6] Etudie cette carte (utilise le vocabulaire approprié). N'oublie pas de toujours justifier tes idées par des chiffres de la carte.

\begin{solution}
\vspace{5cm}
\end{solution}



 \section*{Rédiger un texte construit et argumenté en utilisant le vocabulaire géographique spécifique.}
 
 \question[5] Explique le développement durable (constat, causes, conséquences)
 
 \begin{solution}
 \vspace{10cm}
 \end{solution}
 
 
 % I/ Pop augmente
 % A/ constat : chiffre aujourd'hui, dans 50 ans / où augmente : ds les pays pauvres (moins dvpt)
 % B/ Causes : transitions démog pas finis dans certains pays -> niveau de dvpt pas atteint (niveau culture, enseignement). Pas de pol de natalité qui fonctionne.
 % C/ les csq : idée de surpopulation ? de ressource plus disponible après.
 
 % II/ Du dvpt durable aux modes durables de dvpt
 %A/ une naissance dans les pays du nord après leur transition démog
 % B/ le dvpt remis en cause dans les pays pauvre.
  % C/ faut penser local : ne fait pas la même politique dans un pays où pop augmente énormément et dans un autre où augmente pas du tout.
 
 \end{questions}
 
\end{document}