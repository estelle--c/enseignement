
\documentclass[12pt,a4paper,landscape,twocolumn]{article}
%\usepackage[utf8]{inputenc}
\usepackage[francais]{babel}
%\usepackage[T1]{fontenc}
\usepackage{graphicx}
\usepackage[left=0.5cm,right=0.5cm,top=0.5cm,bottom=0.5cm]{geometry}
 \usepackage{fontspec}  %pour xelatex
 \usepackage{xunicode}  %pour xelatex
 \usepackage{xltxtra}
  \setmainfont[Mapping=tex-text]{Century Schoolbook L}

\begin{document}


\fbox{
\begin{minipage}{13.5cm}
TITRE
\vspace{0.7cm}
\end{minipage}
}

\vspace{0.2cm}

\fbox{
\begin{minipage}{13.5cm}
PERSONNAGES \\


\vspace{3.3cm}

\end{minipage}
}

\vspace{0.2cm}

\fbox{
\begin{minipage}{13.5cm}
DÉFINITIONS \\
\vspace{4cm}
\end{minipage}
}

\vspace{0.2cm}

\fbox{
\begin{minipage}{7cm}
MÉTHODES ET CAPACITÉS \\
\begin{itemize}
\item Etudier un graphique : prélever, hiérarchiser et confronter des informations.
\item Situer et caractériser une date dans un contexte chronologique (la grande famine dans l'explosion migratoire des irlandais)
\item Rédiger un texte construit et argumenté en utilisant le vocabulaire historique spécifique : Lier les événements (constats-causes-conséquences)
\end{itemize}

\end{minipage}
}
%\hspace{0.2}
\fbox{
\begin{minipage}{19.8cm}
CARTE MENTALE
\vspace{8cm}
\end{minipage}
}

\newpage
\fbox{
\begin{minipage}{13cm}
CHRONO DE REPÉRAGE %dates importantes, périodes...

\vspace{1cm}

\includegraphics[width=13cm]{trait-de-chrono-vierge.eps}

\vspace{1cm}
%\includegraphics{trait-de-chrono-vierge.eps}

\end{minipage}
}

\vspace{0.3cm}

\fbox{
\begin{minipage}{13cm}
CARTE DE REPÉRAGE

\includegraphics[width=10cm]{croquis-monde.eps}

\vspace{1.2cm}
\end{minipage}
}



\end{document}