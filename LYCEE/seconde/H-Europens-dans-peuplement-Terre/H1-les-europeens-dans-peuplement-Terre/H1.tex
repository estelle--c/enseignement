 \documentclass{beamer}
  %\usepackage[utf8]{inputenc}
  \usepackage{fontspec}  %pour xelatex
 \usepackage{xunicode}  %pour xelatex
  \usetheme{Montpellier}
   \usepackage{color}
   \usepackage{xcolor}
   \usepackage{graphicx}
   \usepackage{ulem}
%   \usepackage{xkeyval}
   \usepackage{pst-tree}
   \usepackage{tabularx}
   \usepackage[french]{babel}
 %  \usepackage{pstcol,pst-fill,pst-grad}
  \setbeamercolor{normal text}{fg=black}
 \setbeamercolor{section in head/foot}{fg=black}
  \setbeamercolor{subsection in head/foot}{fg=blue}
\beamerboxesdeclarecolorscheme{blocbleu}{black!60!white}{black!20!white}
\beamerboxesdeclarecolorscheme{blocimage}{black!60!white}{black!10!white}
\setbeamercolor{section in toc}{fg=black}
\setbeamercolor{subsection in toc}{fg=blue}
\date{}



\AtBeginSection[]
{
 \begin{frame}
  \tableofcontents[currentsection,hideallsubsections]
 \end{frame}
}

\AtBeginSubsection[]
{
  \begin{frame}
  \tableofcontents[currentsection,currentsubsection]
  \end{frame}
}
  \title{{\textcolor{red}{Thème introductif - Les Européens dans le peuplement de la Terre \\ La place des populations de l'Europe dans le peuplement de la Terre.}}}


\begin{document}

\newcommand{\df}[2]{\textcolor{red}{\underline{#1}: #2}}

\newcommand{\doc}[1]{
\begin{flushright}
\fbox{Documents : #1}
\end{flushright}
}

\newcommand{\con}[1]{\textcolor{blue}{\underline{Consigne}: #1}}

\newcommand{\rep}[1]{\textcolor{green}{\underline{Réponse}: #1}}

\begin{frame}{Thème général de l'année}
\begin{center}
{\Huge \textcolor{red}{Les Européens dans l'histoire du monde}}
\end{center}
\end{frame}




\begin{frame}
 \titlepage %{CHAPITRE 2 - LES IDENTIT�S MULTIPLES DE LA PERSONNE}
 \end{frame}

\begin{frame}{Manuel p. 8-33}

\end{frame}


\begin{frame}{Fils directeurs: }
\begin{itemize}
\item Comment évolue la population européenne entre l'Antiquité et le XIXe siècle ?
\item Pourquoi étudie-t-on cette évolution ?
\item Que nous apprend l'exemple des Irlandais sur les migrations européens vers le monde au XIXe siècle ?
\end{itemize}
\end{frame}
itemize
\section{I/ La croissance de la pop européenne de l'Antiquité au XIXe siècle}


\begin{frame}
\df{Croissance démog}{augmentation de la pop d'un pays, soit / accroissement naturel ou par l'arrivée de nouveaux hab.}

\vfill

\df{Accroissement naturel}{augmentation de la pop. Calcule : nb de naissance - nb de morts pour un an sur 1000 hab.}
\end{frame}


\subsection{A/ La permanence du foyer européen.}

\begin{frame}{Le foyer européen}

\doc{1 p. 10}

 \con{Explique la permanence du foyer européen (n'oublie pas d'utiliser les chiffres)} \\
 \rep{vers 0 : une 30 aines de points -> 30 millions d'hab. Peuplement déja présent. continue d'augmenter pdt tt le MA et l'époque moderne. deb 19e : plus d'une centaine de points --> augmentation de la pop. Permanence car tjs présent.}
 
 \vfill
 
 \con{Cite les autres foyers de peuplement et explique leurs importances.} \\
 \rep{Chine et Inde : deux 1ers foyers de peuplement au monde.}
\end{frame}

\begin{frame}{Résumé}
\begin{itemize}
\item 3 foyers pcpaux de peuplement dans le monde à travers les siècles : la Chine, l'Inde et l'Europe. Europe est le 3e (derrière Chine et Inde).
\item jusqu'au XVIIIe : évolution lente (augmente peu). Par contre, à partir du XVIIIe, forte augmentation de la pop.
\end{itemize}

\df{Foyers de peuplement}{Région où la densité de pop est forte.}

\end{frame}

\begin{frame}{Une évolution constante ?}
\doc{3 p. 11}
\con{Etudie le graphique} \\

Méthode du graphique : d'abord regarder le titre, puis ce qui est évalué dans le graphique et enfin le graphique lui-même.

\rep{évolution lente jusqu'au XIXe siècle puis accélération au 19e. Passe de 32 millions de personnes en 0 -> multiplie par 2 au MA, par 3 à ép moderne. Mais à partir du XVIIIe siècle --> brutale accélération. Suit la courbe de la pop mondiale => phénomène qui se retrouve partout dans le monde.}
\end{frame}

\begin{frame}{Résumé}
\begin{itemize}
\item jusqu'au XVIIIe : évolution lente (globalement stable : forte natalité (env. 10 enfants / fe mais forte mortalité infantile)). Par contre, à partir du XVIIIe, forte augmentation de la pop.
\end{itemize}

\end{frame}

\subsection{C/ Malgré tout, crises et croissances}

\begin{frame}{L'exemple du MA.}
\doc{1 + 2 + 3 p. 12}

\con{Présente les documents. Montre, grâce au graphique, les aléas de la croissance démographique puis explique ses causes.} \\
\rep{Entre 1150 et 1400 : forte augmentation de la pop parisienne : passe de 50.000 à 250.000. Ensuite, XIV-XVe : très forte baisse. reprise au XVe.}
\rep{cause augmentation : Au MA, amélioration des techniques agricoles + climat + chaud => augmentation de la pop. Cause baisse : la peste -> manque d'hygiène, prob de maladie, prob de la médecine.}

\end{frame}

\begin{frame}
\df{Crise démographique}{Hausse brutale du nb de morts. On est en crise dès que le nb de mort double.}
\end{frame}

\begin{frame}{Résumé : comparaison entre 1 période d'esor démog et 1 grave crise démo.}
- Place du foyer européen relativement stable jusqu'au 18e / rapport aux espaces chinois et indiens.
- permanence pas = à absence d'évolution. Alterne période de croissance et de repli.  Faut analyser causes.
\end{frame}


\subsection{B/ A partir de la 2e moitié du XVIIIe siècle : accélération de la croissance démographique.}

\begin{frame}
- phénomène touche tous les continents => re-regarde 1er graphique
- faut expliquer pourquoi nettement + marqué en Europe --> part dans pop mondiale passe de 15\% à 25\%. 
- conduit à interrogation sur la transition démog et ses facteurs explicatifs.
\end{frame}

\begin{frame}{Exercice : les causes de la transition démog.}{prélever, hiérarchiser et confronter plusieurs documents \\ cerner le sens général d'un document et le mettre en relation avec la situation historique étudiée.}

\doc{2 p. 19. Louis Pasteur explique comment éviter les infection dus aux microbes}
\doc{3 p. 19. L'industrie chimique au service des progrès alimentaires.}
\doc{2 p. 17. le modèle de la transition démog}

\con{Après avoir présenté ces deux documents, tu expliquera l'augmentation de la population européenne au XIXe siècle et ses causes}



\end{frame}

\section{II/ L'émigration d'Européens vers d'autres continents au XIXe siècle}

\subsection{A/ L'émigration des Irlandais aux USA.}

\begin{frame}{Constat} %{Prélever les informations \\ Créer un organigramme afin de comprendre les causes à effet.}

\doc{1 p. 24; 2 p. 27; 2 p. 30}
1 p. 20
1 et 2 p. 24 : comprendre que irlandais pdt tout le 19e et deb 20e. repérer le pic de la gde famine : 45-49. Voir comment on fait (n°3)
2 p. 30 causes : 1 noble organise le départ de certains de ses fermiers non solvables. migrante --> bcp de femmes. 
2 p. 27 : cause : gde famine (def p.27)



\con{Présente les trois documents. Explique l'émigration irlandaise aux USA. Attention de hiérarchiser tes idées.}
-> hiérarchiser : constat, causes, conséquences.

CONSTAT : 1 p. 20 / 1 p. 24
CAUSES : 2 P. 30 / 2 P. 27 / 2 P. 30
CONSÉQUENCES : 2 p. 24 / 2 p. 30

\end{frame}

\begin{frame}
\df{Emigration}{Quitter son pays pour un autre pays, de façon définitive ou temporaire, pour des raisons diverses.}
\end{frame}

\begin{frame}
Grande famine entre 1845-1852.
Causes : politique désastreuse de la GB, méthodes agricoles inappropriées, apparition du mildiou -> parasite qui détruit les pommes de terre (nourriture pcpale des paysans).
Dep 2 siècle, 1 act a été mis en place / anglais : au lieu de transmettre 1 terre à l'aîné, l'a transmet à ts les garçons --> division des propriétés, -- facile de cultiver. Y met de la pomme de terre pr que ce soit + simple à cultiver.
baisse pomme de terre -> famine.
\end{frame}

\subsection{B/ Autres populations et lieux d'émigration}

\begin{frame}{-}{Prélever des informations d'une carte.}

\doc{2 p. 9}

\con{Après avoir présenté la carte, tu expliquera quels sont les différents lieux d'émigration et les différentes nationalités qui y partent.}

\end{frame}

\begin{frame}
\begin{itemize}
\item Aux XIX-deb XX : essor considérable des migrations européennes vers autres continents. Env. 60 millions de migrants quittent l'Europe entre 1820 et 1914. Leurs raisons sont variés (voir ex. sur Irlande), les migrations sont temporaires ou durables. \\
\item La principale raison est toute fois économique (pauvreté ou recherche de nouvel argent). 
\item Ces émigrations ne sont pas un phénomène ponctuel : c'est un processus à long terme -> on émigre souvent avec des parents ou des amis de parents déjà sur place => chaîne de personnes.
\item Difficile de décider de migrer : décision difficile à prendre. Bcp de personnes n'y arrivent pas -> 1/4 des migrants retournent chez eux.
\end{itemize}
\end{frame}

\begin{frame}
\begin{itemize}
\item essor considérable des migrations europ vers autres continents
\item Causes :  hausse brutale de pop, dvpt gal de la mobilité, mutations éco et sociales et progrès des transports.
\item env 60 millions de migrants quittent l'Europe entre 1820 et 1914, surtt vers Am. 
\item Raisons diverses
\begin{itemize}
\item diff de dvpt éco entre régions d'origines
\item influence de logiques collectives (familiales ou régionales)  ou motivation individuelles.
\item migrations temporaires ou durables.
\end{itemize}

\item permanences se dégagent qd même : 
\begin{itemize}
\item prépondérance du facteur éco
\item difficulté de la migration -> nb important de retour (1/4 des migrants).
\end{itemize}

\item Ce qu'il faut faire comprendre : émigration pas phénomène ponctuel ms processus à moyen ou long terme. Décision de migrer répond à gde complexité de facteurs.
\end{itemize}


\end{frame}

  \end{document}