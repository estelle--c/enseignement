 \documentclass{beamer}
  %\usepackage[utf8]{inputenc}
  \usepackage{fontspec}  %pour xelatex
 \usepackage{xunicode}  %pour xelatex
  \usetheme{Montpellier}
   \usepackage{color}
   \usepackage{xcolor}
   \usepackage{graphicx}
   \usepackage{ulem}
%   \usepackage{xkeyval}
   \usepackage{pst-tree}
   \usepackage{tabularx}
   \usepackage[french]{babel}
 %  \usepackage{pstcol,pst-fill,pst-grad}
  \setbeamercolor{normal text}{fg=black}
 \setbeamercolor{section in head/foot}{fg=black}
  \setbeamercolor{subsection in head/foot}{fg=blue}
\beamerboxesdeclarecolorscheme{blocbleu}{black!60!white}{black!20!white}
\beamerboxesdeclarecolorscheme{blocimage}{black!60!white}{black!10!white}
\setbeamercolor{section in toc}{fg=black}
\setbeamercolor{subsection in toc}{fg=blue}
\date{}



\AtBeginSection[]
{
 \begin{frame}
  \tableofcontents[currentsection,hideallsubsections]
 \end{frame}
}

\AtBeginSubsection[]
{
  \begin{frame}
  \tableofcontents[currentsection,currentsubsection]
  \end{frame}
}
  \title{{\textcolor{red}{Thème introductif - Les enjeux du développement. \\ Du développement au développement durable.}}}




\begin{document}

\newcommand{\df}[2]{\textcolor{red}{\underline{#1}: #2}}

\newcommand{\doc}[1]{
\begin{flushright}
\fbox{Documents : #1}
\end{flushright}
}

\newcommand{\con}[1]{\textcolor{blue}{\underline{Consigne}: #1}}

\newcommand{\rep}[1]{\textcolor{green}{\underline{Réponse}: #1}}

\begin{frame}{Thème général de l'année}
\begin{center}
{\Huge \textcolor{red}{Société et développement durable}}
\end{center}
\end{frame}




\begin{frame}
 \titlepage %{CHAPITRE 2 - LES IDENTIT�S MULTIPLES DE LA PERSONNE}
 \end{frame}

% Ce chapitre va répondre aux questions majeurs qui se posent à humanité



\begin{frame}{Fils directeurs: }
\begin{itemize}
\item Comment réduire les inégalités entre les populations ?
\item Comment faire face aux besoins croissants d'une humanité toujours plus nombreuse ?
\item Comment mettre en oeuvre des modes durables de développement ?
\end{itemize}
\end{frame}


  \end{document}