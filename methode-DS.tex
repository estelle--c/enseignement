\documentclass[a4paper,12pt]{exam}

%printanswers % Pour ne pas imprimer les réponses (énoncé)
\addpoints % Pour compter les points
\pointsinrightmargin % Pour avoir les points dans la marge à droite
%\bracketedpoints % Pour avoir les points entre crochets
%\nobracketedpoints % Pour ne pas avoir les points entre crochets
\pointformat{.../\textbf{\themarginpoints}}
% \noaddpoints % pour ne pas compter les points
%\qformat{\textbf{Question\thequestion}\quad(\thepoints)\hfill} % Pour définir le style des questions (facultatif)
%\qformat{\thequestiontitle \dotfill \thepoints}


\usepackage{fontspec}
\usepackage{amsmath}
\usepackage{amssymb}
\usepackage{wasysym}
\usepackage{marvosym}
\usepackage{cwpuzzle}
\defaultfontfeatures{Mapping=tex-text}
%\setmainfont{Linux Libertine}
\setmainfont{Century Schoolbook L}
%\usepackage[margin=1cm]{geometry}
    \usepackage[francais]{babel}
    \title{DS - Histoire - L'Orient ancien}
   
     
    % Si on imprime les réponses
    \ifprintanswers
    \newcommand{\rep}[1]{}
    \newcommand{\chariot}{}
    \else
    \newcommand{\rep}[1]{\fillwithdottedlines{#1}}
    \newcommand{\chariot}{\newpage}
    \fi


\makeatletter
\renewcommand\section{\@startsection
{section}{1}{0mm}    
{\baselineskip}
{0.5\baselineskip}
{\normalfont\normalsize\textbf}}
\makeatother
%\usepackage{titling}
%\renewcommand{\maketitlehooka}

 
\begin{document}

\begin{minipage}{4cm}
  Nom :
  
  Prénom :
  
  Classe : 
  
  Date : 
\end{minipage}
\hfill
\begin{minipage}{3.5cm}

{\small \begin{questions} \question[1] Orthographe et expression
\question[1] Présentation \end{questions}
}
\end{minipage}


\vspace{1cm}

\begin{center}

{\Large DS - Histoire - Les difficultés de la monarchie de Louis XVI}

\vspace{0.5cm}
  \end{center}
Appréciation : \hfill {\large …/\numpoints\ } %\quad\quad …/\textbf{20}


\begin{questions} % Début de l'examen. Débute la numérotation des questions
\section*{Connaissances}
 
\begin{multicols}{2}
\PuzzleUnsolved
\begin{Puzzle}{20}{14}
|{} |{} |{} |{} |{} |{} |{} |{} |{} |{} |{} |{} |{} |{} |{} |{} |{} |{} |{} |{} |
|{} |{} |{} |{} |{} |{} |{} |{} |{} |{} |{} |{} |{} |{} |{} |{} |{} |{} |{} |{} |
|{} |{} |{} |{} |{} |{} |{} |{} |{} |{} |{} |{} |{} |{} |{} |{} |{} |{} |{} |{} |
|{} |{} |{} |{} |{} |{} |{} |{} |{} |{} |{} |{} |{} |{} |{} |{} |{} |{} |{} |{} |
|{} |{} |{} |{} |{} |{} |{} |{} |{} |{} |{} |{} |{} |{} |{} |{} |{} |{} |{} |{} |
|{} |{} |{} |{} |{} |{} |{} |{} |{} |{} |{} |{} |{} |{} |{} |{} |{} |{} |{} |{} |
|{} |{} |{} |{} |{} |{} |{} |{} |{} |{} |{} |{} |{} |{} |{} |{} |{} |{} |{} |{} |
|{} |{} |{} |{} |{} |{} |{} |{} |{} |{} |{} |{} |{} |{} |{} |{} |{} |{} |{} |{} |
|{} |{} |{} |{} |{} |{} |{} |{} |{} |{} |{} |{} |{} |{} |{} |{} |{} |{} |{} |{} |
|{} |{} |{} |{} |{} |{} |{} |{} |{} |{} |{} |{} |{} |{} |{} |{} |{} |{} |{} |{} |
|{} |{} |{} |{} |{} |{} |{} |{} |{} |{} |{} |{} |{} |{} |{} |{} |{} |{} |{} |{} |
|{} |{} |{} |{} |{} |{} |{} |{} |{} |{} |{} |{} |{} |{} |{} |{} |{} |{} |{} |{} |
|{} |{} |{} |{} |{} |{} |{} |{} |{} |{} |{} |{} |{} |{} |{} |{} |{} |{} |{} |{} |
|{} |{} |{} |{} |{} |{} |{} |{} |{} |{} |{} |{} |{} |{} |{} |{} |{} |{} |{} |{} |



|[4]L |A |* |[5]O |T |A |.
|{} |[6]L |I |T |* |{}|.
\end{Puzzle}
\begin{PuzzleClues}{\bfHorizontal}
\Clue{1}{MOLE}{Quantité de matière}
\end{PuzzleClues}
\begin{PuzzleClues}{\bfVertical}
\Clue{1}{MAL}{Pas bien}
%
...
\end{PuzzleClues}
\end{multicols}
 
 
\question[1] Métèque : 
% \begin{solution}
% Solution
% \end{solution}
\rep{2cm}
 
\question[1] Citoyen : 
% \begin{solution}
% \end{solution}
\rep{2cm}

\question[1] Agora : 
% \begin{solution}
% Solution
% \end{solution}
\rep{2cm}
 
\question[1] Démocratie : 
% \begin{solution}
% Solution
% \end{solution}
\rep{2cm}
 
     
\question[1] Acropole : 
% \begin{solution}
% \end{solution}
\rep{2cm}

\question[1] Hoplite : 
% \begin{solution}
% Solution
% \end{solution}
\rep{2cm}

 
 \end{questions}
 \begin{questions}
     \section*{Choix multiple, vrai ou faux}
     %\subsection*{Choix multiple}
 \question [1] Pour quelle dieu/déesse la fête des Panathénées a-t-elle lieu ?\\
 \begin{oneparchoices}
\choice Apollon
 \choice Athéna
 \choice Zeus
 \choice Héra
 %\CorrectChoice Socrates
 \end{oneparchoices}
 \vspace{0.5cm}

\question [1] Que veut-dire déesse poliade ?
  % \begin{solution}
  % Solution
  % \end{solution}
  \rep{1cm} 
 \vspace{0.5cm}
 
 \question [2] Vrai ou Faux. Le Parthénon se trouve sur l'Agora.\\
  \begin{oneparchoices}
 \choice Vrai
  \choice Faux
   \end{oneparchoices}
  \vspace{0.5cm}

Justifie ta réponse
 % \begin{solution}
 % Solution
 % \end{solution}
 \rep{1cm} 
\vspace{0.5cm}
 \question [2] Vrai ou Faux. L'Attique est le territoire de la cité d'Athènes.\\
   \begin{oneparchoices}
  \choice Vrai
   \choice Faux
    \end{oneparchoices}
 \vspace{0.5cm}
 
 Justifie ta réponse
  % \begin{solution}
  % Solution
  % \end{solution}
  \rep{1cm} 
 
 \end{questions}

\begin{questions} 
\section*{Je sais raconter dans un texte un débat à l'Ecclésia}
\question[6] Paragraphe 1 : Situe les débats : dans quelle cité, à quelle époque ?\\
Paragraphe 2 : Quelle est l'espace principal où se situe l'Ecclésia ?\\
Paragraphe 3 : Quels sont les trois types d'individus qui siègent à l'Ecclésia ?\\
Paragraphe 4 : Quels sont les trois but différents des citoyens ?\\
% \begin{solution}
% Solution
% \end{solution}
\rep{10 cm}
 \end{questions}

\end{document}