\begin{frame} 
\end{frame}

sommaire : 
\tableofcontents

couleur: 
\textcolor{black}{ }

souligné
\uline{ }

saut de ligne
\vfill

boite pr image ou exemple
\begin{frame}
\begin{beamerboxesrounded}[scheme=blocimage]{Doc 1. Autoportrait de Léonard de Vinci, 1512-1515} 
\end{beamerboxesrounded}
\end{frame}

inclure une image
%\includegraphics[scale=0.30]{}

alinéa
\setlength{\parindent}{1cm}

exposant
\up{exposant}

faire un tableau
\begin{tabular}{|l|r|}
\end{tabular}

faire un dessin
\begin{picture}(4,2)
\end{picture}

faire une boite dans un dessin
\put(0,2.4){\framebox(3,2)[t]{\shortstack{Lieux de \\conception \\
\pause[1]\textcolor{blue}{Californie}}}}

faire une flèche dans un dessin
\put(7,3){\vector(1,0){1}}

choisir la grosseur de la flèche
\linethickness{0.8mm}
\put(9,-0.2){\vector(0,-1){0.7}}

saut de page
\newpage
