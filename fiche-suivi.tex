\documentclass[a4paper,oneside,landscape,10pt]{article}
\usepackage{fontspec}
\usepackage{xltxtra}
\setmainfont[Mapping=tex-text]{Century Schoolbook L}
 \usepackage[francais]{babel}
 \usepackage{tabularx}
 \usepackage{geometry}
 \geometry{hmargin=0.5cm, vmargin=0.5cm } 

\makeatletter
\renewcommand\section{\@startsection
{section}{1}{0mm}    
{\baselineskip}
{0.5\baselineskip}
{\normalfont\normalsize\textbf}}
\makeatother

\setlength{\extrarowheight}{35pt}

\begin{document}
\begin{minipage}{5cm}
NOM
\end{minipage}
\hfill
\begin{minipage}{5 cm}
\begin{center}
\textbf{Fiche de suivi en classe \\
Du  décembre 2013 au 13 décembre 2013}
\end{center}
\end{minipage}

\vspace{0.5cm}
Cette fiche a pour but de noter pendant la semaine, le comportement d'Antoine en classe.
\vspace{0.5cm}


\begin{minipage}{20cm}
Pour Antoine : Je m'engage à être plus sérieux en classe, à plus me concentrer lors des leçons, à ne pas me laisser distraire en bavardant.
\end{minipage}
\hfill
\begin{minipage}{3cm}
\begin{flushright}
Signature : 
\end{flushright}
\end{minipage}



\vspace{0.5cm}

\begin{minipage}{20cm}
Pour les professeurs : chaque remarque doit y être inscrite, négative comme positive.
Antoine doit, chaque jeudi, remettre cette fiche à son professeur principal afin de faire le point. 1 ou 2 points pourront être rajoutés en fin de trimestre sur la note de vie scolaire d'Antoine en cas de progression.
\end{minipage}
\hfill
\begin{minipage}{3cm}
\begin{flushright}
Signature : 

\end{flushright}
\end{minipage}
\vspace{0.5cm}

\begin{minipage}{20cm}
Pour les parents : Antoine devra, chaque soir, vous montrer la feuille afin que vous puissiez juger des progrès réalisés.
\end{minipage}
\hfill
\begin{minipage}{3cm}
\begin{flushright}
Signature : 
\end{flushright}
\end{minipage}
\vspace{0.5cm}

\begin{tabular}{|p{1cm}|p{5cm}|p{5cm}|p{5cm}|p{5cm}|p{4.8cm}|} \hline
M1&  &  &  &  &  \\ \hline
M2& &  &  &  &  \\ \hline 

M3& &  &  &  &  \\ \hline 

M4& &  &  &  &  \\ \hline

S1 & &  &  &  &  \\ \hline

S2 & &  &  &  &  \\ \hline
S3& &  &  &  &  \\ \hline 
S4 & &  &  &  &  \\ \hline 
\end{tabular}




\end{document}