\documentclass[12pt]{article}
\usepackage{fontspec}
\usepackage{xltxtra}
\setmainfont[Mapping=tex-text]{Century Schoolbook L}
 \usepackage[francais]{babel}
 
 \usepackage{geometry}
 \geometry{ hmargin=0.5cm, vmargin=0.5cm } 

\makeatletter
\renewcommand\section{\@startsection
{section}{1}{0mm}    
{\baselineskip}
{0.5\baselineskip}
{\normalfont\normalsize\textbf}}
\makeatother



\begin{document}

\section*{L'accompagnement personnalisé}

\textbf{Dans programme : }
\begin{itemize}
\item Partie 2 - Pédagogie
\item Section 1 : aide à l'élève dans son travail personnel. \\
\end{itemize}

\textbf{Mots-clés : } 

\begin{itemize}
\item difficultés d'apprentissage
\item 
\item 
\item  
\end{itemize}

\textbf{Manuel, 174}

\begin{itemize}
\item Objectifs : 
\begin{enumerate}
\item Mettre en œuvre des modalités d'accompagnement adaptées aux profils cognitifs des élèves. \\
\end{enumerate}

\item dep 10 ans, neau voc pr nommé aide aux élèves : aide, accompagnement, soutien, modules, remise à niveau, consolidation, tutorat, étude dirigée... Termes svt qualifiés par << personnalisé >>, <<individualisé >> ou << différencié >>.\\
\item + récemment, notion a 2 termes : 
\begin{enumerate}
 \item accompagnement = être aux côtés de ts les élèves (pas juste ceux qui st en difficultés). Au collège, parle de accompagnement éducatif.\\
 \item personnalisé = démarche individuelle ou en petit groupe. A école primaire, ex: les PPRE (programme personnalisé de réussite éducative).\\
\end{enumerate}

\item extension de accompagnement à ts les niveaux de la scolarité dep années 2000.\\
\begin{enumerate}
\item Ecole maternelle et élémentaire :  élève qui a prob ds apprentissage peut bénéficier d'1 AP de 2 heures / sem, de travail en petit groupe + stage de remise à niveau à le fin du cycle 2.\\
\item collège, plusieurs dispositifs pr la personnalisation des enseignements et des parcours : PPRE, AP en 6e, PPRE passerelle et AE (Accompagnement éducatif). \\

\item qu'est-ce que l'AP ? \\

\begin{enumerate}
\item dispositif mis en place à rentrée 2007 ds les collèges de éduc prioritaire puis généralisé à rentrée 2008 \\

\item 2h/jr. En collège, annualisé, de préférence en fin de journiée, 4 jrs / sem. \\

\item 

\end{enumerate}


\end{enumerate}


\end{itemize}

\textbf{Bibliographie : }
\begin{itemize}
\item LIEURY, FENOUILLET,\textit{ Motivation et réussite scolaire}, 2006. 370.15 LIEU \\
\item RAYOU, \textit{Faire ses devoirs, enjeux cognitifs et sociaux d'une pratique ordinaire}, 2010 \\
\item THÉLOT, \textit{Pour la réussite de tous les élèves (rapport de la commission nationale du débat sur l'avenir de l'école)}, 2004
\end{itemize}


\end{document}