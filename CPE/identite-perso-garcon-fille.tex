\documentclass[12pt]{article}
\usepackage{fontspec}
\usepackage{xltxtra}
\setmainfont[Mapping=tex-text]{Century Schoolbook L}
 \usepackage[francais]{babel}
 
 \usepackage{geometry}
 \geometry{ hmargin=0.5cm, vmargin=0.5cm } 

\makeatletter
\renewcommand\section{\@startsection
{section}{1}{0mm}    
{\baselineskip}
{0.5\baselineskip}
{\normalfont\normalsize\textbf}}
\makeatother



\begin{document}



\textbf{La construction de l'identité personnelle du jeune vers l'âge adulte, garçon et filles, 156} \\

\textbf{Mots clés : }
\begin{itemize}
\item les différents collectifs (les << nous >>)
\item 
\item 
\item 
\item 
\item 
\end{itemize}

\vspace{0.5cm}

\textbf{Objectif :}


\begin{enumerate}
\item comprendre spécificités de enfance et de ado
\item connaître déterminants sociaux du passage d'1 âge à l'autre
\item identifier invariants de construction de autonomie de la personne
\item comprendre rôle de école ds dvpt individuel \\
\end{enumerate}

\textbf{1. La difficile sortie de l'enfance} \\

\begin{itemize}
\item pdt enfance, individu se perçoit comme membre de sa famille. Famille = 1er collectif au sein duquel enfant se pense exister.
\begin{enumerate}
\item Ado modifie ce schéma. Apparition d'autres collectifs. << Nous >> familial destabilisé / << nous >> générationel. Conflit des générations : << nous avec copains, on fait pas pareil que nous avec parents >>. \\
\end{enumerate}

\item conquête d'autonomie fonctionnera pas si c'est dans conflits permanent avec parents, entre jeune et monde des adultes, qui doit devenir son monde. \\
\begin{enumerate}
\item ici que école (et surtout collège) a rôle important. Fait vivre << communauté éducative >> autour du jeune. \\
\item c'est à entrée au collège que jeune s'aperçoit que appartenance familiale relative.\\
\item \textbf{collège = lieu où ado est écouté / d'autres que ses parents}. Lieu d'expérimentation, sous regard de ses pairs et d'autres adultes. \\
\item CPE au coeur de articulation de ts ces regards. Doit écouter, corriger appréciations erronées, aider jeune à devenir autonome, tt en respectant autres. Doit comprendre le malaise qd existe et recourir à famille s'il le faut.\\
\end{enumerate}

\fbox{
\begin{minipage}{18cm}
\textbf{La sortie de l'enfance passe par la conquête de l'autonomie.}\\
DE VOS Bernard, Actes de la table ronde bruxelloise, 5 novembre 2008\\

"Il est important d'identifier le malaise que les jeunes peuvent ressentir. [...] Il faut favoriser la participation des jeunes, reconnaître leurs compétences, poser un regard positif sur eux et les prendre au sérieux plutôt que les prendre au mot. Ce qu'il manque aux jeunes in fine, c'est l'attachement, ie le sentiment d'être important dans le regard de quelqu'un. >>
\end{minipage}
}

\vspace{0.5cm}

\textbf{2. Individualisation et socialisation}\\

\item recul des repères qui fondaient identité collective de nos société, lié à montée de individualisation.\\
\begin{enumerate}
\item au nom du principe qui veut qu'à tout âge on soit une personne pourvue de droit et devoirs individuels, société tend à atténuer frontières entre divers âges de la vie. << flou des âges >>. \\
\end{enumerate}

\fbox{
\begin{minipage}{19cm}
\textbf{Le flou des âges et l'individualisation} \\
SINGLY, François de, \textit{Les Adonaissants}, 2006, p. 12-13. \\

\item adultes auraient perdus leurs repères, cherchant à s'aligner sur les ados. Peur de inversion (ado adultes, adultes ado) = traduit résistance à individualisation.
\item individualisation = droit pour tout individu de ne pas être défini seulement / une place (dans ordre des générations, des sexes, des institutions). 1 garçon ou 1 fille n'est pas seulement << fille de >> avec ses parents, tt comme une femme n'est pas juste << épouse de >> avec son mari.\\
\begin{enumerate}
\item chacun peut être considéré comme individu à part entière, en tant que personne. \\

\item Nlle manière de définir les individus bouleverse les barrières traditionnelles entre les âges, genres, orientations sexuelles. \\

\item Engendre un flou : enfants ont droits sans attendre l'âge adulte. Mais signifie pas que enfant devient un adulte.
\end{enumerate}
\end{minipage}
}

\vspace{0.5cm}

\item PROBLEME : cette recherche de collectif auxquels s'assimiler est \textbf{processus nécessaire au dvpt de ado vers autonomie}, vers construction de individualité = résultat d'une succession d'adhésion de soir à différents collectifs.\\

\item Conception de processus d'individualisation à partir de collectifs trouve origines ds philosophie des Lumières. Ts individus doivent être autonomes, penser / eux-mêmes.

\vspace{0.5cm}

\fbox{
\begin{minipage}{18cm}
\textbf{Les lumières à l'origine de l'individualisation} \\

KANT Emmanuel, Qu'est-ce que les Lumières ?, 1784. \\

<< Sapere aude ! [Ose penser ! ] Aie le courage de te servir de ton propre entendement. Voici la devise des Lumières ! >> Homme doit sortir de sa minorité / sa capacité à pvr penser sans la direction d'autrui.\\
\end{minipage}
}

\vspace{0.5cm}

\item Pr le faire, jeune doit disposer des conditions nécessaires pr échapper aux tutelles. 1 des missions de école. Philosophie de éducation pense que école sert à émanciper le jeune : 
\begin{enumerate}
\item Mission d'émancipation : faire de individu quelqu'un de pensant par lui-même grâce à l'instruction. Mission qui entre en conflit ac mission de socialisation. \\
\item permettre à individu de devenir 1 être social (Emile Durkheim) comme membre d'1 ou plusieurs collectifs (famille, milieu de vie, catégorie sociale, culture...)
\end{enumerate}

\item époque actuelle : mise en tension des deux missions car affaiblissement des repères d'identifications collectives portée / école + montée de individualisation ds mondialisation.\\

\item pr bcp de jeunes, école peine à être lieu où acquiert sa liberté et autonomie au sein de la Cité. perçue comme celle qui condamne à devenir individu qui réussit ou qui échoue. \\

\textbf{3. Le rôle de l'école.} \\

\item ts membres de communauté éduc doivent se mobiliser pr inverser tendance et venir en aide aux enfants les + exposés.\\

\fbox{
\begin{minipage}{19cm}
\textbf{Crise de l'adolescence et rôle de l'école.} \\

JEAMMET Philippe (dir), \textit{Adolescences}, 2002, p. 176-177 \\
 
<< société a changé, et avec elle relations qu'entretenaient enseignants et élèves, ens et parents, parents et enfants. Ces relations st affectée / modification de nos valeurs, en particulier rapport à la << liberté >>, à << autorité >> dans un contexte éco qui fait de la scolarité secondaire la mise en orbite soit de l' <<échec>>, soit de la <<réussite>>. [...] \\
Faire fonctionner le << groupe classe >>, pour que les jeunes se soutiennent entre eux face aux angoisses à propos de l'avenir, et être vigilant à ces groupes << SOS >> de fond de classe, dt agitation témoigne du mal-être.>> \\
Dans étab où équipe enseignante soudée, infos s'échangent sur élèves. Cela permet de mieux appeler l'attention des parents ou du médecin scolaire sur évntuels troubles de personnalité.\\
\end{minipage}
}

\item faut faire comprendre aux élèves qu'ils ne peuvent pas se construire sans la concurrence aux autres.\textbf{ Ecole est lieu où on apprend à connaître les inégalités de soi aux autre pr se reconnaître soi-même}, différent et unique, ms pas seul au monde. \\

\item Comment école peut atteindre ce double objectif ?
\begin{enumerate}
\item en s'opposant, / dispositifs éducatifs, à confusion des âges. \\
\begin{enumerate}
\item A âge ado, important de marquer fin de enfance. Ms dep 20 ans, disparitions des rituels d'intégration sociaux. flou règne entre 12 et 25 ans.\\
\item Moment sans limite précise de sortie de enfance : 10-12 ans. Mutation de société engendre difficultés pr certains jeunes : 
\begin{enumerate}
\item passage de celui qui gère la cour d'école à celui qui subit la cours du collège. Possède pas ttes les clés : peut générer certaine angoisse.\\
\item passage de reconnaissance comme << grand de primaire >> au vécu du << petit du collège >>. Jugent svt mal accueillis / élèves plus grands. Crise d'identité générée / ce changt de perspective peut être d'autant plus grave qu'elle se situe au début de ado.\\
\end{enumerate}

\item pose question des rituels d'intégration sociale
\begin{enumerate}
\item pr marquer sortie de enfance et entrée ds ère de responsabilisation. 13 ans = juridiquement en France, âge de responsabilité pénale.
\item pr marquer entrée ds âge adulte. étab scolaire, mairies doivent organiser des cérémonies pr marquer ce moment décisif de rupture.
\end{enumerate}
\end{enumerate}
\end{enumerate}

\item svt familles ne se rendent pas compte qu'une travail bien encadre sur limites, dangers, risques, peut permettre d'éviter que jeune n'aillent rechercher sensations extrêmes.\\
\item \textbf{transgresser : pr ado, moyen de prospecter les limites, tester, mesurer les interdits.} Important que adulte ne se laisse pas prendre au jeu de transgression qu'expérimente ado. S'agit pas d'être laxiste ms travailler sur limites et régulation possible.\\

\item qd interdit, au nom du principe de << précaution >> ds cours de récréation jeux de balles, rallyes d'orientation nocture... qui amène à travailler avec le jeune les peurs et dangers, on ne s'étonne pas du résultat.
\begin{enumerate}
\item pr qq cas médiatisés, regrettable, on empêche l'ado de se préparer à gérer son passage à la maturité.\\
\end{enumerate}

\textbf{4. l'importance des relations école-familles pour construire l'identité de l'élève} \\
\textit{4.1 L'ado entre identité de l'élève et identité de personne.} \\

\fbox{
\begin{minipage}{19cm}
\textbf{Favoriser de bonnes relations parents-enseignants} \\

\textit{Circulaire du 25 août 2006 sur le rôle et la place des parents à l'école.}

\begin{enumerate}
\item Tt mettre en oeuvre pr que parents puissent pde connaissance des résultats scolaires de enfant.\\

\item parents doivent être prévenus rapidement de tte difficultés rencontrée / élève (scolaire ou comportementale). Question de assiduité scolaire (élément fondamental de la réussite) attention particulière. Utilisation des SMS et autres moyens d'Internet doivent permettre des échanges + rapides avec les parents (absences, réunions...)
\begin{enumerate}
\item \textit{ce que circulaire dit pas : si sanctions prononcées sans dialogue suffisant au préalable, svt effet contraire : radicalisation, marginalisation, exclusion}.
\end{enumerate}

\item Organisation de rencontres collectives : pr ensemble des parents (info de rentrée, parents d'élèves nllement inscrits), pr 1 groupe de parents (par classe ou en sous-groupe).\\
\begin{enumerate}
\item \textit{Nécessité d'adapter taille du groupe de parents au prob à traiter, pr éviter de stigmatiser familles dt enfants rencontre difficultés.}
\end{enumerate}

\item Rencontre individuelle ac enseignants et autres personnels. Cadre adaptés à demande, respect de confidentialité des propos échangés. \\
\begin{enumerate}
\item \textit{certaines familles ont pas habitude des rencontres ou maîtrise pas bien langue française. Pierre Bourdieu : ont << habitus primaire >> éloigné de << habitude secondaire >> imposé / école. Doit prendre en compte cette distance, faire effort d'adaptation nécessaire}  \\
\end{enumerate}

\item dialogue avec parents fondée sur reconnaissance mutuelle des compétences et missions des uns et des autres (professionnalisme des enseignants ds fonction, responsabilités éducatives des parents) + souci commun respect de personnalité de élève.\\
\begin{enumerate}
\item \textit{ATTENTION : ne veut pas dire confondre les rôles. ce n'est pas pr prof ou CPE abdiquer ses prérogatives. Faut que chacun connaissent et reconnaisse clairement fonction et prérogative de autre.} \\
\end{enumerate}

\end{enumerate}
\end{minipage}
}

\vspace{0.5cm}

\fbox{
\begin{minipage}{19cm}
\textbf{Trois concepts fondamentaux caractérisent la théorie de la reproduction. \\ }

BOURDIEU Pierre, PASSERON Jean-Claude, \textit{La Reproduction, éléments pr une théorie du système d'enseignement}, 1970.\\

\textit{distance habitus primaire et secondaire responsable de la \textbf{violence symbolique} exercée / école à son insu envers enfants des milieux moins favorisés. Rend compte à elle seule du phénomène de \textbf{reproduction des inégalités sociales} au coeur du fonctionnement de école.} \\

\begin{enumerate}
\item \textbf{capital culturel} : tte ressources culturelles d'un individu (bien, diplômes, rapport au savoir et école). dépend du milieu social. corrélé au capital éco (revenus, patrimoines) et social (relations sociales). \\
 \item \textbf{habitus} : système de représentations de individus, oriente son comportement, ambition, projets. façon de s'habiller, d'évoluer, de penser le monde.  Construite pdt phase de socialisation ds famille puis école.\\
 \item \textbf{Violence symbolique} : façon dt s'exerce fonction de reproduction de école. Fonction de reproduction des inégalités de école passe inaperçu à cause d'1 pensée : si je ne réussit pas, c'est que je ne suis pas doué. Selon Bourdieu, échec scolaire dû à trop grande distance chez enfant de classe dominée, entre habitus primaire (famille) et secondaire (école). Distance fait obstacle à intériorisation de habitus secondaire.\\
\end{enumerate}
\end{minipage}
}

\vspace{0.5cm}

\item Sur plan dvpt psycho de ado, pcpal enjeu de bonnes relations entre collège-parent : pas s'installer \textbf{dichotomie entre personne de enfant et personne de élève.} Famille : fil de  qq'1, à école : élève de classe.\\

\item relation parent-école ont fonction de \textbf{prévenir césure psycho} ou réagréger personnalité du jeune en lui montrant que parents et enseignants st acteurs conscients et partenaires de son dvpt.\\

\item Apprendre à ado qu'on est 1 tt en évoluant dans monde différent (école, famille, club de sport...) sans rompre identité perso. \\

\fbox{
\begin{minipage}{19cm}
\textbf{Ambivalence du concept de communauté éducative.} \\

DE SINGLY François, <<Communauté éducative ou société scolaire démocratique ? >>, MADIOT Pierre (dir), \textit{Enseignants, parents, réussite des élèves, quel partenariat ?}, 2010 \\

\begin{enumerate}
\item Pas 1 seule communauté éduc :1 scolaire, 1 familiale. parents 2nd ds la scolaire et 1er ds la familiale.\\
\item << Tant que nous raisonnerons avec une catégorie inexacte << communauté éducative >> pour ne désigner que la << communauté éducative scolaire >>, tensions resteront. >>
\end{enumerate}
\end{minipage}
}

\vspace{0.5cm}

\textit{4.2. Les parents connaissent mal l'école de leurs enfants} \\

\item Indispensable que parents et enseignants partagent même valeurs constitutives du << contrat social >> français : refus de violence, refus des préjugés et des discrimination, dvpt du sens des responsabilités, de la solidarité entre les personnes. \\
\begin{enumerate}
\item qd pas le cas, étab scolaire peut pde initiatives citoyennes (organisation de débats, d'info impliquant parents travaillant ds secteurs de vie en société : justice, santé, aide sociale...).
\end{enumerate}

\item pr certaines familles menacées de rupture sociale, compréhension des missions de école difficile. \textbf{Faut donner à voir aux élèves et parents la légitimité de école}. But : aider personnel d'éducation à déminer tt ce qui peut faire obstacle aux bonnes relations entre parents et école.\\
\begin{enumerate}
\item niveau CPE : qualité du dialogue avec parents pt central de sa mission. Si veut aider ado à construire leur identité personnelle pr construire 1 projet professionnel.\\
\end{enumerate}

\item construction d'1 école de réussite pr ts implique que : au lieu de se regarder en chiens de faïence, parents et personnel s'épaule ds respect de complémentarité de leurs rôles respectifs.\\ 
\begin{enumerate}
\item chacun veut réussite des enfants. ts font de leur mieux et quand ils se rencontrent : svt dialogue de sourds. Svt, CPE parle et parents écoutent, inquiets ou affectants de l'être, face à ce qu'ils entendent.\\
 \end{enumerate}

\item meilleur présence des parents d'élèves ds école : permet soutien réciproque. Peut améliorer : 
\begin{enumerate}
\item condition d'accomplissement missions de étab scolaire \\
\item mise en situation d'apprentissage des jeunes \\
\item choix d'orientation désirés des jeunes pdt cursus. \\
\end{enumerate}

\item démocratiser école, c'est partager savoirs concernant son organisation, son fonctionnement, ses programmes.\\

\item 

\end{itemize}
\end{document}