\documentclass[12pt]{article}
\usepackage{fontspec}
\usepackage{xltxtra}
\setmainfont[Mapping=tex-text]{Century Schoolbook L}
 \usepackage[francais]{babel}
 
 \usepackage{geometry}
 \geometry{ hmargin=0.5cm, vmargin=0.5cm } 

\makeatletter
\renewcommand\section{\@startsection
{section}{1}{0mm}    
{\baselineskip}
{0.5\baselineskip}
{\normalfont\normalsize\textbf}}
\makeatother



\begin{document}
%\renewcommand{\figurename}{Doc.}
NOUVEAU CONCOURS
\begin{itemize}
\item 2 épreuves d'admissibilité
\begin{itemize}
\item \underline{Epreuve de maîtrise des savoirs académiques}. 4h. Dissert (connaissances de sciences humaines, histoire, socio de éduc, psycho de enfant et ado, philo de éduc, socio.). Connaissance des grands enjeux de éduc et des évolution du système éducatif + csq sur fonctionnement de étab scolaire et sur rapports des élèves aux apprentissages.
\item Etude de dossier sur pol éducatives. Analyse de doc (origine et statuts variés). Note de synthèse répondant à 1 questionnement précis. 5h. \\
\end{itemize}

\item Orale d'adm.
\begin{itemize}
\item Mise en situation professionnelle. Dossier de 10 pages max élaboré / candidat (par ex. à partir de ses travaux de recherche. Dossier porte sur une situation professionnelle pouvant être rencontrée / un CPE. 1h : exposé de 10 min + entretien de 50 + tps de préparation de 30 min.) \\

\item Entretien sur dossier. dossier de 5 pages max, 1 ou plusieurs docs. Problématique éducative que candidat devra approfondir / recherche perso (aura ordi connecté à internet. Prep de 1h30. exposé de 20 min + entretien de 40.). \\
\end{itemize}
\end{itemize}

CPE / Régis REMY.

Introduction. Du surgé au CPE
\begin{itemize}
\item CPE vecteurs des normes du milieu social.
\end{itemize}

Partie 1 - Historique.

Chapitre 1 - L'arrivée du surgé.
\begin{itemize}
\item crée au 19e : agents de transmission et d'apprentissage des valeurs que institution veut inculquer aux enfants.\\
\item mission évoluent ac transformations mentalités de société.

Le lycée impérial, berceau du SG.
\item nait en 1847. selon décret: hiérarchie (ap prof ms avt Me élémentaires.). Au moment d'arrivée, lycée en mutation (passe du militaire au laïc et publics). Discipline militaire (clairon). Punition peu nbeuses ms sévères (armée). Sanctions humiliantes.


Le statut initial.
\item décret 17/11/1847 crée fct° SG ds lycée. circulaire ministérielle du 20/12/1847 définiti mission.
\item nommé / ministre. Mission : auxiliaires des censeurs d'étude. doivent diriger les répétiteurs. attribuent note de conduite. st membres conseil de disciplines. Licence. Ap 5 ans d'expérience, peuvent postuler à poste de censeurs des études.

Mission triste du SG.
\item doit loger au lycée. chargé du bon ordre de étab, application règlement et sanctions, discipline, absences, retards, tenue et propreté, politesse.... Tient cahier de punitions.
\item va mettre en oeuvre des récompenses.
\item notion de vie scolaire (actuelle) inexistante. jeunes st disciples. éducation / contrainte. objet de crainte, mépris, haine. \\

Chapitre 2 - L'arrivée d'un surveillant général éducateur. \\

\item fonction évolue ac société. statut figé J-> fin 2GM. Ap 2GM : nb d'étab augmente. dvpt enseignement technique public.

1- Les étab publics d'enseignement ap guerre.
\item 3 types d'étab : 
\begin{itemize}
\item cours complémentaires (futurs collèges). but : brevet (futur des collèges) pas de SG
\item lycée classique et modernes. 10 \% classe d'âge de 6e à term. examen d'entrée en CM2. Trouve surveillants généraux dits de lycée.
\item étab techniques.
\begin{itemize}
\item lycées techniques. peu nbeux. brevets techniques. 6e-term. surveillants généraux de lycée.
\item centres d'apprentissages (futurs collèges d'enseignement technique). plein dvpt. ap école primaire. préparer CAP (Certificat d'aptitude professionelle). trouve surveillants généraux dits de CET (svt ac internat). évolution de leurs pratiques : voit apparaître notion d'animation (plus tard va se fondre ds celle plus large de vie scol).\\
\end{itemize}
\item Etat fait bcp d'efforts pr eux. but : promouvoir élite ouvrière.
\end{itemize}

L'encadrement éducatif (hors enseignement)
\item étab pas mixte. lycée garçons dirigé / proviseur homme + censeur (adjoint). ds technique pas d'adjoint : SG assume fonction.
\item ds étab : aussi SG en 2 catégories : 
\begin{itemize}
\item Catégorie des SG de lycée. Même carrière que adjoint d'enseignement. choisi sur liste d'aptitude (tableau d'avancement). faut avoir 25 ans, une licence, 3 ans d'enseignement. Nommé / ministre, stagiaire et titul au bout d'1 an. pas de formations.
\item Surveillants généraux de CET. liste d'aptitude. bachelier, 5 ans de service, 28 ans. Stagiaire et titul. Pas formation.
\end{itemize}

\item Répétiteurs. extinction. Fin : deb 60's. Encadre études et permanences.
\item Me d'internat et surveillants d'externat (MI/SE). Nouvellement créer. Remplace répétiteurs. Jeunes étudiants. Vecteurs changt du climat des étab. Licence. 40h / sem. Svt Motivés et compétents. Pas formation ms svt encadrés camps de vacances (boom vacances).
\item équipes de surveillance de qualité et à haut potentiel. Donne formation. MI arrivent ac capacité et réserves d'animation : activités de veillées, jeux d'intérieurs et d'extérieur... + désir de mettre en oeuvre.


2 - Rencontre d'une pop nlle et de éducation populaire.

Internats font plein.
\item car peu moyens de transports. Internat aussi week-end.
\item liste d'attente pr internat ds certains étab.

La vie du lycéen interne.
\item long moment à étudier seul.

L'interne de CET.
\item impossible de faire vivre technique sur même mode que général..
\item qualité de vie et climat à internat influencent ceux de étab tt entier. SG vt accorder bcp d'importance à organisation de vie collective à internat.

Le surveillant général animateur.
\item lors week-end, prob de encadrement des internes au CET s'amplifie : tps de loisir et récréation + nbeux. A côté, promenade traditionnelles et activités sportives se dvpt animations socio-éducatives.
\item MI prennent place prépondérante ds création et animation de ces activités nlles. Faut restreindre risque de tensions. Apporte détente. Meilleur qualité relationnelle.
\item Apparition de clubs (ciné, photo, journal, ping-pong...) Sommet en 1955-1965.
\item recours à pratique professionnelles innovantes qui modifie méthodes trad du maintien de l'ordre et discipline. Introduit début de dialogue. Jugé dangereuse / collège de général. 

Les associations d'éducation populaire.
\item prospère fin 50's-deb 60's. ex: UFOLEIS : union des fédérations des oeuvres laïques pour l'enseignement de l'image et du son.
\item SG font largement appel à ces assoc pr org et fonctionnement des activités socio-éduc de leurs étab.

l'AROEVET et la FOEVET.
\item Assoc loi 1901 (FOEVET : Fédération des oeuvres éducatives et de vacances de l'enseignement technique). but : fédérer activités socio-éduc. Siège : ministère.
\item relais académique : AROEVET (Assoc régionale des oeuvres éduc et de vacances de enseignement technique). 
\item Institution reconnaît travail ds CET. Reconnaissances travail SG.

le FSE et la cogestion comme instruments d'éducation.
\item AROEVET et FOEVET invintent 1 modèle d'assoc loi 1901 : foyer socio-éduc. Président : chef d'étab. géré / CA et 1 bureau où entrent parents et élèves (ac même droit que adultes). Révolution : faire participer jeune à sa formation. Apprentissage de responsabilité et autonomie. Notions nlles.
\item retombée évènements Mai 68 : 1ers textes officiels concernant FSE.
\item gde découverte des FSE : \textbf{cogestion}. Activités en collab jeune-adultes. projets discutés en commun.
\item SG a origine de plupart des FSE. pratiques professionnelles modifiées : amener à confier naturellement plus responsabilités aux jeunes. objectif acquisition autonomie. Autodiscipline : but et non moyen. Apprentissage nécessaire car liée notion responsabilité.
\item ds étab où FSE : s'instaure autre climat. favorise relation de confiance.

Le FSE, un enjeu dans l'enseignement technique.
\item Là où existe, SG l'utilisent comme 1 outil d'éducation. pratique professionnelles nlles et innovantes -> fractures. Amène SG à réfléchir à identité.

L'apport des centres de vacances coll.
\item 50-fin 60's : centres de vacances coll pr jeunes : large succès. vocation sportive ou découverte. Vie en commun.

L'apport des mouvements << d'éducation nouvelle >>
\item moins d'influence que dans le primaire. ms ds pédagogique : méthodes et techniques innovantes. Elève considéré comme acteur participant à son appréhension du savoir.

Chapitre 3 - Les espoirs d'une fonction nouvelle.

Le décalage lycée / CET s'accroît.
\item 60's : mesure combien pratiques et comportements ont évolués. Ms ds lycée, rien change. Peu ont mis en place FSE.
\item conception nlle de "surveillance gal" ds CET. Transforme objectifs, méthodes et pratiques professionnelles.

Rôle moteur de Inspection gale.
\item Inspection générale de adm scol voit son secteur d'intervention étendu à "vie scol" en 65. avait en charge recrutement et formation SG.
\item IG spécialise qq insecteurs gaux ds secteurs de vie scol. 4 importts : GALLI (crée 1ers stages de formations), BRUNSCWIG, DEJEAN, HATINGUAIS.
\item but : empêcher que CET joue apprentis sorcier en donnant trop responsabilité à ado.

Les 1ères formations.
\item 59-60 : GALLI met stages de formation pr nllement recrutés. 8-10 jours au CET de Reims et Mans (FSE bonne qlité). Stage importt ds nlles pratiques.
\item effets de formations rapidement perseptibles : 
\begin{itemize}
\item Crée FSE ds lycées.
\item 65 : ministère sup cloisonnement technique, moderne et classique. FOEVET et AROEVET s'ouvrent aux lycées et collèges : deviennent FOEVEN et AROEVEN (éducation nationale).
\item circulaire qui définit mission des personnels d'éduc. SG ont mission sup (org de vie scol) + possibilité d'action pédagogique et éducative.
\end{itemize}

\item 3 phénomènes vt peser sur pratique SG et amener transformations cadre de vie scol : 
\begin{enumerate}
\item explosion des collèges. car explosion démog. pop différente des CET (pré-ado). pratiques doivent s'adapter.
\item qlité des équipes éducatives : compétences certaine, travail et réflexion en équipe, responsabilisation, volonté de prise en charge des besoins et intérêt des ados. But : faire de éducation. Ds certains étab, internat cogéré / conseil d'internat composé d'élèves élu et Me d'internat.
\item apparition de mixité. SG s'adaptent bien.
\end{enumerate}

La tentative manquée des adjoints d'éduc.
\item Certains MI/SE se complaisent ds anim. Négligent leurs études. Institution envisage pr eux statut d'1 corps "d'adjoints d'éducation".
\item pas de suite.

Les FSE piétinent.
\item vie scol de meilleur qlité. climat général mieux. Ms FSE pas partout.
\item moyens de communication se dvpe. Internes diminuent.

p. 49

\end{itemize}



\end{document}