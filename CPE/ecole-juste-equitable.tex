\documentclass[12pt]{article}
\usepackage{fontspec}
\usepackage{xltxtra}
\setmainfont[Mapping=tex-text]{Century Schoolbook L}
 \usepackage[francais]{babel}
 
 \usepackage{geometry}
 \geometry{ hmargin=0.5cm, vmargin=0.5cm } 

\makeatletter
\renewcommand\section{\@startsection
{section}{1}{0mm}    
{\baselineskip}
{0.5\baselineskip}
{\normalfont\normalsize\textbf}}
\makeatother



\begin{document}



\textbf{Une école juste, inclusive et équitable pour tous, 8} \\

\textbf{Mots clés : }
\begin{itemize}
\item égalité et démocratie
\item notion de mérite / méritocratie
\item << égalité des chances >>
\end{itemize}

\vspace{0.5cm}

\textbf{Objectif :}


\begin{enumerate}
\item connaître concepts philosophiques en démocratie : égalité, équité
\item compde conditions historiques et sociologique d'émergence des modèles juridiques de égalité
\item identifier enjeux des questions actuelles et nécessité d'1 approche pragmatique de action éducative.
\item comprendre rôle de école ds dvpt individuel \\
\end{enumerate}

\begin{itemize}
\item Historiquement, sorte d'équivalence entre justice et égalité. Longue liste d'égalité des droits (DDHC, droit à éducation, logement, chances...)\\

\item dep 80's, égalité des chances revient régulièrement ds bouche hommes politiques. \\

\textbf{1. Démocratie - Egalité d'éducation}
 
\textit{1.1 L'équivalence entre égalité et démocratie trouve son origine ds pensée philo du 18e.} \\

\item philosophes du Contrat (Samuel von Pufendorf, Christian Wolff puis Rousseau), \textbf{égalité = droit naturel}. Exister à état de nature avt que hommes décident de coopérer.\\

\fbox{
\begin{minipage}{19cm}
\textbf{L'égalité ds état de nature et sa fin} \\

ROUSSEAU, Jean-Jacques, \textit{Discours sur l'origine et les fondements de l'inégalité parmi les hommes}, seconde partie, 1755. \\

\begin{enumerate}
\item \textit{ds ce tps précédent le tps historique, hommes dépendaient pas 1 des autres. Assurer subsistance et satisfaction besoins. Ensuite à succéder tps de guerre : forces antagonistes, conflits entre individus et groupes. Perte de égalité naturelle.}\\

\item << Tant que hommes se contentent de cabanes rustiques, se bornent à coudre habits de peaux, perfectionner leurs arcs... Tant qu'ils s'appliquèrent qu'à des ouvrages qu'un seul pouvait faire [...] ils vécurent libres, sains, bons, heureux. [...] Mais dès instant qu'un homme eut besoin du secours d'un autre; dès qu'on s'aperçut qu'il était utile à un seul d'avoir des provisions pour deux, l'égalité disparut, la propriété s'introduisit, le travail devint nécessaire, et les vastes forêts se changèrent en des campagnes riantes qu'il fallut arroser de la sueur des hommes. >> \\
\end{enumerate}
\end{minipage}
}

\vspace{0.5cm}

\item Rousseau présente 1 contrat qui veut pas revenir à état de nature ms vu comme correction des inégalités issues des rapports de force / 1 pacte social.\\ 
\begin{enumerate}
\item ce neau droit = base de intérêt général (pas conçu comme utilité collective d'1 ensemble de personnes, ms comme expression de volonté de souveraineté des citoyens reconnus égaux en droits) \\
\end{enumerate}

\item Rousseau indique pas si citoyens en droit = égalité d'éducation. Diderot est le 1er qui applique égalité des citoyens en droit à éducation. \\

\fbox{
\begin{minipage}{18cm}
\textbf{L'égalité d'éducation contribue à l'intérêt général} \\

DIDEROT, DENIS, \textit{Essai sur les études en Russie}, 1775-1776 \\

\begin{enumerate}
\item \textit{compose à demande de Catherine 2, plan d'éducation publique.} \\

\item << Le nb de chaumières et des autres édifices particuliers étant à celui des palais dans le rapport de 10.000 à 1, il y a 10.000 à parier contre 1 que le génie, les talents et la vertu sortiront plutôt d'une chaumière que d'un palais. [...] \\

\item  <<université est une école dt la porte est ouverte indistinctement à ts les enfants d'une nation. [...] Ces basses écoles sont pour le peuple en général, parce que, depuis le Premier ministre jusqu'au dernier paysan, il est bon que chacun sache lire, écrire et compter. >> \\
\end{enumerate}
\end{minipage}
}

\vspace{0.5cm}

\item Diderot estime que Etat a intérêt à faire sortir de ombre ts les talents cachés pr que société tire profit de ttes les intelligences.
\begin{enumerate}
\item 1ère raison : utilité collective : estime que autt de chance de rencontrer talents chez laboureurs que ministres. \\
\item 2nde raison: argument moral. << cruel de condamner à ignorance conditions subalternes de société >>. Idée de justice. Inspire projet de Encyclopédie.\\
\item 3e raison : idée de diffusion des savoirs : moyen de fonder 1 société + vertueuse faite d'hommes libres. Idée d'émancipation des chaînes de l'oppression sociale / l'instruction (plus difficile de malmener qq'1 qui sait lire et écrire). \\

\end{enumerate}

\textbf{1.2. Une égalité de base pour produire des inégalités sociales ?}

\item cpdt, ds sa double finalité d'efficacité et de justice, éducation vise 1 but opposé à égalité : créer de nlles inégalités liées aux différences de capacités et de <<réussite des élèves >>.


\fbox{
\begin{minipage}{19cm}
BOLTANSKI Luc, THÉVENOT Laurent, \textit{De la justification : les économies de la grandeur}, 1991. \\

\begin{enumerate}
\item caractérisent cette articulation (égalité dvt éducation ms inégalités des conditions sociales crée / éducation). Identifient 2 pcpes établis / RF : 
\begin{enumerate}
\item Le principe de commune humanité.  ts appartiennent à même humanité :  fonde égalité des citoyens en droits (en dépit des différences de milieux de naissance...) Individus st divers. Ms si érige leur différences en inégalité -> plus démocratie.\\
\item  pcpe d'ordre. Etablit une hiérarchie entre individus, ms fondée sur la justice. rendu nécessaire / division du travail et efficacité économique pr le bien commun. 2 conditions pr qu'il rentre pas en conflit ac pcpe de commune humanité : 
\begin{enumerate}
\item existe 1 pcpe supérieur commun susceptible de mesure grandeur des personnes\\
\item différences entre individus doivent être provisoires et pvr être << rejouées >> périodiquement. si différence définitives, certains individus seraient indéniablement supérieur : reviendrait au monde de violence. Plus de commune humanité.
\end{enumerate}
\end{enumerate}
\end{enumerate}
\end{minipage}
}

\vspace{0.5cm}

\item Sociologue Derouet a appliqué cette analyse à éducation. \\

\fbox{
\begin{minipage}{19cm}

\textbf{Comment l'égalité initiale dvt l'instruction démocratique peut-elle aboutir à créer des distinctions entre les personnes ?} \\

DEROUET Jean-Louis, \textit{Ecole et justice, De l'égalité des chances aux compromis locaux ?}, 1992, p. 82-83. \\

\begin{enumerate}
\item \textit{pr mettre en pratique ces 2 pcpes sans qu'ils s'opposent, éducation a pr 1er but de" \textbf{créer sentiment d'appartenance de chacun à commune humanité}, à égalité de dignité et de droits.} C'est pourquoi Educ Nat offre a chacun 1 bagage commun de lumière qui fait d'1 peuple 1 nation.\\
\item Nécessairement 1 moment où on passe d'1 école pr tous à des formes diversifiées d'éducation selon différence entre élèves : << principe d'ordre >>. Ms <<meilleurs>> pas désignés / naissance ni fortune.  \\
\item << A partir d'1 certain moment, il est nécessaire de passer d'une école pour tous à une école pour les meilleurs, ceux qui méritent la promotion sociale. [...] L'important est qu'il crée une distinction et que cette distinction soit justifiée [...] Une organisation scolaire doit donc satisfaire à deux conditions : d'une part rendre l'école pour tous réellement accessible à tous, en compensant les inégalités d'implantation géographiques, de fortunes, qui peuvent éloigner certains enfants de l'école; d'autre part, faire accord sur le pcpe de sélection qui permet de passer de l'école pr ts à l'école pr les meilleurs. >>
\end{enumerate}

\end{minipage}
}


\vspace{0.5cm}

\item C'est à l'école de choisir (et pas de recevoir) les individus qui peuvent se montrer les plus efficace pr les professions futures. Ms pr que école puisse jouer ce rôle, doit être 1 affaire d'Etat sous contrôle de svraineté populaire.

\textbf{2. Ethique et égalité} \\

\textit{2.1. De l'égalité formelle d'éducation à l'égalité des chances.} \\

\item Pcpe du 18e a replacer dans contexte : veulent passer de Ancien Régime (ordre où position sociale définie / naissance ds castes héréditaires) à société où positions sociales acquises / mérite individuel.\\
\begin{enumerate}
\item semblait nécessaire de soumettre futurs citoyens aux mêmes conditions et contenus d'enseignement. Ms RF a pas eu le temps de mettre en palce cette égalité. \\
\item 2nde moitié 20e : égalité formelle des individus dvt offre scolaire mise en place. Prob d'équité revient au moment de époque de massification des études secondaires.\\
\end{enumerate}

\item Pr Rodrigo Roco Fossa, égalité d'accès à école + égalité de traitement de ts st devenu 1 << droit de base >> seulement destiné à légitimer compétition des individus en son sein.\\

\fbox{
\begin{minipage}{19cm}
ROCO FOSSA Rodrigo, \textit{De l'égalité en éducation} in DROUIN-HANS Anne-Marie, \textit{La Philosophie saisie par l'éducation}, 2005, p. 38-39.

\begin{enumerate}
\item << égalité d'accès à école = droit de base nécessaire au bon fonctionnement de la structure éco et sociale. L'égalité de traitement (accès aux même contenus, procédures, épreuves...) serait surtt le moyen de parvenir à extraire de la société les meilleurs éléments, qui auront le droit légitime d'accéder aux meilleurs positions.>> \\
\item  << idéologie méritocratique = exige que les positions des individus par rapport au pouvoir, aux richessesn aux statuts, au savoir approfondi, etc, se méritent. D'abord à l'école, et grâce à elle, à la vie en général. >>.
\end{enumerate}
\end{minipage}
}

\vspace{0.5cm}

\textit{2.2. L'égalité des chance accusée d'iniquité} \\

\item dep Waux des années 60's (sociologie critique : Pierre Bourdieu, JC Passeron) : procès d'iniquité fait à << idéologie méritocratique >>.
\begin{enumerate}
\item école démocratique <<égale pr ts >> fondée sur récompense du << mérite >>.  Or elle produit inégalités sociales à la sortie. MAIS en plus, elle reproduit inégalités sociales des enfants à entrée et les justifie en rejetant sur les usagers de l'école la responsabilité de leurs échecs.\\
\end{enumerate}

\item 80's : nel objectif assigné à l'école : \textbf{l'égalité des chances.} Neutralisation de ttes les inégalités liée à la naissance. Ms égalité des chances commande inégalité des résultats. Mérite = mesure de ce qui serait dû à chacun.\\

\underline{2.2.1 Que signifie le mérite ?} \\

\item + svt : élève << méritant >> : celui qui obtient les meilleurs résultats et pas celui qui fait des efforts pr réussir. Notion de mérite recouvre celle de compétence : seule récompensée ds course aux qualifications. pourtt, celui qui fait plus d'éffort n'est-il pas plus méritant que celui qui réussit sans en fournir ?\\

\item celui qui a la force de ne pas céder à la fatique n'est-il pas simplement gâté / la chance ? A-t-il plus de mérite que celui qui est obligé de se faire violence, qui est rempli de paresse et qui n'y parvient pas tjs ? \\

\underline{2.2.2. Que signifie la récompense du mérite ?} \\

\item Suivre une formation artisanal, est-ce échouer ? Faire une licence d'histoire, est-ce la récompense d'un mérite supérieur ? 

\underline{2.2.3. L'égalité des chances n'est pas vraiment atteinte}\\

\item Le système perpétue la reproduction des privilèges. l'égalité des chances devrait renversé complètement l'ordre social et établir l'égalité de toutes les conditions sociales. C'est pourtant pas sa finalité. Au contraire.

\fbox{
\begin{minipage}{19cm}
\textbf{L'égalité des chances : une utopie don personne ne souhaite l'accomplissement ?} \\

DROUIN-HANS Anne-Marie, << L'égalité des chances, une idée bouleversante ? >>, \textit{La Philosophie saisie par l'éducation}, p. 79 \\

\begin{enumerate}
\item << Vouloir une égalité des chances dans une société qui est fondée sur l'inégalité [...] ? On fait tout pour rendre effective l'égalité en souhaitant très fort qu'elle ne se réalise pas sous peine de devoir révolutionner les organisations sociales et idéologiques. >>\\
\item a un but et se désole qu'il soit si difficile à atteindre. \\
\end{enumerate}
\end{minipage}
}

\textbf{3. Vers une conception pragmatique de l'égalité}\\

\item débat philosophique aboutit à discréditer l'égalité des chances, les analyses sociologiques y voient 1 tromperie pr les catégories défavorisées et 1 moyen de conforter la position dominante des classes les plus favorisées. \\

\item 1 approche + pragmatique : considérer égalité des chances pas comme absolu à atteindre ms comme appellation générique de ttes les politiques qui s'efforcent de réduire les inégalités dvt l'école.\\

\item La réalité est que : << massification >> des études n'a cessé de progresser depuis 1/2 siècle. atteint aujourd'hui enseignement sup.


\vspace{0.5cm}

\fbox{
\begin{minipage}{19cm}
\textbf{La massification des études durant période 1950-2000} \\

\begin{tabular}{|c|c|c|c|p{2cm}|}
\hline  & 1950 & 1965 & Rentrée 2000 &  \\ 
\hline secondaire & 770.000 & 2,4 M & 5,5 M (dt 3,2 M en collège) & prolongement de scolarité. stable dep 97 \\ 
\hline Apprentissage (dt CFA) & ? & ? & 400.000 & en hausse dep rentrée 2000 \\ 
\hline Total (maternelle-univ) & 3,3 M (24\%) & 7,7M (45\%) & 14,3 M (93\%) & en \% des 2/22 ans \\ 
\hline 
\end{tabular} 
\end{minipage}
}

\item Institution scolaire pd en charge 90\% de pop française. Corrélativement, son pvr de qualification et d'insertion professionnelle n'a jamais été aussi grand.\\

\fbox{
\begin{minipage}{19cm}
\textbf{L'égalité des chances est une fiction nécessaire} \\

DUBET François, \textit{L'école des chances}, 2004, p. 51-52 \\

\begin{enumerate}
\item <<admet que égalité des chances et recherche du mérite sont des fictions nécessaire, ie qu'elles sont à la fois désirables et inévitables ds une société démocratique tenue d'articuler l'égalité des indiv à l'inégalité des positions sociales, il faut tt faire pr s'en approcher. [...] Une certaine hypocrisie face à la ségrégation scolaire, aux coûts et aux bénéfices privés des études peut être levée. Il faut probablement dvper 1 politique de discrimination positive ciblée sur indiv autant que sur étab fragiles. Dans la mesure où la mobilisation des élèves et celle des parents st indispensables à la réussite, l'information et la capacité de circuler doivent être dvpée, rompant ainsi avec 1 fausse image du sanctuaire scolaire. >>
\end{enumerate}
\end{minipage}
}

\vspace{0.5cm}

\item Elèves, usagers de école, dt des êtres sensibles et de futurs citoyens qui méritent, quels qu'ils soient et d'où qu'ils viennent, la prise en considération de leur destin. L'heure des compromis locaux pd le pas sur les grands modèles.

















\end{itemize}
\end{document}