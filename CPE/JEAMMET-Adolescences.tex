\documentclass[12pt]{report}
\usepackage{fontspec}
\usepackage{xltxtra}
\setmainfont[Mapping=tex-text]{Century Schoolbook L}
 \usepackage[francais]{babel}
 \usepackage{hyperref}
 \usepackage{geometry}
 \geometry{ hmargin=0.5cm, vmargin=0.5cm } 
\usepackage{xfrac}
\makeatletter
%\renewcommand\section{\@startsection
%{section}{1}{0mm}    
%{\baselineskip}
%{0.5\baselineskip}
%{\normalfont\normalsize\textbf}}
%\makeatother

\title{Adolescences. Repères pour les parents et les professionnels}
\author{JEAMMET Philippe}
\date{2002} 


\begin{document}
\maketitle


\tableofcontents

\chapter*{Auteurs, résumés, mot-clés.}

\section*{Philippe JEAMMET (dir)}

\begin{itemize}
\item chef du service de psychiatrie de l'ado et du jeune adulte, Paris. \\
\item Auteurs sous direction : médecins, psychologues, psychiatres, journalistes. \\

\section*{Résumé}
\item Ado : période de chgt. Risque de voir se figer conduites négatives qui peuvent empêcher dvpt harmonieux du jeune.
\begin{enumerate}
\item But du livre : mieux comprendre cette période, l'ado, l'aider à consolider ses acquis, lui donner de nlles chances de trouver 1 issue positive à ce qui a déjà marqué son enfance.
\end{enumerate}

\section*{Mots-clés}
\item 
\item 
\item 
\item 

\chapter{Transformations, 15}

\item Au sortir de ado: jeune adulte a bonne image de lui-même, s'adapte, se projette dans avenir, est autonome. \\

\item Ado : passage pr y arriver. Construction de identité et reconnaissance de son identité ne se font pas sans turbulences. \\

\subsection{Des chgts imposés, 15}

\item Métamorphose de ado imposée. Ne choisit pas lieu, quand, comment.\\
\item Multiples transformations physiques, psy, affective et sociale. Au coeur de métamorphose : rencontre ac corps sexué. Nlle fonction : assurer reproduction de espèce.\\
\item Renoncer au corps d'enfant pas facile. Puberté remet en cause. Jeune sais qu'il est plus un enfant, ms arrive pas à donner un sens à ce qui se passe. \\

\subsection{Des questionnements multiples, 16}

\item Qui suis-je ? Suis-je normal ? Comment me situer par rapport aux autres ?\\
\item Triple remaniement : relation avec son corps, son identité psychique et son environnement.
\begin{enumerate}
\item redéfinition de la relation à l'entourage et aux parents. \\
\item Réactivation du passé, force du présent, interrogation sur futur, le temps de l'ado est pas linéaire. Données s'enchevêtrent, avancée et reculs. \\
\item Parents et éducateurs doivent accompagner l'ado dans cet itinéraire.\\
\end{enumerate}

\section{Se métamorphoser, 16}

\item Apparition 1ers signes prépubaire sur corps : début de ado. Chgt travaille corps de enfant, bouleverse repère et dvpt physique.

\subsection{Des évènements qui s'imposent, 17}
\item Puberté peut être paisible ou vécu comme traumatisme. 
\begin{enumerate}
\item Phénomène naturel qui arrive qd il veut. Peut être ressentie comme subie.\\
\item (def) Puberté = ensemble des chgts, surtt biologique et anatomiques, qui aboutissent à la capacité de reproduction. Limitée ds temps (alors que ado est plus variable). Pas même début pr ts. \\
\item Plus précoce pr fille. Durée variable ms  séquence stéréotypée. \\
\end{enumerate}

\item Plan biologique : modifications hormonales connues. Inondation de organisme / hormones en cascade à partir d'une impulsion du cerveau.  L'hormone LH-RH excite la production de 2 autres hormones / hypophyse. Elles ont pr cibles les glandes sexuelles (ovaire et testicules) où elle => sécrétion d'hormones sexuelles. \\

\item 1er signe visible chez fille : apparition bourgeons mammaires (env 11 ans et demi). Ensuite, règles. Dvpt complet seins jusqu'à aspect adulte pd env 4 ans. Pilosité atteinte en 2 ans \sfrac{1}{2}. 1ères règles vers 12 ans \sfrac{1}{2}. \\

\item chez le garçon : augmentation du volume testiculaire (environ 11 ans \sfrac{1}{2}). Augmentation longueur et volume de verge (entre 12 ans \sfrac{1}{2} et 13 ans). Pilosité pubienne. Peut commencer vers 9 ans. Se termine vers 14. \\

\item Période marqué / accélération de vitesse de croissance. Croissance la plus forte : le thorax (25 cm entre 10 et 18 ans chez le garçon).

\subsection{Méconnaissables à eux-mêmes, 17}

\item plan morpho : modelage du cours de la jeune fille = féminisation (élargissement des hanches, dvpt de masse graisseuse) et garçon (augmentation largeur épaules) + dvpt masse musculaire, mue de voix et pilosité du visage. \\

\item vrai métamorphose. Peut rendre méconnaissable pr proche et pr lui-même (gauche, maladroit, ne sachant plus très bien qui il est). \\

\item Perception du corps : 68\% de jeunes se trouvent << bien >> pr poids. Mais 21 \% se trouvent trop gros, 11 \% trop maigre (enquête de 93). \\

\item 84\% se disent bien dans leur peau et 53\% sur d'eux-même. Garçon plus à l'aise que fille.

\subsection{La perte de la quiétude, 20}

\item Processus : rupture avec enfance.Doit faire deuil (statut de enfant, privilèges). 
\begin{enumerate}
\item doit abandonné une partie de ce qui le rattache au passé tt en intégrant 1 foule de données nouvelles. \\
\item perception qu'il a de son corps : participe à nlle identité. \\
\item 3 dimensions de transformation du corps.
\begin{enumerate}
\item plan statique : modification morphologiques => confrontation de ado a nlles réalités anatomiques. Vit dans monde d'apparence où formes ont importance. S'interroge sur aspect visuel, sensations cognitives, affectives, émotionnelles. Besoin de se reconnaître, de s'estimer par rapport aux autres. Soit il accepte l'image, soit il la refuse et focalise sur certaines parties de son nouveau corps (taille, poids, forme du nez, poussée d'acnée). Peut devenir une vrai phobie + dévalorisation de image de son corps.\\

\item plan dynamique : perte de stabilité. Altération des repères spatiaux lié à poussée de croissance. Recul de performance (certains renoncent à loisir de enfance car trop maladroit) : peut faire souffrir. Famille a du mal à admettre qu'il se désintéresse d'1 loisir.\\

\item plan intéractif : ado doit faire travail d'acceptation de son nouveau corps. Peut avoir certaines attitudes contrastées : prise de distance, provocation, séduction.\\
\end{enumerate}
\end{enumerate}

\subsection{Discerner le positif, 21}

\item Mutation => évolution de pensée et du jugement.
\begin{enumerate}
\item devient critique car monde des adultes est pas celui qui imaginait.\\
\item Eprouve moment de déception => période de rébellion enflammée.
\end{enumerate}

\item Devient + distant, secret. Exprime moins facilement ses besoins affectifs.
\begin{enumerate}
\item Supporte + difficilement règles et contraintes. Adulte tolère mal qu'elles soient enfreintes. Conflit. Ado est à la fois égoïste et généreux, peureux et téméraire, enthousiaste et découragé...\\
\end{enumerate}

\item Malgré tt ça, importt de savoir montrer le positif. Ex: Quel étonnement de le ou la voir organiser une petite réception pour fêter ses 16 ans alors qu'il lui semble si difficile de donner 1 coup de main à la cuisine !

\section{Miroir du corps, 22}

\item pdt enfance, individu a une image de lui-même qu'il se constitue pas rapport au regard des autres. Image continue de se construire ac transformations de son corps. Ms au moment de puberté, transformations s'accélère et tête ne suit plus. Période de décalage entre les 2 : image de soi ne correspond plus à réalité. \\

\subsection{Le décalage, source d'angoisse, 23}

\item Décalage source d'angoisse => recherche d'adaptation qui produit chgts ds attitude psychique et ds comportements. \\

\item Angoissé => peut devenir coléreux, fugueur, solitaire... \\

\item Agir est un moyen de décharger son anxiété : bouder, se mettre en colère, obtenir de mauvais résultats scolaires...\\

\item L'importance : arriver à établir des liens entre des phénomènes ordinaires et leurs conséquences 1 peu extraordinaire. Eviter de se laisser tromper / comportements incompréhensibles à nos yeux au point d'oublier l'importance d'un bouton sur le nez. \\

\subsection{Il aimerait être << comme >>, 23}

\item Ado se cherche. Mal dans sa peau, en quête d'1 autre image plus acceptable.
\begin{enumerate}
\item Ne faut pas s'étonner si jeune s'enferme des heures dans salle de bain, change de coiffure 3 fois / jours, revendique goûts vestimentaires contestables. Il est en recherche. Inquiet et cherche issue.\\
\end{enumerate}

\item A défaut de savoir qui il est, il va chercher à ressembler à tel chanteur. S'identifie à personnages. \\

\item Si regard de adulte à ce moment-là, sans compde que identité provisoire, et manifeste interêt ou agressivité, risque fort de s'installer un peu plus. Si attitude ne provoque pas de réaction chez entourage, elle fluctue et varie en une journée. \\


\subsection{Maltraiter sa propre image, 24}

\item Certains, / coiffure ou vêtement, vt jusqu'à se réfugier ds attitude de provocation. Pousse recherche de laideur ou disharmonie à son maximum. Pourquoi ?
\begin{enumerate}
\item préfère maltraiter leur image que de l'accepter. Au moins, ne risque pas de se plaire, ils sont tranquilles.\\
\end{enumerate}

\item Recherche du jugement des parents pas absente. Vont-ils s'en aperçevoir ? Qu'est-ce que ça leur fait ? Se pose question des limites et de franchissement.

\section{Qui suis-je ?, 25}

\item En même temps que cherche à s'approprier son nouveau corps, en quête de son identité psychique.
\begin{enumerate}
\item A question << Qui suis-je >> Répond : je suis un bon ou mauvais élèves, je suis celui qui fait rire, je suis solitaire ou chef de bande.\\
\item Faut veiller à ce que cette identité provisoire ne soit pas fixée / parole des adultes. Parfois, ado ont difficultés à se sortir de réputations qu'ils ont mis en place. \\
\end{enumerate}

\subsection{Partager la charge, 25}
\item pr mieux supporter choix de incertitude, certains cherchent soutient ds 1 groupe (musique, sport, idéologie) à solitude.\\
\begin{enumerate}
\item En partageant la charge de se demander qui ils sont, trouvent 1 apaisement provisoire.\\
\item face à phénomène de groupe, adultes svt irrités du comportement des ado (vêtement, musique...). Ne seront pas forcément plus tard fermés ds comportements stéréotypés. En ont juste besoin sur l'instant.\\
\item Faut aussi pde en considération façon dt entourage à envisager avenir du jeune.\\
\item => c'est à partir de ttes ces dimensions que personnalité du jeune se contruit. \\
\end{enumerate}


\section{Présence du passé, 27}
\subsection{Entre imaginaire et réalité, 27}

\item Concordance entre enfance et ado.
\begin{enumerate}
\item Avt naissance, parents imaginent enfant.\\
\item naissance et 1ers temps : concordance et divergence entre ce que les parents avaient imaginés et bébé réel. Avenir du bébé s'organise ds conjugaison de ce qu'on attend de lui et de ce qu'il est vraiment. \\
\end{enumerate}

\item Ado :  même mécanisme se reproduisent. Nouveau décalage entre ce que rêvent les parents pr leur enfant et ce qu'il veut.
\begin{enumerate}
\item 1 identité se dessine, qui ne ressemble pas à celle que parents souhaitait.\\
\item Certains parents vt avoir mêmes difficultés à reconnaître leur ado. Certains préfèreront continuer de rêver sans tenir compte de ce qu'il est. \\
\end{enumerate}

\item Tt se joue ds accordage entre rêve et réalité. Les 2 sont importants : 
\begin{enumerate}
\item jeune suscite désirs et espoirs : moteur de son dvpt. Faut pas y renoncer.
\item Ms peut pas non plus refuser de voir enfant réel

\end{enumerate}


\subsection{Les accidents de parcours, 28}

\item Au moment de naissance comme pr ado, difficultés imprévues peuvent surgir (ex: couveuse pr nourrisson, convulsion.)
\begin{enumerate}
\item Ds ces moment de fragilité, ajustement entre enfant imaginé et réel en train de se jouer. Evènement vt imprimer leur marque. \\
\end{enumerate}



\subsection{Des émotions à la pelle, 29}

\item choc de naissance ou de transformation de puberté délie langues. 

\subsection{Un couffin symbolique, 29}

\item qd nouveau-né pleure sans raison, parents tendance à prendre ds bras, envelopper corporellement, à lui parler. ds lit trop grand, tendance à ramper pr trouver des parois, limites / un contact contre sa peau. \\

\item ado a besoin d'1 couffin symbolique : doit pouvoir se représenter limites (sous forme d'interdits, règlements, contrats passé avec lui).  A besoin d'1 couffin matérialisé (chambre ou espace personnel).
\begin{enumerate}
\item Ne sait plus d'où il vient et où il va.  risque l'éclatement. En faisant de sa chambre 1 endroit protégé, il crée 1 espace clos où réunir son corps et son esprit. Malgré demande d'indépendance, a besoin d'être maintenu (ms avec souplesse).
\end{enumerate}

\subsection{Dépasser les bornes, 30}

\item Expérimente possibilités de satisfaire ses désirs qd il le veut. Ds enfance apprend que ce n'est pas parce qu'on lui refuse un cadeau qu'on ne l'aime pas, qu'un cadeau n'est pas dû, qu'il peut être différé.  A ado, remet ces questions à ordre du jour. \\

\item ado tentera de transgresser règles.  Si les parents conservent la même position, acceptera idée d'1 refus. \\

\item  si enfant a tjs obtenu ce qu'il veut, ado ne pourra pas comprendre 1 revirement des parents.  \\


\subsection{Le retour en arrière, 31}

\item qd ado malmené, peut être tenté d'1 retour en arrière, revient à une situation connue de enfance. \\
\item ex : propreté à enfance et hygiène corporelle à ado.  méticulosité excessive ou négligence et saleté : plus l'entourage réagit avec irritation, plus manoeuvre réussit. \\
\item les régressions st transitoires auxquelles il faut laisser libre cours tant qu'elles ne s'installent pas durablement. st liées aux difficultés de la puberté. \\

\subsection{Un moyen de se réassurer, 32}
\item / comportement de régression, ado recherche aspect rassurant : certaine maîtrise de relation à l'autre. 


\subsection{Un équilibre à apprécier, 33}
\item comportement passager.  2 écueils à éviter :  trouver tte manifestation de ce type anormale ms aussi le taire et le banaliser.  Ado a besoin des commentaires de ses parents. Il ne faut pas que ce soit humiliant ni 1 jugement péremptoire. \\


\chapter{Besoins, 35}

\item pr devenir adulte, enfant se dvpe sur plan psychique selon 2 pcpes : se nourrit des apports de environnement (surtt parents) + doit trouver sa différence et son autonomie. \\

\item  pdt période de latence (entre 8 et 12 ans), enfant a acquis contrôle sur son corps + maîtrise de lui-même et de son environnement. Ac puberté, situation nlle qu'il n'arrive pas à dominer. Tt change : corps, sentiment, émotions, pensées. Se trouve en situation de dépendance vis-à-vis des adultes, qui lui paraît insupportable. \\

\item Ado est poussé vers activité et autonomie.  Doit puiser ds divers savoir-faire et savoir-être.
\begin{enumerate}
\item selon qualité des relations dt il aura bénéficié ds passé, l'ado abordera + - facilement l'inconnu. 
\end{enumerate}

\subsection{Des situations de malaise, 36}

\item pr expérimenter nouveauté, ado revendique liberté. Ms pas encore capable de l'assumer. Tjs besoins de présence et affection de ses parents. Il est conscient de ses potentialités, pourrait mener une vie d'adulte ms pas psycho et sociologiquement prêt.
\begin{enumerate}
\item Relations familiales se sexualisent : dimension de gêne +  de proximité physique vite insupportable pr ado. Y répond ac distance exagérée.\\
\end{enumerate}

\item Pr se dvper, ado obligé de s'éloigner de ses parents. Il va devoir pde la mesure de ce qu'il peut faire par lui-même.\\
\item Situation génératrice d'inquiétude. Paradoxe : plus ado a besoin de soutien et plus celui-ci est une menace pour son autonomie. Plus il se croit dépendant des adultes, moins il le tolère. Ado dénudé, mis à vif par cette immense avidité affective qui le rend hypersensible. \\

\subsection{Attente et déception, 36}

\item Parfois, pense que estime de soi dépend que du regard extérieur. Ms risque si on a trop besoin de autre : être déçu. Si autre le comble, il est adoré, s'il le déçoit, il est rejeté.

\subsection{Opposition et vulnérabilité, 37}

\item Dvt nécessité de s'affirmer, ado va porter ses attaques contre ce qui faisait l'objet jusque-là, pour lui, d'un intérêt particulier : rejet d'une relation privilégiée avec 1 parent, abandon d'1 loisir ou de réussite scolaire. Mais résultat : s'affaiblit et devient encore plus vulnérable.

\subsection{Les fantômes du passé, 38}

\item pr parent, période où ils revivent ce qu'ils ont vécu avec leur propre parents. Ca brouille relation directe avec jeune.

\section{Prendre ses distances, 38}
\subsection{Besoins d'attachement, besoins d'autonomie, 38}

\item ds nbeux exemples, tt se passe bien : ado a ++ d'autonomie, parents prêts à laisser ado aller vers de nouveaux besoins : dialogue s'établit, accord à propos d'une sortie se négocie sans dommage.

\subsection{Au coeur des ambiguités, 39}

\item Ado a besoin de ses parents ms il doit s'en éloigner. Dépendant d'eux matériellement et affectivement. Ms ressent (à cause transformation de puberté) nécessité de les quitter pr pvr devenir adulte. \\

\item Corps de ado le trahit (trop grand, trop maigre, trop gros). Attire attention des adultes, les commentaires, les regards. Parents deviennent des êtres sexués, gêne, on se touche à peine.\\

\item proximité = danger. Plus d'entrée intempestive dans la salle de bain. Il a pourtant besoin de contact, ms il le trouvera plus facilement auprès de tiers (groupe, copine, amis des parents, gds-parents)...

\subsection{Une manière de se protéger, 40}


\item Ttes formes de conflits seront autant de barrage entre soi et parents. Manière pr jeune de poser bornes.

\subsection{Ca suffit maintenant !, 41}

\item Pr parents, important c'est la limite : ce que l'on accepte et ce que l'on défend.\\

\item Plus relation entre parent et ado insatisfaisante, plus difficile de se séparer. Peut de perdre l'autre en partant car on se quitte mal (s'engueule).  On reste ms on souffre. Ado a impression qu'il est mauvais et risque de perdre ses parents, parents craignent drames s'ils surveillent pas. Cercle vicieux. Appelle à tiers. 

\subsection{Le risque nécessaire, 41}

\item Prise de risque nécessaire ms attention qu'il soit pas trop grand. Svt, parent n'agit pas vraiment en fonction du fait objectif, ms de sa propre subjectivité, de son anxiété, de son passé. 

\subsection{La confiance et sa limite, 42}

\item ado dépendent de leurs parents ms ont du mal à le reconnaître. Idée de confiance et limite de confiance. Pvr reconnaître les capacités du jeune, ms aussi ses manques. Faut savoir poser la limite, être capable de dire on, d'affronter la colère ou la déception du jeune. 

\subsection{Sans perdre la face, 43}

\item Interdiction du parent viendra soulager le jeune si celui-ci ne se sent pas prêt à prendre le risque. Important que climat soit créé, où ado puisse sentir ce qu'il est en mesure de faire, à son propre rythme. Doit pas être poussé ni bloqué / parent qui n'agirait pas au nom de ado ms en fonction de ses propres inquiétudes. Ado doit pas sentir qu'il est pas traité comme un sujet à part entière, sinon se sent dévalorisé car constate que pas de rencontre entre lui et le parent.

\subsection{Les champs d'intérêt, 43}

\item intéressé / tout ou par rien. Change de loisir ts les 2 jours. se passionne pr une matière scolaire ou la rejette en bloc.\\
\item Interêt que ado porte à telle ou telle matière scolaire ou activité est indissociable de la relation qui s'établit à ce propos avec le parent (comme lorsqu'il était enfant avec les 1ers apprentissages).\\
\item Si ado << plonge >> avec de mauvais résultats scolaire, faut pas fixer sur abandon ou échec ms regarder ds quel climat relationnel il survient.
\begin{enumerate}
\item de manière générale, enfants ont impression qu'ils enlèvent à leurs parents ce qu'ils gagnent pr eux-même. Le jeune a besoins d'1 investissement qu'il réalise / lui-même, soutenu / un adulte qui ne tient pas 1 place essentielle. Adulte ne doit pas s'impliquer à la place du jeune.
\end{enumerate}

\subsection{Jardin secret, 46}


\item Equilibre à trouver : ado besoin de rester centre d'intérêt des parents ms veut ses zones de secret. Preuve de confiance de le laisser librement gérer son espace personnel. \\
\item Ms confiance ne se décrète pas. Il faut l'éprouver ce sentiment. Confiance ne s'établit pas à partir de l'attitude du jeune qui la mériterait ou non. Se joue dans une relation avec parents. Il ne faut pas que parents insèrent leur propre souvenir dans la relation. 

\subsection{Le parent, garant de la limite, 46}

\item pas éducation idéale, bonne ou mauvaise. parle plutôt de dynamique qui rend possible une évolution, sans que parents renoncent à leur rôle. St gardiens d'une limite, à l'intérieur desquelles ado peuvent créer 1 espace d'autonomie.\\
\item autorité trop forte ou suivisme (parents offrent plus de résistance) st 2 attitudes qui empêchent autonomisation.

\subsection{Trouver la liberté intérieure, 47}


\item nature de relation entre parents sera déterminante. Si complémentaires (ie chacun accepte de ne pas détenir tt dans le couple, chacun reconnaît ses limites), ado invité à devenir à son tour lui-même.

\section{S'identifier, 47}

\item Enfant se nourrit de ses deux parents. Enfant se construit dans la diversité. 

\subsection{Le corps, reflet de la relation avec le parent, 48}

\item relation avec le corps est le reflet de la relation avec les parents. Ex: fille qui refuse sa féminité. Si lui dit "tu es plus belle en robe" : elle va refuser l'idée.  Attitude des parents l'1 vis-à-vis de l'autre (quelle place occupe la féminité ds rapports : reconnue, valorisée, objet de mépris ?) sera bcp plus déterminante.

\subsection{S'identifier n'est pas imiter, 50}

\subsection{Se montrer authentique, 50}

\item parent doit trouver capacité à supporter agressivité de ado et ses écarts. Parents peuvent être contents de ce que le jeune réalise ou déçu / ses échecs scolaires, ses difficultés d'insertion. Qq fois, vaut mieux se montrer atteint, déprimé mais authentique.

\section{Prendre du plaisir, 51}

\item en lien direct avec autonomisation du jeune.

\subsection{De la pulsion aux attentes profondes, 51}

\item Plaisirs élémentaires (besoin de manger, dormir), d'autres st plus complexe (relation à autre), d'autres ont des attentes + profondes (être capable de répondre à 1 idéal 1 souhait des parents : sentiment d'être aimable et aimé). \\

\item plaisir pas que pulsion. Peut aller plus loin sinon il disparaît juste après. Exemple : fête après exam.
 

\subsection{Un révélateur de l'autonomie, 52}

\item capacité de se donner droit de vivre et d'être heureux lié à qualité relationnelle de ado avec entourage. Si pd plaisir en dehors parents, on peut donc se passer d'eux. Ms peux se retrouver seul -> plaisir associé à idée de perte de protection. svt suivi d'un moment dépressif. Peux être difficile à vivre. 

\subsection{Des plaisirs destructeurs, 52}

\subsection{L'expérience solitaire, 54}
\subsection{Un climat de confiance, 54}

\item chq ado unique.  pas de recette pr lui indiquer comment vivre sa sexualité. ms adultes st artisans du climat créé autour des enfant. Faut travailler à établir un climat de confiance pr que jeune puisse exprimer ce qu'il ressent.\\

\item svt choquant pr ado que parent exhibe vie sexuelle dvt lui.  Perd dimension idéalisé de parents, qui lui est encore nécessaire pr soutenir son estime de lui-même.\\

\item certains ado vulnérables refusent tt ce qui est de l'ordre du corps (ascèse) ou l'exhibe. traduisent difficulté pr ado de créer son propre T et vivre ses potentialités.

\section{Le besoin de s'intégrer, 57}

\item évolution naturelle de ado : s'intégrer à société des adultes. Besoin fondamental. Marginalisation pas un choix = témoigne de intensité de envie et de attente à égard de société.\\

\item 1ère intégration : découverte de autre sexe. s'ouvrir à la différence. Etape que enfant doit franchir : école, copains, clubs, groupements divers. utilisent puis laissées. Différence avec secte : totalitaire, solutions ne st pas étapes sur chemin ms vu comme aboutissement.

\chapter{Confrontations, 59}

\item Opposition : moyen privilégié / ado pr se situer dans relation ac adultes qui l'entourent. Amorce alors mvt de séparation.\\

 \item Ms attention : se séparer sans pr autant se retrouver seul. \\
 
 \item Si milieux familial a su lui  proposer dès enfance cadre précis, 1 ouverture assez large pr se différencier, ado sera en mesure de trouver ses propres limites sans conflits excessifs. Bcp jeunes st dans ce cas. Opposé : ceux qui ont pas pu exprimer aspirations différentes de parents sans provoqué intolérance, st demeurés très dépendants du milieu familial : arrivent pas à identifier leur désir comme le leur. Pr sortir de la confusion :  comportement provocants.
 

\subsection{La confusion des désirs, 60}

\item Si pas limites ou trop de limites, ado empêché d'être lui-même.

\subsection{Des moyens indirects d'opposition, 60}

\item s'engagera alors ds conduites négative car pourra les revendiquer comme siennes. Conduites d'attaques contre le corps (suicide, automutilation) => parents doivent réagir : ado rencontre alors une résistance.\\

\item exemple : la fugue. ne sait pas où on va. ms important est d'échapper à un mode qu'il ressent comme clos.\\

\item qd ado se sent absorbé / désir des parents, cherche moyen pr le protéger de cette fusion : alcool, drogue, sectes. But : se différencier avec les parents. \\

\subsection{Une confrontation tolérable, 61}


\section{Maintenir un lien, 62}

\item droits et devoirs de vie quotidienne fournissent à ado multiples occasions de partir à recherche de lui-même et de son environnement : pdt discussions familiales, sorties scolaires, fréquentations. Teste marge des manoeuvre dt il dispose, repère obstacles.\\

\item règles peuvent pas être fixé une fois pour toutes. Elles doivent évoluer dans la négociation : où échange et prise en considération de l'autre << différent >> prend pas sur rapport de forces.\\

\item trop de liberté peut être véu comme abandon, trop peu comme 1 contrainte insupportable. En s'opposant, ado éprouve solidité du lein qui le relie à ses parents.

\section{Les obstacles à la confrontation, 63}

\item confrontation constitutive de l'identité du jeune.

\subsection{Entre trop près et trop loin, 63}

\item pr devenir adulte, a besoin d'un espace personnel et d'une certaine quiétude. distance à respecter.

\item les conduites de descolarisations : 
\begin{enumerate}
\item tx de redoublement reflète échec des acquis scolaires. Ms conduites de descolarisation (fait d'arriver en retard, manquer des cours) : expriment désinvestissement scolaire qui peut mener au redoublement ou exclusion scolaire.\\

\item en 10 ans, moyenne a pas évolué. Constate diminution chez moins de 16 ans et augmentation chez + de 16 ans. jeunes en scolarité obligatoire plus assidu qu'il y a 10 ans. proportion élèves de 18 ans svt en retard est passé de 12 à 16 \% chez garçons et 6 à 10\% chez filles (entre 1993 et 2003)
\end{enumerate}

\subsection{La dépression parentale, 67}

\item si parent déprimé, jeune va pas vouloir rajouté encore de la peine : peur de se voir comme le méchant qui fait plus de peine au parent. Ado entre alors ds processus où il s'auto-confirme / 1 comportement provocants ds l'image négative de lui-même.

\section{Fuir les autres et soi-même, 68}
\item qd confrontation devient trop dangereuse, ne reste plus qu'à l'éviter.  ado adopte conduites pr fuir relation. trouve illusion de pvr sur son environnement.

\subsection{S'isoler, se mettre en retrait, 68}

\item replié sur lui-même, refuse échanges. solitude = fuite active face à situation qu'il ressent bloqué et violente. Traduit gde demande à égard des autres et intolérance à solitude.

\subsection{Un évitement de soi-même, 68}

\item peut être difficile d'accepter son incapacité à affronter les autres. en se fuyant soi-même, s'isolant, on évite de pde conscience de cette réalité.
\begin{enumerate}
\item : ex : l'ascèse : ado s'épuise ds conduites exigeant de renoncement. se prive de nourriture, se dépense physiquement bcp trop, se soumet à rituels de travail exigeants. se protège de sa vie intérieure / une somme de règles extérieures.\\
\item ex: recherche débridée de sensation (fait de rouler trop vite en voiture...). Craint de sombrer dans dépression s'il n'est pas survolté. Excès du comportement vient remplacer vide dû à absence de relation satisfaisante avec les autres.\\
\end{enumerate}

\subsection{Les fausses maîtrises de l'environnement, 69}

\item toxicomanie, alcoolisme, dispersion sexuelle, troubles de la conduite alimentaire, abus de médicament ont problématiques communes :  comblent ds vie de ado l'espace blanc du vide relationnel. pdt 1 temps, lui donne sentiment d'autonomie. ms se piège à son insu et retrouve dans cette fuite la dépendance affective vécue avec son entourage.\\

\item derrière comportement, trouve tjs mauvaise image de lui-même : manque estime de soi, fond dépressif.

\section{La définition des limites, 71}

\item comportements pathologiques viennent de difficultés des ado à trouver autorité des parents (qu'elle n'existe pas ou qu'elle soit trop forte). Opposition constructive commencera si cadre éducatif défini dès enfance, avec fermeté et souplesse, pr permettre constitution d'1 rapport de négociation.

\subsection{Trouver sa position, 71}

\item dvt non, ado voient svt que mauvais vouloir des parents => occasion d'excitation et de révolte.

\subsection{L'interdit, non le jugement, 72}

\item interdit : << j'ai peut-être tort, mais je n'accepte pas que tu fasse cela. Je ne peux pas dire oui. Tu verra plus tard. >> Position qui laisse autonomie potentielle à ado. Jeune pas disqualifié. Limite diffère juste réalisation de ces désirs.\\

\item jugement : << tu n'as pas honte ! tu me le demande mais tu n'en est pas capable ! >> Verdict sur ado.  pr soumettre ado à son pvr, a besoin de le dévaloriser. se sent meurtri et peut-être nourrira-t-il sentiment de révolte.\\

\item en posant limites, parent à svt peur de s'entendre dire : tu ne m'aime pas. comme s'il avait besoin d'être rassuré dans son rôle.\\

\item qd on dit non à ado, faut supporter qu'il claque la porte. Il ne faut pas changer d'avis. Ado a besoin de se révolter, si on va dans son sens, il peut pas et il est frustré. 

\section{Un << toucher >> réciproque, 74} 

\item faut instaurer précocement un dialogue avec enfant. 

\subsection{Attendons demain, 75}

\item faut savoir différé un conflit et ne pas le pousser à bout.  temps joue en faveur de apaisement : nous en reparlerons demain, avec ton père ou ta mère. Attendons demain.


\chapter{Générations, 77}
\subsection{La mémoire et le réveil, 77}

\item svt garde peu de trace de ado : tellement chargé en émotion, qu'on préfère l'oublier. S'en rappelle quand leur enfant est ado ou crise de la quarantaine.

\subsection{Une transmission inconsciente, 78}

\item crise de la quarantaine : se rappelle de ado. pose question : possible de repartir à zero ?

\section{Un héritage, 80}

\item alors que société change, ado éprouve pr sexualité même question que génération passée. 

\subsection{Le passé raconté, 80}

\item ado doit savoir d'où il vient (histoire familiale racontée) pr savoir où il va.  Il relativisera alors plus facilement ses difficultés actuelles s'il s'aperçoit que tt le monde a dû passer par là. 


\subsection{Les silences et les oublis, 81}

\item certains secrets volontairement gardés / adultes.  craint jugement de famille. redoute transmission. Peur de léguer échec et déceptions.  Doit-on partir à recherche de ce qu'on ignore ? Tout n'est pas bon à dire.

\subsection{Le fils de son père, 83}

\item ado savent très bien pousser parents à dire ce qu'ils ne veulent pas révéler. Tentative de suicide, prob scolaires, disciplinaires peuvent avoir cette fonction d'appel à la parole. Ado mène enquête, curieux. pdt recherche de filiation, ado trouve points d'appui dans identification à personnes qui se situent ds ses lignées et en dehors. Il va se construire à travers ttes les ressemblances. 

\section{Le <<métier>> de parent, 85}
\subsection{Des positions profondes, 85}

\item si ado est en difficulté scolaire, parent doit se souvenir de sa propre scolarité.  En la reprenant à son compte, le jeune lui montre son attachement. Si reproduction pas identique de génération en génération, c'est grâce à l'autre ds le couple (avec sa propre histoire, différente). 

\subsection{Des qualités et des défauts, 86}
\item illusoire de penser devenir de très bons parents. on a qualités et défaut. vouloir donner à ado illusion que adulte est être idéal : lui rend cet état inaccessible. \\

\item faut montrer qu'on est imparfait, afin que ado accepte ses propres difficultés. \\


\subsection{Pour souffler un peu, 87}
\item face aux difficultés de ado, possible de lui avouer son impuissance à aider directement. pas hésiter à rencontrer un professionnel. Adultes (proches, oncles, tantes) peuvent accueillir ado un moment. Hors maison, représentations changent (ado voit d'autres systèmes familiaux). Objectif : que jeune et parents se connaissent mieux. Voir ce qui l'intéresse. 



\section{Le parent du parent, 87}
\item Ado provoque bouleversement, questionnement sur couple et parentalité.  gds-parents st potentiel sous employé dans relation parent/ado


\section{La crise du milieu de la vie, 88}

\item enfant pubère, / apparition de sa capacité à devenir géniteur, fait basculer adulte ds autre génération. parent devient gd-parent potentiel. adulte préfère pas aborder de front csq de ces chgts. 

\subsection{Parent autonome, 90}

\item comment pouvons nous devenir autonome à égard de nos enfants ? Que devenons-nous hors de notre statut de parent de jeunes enfants ? 
\begin{enumerate}
\item réapprentissage de vie quotidienne du couple fera pas économie de sa sexualité. \\
\item parent tt seul se sent-il capable d'organiser différemment son temps ? quels sont ses champs d'intérêts ?
\end{enumerate}

\subsection{Le couple en face, 91}

\item aujourd'hui, constate retard dans prise de distance avec les enfants (jeunes adultes restent ++ dépendants). 2 réalités qui se font écho : ado essait de faire crise ++ tard (rester dans enfance), adulte essai de garder ado ++ tard (pr retarder sa propre crise). 

\subsection{L'idéal à propos de l'enfant, 91}

\item constat : adultes ne reconnaissent plus leur enfant (chgt physique, mû, attitude). Renoncement à leur idéal (pr certain, déception).  comment surmonter la déception ? Fait de s'avouer écart entre enfant et idéal =>  permet d'être moins blessé.

\section{Avoir un autre enfant, 93}
\item afin de différer ces 2 réalités, peut décider d'avoir 1 autre enfant. 

\subsection{Une évidence qui s'impose, 93}

\item nlle maternité place ado devt constat : père et mère st être sexués.  choquante pr ado. 

\subsection{Se dégager de la menace, 94}

\item doit se défendre contre ce risque potentiel : chercher à s'éloigner du parent, prendre ses distances / conflit, se place en position de 2nd mère. 

\section{Les ruptures dans la famille, 94}
\subsection{Question de confiance, 94}

\item quels sont à ado, effet d'une rupture des parents pdt tendre enfance ? que se passe-t-il si rupture survient à ado ? confiance à ado dans sa capacité à comprendre ce qu'il se passe : DETERMINANT. Ne doit pas lui demander son avis ms doit le comprendre. explication ne doivent pas être bidons. passage de responsabilité à culpabilité vite franchi.\\


\subsection{Sortir de l'engrenage, 95}

\item si se sent coupable, risque, ds sa vie futur, de répéter le scénario pr se l'approprier.  Ado : doit mettre des mots sur ce qui s'est passé.  de part parent, signe de confiance ds nlles capacité de enfant à comprendre. Important que 2 parents puissent s'exprimer. 

\subsection{Des éléments corrects de compréhension, 96}

\item ado ni sourd ni aveugles, ca sert à rien de leur mentir --> ils ne se sentiront que plus floués.  Plutôt que laisser ado en proie à son imagination et à sentiment de culpabilité, préférable que cheminement de crise à parole et à décision soit visible. 

\subsection{Le dégager de sa responsabilité, 96}
\item but : dégager responsabilité qu'il est prêt à endosser. 

\subsection{Le droit d'aimer les deux, 97}
\item parents ont leur raisons de se séparer. Ms doivent permettre à enfant de continuer à les aimer ts les 2.  Enfant n'a pas a choisir de camp.  doit avoir sentiment de conserver sa place dans esprit du parent qui a refait sa vie.


\chapter{Troubles, 99}
\subsection{Des transformations multiples, 100}
\subsection{Un bain émotionnel, 101}
\subsection{Les résistances au changement, 101}
\section{La peur de changer, 103}
\subsection{Des manifestations tangibles, 103}
\subsection{Des ado sur la défensive, 104}
\subsection{Le calme et la désintérêt, 105}
\subsection{Une trop grande nervosité, 106}
\subsection{Des réponses d'ajustement, 107}
\subsection{La recherche des solutions, 107}
\section{L'appréhension de la vie sexuelle, 108}
\subsection{Des craintes partagées, 109}
\subsection{Dire ce qu'il faut, mais pas plus, 109}
\subsection{En fuite de l'autre sexe, 112}
\subsection{Le désir d'attendre un enfant, 113}
\subsection{Le trop-plein ou le vide, 113}
\subsection{Menaces incestueuses, 114}
\section{Le manque de confiance en soi, 115}
\subsection{Des alternances d'excès, 116}
\subsection{La vulnérabilité à la déception, 117}
\subsection{Des clignotants qui s'allument, 117}

\chapter{Amours, 119}
\subsection{Richesse et danger, 119}
\subsection{Le cheval de Troie, 120}
\subsection{La sexualité comme quête affective, 121}
\section{Une vieille histoire, 121}
\subsection{L'ambivalence des sentiments, 122}
\subsection{Une conduite d'essais/erreurs, 122}
\subsection{Exprimer l'indicible, 123}
\subsection{A la découverte de l'autre, 123}
\section{Le discours sur la sexualité, 124}
\subsection{Prendre en compte le sentiment, 125}
\subsection{Tout en abordant les risques, 126}
\section{L'information en question, 127}
\subsection{Des maladies sexuellement transmissibles, 127}
\subsection{La menace du SIDA, 128}
\subsection{Les jeunes en pleine évolution, 129}
\subsection{Utile, rassurant, efficace, 130}
\subsection{Des obstacles qui demeurent, 132}
\subsection{Au plus près de la réalité, 133}

\chapter{Communication, 135}
\subsection{La communication, un révélateur, 136}
\subsection{Etre proche et différent, 136}
\section{La communication ne se décrète pas, 137}
\section{La distance relationnelle en question, 138}
\subsection{Tu ne me comprend pas, 139}
\subsection{Ma mère, je lui dis tout, 139}
\subsection{Alors cette sortie, Raconte ?, 140}
\subsection{Je te demande pas ton avis !, 140}
\section{Perturbations, 141}
\subsection{Parler avec les maux, 141}
\subsection{Des actes comme langage, 142}
\subsection{Choses cachées, inventées ou volées, 142}
\subsection{Secret de famille, 144}
\section{Quand le message est reçu, 145}
\subsection{Etre bien ensemble, 146}

\chapter{Maladies, 147}
\subsection{Une période sensible, 147}
\section{Patraque ou malade, 149}
\subsection{Anomalies de la puberté, 149}
\subsection{Du retard simpl au retard pathologique, 150}
\subsection{Maladies chroniques et handicaps, 152}
\subsection{Vivre dans la différence, 154}
\subsection{La recherche d'un certain << bien-être >>, 155}
\subsection{La souffrance des parents, 156}
\subsection{La juste place de la maladie, 157}
\section{Désordres du corps et de l'esprit, 158}
\subsection{Les troubles du comportement, 159}
\subsection{Des actes qui appellent, 160}
\subsection{La passion de la nourriture, 161}
\subsection{La recherche de l'ivresse, 163}
\subsection{La fuite et le silence, 164}
\subsection{Une absence de désir, 165}
\subsection{Les souffrances psychiques, 165}
\subsection{Des manifestations parfois spectaculaires, 166}
\subsection{Du spleen à la dépression, 167}
\subsection{Le corps comme langage, 167}
\subsection{Derrière la plainte, 168}
\subsection{La demande d'aide, 169}
\subsection{Eviter à tout prix l'enfermement, 170}
\chapter{Espaces, 171}
\subsection{La parole et le temps, 172}
\subsection{Savoir écouter, 173}
\subsection{L'ado a besoin de quelqu'un à qui parler, 173}
\section{Le médecin généraliste, 173}
\section{Le pédiatre, 175}
\section{Le dermato, 177}
\section{Le gynéco, 177}
\section{Les enseignants, 178}
\section{L'infirmière scolaire, 180}
\section{Le médecin scolaire, 183}
\section{L'assistante sociale scolaire, 183}
\section{Les associations de parents d'élèves, 187}
\section{Le psy, 187}
\section{Le psy pr ado, 190}

\chapter{Table ronde : et la santé, ça va ?, 197}


















\end{itemize}

\end{document}