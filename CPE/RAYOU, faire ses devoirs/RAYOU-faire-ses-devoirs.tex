\documentclass[12pt]{article}
\usepackage{fontspec}
\usepackage{xltxtra}
\setmainfont[Mapping=tex-text]{Century Schoolbook L}
 \usepackage[francais]{babel}
 \usepackage{enumitem}
 \usepackage{pifont}
 \usepackage{geometry}
 \geometry{ hmargin=0.5cm, vmargin=0.5cm } 

\makeatletter
%\renewcommand\section{\@startsection
%{section}{1}{0mm}    
%{\baselineskip}
%{0.5\baselineskip}
%{\normalfont\normalsize\textbf}}
%\makeatother

\title{Faire ses devoirs. Enjeux cognitifs et sociaux d'une pratique ordinaire}
\author{RAYOU}
\date{2010} 
\frenchbsetup{StandardLists=true}

\begin{document}
\maketitle
\section*{Emplacement :} 370.15 BAUT

\section*{Auteurs : }
\begin{itemize}
%\item Elisabeth BAUTIER : profs science éduc Paris 8
\item Patrick RAYOU : prof science éduc Paris 8.
\begin{enumerate}
\item Au début, recherches sur la sociologie de enfance. Depuis 2 nouveaux thèmes : formation et professionnalisation des enseignants / inégalités d'apprentissages.
\end{enumerate}
\end{itemize}


\section*{Résumé}
\begin{itemize}[label=\ding{43}]
\item << Travail hors classe >> (tps qui pd suite du W d'apprentissage réalisé en classe). Fait objet d'interrogation et de critiques. Comment comprendre que, alors que limité dans la loi, il est très utilisé en vrai ? \\

\item S'appui sur 2 enquêtes : 
\begin{enumerate}
\item Réalisée / équipes de chercheurs des IUFM de Créteil, Aquitaine, Versailles de 2005 à 2007
\item 2008 / appel du Centre Alain Savary de INRP.\\
\end{enumerate}


\item Terrains de enquête volontairement multiples : ZEP ds Val-de-Marne et Seine St Denis, Somme, collèges et écoles de classe moyenne, voire moyenne sup ds banlieue de Bordeaux et Val-de-Marne. \\


\item Ouvrage a 2 parties : 
\begin{enumerate}
\item Mise en perspective des cadres historiques, institutionnels et sociaux du TH (W hors classe)
\item réflexion sur THC 
\begin{enumerate}
 \item \textbf{THC = tâches orales ou écrite données / enseignants à élèves et effectuées hors de leur regard et de leur soutien directe, ds école (études surveillées, aide aux devoirs...) ou hors de école (ds familles, au sein d'assoc.). }
\end{enumerate}
\end{enumerate}

\item Intro : P. Rayou demande si maintien des devoirs relève d'une << croyance partagée du monde de l'éducation >> ?\\

\item Valérie Caillet et Nicolas Sembel : confrontent pts de vue enseignants, parents et élèves. Montre que << bonnes raisons >> de demander THC l'emporte sur critiques.
\begin{enumerate}
\item Bonnes raisons : fixer apprentissages, identifier difficultés élèves et ajuster enseignement en retour, dvper autonomie élève, impluquer parents ds scolarité des enfants. \\
\item critiques des acteurs : interrogation sur formes du travail (répétitions, apprentissages / coeur, lourdeur des devoirs)
\end{enumerate}


\item Frédéric Charles : s'interesse à manière dt 1 promotion de stagiaire perçoivent à fin de stage le THC : rapport positif. 96 \% st ok avec pratique car permet de << réviser et mémoriser savoirs appris en cours >>

\newpage
\section{oi}
\section{oi}

\newpage

\section{Le point de vue des élèves}
\subsection{Un rituel de transition de l'école vers la famille, 48}
\subsubsection{Au centre de l'emploi du temps de l'élève et de l'enfant, 48}

\item pr ts élèves qui sortent sans avoir fait devoir, emploi du tps : THC. Tjs vécu comme une obligation, voire une contrainte.

\subsubsection{En lien avec l'autorité parentale, 49}

\item THC = occaz pr parents de faire autorité parentale, exercer droit de regard et vérification. Présence parents nécessaire aux élèves même qd enfants n'a pas  de difficultés scolaires. \\

\item THC reflète choix de mode éducatif, de partage de expérience scolaire des enfants ac parents.

\subsection{Un rituel de transition entre travail scolaire et loisirs, 50}

\item THC empêche que monde extrascolaire soit 1 monde sans aucun lien ac W scolR. S'inscrit dans structuration du tps quotidien de enfant (il grandit grâce à articulation de contraintes et de plaisir). 

\subsection{Un révélateur du travail personnel de l'élève, 51}

\item Ac THC, élève doit se construire cô sujet de sa scolarisation, se motive pr faire ses devoirs et se concentrer sur possibilité d'1 travail autonome. \\
\begin{enumerate}
\item Elève a choix de faire ou pas W, de le faire en avance, retard.., d'en faire plus.
\end{enumerate}

\subsection{Un révélateur du travail enseignant, 52}
\item THC peut se substituer au W ds classe. Espace possible pr construction de autonomie et subjectivation de élève.

\section{Les principales tensions du travail hors classe, 53}

\subsection{Les enseignants : différenciations et interrogations, 53}
\subsubsection{Les tensions du quotidien : différences, comparaisons, croyances, 53}

\item Diff liée aux incertitudes réponses que veut prof. Porte sur méthode : apprendre / coeur ou non ?\\

\item prob d'absence d'échange sur THC. Devoirs rarement mené / pol d'étab.  Guidés / choix indiv. Si interdisait THC, prof s'en sortiraient qd même. \\

\subsubsection{Des tensions structurelles : interrogations et dysfonctionnements}

\item questions pr THC : aide parentale (ne sait pas si élève a travaillé seul ou pas), tps consacrés, manière dt élève le réalisent (recopient-ils bêtement ce qu'ils ont appris ?), difficultés de certains élèves.\\

\item selon certains profs, disparition THC risquerait de creuser inégalités sociales. Selon d'autres, c'est l'inverse.

\subsubsection{Au collège, 57}

\item profs donnent parfois trop de THC. Inégalité d'aide au travail en dehors de classe.\\

\item tensions du THC => revers de individualisme des profs. continuent de croire en le THC tt en étant pas dupes. Critique svt externalisée et centrée sur organisation du W, sur division des tâches, missions et système éduc en général : comment motiver élèves ? noter leur travail ? réduire échec scolaire ?

\subsection{Les parents : du malentendu au différend, 58}

\item tension sur famille lié à intrusion du THC ds familles.

\subsubsection{Le malentendu, expression des tensions du quotidien, 58}

\textbf{La réception parentale du THC, 58} \\

\item différence entre prof (fait de donner ou non du THC) perçu comme irrégularité entre classe, source de sentiment d'injustice et de critique.\\

\item Incompréhension des parents sur devoirs à faire. Parents endossent tant bien que mal le rôle de prof auprès de enfant : st ds pédagogie de réponse. se trouvent désarmés dvt 1 pédagogie de questionnement ou découverte.\\

\item tps de W réel à la maison svt sup à celui imaginé / prof (normalement 20 min : ms correspond plus à W d'1 bon élève que d'un élève normal).\\

\item Prof perçoivent pas situation parents << débordés >> comme prob. \\

\textbf{Le brouillage des rôles, 59} \\

\item THC brouille distinction école-famille sur plan de instruction (théoriquement école) qui entre au domicile. Parents s'implique trop ou pas assez. Attention à ce que THC ne deviennent pas source de conflit à l'intérieur des familles.\\

\subsubsection{Le différend, expression de tensions structurelles, 60}

\item Malentendus (prof-parent) peut devenir 1 conflit autour de 2 points : 
\begin{enumerate}
\item consumérisme parental. Réduit le THC à 1 << prestation de service >> alors que école devrait pas être 1 service ms un lien.\\
\item mauvais résultats scolaires.\\
\end{enumerate}

\textbf{Une critique parentale ambivalente}\\

\item certains parents confondent devoirs et leçons. Certains demandent arrêt complet des devoirs, d'autres encouragent les enfants à les faire.

\item parents <<débordés>> / THC st svt familles favorisées qui possédaient jusqu'à présent ressources pr éviter de se retrouver ds cette situation. Enfant de parents débordés, travaillent svt moins seuls, se concentrent plus. Retrouve bonne volonté pédagogique des catégories pop.\\

\item parents demandeurs de THC ft partie classe très favorisées. Enfants diversifient plus travail, connaissent peu difficultés.\\

\item les 2 types de parents st pris ds logique consumériste qui touche ensem des parents et élèves. Famille en attente de réussite scolaire. Ouverture culturelle passe après.\\

\textbf{Tensions liées à la difficulté scolaire de l'élève.}\\

\item critique parentale augmente qd résultats baissent. Difficulté scolaire \ding{233} différends entre parents (W comme un autre) et enseignants (W à relativiser / rapport réussite de élève).\\

\item En cas de difficulté scolaire, parents amenés à discuter + du THC ac prof (par pr critique ms pr demander conseils). Tensions sur THC varie selon niveau scolaire de élève : augmente qd difficultés. Tension \ding{233} critiques (demande parentale exigeante). Difficulté scolaire fragilise voir marginalise élèves qui la connaissent

\subsection{Les incohérence de la circulation du W des élèves, 64}
\item Tension pcpale du THC = accumulation de petites tensions lié à son pcpe : sa circulation en dehors de classe, la mise en oeuvre. Tension \ding{218} ac incohérence perçue. \\

\subsubsection{D'un enseignant à autre, 64}

 \item THC = moment pr élève de comparer différence ds W scol des profs et entre classe. Tte différence vue comme injustice. Futur W au collège peut \ding{233} stress ou découragement

\subsubsection{De la classe à l'étude surveillée, 64}

\item ambiguité du statut d'étude surveillée en primaire.  Appelé aussi garderie. Diff entre bons élèves qui veulent se débarrasser du W au + vite, entre ceux qui en ont pas à faire, et ceux qui st à étude surveillée parce que leurs parents peuvent pas venir les chercher.\\

\item limite de étude surveillée : dimension + gde que juste THC. construction individuelle d'1 autonomie à la fois sans les pairs et à la périphérie des institutions. \\

\subsubsection{De l'étude à la famille, 66}

\item THC empiète sur vie quotidienne. coexistence W-loisir peut être compliqué. Difficulté du W solitaire. Comparaison aussi entre lieux de W (pr divorcé) \ding{233} élève doit s'adapter aux liens et personnes qui l'entourent.


\subsubsection{Le retour en classe, 67}
\item si THC pas fait, élève risque punition en classe. Soit c'est très mauvais élève et laisse couler, soit punit. Bon élève : le font pas exprès. Ms en même tps, profs lui accordent une valeur relative.

\item Conclusion pr élèves : Le W juste =  
\begin{enumerate}
\item doit être identique ds ttes les classes, en qtité réelle ms limitée pr q chacun puisse se l'approprier individuellement avec le moins de tensions possible et commencer à construire son autonomie.
\item 
\end{enumerate}

\section{Entre consensus sociale, pli cognitif et autonomie intellectuelle, 68}

\item modèle fait pr les classes moyennes par les classes moyennes, réponse externalisée à massification et échec scolaire.\\

\item fondement informel et ancien dy système éducatif, << désirée et rejetée, nécessaire et inutile, efficace et inefficace, sécurisante et source de tensions>> (Besson, Glasman, 2004) \ding{233} << THC pas représentatif de évolution de institution scolaire ms correspond plus au vécu et attentes des parents >>. Peu importe qu'il améliore pas la réussite scolaire, tant qu'il maintient le niveau actuel.

\chapter{Les futurs professeurs et leurs rapports avec le THC / Frédéric Charles, 71}


\newpage









\end{itemize}



\end{document}