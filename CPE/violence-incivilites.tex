\documentclass[12pt]{article}
\usepackage{fontspec}
\usepackage{xltxtra}
\setmainfont[Mapping=tex-text]{Century Schoolbook L}
 \usepackage[francais]{babel}
 
 \usepackage{geometry}
 \geometry{ hmargin=0.5cm, vmargin=0.5cm } 

\makeatletter
\renewcommand\section{\@startsection
{section}{1}{0mm}    
{\baselineskip}
{0.5\baselineskip}
{\normalfont\normalsize\textbf}}
\makeatother



\begin{document}



\textbf{Violence  et incivilité, 91} \\

\textbf{Mots clés : }
\begin{itemize}
\item rupture scolaire
\item délinquance
\item zone sensible
\item violence
\item notion de << comportement >> (Mille, Thin)
\item climat scolaire
\end{itemize}

\vspace{0.5cm}

\textbf{Objectif :}


\begin{enumerate}
\item Définir ce qu'est la violence en milieu scolaire.\\
\item Voir causes de violence pr établir plan d'action à échelle de étab.\\
\item connaître pcpes du droit pr sanctions. \\
\item Situer plans français de lutte vs violence ds cadre européen.\\
\end{enumerate}

\textbf{1. Etat des lieux sur les violences à l'école.}

\begin{itemize}
\item ++ préoccupante. Comme cause, évoque déclin des solidarités, fragilisation lien social, exclusion sociale, chômage, sentiment de galère ds zone sensibles, décrochage élèves, absentéisme. \\

\item Pr saisir phénomène, faut pde en compte représentations et identités des différents membres de communauté éducative.

doc 1. Les incidents graves selon leur nature.
%\includegraphics[scale=0.20]{file}

\item Graphique (entre 2007 et 2009) à partir de sources données / ministère de Education nationale.
\begin{enumerate}
\item aggravation des faits les + graves. Violence verbale en diminution. Agression physique + fréquentes (presque 50 \% des faits de violence déclaré) \\
\item considéré avec prudence données fournies. Incident répertoriés / ministère recueillis à partir de 2 questionnaires envoyé chq trimestre à 1 échantillon de 1000 étab : 
\begin{enumerate}
\item 1er : recense nb et type d'incidents pdt le trimestre.\\
\item 2nd : demande appréciation du chef d'étab sur ambiance gal et sécurité.\\
\end{enumerate}

\item nb de réponses diffèrent selon type d'étab (collège répondent plus que lycée), selon moment ds année (à partir de mars, répond moins). \\
\begin{enumerate}
\item publication palmarès lycée + mise en concurrence étab ds système éducatif ++ piloté / loi du marché = soupçon d'insincérité sur déclaration de faits de violence de part des chefs d'étab.\\
\end{enumerate}

\end{enumerate}

\item presse écrite relate que certains nb de faits, gravissimes. ms pas quotidien des étab. en 2004 :  étab ft remonter 240.000 déclarations d'incidents / trimestres dt 3\% st faits graves.\\

\item donnée importte :  63\% des coups et blessures contre élèves ou adultes en milieu scolaire st fait d'élèves de l'intérieur. -> représentation de hordes barbares déferlant sur étab pas fondée.\\
\begin{enumerate}
\item faut pas faire des étab des forteresses ms plutôt aller vers une ouverture organisée. Trop de fermeture peut préciciper la coupure sociale.\\
\end{enumerate}

\item violence très sexuée. Surtt garçons.\\

\item  personnels st victimes de violence ds proportion inquiétante.\\

\item typologie des faits de violences : 
\begin{enumerate}
\item violences pénalisables.
\begin{enumerate}
\item 1er niveau. crimes et délits : vols, extorsions, coups et blessures, trafic et usage de stupéfiants...\\
\end{enumerate}
\item violences non pénalisables.
\begin{enumerate}
\item incivilités (bruit, vandalismes, injures...)
\begin{enumerate}
\item codes élémentaires de la vie en société qui est pas respectée. Peuvent paraître comme menace contre ordre établi. \\
\item violence grave et révélatrice d'une crise forte du lien social. Dominante en milieu scolaire. Explique malaise actuel plus que les violences.\\
\end{enumerate}

\item sentiment d'insécurité, << de violence >>. svt ressentie / personnes pas victimes de faits violents ms qui ont peur de l'être.\\

\item certains actes violents (racket, incivilités) commencent à toucher école primaire.\\

\end{enumerate}
\end{enumerate}

\item en lycée, chiffres baissent.\\
\begin{enumerate}
\item avec âge, jeunes se construisent une identité autrement que / violence.\\
\item 1 partie des éléments les + durs ont quitté système éducatif.\\
\end{enumerate}

\textbf{2. Les causes sociales de la violence.} \\

\item faits de violence en augmentation ds zones sensibles.\\

\fbox{
\begin{minipage}{19cm}
\textbf{DEBARDIEUX Eric, \textit{Les dix commandements contre la violence à l'école}, 2008, p. 96} \\
\begin{enumerate}
\item \textbf{élèves situés en zones sensibles st ++ svt victimes des faits de violence.} \\
\item augmentation du sentiment d'insécurité lié à augmentation de l'intensité des victimations pr 1 nb restreint de victime plus durement agressée. \\
\item Vrai changement : augmentation des agressions en groupe, contre des victimes isolées, sur critère identitaire.


\item Debardieux est 1 membre de l'Observatoire européen de la violence scolaire. Membre de American Society of Criminology.

\item selon lui,  baisse du nb de faits déclarés de violence + aggravation de nature des violences.  Ccls : 
\begin{enumerate}
\item dep 10 ans, augmentation faits + graves. Faits moins graves st en baisse. Faits de violence dans ensemble baissent.\\
\item cadre des faits + graves ++ des lieux d'exclusions. Fréquences diminue ds étab ds secteurs non sensibles.\\
\item augmentation actes de violence pr raison religieuse, ethniques...\\
\item augmentation nb d'agressions commises en groupes.\\
\end{enumerate}
\end{enumerate}
\end{minipage}
}


\item \textbf{Aggravation des inégalités sociales devant faits de violences}. \\

\fbox{
\begin{minipage}{19cm}
\textbf{DEBARDIEUX Eric, \textit{Les dix commandements contre la violence à l'école}, 2008, p. 96} \\

\begin{enumerate}
\item \textbf{Violence contre personnels et institution augmente ds zones prioritaires}

\item source : recensement 2005-2006 de SIGNA : 
\begin{enumerate}
\item + 7\% de violence vs profs.\\
\item + 23\% par rapport à 2002-2003 sur une longue durée pr personnel responsable de l'ordre au quotidien : CPE ou surveillants. \\
\end{enumerate}

\item écart collège ZEP et autre se creuse. Plus forte agressivité contre enseignants. \\
\end{enumerate}
\end{minipage}
}

\vspace{0.5cm}

\item  violence frappe élève à double titre.
\begin{enumerate}
\item crée sentiment d'insécurité qui détériore conditions de leurs apprentissages.\\
\item éloigne d'eux enseignants les + expérimentés, dt l'ancienneté de services leur confère 1 barème suffisant pr obtenir 1 mutation dans lieux moins exposé aux violences.\\
\end{enumerate}

\item Causes ?
\begin{enumerate}
\item selon sociologue, lien avec exclusion. Ms faut pas tomber dans << criminalisation de la misère >>.
\item Etude menée / univ  de Bordeaux pdt 32 ans sur 545 élèves. Montrent concomitance facteur de marginalité (pauvreté, chômage récurrent, père violent ou absent) empêche pas bcp d'enfants d'échapper aux prédictions de violence ou de s'adapter aux exigences de institution.\\
\item ds certains cas, \textbf{étab médiateurs essentiels ds transactions des enfants ac leur environnement}. Prouvée que déscolarisation corrélée avec délinquance.
\end{enumerate}

\fbox{
\begin{minipage}{19cm}
\textbf{GLASMAN Dominique et DOUAT Etienne, <<Qu'est-ce que la << déscolarisation >> ?, in GLASMAN et OEUVRARD Françoiase (dir), \textit{La déscolarisation}, 2011, p. 71-73.}

\begin{enumerate}
\item parmi jeunes incarcérés, ceux qui avaient été déscolarisés st surreprésentés.\\
\end{enumerate}
\end{minipage}
}
\vspace{0.5cm}

 \item conclusion : exclusion prolongée du collège ou lycée = mesure contre-productive. augmente risque de plongée ds délinquance pr jeunes en situation de rupture scolaire ou de violence à l'école.\\


\textbf{3. Ruptures scolaires, incivilités et violences} \\
\textit{3.1. Les facteurs qui favorisent la violence en zone sensible}.\\

\item Ne doit pas conclure que violence en milieu scolaire est une \textbf{fatalité} ds zone sensible.\\

\item Certains étab, même si situés en secteurs défavorisés, échappent à ces logiques. Ils ont : 
\begin{enumerate}
\item moins de 500 élèves.
\item << loi >> bien appliquée
\item équipes péda unies et motivées
\item équipe de direction volontaire, fiable et s'enferme pas ds tâches adm.
\end{enumerate}
\fbox{
\begin{minipage}{10cm}
cf. OBIN Jean-Pierre [dir], \textit{Pour un étab mobilisé contre la violence}, site internet education.gouv.fr
\end{minipage}
}

\vspace{0.5cm}

\item  Les plus concernés st : 
\begin{enumerate}
\item effectifs lourds (plus de 600 élèves)
\item élèves ignorent le bureau du principal qu'ils considère vide de sens.
\item équipes st enfermés ds conflits d'adultes
\item désordre quotidien permanent et non  régulé.
\item Perte générale de confiance ds capacité des adultes à instituer l'ordre. \\
\end{enumerate}

\textit{3.2. L'enchaînement ruptures scolaires - délinquance.} \\

\fbox{
\begin{minipage}{19cm}
\textbf{MILLE Mathias, THIN Daniel, \textit{Ruptures scolaires : l'école à l'épreuve de la question sociale}, 2010, p. 157}\\
\item Différentes façons de voir les collégiens en rupture scolaire.
\begin{enumerate}
\item Les plus contraires au règles scolaires. Comportement perturbateurs de l'ordre scolaire (secondairement absentéisme).
\item Notion de << comportement >> omniprésente ds appréciations scolaires des collégiens. Distingue les << bons comportements >> (conformes aux règles et morales scolaires) des << mauvais comportements >> (perturbant ordre scolaire et entravant action péda.)
\begin{enumerate}
\item Ce seul mot suffit à justifier de sanctions
\end{enumerate}
\end{enumerate}

\item parle de effet nocif d'une norme scolaire illisible. Loi doit être bien identifiée. Régime de sanction progressif. Jeune sanctionné pr comportement : trop illisible. Comme automobiliste sanctionné pour inconduite sans plus de précisions.
\begin{enumerate}
\item Effet de cet stigmatisation : enchaînement qui conduit au rejet de école, aux incivilités et aux violences.
\end{enumerate}

\item auteurs insistent sur fait qu'au départ de ces problèmes, très svt des difficultés d'apprentissages : 
\begin{enumerate}
\item difficultés d'apprentissage
\item pratique hétérodoxes dans espace scolaires
\item stigmatisation provoquée / sanction pour << comportement >>.
\item hypoactivité scolaire manifestée / collégiens.
\item évitement et pratique de survie. cours - en - supportables. présence en classe perd son sens.
\item perturbations de ordre scolaire + relations conflictuelle ac profs.
\item comportements engendrés => aggravation des performances scolaires.
\end{enumerate}
\end{minipage}
}

\vspace{0.5cm}

\item Peux pas réduire question de violence à problème pédagogique. Ms certain que échec scolaire non traité conduit à rupture scolaire et violence contre institution et personnel.\\

\item svt 1 solution péda appliquée précocémment meilleure des solutions. Ici que rôle CPE essentiel : très différent du surgé : 
\begin{enumerate}
\item dépister, derrière << problèmes de comportements >> : éventuelles difficultés d'apprentissages\\
\item mettre en place qd possible, projet péda d'aide et soutien aux élèves pr restaurer estime de soi et réconcilier élève ac école.\\
\end{enumerate}

\textbf{4. Les principes des mesures disciplinaires} \\

\item Ne faut pas se désintéressé de question des sanctions.
\begin{enumerate}
\item \textbf{sanction disciplinaire : objet de réflexion approfondie dep 15 ans}.

\fbox{
\begin{minipage}{15cm}
cf travaux de Olivier REBOUL, Eirick PRAIRAT, Eric DEBARBIEUX, Gilles FERREOL\\

\textbf{PRAIRAT, \textit{La sanction en éducation}, 2003}\\
\begin{enumerate}
\item rappelle méfaits des châtiments corporels, administrés / enseignants ou parents.\\
\end{enumerate}
\end{minipage}
}

\end{enumerate}

\vspace{0.5cm}

\item Elèves doivent être associés à tte phases de conception et application des sanctions pr initier à vie démocratique.

\item circulaire, 1er août 2011 << L'organisation des procédures disciplinaires dans les collèges, lycées et les EREA >> distingue punitions scolaires et disciplinaires.
\begin{enumerate}
\item punition scolaire =  acte pédagogique : manquements mineurs des élèves, prononcé / personnels.
\item punition disciplinaires = dimension éducative. prononcées / chef d'étab ou représentant. figurer au règlement intérieur. \\
\end{enumerate}

\textit{4.1. Principe de la légalité des sanctions disciplinaires} \\

\item pas rétroactives. Peuvent faire objet d'1 recours administratif interne ou d'1 recours dvt juridiction administrative. \\

\textit{4.2. Principe du contradictoire} \\

\item Avt tte décision à caractère disciplinaire, impératif d'instaurer dialogue ac élève. Sanction doit se fonder sur élément de preuve. Peut faire objet d'1 discussion avec parents.\\

 \item procédure contradictoire : chacun doit pvr s'exprimer. Elève peut se faire assister de personne de son choix (élève ou délégué de classe).\\
 
 \item représentants légaux du mineurs informés de procédures et aussi entendus s'ils veulent. Tte sanction motivée et expliquée. \\
 
 \textit{4.3. Principe de la proportionnalité de la sanction}\\
 
 \item Finalité de la sanction :\textbf{promouvoir une attitude responsable de l'élève.}
 
 \item Impératif qu'elle soit graduée selon gravité du manquement à la règle.\\
 
 \item hiérarchie entre : 
 \begin{enumerate}
 \item atteintes aux personnes et atteintes aux biens\\
 \item infractions pénales et manquement au règlement intérieur. \\
 
 \end{enumerate}

\item Utile de se référer au registre des sanctions disciplinaires pr éviter des distorsions ds traitement d'affaires similaires. \\

\textit{4.4. Principe de individualisation des sanctions.}\\

\item En aucun cas collective.
\item Individualiser = tenir compte du degré de responsabilité de l'élève, son âge, implication ds manquements, antécédents en matière de discipline.\\
\begin{enumerate}
\item décret du 24 juin 2011 : prévoit obligation d'1 action disciplinaire ds certains cas de violence verbale, physique. \\
\end{enumerate}

\item sanctionne pas que acte commis, ms sanctionne en regardant personnalité de élève, contexte de chq affaire.\\

\item Tte sanction, hors exclusion définitive, est effacée du dossier administratif de élève au bout d'un an.\\

\textit{4.5. Les mesures de prévention}\\

\item Mesure inscrites au règlement intérieur. Vise à prévenir acte répréhensible (ex: confiscation d'un objet dangereux). \\

\item Obtenir engagement de élève sur objectifs précis en termes de comportement. Rédaction d'1 document signé par l'élève.\\

\textit{4.6. Les mesures de responsabilisation} \\

\item Faire participer l'élève à des activités en dehors des heures d'enseignement au sein de étab, d'assoc agrée ou coll Tale. \\

\item Aucune tâche dangereuse ou humiliante.\\

\textit{4.7. L'exclusion temporaire de l'élève} \\

\item durée maximum 8 jours. Prononcée / chef d'étab.  Exclusion de étab ou de classe avec présence obligatoire ds étab.\\

\textbf{5. Instances impliquées} \\

\textit{5.1. La commission éducative} \\

\item obligatoire ds tt étab. Représentant de ttes les composantes de communautés scolaire. But : examiner situation d'élèves en rupture ac règles de vie de étab (qd exclusion prononcée). \\

\item Transmet 1 avissur éventuelles suites disciplinaires ou mesures éducative à pde au chef d'étab. En dernier ressort, à pvr de décision.\\

\textit{5.2. Le conseil de discipline} \\

\item Réunion à demande du chef d'étab ds cas d'1 exclusion définitive. Statue sur cette exclusion.\\

\item Tte sanction prononcée peut être référée ds délai de 8 jrs, au recteur d'académie / représentant légal du mineur ou l'élève (si majeur). Recteur décide, après avis d'une commission académique.\\

\textit{5.3. Attributions du chef d'établissement} \\

\item Seul a pvr engager sanction disciplinaires. Prononce tte : de avertissement à exclusion temporaire de classe ou de étab. \\

\item A titre conservatoire, peut interdire accès à étab à 1 élève devant passer en conseil de discipline.\\

\textbf{6. Les plans récents de prévention anti-violence en France} \\

\item dep 20 ans, 9 plans de lutte vs violence élaborés / ministres de Éducation nationale successifs.
\begin{enumerate}
\item mai 92 : plan Lang. Création de 300 postes administratifs + recours à 2000 profs + partenariat EN-police-justice.\\
\item mars 95 : Bayrou. réduction taille étab, fonds d'assurance pr enseignants, postes de médiateurs, n° spécial << SOS violence >>, éducation civique renforcée.\\
\item mars 96 : 2nd volet.  \\
\begin{enumerate}
\item 3 orientation
\begin{enumerate}
\item renforcement de encadrement \\
\item relations élèves-parents \\
\item étab et environnement. \\
\end{enumerate}

\item Recourt à 1200 profs + création neaux postes de personnels de santé + création des classes relais pr élèves en difficultés.\\
\end{enumerate}

\item nov 97. Phase 1 plan Allègre. Moyen sup ds 10 sites sur 6 académies. 98 et 99 : ces étabs ont 485 emplois d'infirmières et assistantes sociales, 100 postes de médecins scol, 400 emplois ATOS, 100 de CPE, 4728 aides-éducateurs.\\

\item + 3 mesures :  \\
\begin{enumerate}
\item aggravation des sanctions pénales pr fait de violences ds étab.
\item signature ds 14 départements d'1 convention ac Institut national d'aide aux victimes et de médiation
\item programme de partition des plus gros collèges.\\
\end{enumerate}

\item janv 2000 : 2nde phase plan Allègre. Création 5 zones d'expérimentation + moyens supp. \\

\item oct 2000 : Lang met en place Comité national de lutte vs violence à l'école. Instaure logiciel SIGNA (recenser actes violents). Diffuse vademecum pr gérer situation de violence. \\

\item mai 2009 : plan Darcos. création d' <<équipes mobiles de sécurité >> ap que Sarko ait plaidé pr << sanctuarisation >> des étab en réactions aux incidents.\\

\item rentrée 2009 :  <<plan de sécurisation >> ac << diasgnostics de sécurité >> pouvant aboutir à installation de clôture et système de vidéosurveillance + plan de formation à gestion de crise et à exercice de autorité. doit toucher 14.000 personnes.\\
\end{enumerate}

\item Actions permettent de \textbf{stabiliser situation globale}. Eviter explosion faits de violence.\\
\begin{enumerate}
\item ms pas éradiquer. alternances pol -> succession de logiques différentes en peu de temps : pol de prévention et réduction effectifs -> répression, déploiement dispositifs policiers et surveillance.\\
\end{enumerate}

\item Déploie éducateurs à gde échelle, ou portiques et système de vidéo... Donne usagers de école impression d'un pilotage pas propre.
 \item peut pas prétendre sanctuariser étab sans traiter prob + gal de exclusion sociale ds ces quartiers périurbains. \\
 
\fbox{
\begin{minipage}{19cm}
Eric DEBARBIEUX, << Violence scolaire : << Je suis pessimiste >> nous dit E. Debarbieux >>, 2012, site du Café pédagogique.\\

\begin{itemize}
\item  Eric Debarbieux mentionne série de points à approfondir : cohérence des sanctions, prise en compte de prévention des violences dans formation des enseignants, qualité des relations entre adultes ds étab et entre enseignants-parents d'élèves, importance trop grande donnée à transmission des savoirs sur la socialisation des jeunes.\\
\end{itemize}

\begin{enumerate}
\item violence scol a plusieurs causes : situation éco, familiale, facteurs liés à institution scol. Forte correlation qualité du climat scol (qualité relation adultes-élèves et entre adultes : capacité à avoir un dialogue avec élèves et pas un affrontement, clarté des règles collectives) et victimisation. Sentiment d'appartenance collective et justice st 2 composante essentielle de ce climat.\\
\item Ne forme pas en France, enseignants à gestion des punitions. Se retranche derrière CPE. S'intéresse plus à transmission du savoir. Oublie importance de identification aux adultes.\\
\item  Tte ne s'explique pas / climat scolaire. Pays qui s'en sortent le mieux face à violence scol st ceux où place des parents est la plus forte.
\begin{enumerate}
\item Ex: de Rio. Violence endémique ds certains quartier, mais ne rentre pas à école car protégé / communauté. 
\item En France, voit parents comme ennemis. Impératif de travailler ensemble face à violence. Très démunis face à elle.
\end{enumerate}
\end{enumerate}

\end{minipage}
}


\end{itemize}

\end{document}