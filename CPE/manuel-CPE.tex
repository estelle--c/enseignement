\documentclass[12pt]{report}
\usepackage{fontspec}
\usepackage{xltxtra}
\setmainfont[Mapping=tex-text]{Century Schoolbook L}
 \usepackage[francais]{babel}
 \usepackage{hyperref}
 \usepackage{geometry}
 \geometry{ hmargin=0.5cm, vmargin=0.5cm } 

\makeatletter
%\renewcommand\section{\@startsection
%{section}{1}{0mm}    
%{\baselineskip}
%{0.5\baselineskip}
%{\normalfont\normalsize\textbf}}
%\makeatother



\begin{document}

\tableofcontents

\part{Principes et valeurs de l'éducation}

\chapter{Une école juste, inclusive et équitable pour tous}

\textbf{Mots clés : }
\begin{itemize}
\item égalité et démocratie
\item notion de mérite / méritocratie
\item << égalité des chances >>
%\end{itemize}

\vspace{0.5cm}

\textbf{Objectif :}


\begin{enumerate}
\item connaître concepts philosophiques en démocratie : égalité, équité
\item compde conditions historiques et sociologique d'émergence des modèles juridiques de égalité
\item identifier enjeux des questions actuelles et nécessité d'1 approche pragmatique de action éducative.
\item comprendre rôle de école ds dvpt individuel \\
\end{enumerate}

%\begin{itemize}
\item Historiquement, sorte d'équivalence entre justice et égalité. Longue liste d'égalité des droits (DDHC, droit à éducation, logement, chances...)\\

\item dep 80's, égalité des chances revient régulièrement ds bouche hommes politiques. \\

\textbf{1. Démocratie - Egalité d'éducation}
 
\textit{1.1 L'équivalence entre égalité et démocratie trouve son origine ds pensée philo du 18e.} \\

\item philosophes du Contrat (Samuel von Pufendorf, Christian Wolff puis Rousseau), \textbf{égalité = droit naturel}. Exister à état de nature avt que hommes décident de coopérer.\\

\fbox{
\begin{minipage}{19cm}
\textbf{L'égalité ds état de nature et sa fin} \\

ROUSSEAU, Jean-Jacques, \textit{Discours sur l'origine et les fondements de l'inégalité parmi les hommes}, seconde partie, 1755. \\

\begin{enumerate}
\item \textit{ds ce tps précédent le tps historique, hommes dépendaient pas 1 des autres. Assurer subsistance et satisfaction besoins. Ensuite à succéder tps de guerre : forces antagonistes, conflits entre individus et groupes. Perte de égalité naturelle.}\\

\item << Tant que hommes se contentent de cabanes rustiques, se bornent à coudre habits de peaux, perfectionner leurs arcs... Tant qu'ils s'appliquèrent qu'à des ouvrages qu'un seul pouvait faire [...] ils vécurent libres, sains, bons, heureux. [...] Mais dès instant qu'un homme eut besoin du secours d'un autre; dès qu'on s'aperçut qu'il était utile à un seul d'avoir des provisions pour deux, l'égalité disparut, la propriété s'introduisit, le travail devint nécessaire, et les vastes forêts se changèrent en des campagnes riantes qu'il fallut arroser de la sueur des hommes. >> \\
\end{enumerate}
\end{minipage}
}

\vspace{0.5cm}

\item Rousseau présente 1 contrat qui veut pas revenir à état de nature ms vu comme correction des inégalités issues des rapports de force / 1 pacte social.\\ 
\begin{enumerate}
\item ce neau droit = base de intérêt général (pas conçu comme utilité collective d'1 ensemble de personnes, ms comme expression de volonté de souveraineté des citoyens reconnus égaux en droits) \\
\end{enumerate}

\item Rousseau indique pas si citoyens en droit = égalité d'éducation. Diderot est le 1er qui applique égalité des citoyens en droit à éducation. \\

\fbox{
\begin{minipage}{18cm}
\textbf{L'égalité d'éducation contribue à l'intérêt général} \\

DIDEROT, DENIS, \textit{Essai sur les études en Russie}, 1775-1776 \\

\begin{enumerate}
\item \textit{compose à demande de Catherine 2, plan d'éducation publique.} \\

\item << Le nb de chaumières et des autres édifices particuliers étant à celui des palais dans le rapport de 10.000 à 1, il y a 10.000 à parier contre 1 que le génie, les talents et la vertu sortiront plutôt d'une chaumière que d'un palais. [...] \\

\item  <<université est une école dt la porte est ouverte indistinctement à ts les enfants d'une nation. [...] Ces basses écoles sont pour le peuple en général, parce que, depuis le Premier ministre jusqu'au dernier paysan, il est bon que chacun sache lire, écrire et compter. >> \\
\end{enumerate}
\end{minipage}
}

\vspace{0.5cm}

\item Diderot estime que Etat a intérêt à faire sortir de ombre ts les talents cachés pr que société tire profit de ttes les intelligences.
\begin{enumerate}
\item 1ère raison : utilité collective : estime que autt de chance de rencontrer talents chez laboureurs que ministres. \\
\item 2nde raison: argument moral. << cruel de condamner à ignorance conditions subalternes de société >>. Idée de justice. Inspire projet de Encyclopédie.\\
\item 3e raison : idée de diffusion des savoirs : moyen de fonder 1 société + vertueuse faite d'hommes libres. Idée d'émancipation des chaînes de l'oppression sociale / l'instruction (plus difficile de malmener qq'1 qui sait lire et écrire). \\

\end{enumerate}

\textbf{1.2. Une égalité de base pour produire des inégalités sociales ?}

\item cpdt, ds sa double finalité d'efficacité et de justice, éducation vise 1 but opposé à égalité : créer de nlles inégalités liées aux différences de capacités et de <<réussite des élèves >>.


\fbox{
\begin{minipage}{19cm}
BOLTANSKI Luc, THÉVENOT Laurent, \textit{De la justification : les économies de la grandeur}, 1991. \\

\begin{enumerate}
\item caractérisent cette articulation (égalité dvt éducation ms inégalités des conditions sociales crée / éducation). Identifient 2 pcpes établis / RF : 
\begin{enumerate}
\item Le principe de commune humanité.  ts appartiennent à même humanité :  fonde égalité des citoyens en droits (en dépit des différences de milieux de naissance...) Individus st divers. Ms si érige leur différences en inégalité -> plus démocratie.\\
\item  pcpe d'ordre. Etablit une hiérarchie entre individus, ms fondée sur la justice. rendu nécessaire / division du travail et efficacité économique pr le bien commun. 2 conditions pr qu'il rentre pas en conflit ac pcpe de commune humanité : 
\begin{enumerate}
\item existe 1 pcpe supérieur commun susceptible de mesure grandeur des personnes\\
\item différences entre individus doivent être provisoires et pvr être << rejouées >> périodiquement. si différence définitives, certains individus seraient indéniablement supérieur : reviendrait au monde de violence. Plus de commune humanité.
\end{enumerate}
\end{enumerate}
\end{enumerate}
\end{minipage}
}

\vspace{0.5cm}

\item Sociologue Derouet a appliqué cette analyse à éducation. \\

\fbox{
\begin{minipage}{19cm}

\textbf{Comment l'égalité initiale dvt l'instruction démocratique peut-elle aboutir à créer des distinctions entre les personnes ?} \\

DEROUET Jean-Louis, \textit{Ecole et justice, De l'égalité des chances aux compromis locaux ?}, 1992, p. 82-83. \\

\begin{enumerate}
\item \textit{pr mettre en pratique ces 2 pcpes sans qu'ils s'opposent, éducation a pr 1er but de" \textbf{créer sentiment d'appartenance de chacun à commune humanité}, à égalité de dignité et de droits.} C'est pourquoi Educ Nat offre a chacun 1 bagage commun de lumière qui fait d'1 peuple 1 nation.\\
\item Nécessairement 1 moment où on passe d'1 école pr tous à des formes diversifiées d'éducation selon différence entre élèves : << principe d'ordre >>. Ms <<meilleurs>> pas désignés / naissance ni fortune.  \\
\item << A partir d'1 certain moment, il est nécessaire de passer d'une école pour tous à une école pour les meilleurs, ceux qui méritent la promotion sociale. [...] L'important est qu'il crée une distinction et que cette distinction soit justifiée [...] Une organisation scolaire doit donc satisfaire à deux conditions : d'une part rendre l'école pour tous réellement accessible à tous, en compensant les inégalités d'implantation géographiques, de fortunes, qui peuvent éloigner certains enfants de l'école; d'autre part, faire accord sur le pcpe de sélection qui permet de passer de l'école pr ts à l'école pr les meilleurs. >>
\end{enumerate}

\end{minipage}
}


\vspace{0.5cm}

\item C'est à l'école de choisir (et pas de recevoir) les individus qui peuvent se montrer les plus efficace pr les professions futures. Ms pr que école puisse jouer ce rôle, doit être 1 affaire d'Etat sous contrôle de svraineté populaire.

\textbf{2. Ethique et égalité} \\

\textit{2.1. De l'égalité formelle d'éducation à l'égalité des chances.} \\

\item Pcpe du 18e a replacer dans contexte : veulent passer de Ancien Régime (ordre où position sociale définie / naissance ds castes héréditaires) à société où positions sociales acquises / mérite individuel.\\
\begin{enumerate}
\item semblait nécessaire de soumettre futurs citoyens aux mêmes conditions et contenus d'enseignement. Ms RF a pas eu le temps de mettre en palce cette égalité. \\
\item 2nde moitié 20e : égalité formelle des individus dvt offre scolaire mise en place. Prob d'équité revient au moment de époque de massification des études secondaires.\\
\end{enumerate}

\item Pr Rodrigo Roco Fossa, égalité d'accès à école + égalité de traitement de ts st devenu 1 << droit de base >> seulement destiné à légitimer compétition des individus en son sein.\\

\fbox{
\begin{minipage}{19cm}
ROCO FOSSA Rodrigo, \textit{De l'égalité en éducation} in DROUIN-HANS Anne-Marie, \textit{La Philosophie saisie par l'éducation}, 2005, p. 38-39.

\begin{enumerate}
\item << égalité d'accès à école = droit de base nécessaire au bon fonctionnement de la structure éco et sociale. L'égalité de traitement (accès aux même contenus, procédures, épreuves...) serait surtt le moyen de parvenir à extraire de la société les meilleurs éléments, qui auront le droit légitime d'accéder aux meilleurs positions.>> \\
\item  << idéologie méritocratique = exige que les positions des individus par rapport au pouvoir, aux richessesn aux statuts, au savoir approfondi, etc, se méritent. D'abord à l'école, et grâce à elle, à la vie en général. >>.
\end{enumerate}
\end{minipage}
}

\vspace{0.5cm}

\textit{2.2. L'égalité des chance accusée d'iniquité} \\

\item dep Waux des années 60's (sociologie critique : Pierre Bourdieu, JC Passeron) : procès d'iniquité fait à << idéologie méritocratique >>.
\begin{enumerate}
\item école démocratique <<égale pr ts >> fondée sur récompense du << mérite >>.  Or elle produit inégalités sociales à la sortie. MAIS en plus, elle reproduit inégalités sociales des enfants à entrée et les justifie en rejetant sur les usagers de l'école la responsabilité de leurs échecs.\\
\end{enumerate}

\item 80's : nel objectif assigné à l'école : \textbf{l'égalité des chances.} Neutralisation de ttes les inégalités liée à la naissance. Ms égalité des chances commande inégalité des résultats. Mérite = mesure de ce qui serait dû à chacun.\\

\underline{2.2.1 Que signifie le mérite ?} \\

\item + svt : élève << méritant >> : celui qui obtient les meilleurs résultats et pas celui qui fait des efforts pr réussir. Notion de mérite recouvre celle de compétence : seule récompensée ds course aux qualifications. pourtt, celui qui fait plus d'éffort n'est-il pas plus méritant que celui qui réussit sans en fournir ?\\

\item celui qui a la force de ne pas céder à la fatique n'est-il pas simplement gâté / la chance ? A-t-il plus de mérite que celui qui est obligé de se faire violence, qui est rempli de paresse et qui n'y parvient pas tjs ? \\

\underline{2.2.2. Que signifie la récompense du mérite ?} \\

\item Suivre une formation artisanal, est-ce échouer ? Faire une licence d'histoire, est-ce la récompense d'un mérite supérieur ? 

\underline{2.2.3. L'égalité des chances n'est pas vraiment atteinte}\\

\item Le système perpétue la reproduction des privilèges. l'égalité des chances devrait renversé complètement l'ordre social et établir l'égalité de toutes les conditions sociales. C'est pourtant pas sa finalité. Au contraire.

\fbox{
\begin{minipage}{19cm}
\textbf{L'égalité des chances : une utopie don personne ne souhaite l'accomplissement ?} \\

DROUIN-HANS Anne-Marie, << L'égalité des chances, une idée bouleversante ? >>, \textit{La Philosophie saisie par l'éducation}, p. 79 \\

\begin{enumerate}
\item << Vouloir une égalité des chances dans une société qui est fondée sur l'inégalité [...] ? On fait tout pour rendre effective l'égalité en souhaitant très fort qu'elle ne se réalise pas sous peine de devoir révolutionner les organisations sociales et idéologiques. >>\\
\item a un but et se désole qu'il soit si difficile à atteindre. \\
\end{enumerate}
\end{minipage}
}

\textbf{3. Vers une conception pragmatique de l'égalité}\\

\item débat philosophique aboutit à discréditer l'égalité des chances, les analyses sociologiques y voient 1 tromperie pr les catégories défavorisées et 1 moyen de conforter la position dominante des classes les plus favorisées. \\

\item 1 approche + pragmatique : considérer égalité des chances pas comme absolu à atteindre ms comme appellation générique de ttes les politiques qui s'efforcent de réduire les inégalités dvt l'école.\\

\item La réalité est que : << massification >> des études n'a cessé de progresser depuis 1/2 siècle. atteint aujourd'hui enseignement sup.


\vspace{0.5cm}

\fbox{
\begin{minipage}{19cm}
\textbf{La massification des études durant période 1950-2000} \\

\begin{tabular}{|c|c|c|c|p{2cm}|}
\hline  & 1950 & 1965 & Rentrée 2000 &  \\ 
\hline secondaire & 770.000 & 2,4 M & 5,5 M (dt 3,2 M en collège) & prolongement de scolarité. stable dep 97 \\ 
\hline Apprentissage (dt CFA) & ? & ? & 400.000 & en hausse dep rentrée 2000 \\ 
\hline Total (maternelle-univ) & 3,3 M (24\%) & 7,7M (45\%) & 14,3 M (93\%) & en \% des 2/22 ans \\ 
\hline 
\end{tabular} 
\end{minipage}
}

\item Institution scolaire pd en charge 90\% de pop française. Corrélativement, son pvr de qualification et d'insertion professionnelle n'a jamais été aussi grand.\\

\fbox{
\begin{minipage}{19cm}
\textbf{L'égalité des chances est une fiction nécessaire} \\

DUBET François, \textit{L'école des chances}, 2004, p. 51-52 \\

\begin{enumerate}
\item <<admet que égalité des chances et recherche du mérite sont des fictions nécessaire, ie qu'elles sont à la fois désirables et inévitables ds une société démocratique tenue d'articuler l'égalité des indiv à l'inégalité des positions sociales, il faut tt faire pr s'en approcher. [...] Une certaine hypocrisie face à la ségrégation scolaire, aux coûts et aux bénéfices privés des études peut être levée. Il faut probablement dvper 1 politique de discrimination positive ciblée sur indiv autant que sur étab fragiles. Dans la mesure où la mobilisation des élèves et celle des parents st indispensables à la réussite, l'information et la capacité de circuler doivent être dvpée, rompant ainsi avec 1 fausse image du sanctuaire scolaire. >>
\end{enumerate}
\end{minipage}
}

\vspace{0.5cm}

\item Elèves, usagers de école, dt des êtres sensibles et de futurs citoyens qui méritent, quels qu'ils soient et d'où qu'ils viennent, la prise en considération de leur destin. L'heure des compromis locaux pd le pas sur les grands modèles.

\end{itemize}

\chapter{Gratuité - obligation}

\textbf{Mots clés : }
\begin{itemize}
\item gratuité scolaire
\item Obligation
\item lutte vs absentéisme scolaire
\item décrochage scolaire
\end{itemize}

\vspace{0.5cm}

\textbf{Objectif :}


\begin{enumerate}
\item compde origine et sens des concepts de gratuité et d'obligation
\item identifier valeurs civiques et sociale qu'impliquent ces deux termes
\item connaître institutions, dispositifs, instructions relatifs à gratuité et à obligation.
\item analyser rôle du CPE ds étab ds prévention des situations de rupture de l'obligation (absentéisme et décrochage)\\
\end{enumerate}

\begin{itemize}
\item Obligation d'instruction en débat dès RF. D'abord pensée comme mesure nécessaire à formation du citoyen.\\

\item Projet de CONDORCET : chq enfant émancipé  des castes héréditaires de AR. doit faire de lui 1 citoyen capable d'exercer librement ces droits -> faut donc éclairer son esprit. Pd soin de distinguer éducation (relève des familles) et instruction (école). \\

\item Sous la Terreur : LEPELETIER de SAINT-FARGEAU. prévoit monopole de Etat. Obligation de enseignement et de internat ds des maisons d'éducation commune. \\

\item malentendu datant de cette époque : obligation d'enseignement est-elle compatible avec liberté d'éducation ds famille ? A dû sans cesse réaffirmer que obligation portait sur instruction et pas sur éducation. \\

\item Distinction instruction-éducation va pas tjs de soi aujourd'hui. Débat relancé sur dvpt des << éducations >> qui incluent des savoirs-êtres : éducation à environnement, à santé, à citoyenneté, au dvpt durable... \\
\begin{enumerate}
\item éducation citoyenne d'aujourd'hui ne paraît pas compatible ac obligation limitée à instruction (ie savoirs). pensent ++ que valeurs font partie de obligation d'enseignement.\\
\end{enumerate}
 
 \item Gratuité liée à obligation. Comment imposer instruction à ceux qui n'ont pas moyens de la donner à leurs enfants ? Déjà discuté sous loi Guizot (obligation aux communes d'entretenir une école primaire de garçon). Posée pdt EDG à propos de école unique.
 \begin{enumerate}
 \item Aujourd'hui, gratuité se pose encore : classes prépa. couteuses. fréquenté essentiellement / enfants de milieux socialement favorisées, doivent-elle resté gratuites ? La gratuité est-elle le meilleur instrument de l'équité ?\\
 \end{enumerate}

\textbf{1. La gratuité scolaire} \\

\textit{1.1 Les textes.} \\

\fbox{
\begin{minipage}{19cm}
\textbf{La gratuité dans les textes en vigeur en 2012}\\

\textit{Code de l'éducation, Art. L132-1 et 132-2.} \\

<< L'enseignement public dispensé dans les écoles maternelles et les classes enfantines et pendant la période d'obligation scolaire est gratuit. \\

L'enseignement est gratuit pr les élèves de collège et lycée publics qui donnent l'enseignement du 2nd degré ainsi que pr les élèves des classes prépa aux gdes écoles et à l'enseignement supérieur des étab d'enseignement public du second degré. >>
\end{minipage}
}

\vspace{0.5cm}

\item pcpe de gratuité de enseignement primaire public : loi Jules Ferry du 16 juin 1881. Etendue au secondaire / loi du 31 mai 1933.\\

\item manuels scolaires fournis aux élèves pdt scolarité obligatoire. Pris en charges / communes pr école primaire et / Etat pr collège. Etab (ac dotation de Etat) les achète et les prête chq année aux élèves. \\

\item ds lycées, depuis élections aux conseils régionaux de 2004, quasi-totalité des régions a pris décisions pr assurer gratuité pr élèves.\\

\fbox{
\begin{minipage}{19cm}
\textbf{La mise en oeuvre du pcpe de gratuité de enseignement scolaire public.} \\

\textit{Circulaire du 30 mars 2001} \\

<< pcpe de gratuité, doit être considéré de manière absolue. Concerne matériel d'enseignement à usage collectif, fournitures à caractères administratifs et dépense de fonctionnement (dont production de photocopies à destination des élèves et de leurs familles, frais de correspondance adressée aux familles, frais de téléphone). En revanche, dépenses afférentes aux activités facultatives (dt voyages scolaires) relèvent pas de ce principe. Peuvent être laissées à charge des familles. >>
\end{minipage}
}

\vspace{0.5cm}

\item si prob (frais de cantine ou participation financière à un voyage), chq étab dotée d'une somme mise à disposition des élèves : fonds social collégien et lycéen. Alloué /  chef d'étab.\\

\item Fond de vie lycéenne : destinée à financer actions en matière de formation des élus lycéens, info des élèves, communication, prévention des conduites à risques, d'éducation à la santé et citoyenneté, animations culturelles ou éducatives.\\

\item CAF verse à chq rentrée, 1 alloc dt somme fixée / Etat. \\

\item Conseil général pr apporter aide pr cas particuliers \\

\item Bourses versées aux familles des élèves de collèges et lycées. sur base des revenus familiaux, barème comportant points de charge (nb d'enfant, situation familiale ...) \\

\textbf{2. La dépense intérieure d'éducation} \\

\item 2008 : 129,4 milliards d'euros. 6,6 \% du PIB. Part de Etat baisse dep 85 et lois de décentralisation. \\

\item parmi dépenses éducatives : soutien scolaire. 75 \% des parents (selon sondage de 2005) sont prêt à recourir au soutien scolaire si enfant à difficulté. nbeuses collectivités Tales subventionnent assoc qui font de << accompagnement à la scolarité >>. \\

\textbf{3. L'obligation scolaire} \\

\textit{3.1. Définition} \\

\fbox{
\begin{minipage}{19cm}
\textbf{L'obligation scolaire ds les textes en vigueur, 2012} \\

\textit{Code de l'éducation \\}

Art. L131-1. << L'instruction est obligatoire pr les enfants des deux sexes, français et étrangers, entre 6 et 16 ans. La présente disposition ne fait pas obstacle à l'application des prescriptions particulière imposant une scolarité + longue. >> \\

Art. L122-3. << Tout jeune doit se voir offrir, avant sa sortie du système éducatif et quelque soit le niveau d'enseignement qu'il a atteint, une formation professionnelle. >> \\

Art. L122-2. << Tt élève qui, à l'issue de la scolarité obligatoire, n'a pas atteint 1 niveau de formation reconnu doit pvr poursuivre des études afin d'atteindre 1 tel niveau. L'Etat prévoit les moyens nécessaires, ds l'exercice de ses compétences, à la prolongation de scolarité qui en découle.\\
Tt mineur non émancpé dispose du droit de poursuivre sa scolarité au-delà de l'âge de 16 ans. \\
Lorsque les personnes responsables d'1 mineur non émancipé s'opposent à la poursuite de sa scolarité au-delà de 16 ans, 1 mesure d'assistance éducative peut être ordonnée dans les conditions prévues aux articles 375 et suivants du code civil afin de garantir le droit de l'enfant à l'éducation. >> \\
\end{minipage}
}

\vspace{0.5cm}

\item Depuis lois de Jules Ferry (1881-1882), enseignement obligatoire. à partir de 6 ans. Dep janvier 59, jusqu'à 16 ans révolus. \\

\item Loi du 18 décembre 1998 : renforce dispositions concernant obligation scolaire pr lutte vs dérives. \\

\item décret du 23 mars 99 : précise contenu des connaissances requis pr enfants instruits ds famille ou dans étab privés hors contrat. Actualisé en 2007 (obligation de posséder connaissance et compétence du socle.) \\

\textit{3.2. La lutte contre l'absentéisme scolaire.} \\

\item Touche en moyenne 5\% des collèges et lycées professionnels. 10 \% des lycées pro ont tx d'absentéisme atteint 10 à 16 \% des élèves. Fléau à combattre. \\

\item Ordonnance de janv 59 : sanction parents fautifs de laisser enfants manquer les cours.
\begin{enumerate}
\item pr année scol 2001-2002 : 2900 familles touchées.
\end{enumerate}


\fbox{
\begin{minipage}{19cm}
\textbf{La mise en oeuvre du pcpe de l'obligation d'assiduité pr les élèves inscrits ds étab scol.} \\

\textit{Loi du 2 janvier 2004 et le décret du 19 février 2004, BO << obligation scolaire : contrôle de l'assiduité scolaire} \\

<< Le chef d'étab pd contact ac parents de élève qui pas régulièrement présent. [...] Si dialogue inefficace, dossier transmis à inspecteur d'académie qui peut invité famille à suivre 1 module de soutien à la responsabilité parentale; si, en dépit de ensemb des mesures, assiduité de l'élève pas restaurée, proc de Rep pourra être saisi. >> \\
\end{minipage}
}

\vspace{0.5cm}

\item amende de 750 euros prévue pr gros cas. Peut aller (selon juge) jusqu'à 30.000 euros et 2 ans de prison. \\

\item 2 outils complètent : 
\begin{enumerate}
\item 1 commission de suivi de assiduité scolaire. Ds chaque département, sous autorité du préfet. Ts partenaires présents. \\
\item 1 dispositif de veille éducative ds sites prioritaires de la politique de la ville, sous responsabilité du maire. Mobiliser intervenants éducatifs et sociaux pr repérer jeunes en rupture. \\
\end{enumerate}

\item Loi en septembre 2010, loi Eric CIOTTI : si total d'absences au moins 4 demi-journée sur 1 mois constaté, directeur de CAF peut suspendre versement de alloc dû pr enfant. Après au moins 1 mois de présence de enfant, versement peut être rétabli.
\begin{enumerate}
\item texte dénoncé / observateurs de éducation. Exemple anglais (répression encore + forte) montre que pas utile, change pas phénomène. Pas forcément << démission parentale >> qui entraîne absentéisme. C'est pas en affaiblissant famille qu'on y remédie.\\ 
\item janv 2011 : Conseil sup de l'éducation a voté à unanimité contre circulation d'application de loi Ciotti = ts personnels de éduc : enseignants, parents, partenaires sociaux, coll Tale. \\
\end{enumerate}

\item Ap élection présidentielle 2012, débat relancé. Chiffres pas convaincants. 17 janvier 2013 : abrogée / Parlement. Nlle approche de lutte vs absentéisme avancée.

\item Nlle loi d'orientation en juillet 2013 : prévoit rencontre avec la famille + nomination d'1 personnel d'éducation référent pr suivre l'élève absentéiste. A suite des préconisation du Conseil européen de 2011 : améliorer offre pédagogique et prévoir intervention individuelle pr lutter vs abandon scolaire. \\

\textbf{4. Le décrochage scolaire} \\

\textit{4.1. Définition} \\

\item \textbf{Accumulation des difficultés d'apprentissage d'un élève qui aboutit à 1 point de rupture pvt conduire à absentéisme, à violence ou à déscolarisation.} \\

\item processus long, commencé en primaire. Retard d'acquisition. Difficultés s'accumulent. Décrochage survient pas d'un coup ms résulte d'un processus qui se construit au cours de scolarité.\\

\fbox{
\begin{minipage}{19cm}
\textbf{Les difficultés cognitives précèdent le décrochage scolaire.} \\

BONNERY Stéphane, << Décrochage cognitif et décrochage scolaire >>, GLASMAN Dominique et OEUVRARD Françoise [dir], \textit{La Descolarisation}, 2011, p. 149-150. \\

\begin{enumerate}
\item d'abord difficulté d'acquisition des savoirs à école primaire. Ces élèves pensent pourtt l'inverse d'eux-même : persuadés que aucun prob en primaire, que ça provient du collège. comprendre la construction du décrochage scolaire et cognitif nécessite de dépasser ces conceptions.
\end{enumerate}
\end{minipage}
}

\vspace{0.5cm}

\item Pense qu'il est possible de prévenir le décrochage. Rôle du CPE =  interface entre élève, enseignant et parents. crucial. \\

\textit{4.2. La lutte contre décrochage scolaire.}

\item 4 piliers : 
\begin{enumerate}
\item individualisation. Importt = comprendre pourquoi élève décroche. Critères sociaux. \\ 
\item Importance du lien avec famille. Restaurer ou maintenir ce lien. favoriser nouveau mode de coopération avec famille. \\
\item effort pédagogique. Individualisation des parcours pr décrocheurs : se demande si dispositif d'enseignement de droit commun est suffisant. \\
\item partenariat entre différents acteurs.
 
\end{enumerate}


\item 60.000 jeunes sortent sans qualification + 60.000 jeunes échouent ds système. \\
\begin{enumerate}
\item pr bcp en lycée pro et ont pas obtenu la filière qu'ils veulent. \\
\item selon études socio, appartiennent majoritairement aux classes sociales défavorisées. celles dt habitus primaire trop éloigné du secondaire.\\
\end{enumerate}

\underline{4.2.1. L'action éducative et pédagogique au sein de l'établissement scolaire} \\

\item CPE à écoute des élèves : voit ts les jours.  reçoit élèves en retard le matin, observe comportement d'isolement ou d'agression pdt interclasses, siège aux conseils de classe. 
\begin{enumerate}
\item 1ère mission : \textbf{dépistage des élèves décrocheurs}. \\
\item 2nd tps : doit \textbf{alerter les enseignants}, recueillir constats + \textbf{impulser projets éducatifs} au bénéfice des élèves en situation de décrochage. \\
\end{enumerate}

\item \textbf{doit tenter de réconcilier élève décrocheur avec école.} : redonnant du sens à son projet personnel. Entretien avec famille. \\

\item  dernier ressort : CPE peut mobiliser partenaires internes (COP, médecin scol) et extérieur (services sociaux). \\

\underline{4.2.2 Les dispositifs-relais} \\

\item But : remotiver jeune pr qu'il reprenne sa scolarité, pr obtenir diplôme professionnel. Atelier-relais organisés / mvts d'éducation pop pr combattre des tentations d'abandon scolaire / activités sportives et culturelles. \\

\item Création de certains étab pr lutter vs décrochage scolaire : micro-lycée de Sénart, collège << élitaire pour tous >> de Gre, << lycée innovant >> Jean Lurçat de Paris. \\

\underline{4.2.3. Les écoles de la dernière chance.} \\

\item 10aine en France. Jeunes entre 18 et 22 ans ayant pas de diplôme. Formation en alternance de 2 ans ac pédagogie individualisée pr acquérir savoirs fondamentaux indispensables. Contrat entre ts les personnes. \\

\underline{4.2.4. Le dispositif << Défense, 2e chance >>} \\

\item 2006 : 7 centres. 2012 : 18 centres. Objectif : 50 centres pr 20.000 jeunes.\\

\item de 18 à 21 ans. Aucune qualifications. Double formation : remise à niveau / personnels de Educ Nat (dt apprentissage respect et autorité / militaires) + formation professionnelle ds métiers où offre d'emploi insatisfaites (bâtiment, restauration, service à la personne...). Formation en internats, porte uniforme.\\

\underline{4.2.5 Les établissement de réinsertion scolaire.} \\

\item Accueillir élèves perturbateurs qui ne relèvent ni d'une prise en charge thérapeutique, ni d'1 placement ds cadre pénal.\\

\item encadrement : enseignant, assistants d'éducations, personnels de la PJJ (projection judiciaire de la jeunesse). \\

\item ds cadre de prévention du décrochage, accent mis sur sanctions disciplinaires (circulaire 1er août 2011, BO 25 août 2011) : sur \textbf{mesures alternatives à exclusion temporaire}, comme la << mesure de responsabilisation >>.\\
\begin{enumerate}
\item elle consiste à participer, de dehors des heures d'enseignement, à des activités de solidarité, culturelle ou éducative.\\
\item peuvent être réalisé ds étab ou ds cadre d'1 assoc agrée, d'1 collectivité Tale, d'1 groupement rassemblant personnes pub ou d'1 administration de l'Etat.\\
\end{enumerate}

\textbf{Conclusion : obligation, gratuité, droit-créance et lien social} \\

\item compliqué aujourd'hui de penser école démocratique sans l'obligation et la gratuité.
\begin{enumerate}
\item  si école obligatoire, plus comme au temps de Ferry (but : apprendre à école communale petit bagage nécessaire pr entrer ds 1 vie active déterminer / milieu d'origine.) \\
\item ac SCCC (socle commun connaissance compétence), obligation scolaire devenue 1 obligation de Etat de faire acquérir à 1 génération les instruments de sa liberté (vue comme droit-créance pr tous). \\
\end{enumerate}

\item Ms obligation de s'instruire vaut pas que pr intérêt des personnes ms aussi comme 1 tout social.  Indispensable pr tisser lien social et apprendre à vivre ensemble.

%\end{itemize}

\chapter{Formation du citoyen responsable, 28}


\textbf{Mots clés : } \\
\begin{itemize}
\item notion d'autorité
\item notion de citoyenneté
\item 
\item 
\end{itemize}

\vspace{0.5cm}

\textbf{Objectif :} \\
\begin{enumerate}
\item Identifier les mutations dep 1 siècles ds respect de autorité et dans rapport de soir à la solidarité collective.
\item faire un diagnostic de crise actuelle de autorité en milieu scolaire.
\item concevoir dispositions susceptibles de contribuer à éducation citoyenne des jeunes dans étab scolaire. \\
\end{enumerate}


\section{Autorité, discipline, respect de la loi}

\subsection{Le droit dans les sociétés à la solidarité organique (Durkeim)}

%\begin{itemize}
\item Durkheim décrit \textbf{évolution de exercice de autorité dans société moderne.} Analyses validées / sociologues de l'éducation. \\
\begin{enumerate}
\item sociétés industrielles modernes :  division du travail. Conséquence de division éco du travail (plein de personnes de lieux différents pr faire 1 même produit) = division sociale des citoyens. Donc école et société tte entière doivent se préoccuper de fabriquer du \textbf{lien social} entre ts les membres de la société.
\end{enumerate}

\fbox{
\begin{minipage}{19cm}
\textbf{La division du travail exige la mise en place d'une solidarité organique entre ts les membres de la société} \\

DURKHEIM Emile, De la division du travail sociale, 1893, p. 6 \\

\begin{enumerate}
\item 2 sortes de solidarités positives : 1 mécanique et 1 organique
\end{enumerate}

\end{minipage}

}

\section{Education à la citoyenneté, 28}


\chapter{Laïcité et droit aux différences : la place des identités collectives, 33}

\section{L'application du pcpe de laïcité suivant les territoires et les espaces concernés, 37}

\section{La laïcité au collège et lycée : rappel historique, 39}

\section{Le rôle et les obligations de l'école, 40}

\section{La laïcité au péril de l'individualisme, 48}

\chapter{Le défi des inégalités, 51}

\section{Les inégalités sociales, 51}
\section{Les inégalités de genre, 55}
\section{le handicap, 75}

\part{Enjeux de l'éducation : école et société}

\chapter{La communauté éducative : école et famille}

\textbf{Mots clés : }
\begin{itemize}
\item concept de communauté éducative
\item crise de état et de ville / question de mixité sociale
\item lien social
\item 
\item 
\item 
\end{itemize}

\vspace{0.5cm}

\textbf{Objectif :}


\begin{enumerate}
\item Forger sentiment d'appartenance à étab de ts ses acteurs\\
\item Donner du sens au collectif, au bien commun\\
\item Construire étab scol comme espace laïque de savoirs et citoyenneté.\\
\end{enumerate}

\textbf{1. Forger le sentiment d'appartenance à un collectif large en respectant la diversité culturelle} \\

%\begin{itemize}
\item But : étudier conditions favorables et défavorables à constitution du sentiment d'appartenance des familles, élèves et de ts les personnels à communauté éducative (respectueux des différences).\\


\fbox{
\begin{minipage}{19cm}
\textbf{Concept de communauté éducative}
\begin{enumerate}
\item hérité des écoles du 19e. Remis au jours deb 80's avec concept de projet d'étab. But : mieux associer à sa scolarité ts les acteurs de l'éducation d'un jeune.\\
\item Art. L111-3, Code de éducation <<Ds chq école, collège, lycée, la communauté éducative rassemble les élèves et ts ceux qui, dans l'etab ou en relation avec lui, participent à l'accomplissement de ses mission. Réunit personnels des écoles et étab, parents d'élèves, coll Tales + acteurs institutionnels, éco, sociaux, associés au service public d'éducation. >> \\
\item Mobilisation de ts apparâit comme gage de meilleur qualité et efficacité du système éduc.\\
\end{enumerate}
\end{minipage}
}

\vspace{0.5cm}

\item mutations actuelles société ont csq sur vie, fonctionnement des étab. Interroge rôle de la communauté éducative.\\

\item ds la ville, enfants ne se reconnaîssent pas. svt juxtaposition de quartiers où ségrégation sociale et spatiales. cherchent une culture d'identification en se repliant sur groupes ethniques, religieux, voir sectes.\\

\fbox{
\begin{minipage}{19cm}
\textbf{Accepter ou rejeter l'autre : le phénomène << nimby >>} \\
Le Monde, 29-30 décembre 2002.\\
\begin{enumerate}
\item Implantation équipements pr accueillir pop déshérités (pauvres, toxico, tsiganes) se heurte à riverains. << Pas dans mon jardin >>. Attitude en constante augmentation.
\end{enumerate}
\end{minipage}
}

\vspace{0.5cm}

\item ghettoïsation de certains quartiers : refus de autre. produit violence.\\
\begin{enumerate}
\item Y compris pr quartiers aisés qui refusent centre pr handicapés, structure pr toxico à côté.\\
\end{enumerate}

\item crise de nation, des composantes de l'identité nationale. Jeunes ne se sentent ni d'ici ni d'ailleurs. Prob car France régie / droit du sol. Acculturation défi car pr savoir où on va, il faut savoir d'où on vient et ce que l'on veut. \\

\textbf{2. Les mutations des espaces urbains : crise des espace de citoyenneté.} \\

\item Espace pas divisé entre ville et campagne ms entre multitude d'archipel dans le péri-urbain.\\

\item crise de Etat : interrogation sur son rôle, place des pvrs locaux. Etat semble se délité :  transfert compétences au niveau des régions, dep, communes + abandon de souverainaté au profit d'organisme internationaux (Europe, OMC).\\

\item s'y ajoute \textbf{crise de la ville comme espace de mixité sociale}. ++ archipels de quartiers où ségrégation spatiale = ségrégation sociale.
\begin{enumerate}
\item Durkeim et le concept de lien social. Ensemble des droits et devoirs découle de solidarité entre ts. ca doit régler relations sociales. Pr lui et solidariste, école a pr mission de créer ce lien social et montrant que inégalités sociales st justes si st fondées sur différences de produits du travail et si articulée de façon solidaires.\\
\item Or, déplacement de pop, mutation de habitat, dislocation des familles rendent ++ prob notion de << terroirs >> d'origine.\\
\item ex: 1 jeune personne malienne née en France. 10 ans ap, n'est pas totalement Fr (même si nationalité), ni africaine. svt ds situation instable (rejette, oublie ou fantasme communauté de naissance dt elle maîtrise pas ts les codes). Est << Fr >> en Afrique noire et << Afr >> en France. Ne se sent pas insérée (surtt terme de logements et emploi) ds 1 société fr qui respecte pas tjs règles qu'elle édicte pr acceuil des immigrés. \\
\end{enumerate}

\item \textbf{Risque du repli identitaire.}\\
\begin{enumerate}
\item devant ce << flou >>, jeune peut rechercher 1 culture d'identification qui soit ni celle de famille, ni tradition, ni pays d'accueil. peut se forger ds quartier, ds démarche de retour au religieux...\\
\end{enumerate}

\item Pdt longues années, nation (ds fondements familiaux, idéologiques, religieux) a été stable. Reposait sur certains nb de mythes fondateurs et intégrait les nouveaux arrivés (à travers institutions comme Eglise, syndicalisme, partis politique ou assoc sportive : ex avec Italiens ou Polonais).
\begin{enumerate}
\item Sous 3e Rep, école a renforcé la fonction unifiante de la culture scolaire.\\
\item culture scolaire est devenue 1 construction ++ autonome / rapport aux savoirs historiques : qq fois pure invention destinée à répondre au projet d'unification nationale. \\
\end{enumerate}

\item \textbf{Aujourd'hui, refondation d'1 projet national / école est-elle encore possible ?}
\begin{enumerate}
\item pr certains sociologues (dt Mohamed Charkaoui), partout où T nationale pas habité / sentiment d'appartenir à communauté, Etat-Nation risque de devenir 1 coquille vide qui ne peut plus fondée lien social.
\item affaiblissement du repère national comme terroir de vie et espace de valeurs commune complique fabrication du lien social / << école de la République >>.
\end{enumerate}

\item pr identités, superposition et imbrication des T de référence des élèves à différentes échelles (quartier, région, pays d'origine, pays d'accueil, Europe) requiert de école construction d'une culture commune qui ne soit pas une culture unique.\\


\textbf{3. Malaise social, malaise adolescent}\\

\item Jusque 70's, rythmes journaliers et hebdomadaires de vie était très svt les même pr ttes les professions. Différent aujourd'hui.
\begin{enumerate}
\item émiettement des rythmes de vie car aug. durée de déplacement domicile-travail.\\
\item de - en - d'horaires communs.\\
\end{enumerate}

\fbox{
\begin{minipage}{19cm}
\begin{enumerate}
\item 65 \% parents << zones sensibles >> travaillent matin avt 8h30 et 47 \% ap 19h30.
\begin{flushright}
Source : Libé, 1er août 2009
\end{flushright}
\end{enumerate}
\end{minipage}
}

\vspace{0.5cm}

\item pas surprenant que ado se plaignent que parents soient pas là. Demandes des parents : que institutions publiques puissent accueillir leurs enfants. Csq de ces situations sur les enfants ?
\begin{enumerate}
\item questions doivent mobiliser communauté éducative. Consultation des familles et évaluation des ressources. Transmission ensuite des propositions à équipe de direction.\\
\item \textbf{Rôle du CPE ds équipe de direction pr rythme scolaires}
\begin{enumerate}
\item doit se renseigner sur rythme de la majorité des parents pr décider des horaires de réunion. \\
\item Réponse doit être locale : pas la même dans étab d'1 village à 30 km de Toulouse où plupart des parents travaille et ds étab de banlieue proche de T. à proximité d'une station de métro. \\
\end{enumerate}
\end{enumerate}

\item tps des différents acteurs d'1 territoire n'est pas le même.
\begin{enumerate}
\item tps des familles, tps des élèves, tps de institution scolaire et ses rythmes annuels, tps des élus Taux ou nationaux.\\
\end{enumerate}

\textbf{4. Redéfinir le << vivre-ensemble >>} \\

\item  Au niveau philosophique, si école enseigne que ttes les cultures se valent, alors aucuns dialogues possibles entre elles. Danger d'accepter des systèmes de valeurs contraires. \\

\fbox{
\begin{minipage}{19cm}
WALZER Michael, \textit{Pluralisme et démocratie}, 1997, p. 80-81
\item \textbf{Quels sont les conditions du vivre ensemble ?} Selon Michael WALZER, école devrait favoriser chez élève 3 prises de cse : 
\begin{enumerate}
\item \textbf{Pde au sérieux identités, attentes de chq communauté.} Ne st pas nécessairement ascension sociale ou acquisition de ts savoirs scolaires. Ecole doit admettre existe autres sphères sociales où individus égaux entre eux selon d'autres critères\\
\item \textbf{faire compde que chq individu évolue ds plusieurs sphères autonomes} : une région, famille, religion, cat sociale. Propre du citoyen ds socété démoc : conserver faculté d'exercer libre volonté (lui permet de renoncer à 1 telle affiliation). \\
\item \textbf{construire le << soi postsocial >>, condition du vivre-ensemble}. Pr contrer fragmentation de société libérale, faut que individus puisse faire des choix selon conscience du bien commun ou de son propre bien.
\end{enumerate}
\end{minipage}
}

\item construire 1 projet commun pr vivre ensemble : enjeu pr société. Communauté éducative doit se donner moyen de faire coexister ds même espace individus ne partageant pas même convictions, au lieu de les juxtaposer en 1 mosaïque de communautés fermées sur elles-même et exclusives. Ecole doit être 1 moyen de faire coexister individus qui partagent pas mêmes convictions.\\

\textbf{5. Communauté éducative et vie scolaire}\\

\item communauté éducative : label ? voeux pieux ? Vrai concept ?
\begin{enumerate}
\item complexité relation école-famille, mutation que connaissent familles, soupçons sur missions lgtps indiscutables de école -> rend difficile définition de communauté éducative.\\
\end{enumerate}

\item Apparaît pas comme un concept, ms cmme 1 espace partenarial. But : être ce que les acteurs veulent ou peuvent en faire; pr accomplir, chacun dans son rôle mais en complémentarité, la tâche d'éduquer, de dvper qq'1 en devenir.\\

\fbox{
\begin{minipage}{19cm}
PFANDER-MENY Lydie, <<Ecole-famille : vers une nouvelle professionnalité des CPE ? >>, in PICQUENOT Alain, VITALI Christian (dir), \textit{De la vie scolaire à la vie d'élève}, CRDP Bourgogne, p. 112\\

\begin{enumerate}
\item CPE est un professionnel de éducation. Relation constante avec l'élève et sa famille ou ses familles. Fait tenir compte des nouveaux modes de parentalités.\\
<< Le CPE est, dans l'établissement du second degré, un professionnel de l'éducation. Il est à ce titre en relation constante avec l'élève et sa famille ou ses familles. >> \\
\item <<La question centrale posée aux établissements est de savoir comment donner ou redonner sens à la place et au statut des parents ? >> \\
\item 
\end{enumerate}
\end{minipage}
}


\item CPE joue 1 rôle très important.\\
\begin{enumerate}
\item interface entre monde enseignant et famille, intersection entre public et privé. \textbf{Peut animer cette communauté et donner parole à personne concernée} (dt élève.)\\

\end{enumerate}


\item ATTENTION : ne doit pas accaparer rôle éducatif, devenir une sorte de représentants des parents auprès des profs et vice versa.\\

\item n'est pas le spécialiste ms favorise dialogue et coopération.


%\end{itemize}



\chapter{Violence et incivilités}

\textbf{Mots clés : }
\begin{itemize}
\item rupture scolaire
\item délinquance
\item zone sensible
\item violence
\item notion de << comportement >> (Mille, Thin)
\item climat scolaire
\end{itemize}

\vspace{0.5cm}

\textbf{Objectif :}


\begin{enumerate}
\item Définir ce qu'est la violence en milieu scolaire.\\
\item Voir causes de violence pr établir plan d'action à échelle de étab.\\
\item connaître pcpes du droit pr sanctions. \\
\item Situer plans français de lutte vs violence ds cadre européen.\\
\end{enumerate}

\textbf{1. Etat des lieux sur les violences à l'école.}

%\begin{itemize}
\item ++ préoccupante. Comme cause, évoque déclin des solidarités, fragilisation lien social, exclusion sociale, chômage, sentiment de galère ds zone sensibles, décrochage élèves, absentéisme. \\

\item Pr saisir phénomène, faut pde en compte représentations et identités des différents membres de communauté éducative.

doc 1. Les incidents graves selon leur nature.
%\includegraphics[scale=0.20]{file}

\item Graphique (entre 2007 et 2009) à partir de sources données / ministère de Education nationale.
\begin{enumerate}
\item aggravation des faits les + graves. Violence verbale en diminution. Agression physique + fréquentes (presque 50 \% des faits de violence déclaré) \\
\item considéré avec prudence données fournies. Incident répertoriés / ministère recueillis à partir de 2 questionnaires envoyé chq trimestre à 1 échantillon de 1000 étab : 
\begin{enumerate}
\item 1er : recense nb et type d'incidents pdt le trimestre.\\
\item 2nd : demande appréciation du chef d'étab sur ambiance gal et sécurité.\\
\end{enumerate}

\item nb de réponses diffèrent selon type d'étab (collège répondent plus que lycée), selon moment ds année (à partir de mars, répond moins). \\
\begin{enumerate}
\item publication palmarès lycée + mise en concurrence étab ds système éducatif ++ piloté / loi du marché = soupçon d'insincérité sur déclaration de faits de violence de part des chefs d'étab.\\
\end{enumerate}

\end{enumerate}

\item presse écrite relate que certains nb de faits, gravissimes. ms pas quotidien des étab. en 2004 :  étab ft remonter 240.000 déclarations d'incidents / trimestres dt 3\% st faits graves.\\

\item donnée importte :  63\% des coups et blessures contre élèves ou adultes en milieu scolaire st fait d'élèves de l'intérieur. -> représentation de hordes barbares déferlant sur étab pas fondée.\\
\begin{enumerate}
\item faut pas faire des étab des forteresses ms plutôt aller vers une ouverture organisée. Trop de fermeture peut préciciper la coupure sociale.\\
\end{enumerate}

\item violence très sexuée. Surtt garçons.\\

\item  personnels st victimes de violence ds proportion inquiétante.\\

\item typologie des faits de violences : 
\begin{enumerate}
\item violences pénalisables.
\begin{enumerate}
\item 1er niveau. crimes et délits : vols, extorsions, coups et blessures, trafic et usage de stupéfiants...\\
\end{enumerate}
\item violences non pénalisables.
\begin{enumerate}
\item incivilités (bruit, vandalismes, injures...)
\begin{enumerate}
\item codes élémentaires de la vie en société qui est pas respectée. Peuvent paraître comme menace contre ordre établi. \\
\item violence grave et révélatrice d'une crise forte du lien social. Dominante en milieu scolaire. Explique malaise actuel plus que les violences.\\
\end{enumerate}

\item sentiment d'insécurité, << de violence >>. svt ressentie / personnes pas victimes de faits violents ms qui ont peur de l'être.\\

\item certains actes violents (racket, incivilités) commencent à toucher école primaire.\\

\end{enumerate}
\end{enumerate}

\item en lycée, chiffres baissent.\\
\begin{enumerate}
\item avec âge, jeunes se construisent une identité autrement que / violence.\\
\item 1 partie des éléments les + durs ont quitté système éducatif.\\
\end{enumerate}

\textbf{2. Les causes sociales de la violence.} \\

\item faits de violence en augmentation ds zones sensibles.\\

\fbox{
\begin{minipage}{19cm}
\textbf{DEBARDIEUX Eric, \textit{Les dix commandements contre la violence à l'école}, 2008, p. 96} \\
\begin{enumerate}
\item \textbf{élèves situés en zones sensibles st ++ svt victimes des faits de violence.} \\
\item augmentation du sentiment d'insécurité lié à augmentation de l'intensité des victimations pr 1 nb restreint de victime plus durement agressée. \\
\item Vrai changement : augmentation des agressions en groupe, contre des victimes isolées, sur critère identitaire.


\item Debardieux est 1 membre de l'Observatoire européen de la violence scolaire. Membre de American Society of Criminology.

\item selon lui,  baisse du nb de faits déclarés de violence + aggravation de nature des violences.  Ccls : 
\begin{enumerate}
\item dep 10 ans, augmentation faits + graves. Faits moins graves st en baisse. Faits de violence dans ensemble baissent.\\
\item cadre des faits + graves ++ des lieux d'exclusions. Fréquences diminue ds étab ds secteurs non sensibles.\\
\item augmentation actes de violence pr raison religieuse, ethniques...\\
\item augmentation nb d'agressions commises en groupes.\\
\end{enumerate}
\end{enumerate}
\end{minipage}
}


\item \textbf{Aggravation des inégalités sociales devant faits de violences}. \\

\fbox{
\begin{minipage}{19cm}
\textbf{DEBARDIEUX Eric, \textit{Les dix commandements contre la violence à l'école}, 2008, p. 96} \\

\begin{enumerate}
\item \textbf{Violence contre personnels et institution augmente ds zones prioritaires}

\item source : recensement 2005-2006 de SIGNA : 
\begin{enumerate}
\item + 7\% de violence vs profs.\\
\item + 23\% par rapport à 2002-2003 sur une longue durée pr personnel responsable de l'ordre au quotidien : CPE ou surveillants. \\
\end{enumerate}

\item écart collège ZEP et autre se creuse. Plus forte agressivité contre enseignants. \\
\end{enumerate}
\end{minipage}
}

\vspace{0.5cm}

\item  violence frappe élève à double titre.
\begin{enumerate}
\item crée sentiment d'insécurité qui détériore conditions de leurs apprentissages.\\
\item éloigne d'eux enseignants les + expérimentés, dt l'ancienneté de services leur confère 1 barème suffisant pr obtenir 1 mutation dans lieux moins exposé aux violences.\\
\end{enumerate}

\item Causes ?
\begin{enumerate}
\item selon sociologue, lien avec exclusion. Ms faut pas tomber dans << criminalisation de la misère >>.
\item Etude menée / univ  de Bordeaux pdt 32 ans sur 545 élèves. Montrent concomitance facteur de marginalité (pauvreté, chômage récurrent, père violent ou absent) empêche pas bcp d'enfants d'échapper aux prédictions de violence ou de s'adapter aux exigences de institution.\\
\item ds certains cas, \textbf{étab médiateurs essentiels ds transactions des enfants ac leur environnement}. Prouvée que déscolarisation corrélée avec délinquance.
\end{enumerate}

\fbox{
\begin{minipage}{19cm}
\textbf{GLASMAN Dominique et DOUAT Etienne, <<Qu'est-ce que la << déscolarisation >> ?, in GLASMAN et OEUVRARD Françoiase (dir), \textit{La déscolarisation}, 2011, p. 71-73.}

\begin{enumerate}
\item parmi jeunes incarcérés, ceux qui avaient été déscolarisés st surreprésentés.\\
\end{enumerate}
\end{minipage}
}
\vspace{0.5cm}

 \item conclusion : exclusion prolongée du collège ou lycée = mesure contre-productive. augmente risque de plongée ds délinquance pr jeunes en situation de rupture scolaire ou de violence à l'école.\\


\textbf{3. Ruptures scolaires, incivilités et violences} \\
\textit{3.1. Les facteurs qui favorisent la violence en zone sensible}.\\

\item Ne doit pas conclure que violence en milieu scolaire est une \textbf{fatalité} ds zone sensible.\\

\item Certains étab, même si situés en secteurs défavorisés, échappent à ces logiques. Ils ont : 
\begin{enumerate}
\item moins de 500 élèves.
\item << loi >> bien appliquée
\item équipes péda unies et motivées
\item équipe de direction volontaire, fiable et s'enferme pas ds tâches adm.
\end{enumerate}
\fbox{
\begin{minipage}{10cm}
cf. OBIN Jean-Pierre [dir], \textit{Pour un étab mobilisé contre la violence}, site internet education.gouv.fr
\end{minipage}
}

\vspace{0.5cm}

\item  Les plus concernés st : 
\begin{enumerate}
\item effectifs lourds (plus de 600 élèves)
\item élèves ignorent le bureau du principal qu'ils considère vide de sens.
\item équipes st enfermés ds conflits d'adultes
\item désordre quotidien permanent et non  régulé.
\item Perte générale de confiance ds capacité des adultes à instituer l'ordre. \\
\end{enumerate}

\textit{3.2. L'enchaînement ruptures scolaires - délinquance.} \\

\fbox{
\begin{minipage}{19cm}
\textbf{MILLE Mathias, THIN Daniel, \textit{Ruptures scolaires : l'école à l'épreuve de la question sociale}, 2010, p. 157}\\
\item Différentes façons de voir les collégiens en rupture scolaire.
\begin{enumerate}
\item Les plus contraires au règles scolaires. Comportement perturbateurs de l'ordre scolaire (secondairement absentéisme).
\item Notion de << comportement >> omniprésente ds appréciations scolaires des collégiens. Distingue les << bons comportements >> (conformes aux règles et morales scolaires) des << mauvais comportements >> (perturbant ordre scolaire et entravant action péda.)
\begin{enumerate}
\item Ce seul mot suffit à justifier de sanctions
\end{enumerate}
\end{enumerate}

\item parle de effet nocif d'une norme scolaire illisible. Loi doit être bien identifiée. Régime de sanction progressif. Jeune sanctionné pr comportement : trop illisible. Comme automobiliste sanctionné pour inconduite sans plus de précisions.
\begin{enumerate}
\item Effet de cet stigmatisation : enchaînement qui conduit au rejet de école, aux incivilités et aux violences.
\end{enumerate}

\item auteurs insistent sur fait qu'au départ de ces problèmes, très svt des difficultés d'apprentissages : 
\begin{enumerate}
\item difficultés d'apprentissage
\item pratique hétérodoxes dans espace scolaires
\item stigmatisation provoquée / sanction pour << comportement >>.
\item hypoactivité scolaire manifestée / collégiens.
\item évitement et pratique de survie. cours - en - supportables. présence en classe perd son sens.
\item perturbations de ordre scolaire + relations conflictuelle ac profs.
\item comportements engendrés => aggravation des performances scolaires.
\end{enumerate}
\end{minipage}
}

\vspace{0.5cm}

\item Peux pas réduire question de violence à problème pédagogique. Ms certain que échec scolaire non traité conduit à rupture scolaire et violence contre institution et personnel.\\

\item svt 1 solution péda appliquée précocémment meilleure des solutions. Ici que rôle CPE essentiel : très différent du surgé : 
\begin{enumerate}
\item dépister, derrière << problèmes de comportements >> : éventuelles difficultés d'apprentissages\\
\item mettre en place qd possible, projet péda d'aide et soutien aux élèves pr restaurer estime de soi et réconcilier élève ac école.\\
\end{enumerate}

\textbf{4. Les principes des mesures disciplinaires} \\

\item Ne faut pas se désintéressé de question des sanctions.
\begin{enumerate}
\item \textbf{sanction disciplinaire : objet de réflexion approfondie dep 15 ans}.

\fbox{
\begin{minipage}{15cm}
cf travaux de Olivier REBOUL, Eirick PRAIRAT, Eric DEBARBIEUX, Gilles FERREOL\\

\textbf{PRAIRAT, \textit{La sanction en éducation}, 2003}\\
\begin{enumerate}
\item rappelle méfaits des châtiments corporels, administrés / enseignants ou parents.\\
\end{enumerate}
\end{minipage}
}

\end{enumerate}

\vspace{0.5cm}

\item Elèves doivent être associés à tte phases de conception et application des sanctions pr initier à vie démocratique.

\item circulaire, 1er août 2011 << L'organisation des procédures disciplinaires dans les collèges, lycées et les EREA >> distingue punitions scolaires et disciplinaires.
\begin{enumerate}
\item punition scolaire =  acte pédagogique : manquements mineurs des élèves, prononcé / personnels.
\item punition disciplinaires = dimension éducative. prononcées / chef d'étab ou représentant. figurer au règlement intérieur. \\
\end{enumerate}

\textit{4.1. Principe de la légalité des sanctions disciplinaires} \\

\item pas rétroactives. Peuvent faire objet d'1 recours administratif interne ou d'1 recours dvt juridiction administrative. \\

\textit{4.2. Principe du contradictoire} \\

\item Avt tte décision à caractère disciplinaire, impératif d'instaurer dialogue ac élève. Sanction doit se fonder sur élément de preuve. Peut faire objet d'1 discussion avec parents.\\

 \item procédure contradictoire : chacun doit pvr s'exprimer. Elève peut se faire assister de personne de son choix (élève ou délégué de classe).\\
 
 \item représentants légaux du mineurs informés de procédures et aussi entendus s'ils veulent. Tte sanction motivée et expliquée. \\
 
 \textit{4.3. Principe de la proportionnalité de la sanction}\\
 
 \item Finalité de la sanction :\textbf{promouvoir une attitude responsable de l'élève.}
 
 \item Impératif qu'elle soit graduée selon gravité du manquement à la règle.\\
 
 \item hiérarchie entre : 
 \begin{enumerate}
 \item atteintes aux personnes et atteintes aux biens\\
 \item infractions pénales et manquement au règlement intérieur. \\
 
 \end{enumerate}

\item Utile de se référer au registre des sanctions disciplinaires pr éviter des distorsions ds traitement d'affaires similaires. \\

\textit{4.4. Principe de individualisation des sanctions.}\\

\item En aucun cas collective.
\item Individualiser = tenir compte du degré de responsabilité de l'élève, son âge, implication ds manquements, antécédents en matière de discipline.\\
\begin{enumerate}
\item décret du 24 juin 2011 : prévoit obligation d'1 action disciplinaire ds certains cas de violence verbale, physique. \\
\end{enumerate}

\item sanctionne pas que acte commis, ms sanctionne en regardant personnalité de élève, contexte de chq affaire.\\

\item Tte sanction, hors exclusion définitive, est effacée du dossier administratif de élève au bout d'un an.\\

\textit{4.5. Les mesures de prévention}\\

\item Mesure inscrites au règlement intérieur. Vise à prévenir acte répréhensible (ex: confiscation d'un objet dangereux). \\

\item Obtenir engagement de élève sur objectifs précis en termes de comportement. Rédaction d'1 document signé par l'élève.\\

\textit{4.6. Les mesures de responsabilisation} \\

\item Faire participer l'élève à des activités en dehors des heures d'enseignement au sein de étab, d'assoc agrée ou coll Tale. \\

\item Aucune tâche dangereuse ou humiliante.\\

\textit{4.7. L'exclusion temporaire de l'élève} \\

\item durée maximum 8 jours. Prononcée / chef d'étab.  Exclusion de étab ou de classe avec présence obligatoire ds étab.\\

\textbf{5. Instances impliquées} \\

\textit{5.1. La commission éducative} \\

\item obligatoire ds tt étab. Représentant de ttes les composantes de communautés scolaire. But : examiner situation d'élèves en rupture ac règles de vie de étab (qd exclusion prononcée). \\

\item Transmet 1 avissur éventuelles suites disciplinaires ou mesures éducative à pde au chef d'étab. En dernier ressort, à pvr de décision.\\

\textit{5.2. Le conseil de discipline} \\

\item Réunion à demande du chef d'étab ds cas d'1 exclusion définitive. Statue sur cette exclusion.\\

\item Tte sanction prononcée peut être référée ds délai de 8 jrs, au recteur d'académie / représentant légal du mineur ou l'élève (si majeur). Recteur décide, après avis d'une commission académique.\\

\textit{5.3. Attributions du chef d'établissement} \\

\item Seul a pvr engager sanction disciplinaires. Prononce tte : de avertissement à exclusion temporaire de classe ou de étab. \\

\item A titre conservatoire, peut interdire accès à étab à 1 élève devant passer en conseil de discipline.\\

\textbf{6. Les plans récents de prévention anti-violence en France} \\

\item dep 20 ans, 9 plans de lutte vs violence élaborés / ministres de Éducation nationale successifs.
\begin{enumerate}
\item mai 92 : plan Lang. Création de 300 postes administratifs + recours à 2000 profs + partenariat EN-police-justice.\\
\item mars 95 : Bayrou. réduction taille étab, fonds d'assurance pr enseignants, postes de médiateurs, n° spécial << SOS violence >>, éducation civique renforcée.\\
\item mars 96 : 2nd volet.  \\
\begin{enumerate}
\item 3 orientation
\begin{enumerate}
\item renforcement de encadrement \\
\item relations élèves-parents \\
\item étab et environnement. \\
\end{enumerate}

\item Recourt à 1200 profs + création neaux postes de personnels de santé + création des classes relais pr élèves en difficultés.\\
\end{enumerate}

\item nov 97. Phase 1 plan Allègre. Moyen sup ds 10 sites sur 6 académies. 98 et 99 : ces étabs ont 485 emplois d'infirmières et assistantes sociales, 100 postes de médecins scol, 400 emplois ATOS, 100 de CPE, 4728 aides-éducateurs.\\

\item + 3 mesures :  \\
\begin{enumerate}
\item aggravation des sanctions pénales pr fait de violences ds étab.
\item signature ds 14 départements d'1 convention ac Institut national d'aide aux victimes et de médiation
\item programme de partition des plus gros collèges.\\
\end{enumerate}

\item janv 2000 : 2nde phase plan Allègre. Création 5 zones d'expérimentation + moyens supp. \\

\item oct 2000 : Lang met en place Comité national de lutte vs violence à l'école. Instaure logiciel SIGNA (recenser actes violents). Diffuse vademecum pr gérer situation de violence. \\

\item mai 2009 : plan Darcos. création d' <<équipes mobiles de sécurité >> ap que Sarko ait plaidé pr << sanctuarisation >> des étab en réactions aux incidents.\\

\item rentrée 2009 :  <<plan de sécurisation >> ac << diasgnostics de sécurité >> pouvant aboutir à installation de clôture et système de vidéosurveillance + plan de formation à gestion de crise et à exercice de autorité. doit toucher 14.000 personnes.\\
\end{enumerate}

\item Actions permettent de \textbf{stabiliser situation globale}. Eviter explosion faits de violence.\\
\begin{enumerate}
\item ms pas éradiquer. alternances pol -> succession de logiques différentes en peu de temps : pol de prévention et réduction effectifs -> répression, déploiement dispositifs policiers et surveillance.\\
\end{enumerate}

\item Déploie éducateurs à gde échelle, ou portiques et système de vidéo... Donne usagers de école impression d'un pilotage pas propre.
 \item peut pas prétendre sanctuariser étab sans traiter prob + gal de exclusion sociale ds ces quartiers périurbains. \\
 
\fbox{
\begin{minipage}{19cm}
Eric DEBARBIEUX, << Violence scolaire : << Je suis pessimiste >> nous dit E. Debarbieux >>, 2012, site du Café pédagogique.\\

\begin{itemize}
\item  Eric Debarbieux mentionne série de points à approfondir : cohérence des sanctions, prise en compte de prévention des violences dans formation des enseignants, qualité des relations entre adultes ds étab et entre enseignants-parents d'élèves, importance trop grande donnée à transmission des savoirs sur la socialisation des jeunes.\\
\end{itemize}

\begin{enumerate}
\item violence scol a plusieurs causes : situation éco, familiale, facteurs liés à institution scol. Forte correlation qualité du climat scol (qualité relation adultes-élèves et entre adultes : capacité à avoir un dialogue avec élèves et pas un affrontement, clarté des règles collectives) et victimisation. Sentiment d'appartenance collective et justice st 2 composante essentielle de ce climat.\\
\item Ne forme pas en France, enseignants à gestion des punitions. Se retranche derrière CPE. S'intéresse plus à transmission du savoir. Oublie importance de identification aux adultes.\\
\item  Tte ne s'explique pas / climat scolaire. Pays qui s'en sortent le mieux face à violence scol st ceux où place des parents est la plus forte.
\begin{enumerate}
\item Ex: de Rio. Violence endémique ds certains quartier, mais ne rentre pas à école car protégé / communauté. 
\item En France, voit parents comme ennemis. Impératif de travailler ensemble face à violence. Très démunis face à elle.
\end{enumerate}
\end{enumerate}

\end{minipage}
}


%\end{itemize}

\chapter{L'école et ses partenaires}

\part{Transformations : où va l'école ?}

\chapter{L'histoire de la démocratisation de l'école}

\textbf{Mots-clés : } 

\begin{itemize}
\item 
\item 
\item 
\item   
\end{itemize}

%\begin{itemize}
\item Objectifs : 
\begin{enumerate}
\item compde sens actuel des gdes évolutions de école / comparaison ac époques antérieures \\
\item marquer historicité des concepts \\
\item compde gds pcpes constitutifs d'une école démocratique.\\
\end{enumerate}


\section*{Introduction : concept de << démocratisation >>}

\item compdre historicité des concepts pr voir leur sens actuel par rapport à leur sens passé. \\
\item Comprendre ici concept de << démocratisation >>. \\

\subsection*{distinguer histoire du mot et histoire du concept.}

\item Mot << démocratisation >> : attestée depuis fin du 18e. \\

\subsection*{L'élargissement à ttes les couches de la société du droit à ttes les formes d'instruction}

\item Concept construit / historiens et philosophes pr désigner 1 évolution de la société conduisant à élargir bénéfice de instruction à ts futurs citoyens.
\begin{enumerate}
\item pdt EDG, forme plus large de démocratisation de enseignement : égalité des chances d'accéder à aptitudes égales, aux + hautes qualifications, indépendamment de son origine sociale. Appelé << démocratisation de la sélection >>. Politique ont élargit base de recrutement en augmentant nb d'élèves. Parle de massification des études secondaires ou démocratisation quantitative. \\

\item Or cette démocratisation quantitative laisse subsister différences qualitatives entre les filières. Compétition pr accéder aux filières d'excellence tjs socialement inéquitable. \\

\item ds dernier quart du 20e, démocratisation devenue égalité des chances d'insertion professionnelle et sociale, rapportée au thème du projet personnel de l'élève = démocratisation de la réussite. Concerne pas que sélection des élites ms réussite scolaire et professionnelle de ensemble de pop.
\begin{enumerate}
\item Moyen pr y accéder : acquisition universelle du socle commun de connaissance et compétence + dvpt de formation tt au long de vie. \\
\end{enumerate}
\end{enumerate}

\section{Les débuts de l'aspiration à la démocratisation}

\item vient pas de philosophie des Lumières ms de recherche de prospérité collective en un temps où situation éco et militaire de France particulièrement dégradée. Projet vient de pensée libérale appliquée à économie, avt de devenir 1 question formulée en termes de justice et d'équité. \\

\section{La formation du citoyen pdt la Révolution.}

\item Condorcet distingue éducation de instruction. 










\chapter{L'école pour chacun ou l'école pour tous ?}

\part{Gestion : Politique de l'établissement et gestion concurrentielle de l'éducation}

\chapter{L'établissement scolaire}

\textbf{Mots-clés : } 

\begin{itemize}
\item 
\item 
\item 
\item   
\end{itemize}


\item Objectifs : 
\begin{enumerate}
\item compde causes du processus d'autonomisation des étab scolaires.  \\
\item identifier enjeux du projet d'étab \\
\end{enumerate}

\section{Le principe de l'autonomie de l'étab scolaire.}

 \item mise en place école dep années 60's : conséquence : promotion du local. Système éduc lentement unifié sous debuts de 5e Rep (double pression de essor démog et éco). \\
 
 \item Depuis mise en place du collège unique, quasi-totalité chq générations passe / collège. Milieu 80 : système éduc unifié, ac école unique, collège unique, lycée diversifié ms dt ttes les filières conduisent au bac ac ouverture sur formations univ.\\
 
 \item A ce moment la, nlles formes de diversifications apparaissent au plan local. Pourquoi ?
 \begin{enumerate}
 \item 1ère concerne projet d'étab. raison de pilotage et pr raison de rechercher d'efficacité. \\
 \item Ds 1 étab de ZEP ca portera sur intégration des enfants en situation de gde précarité sociale, ds lycée de centre-ville (sur excellence). \\
 \item fait de tenir compte de plusieurs pcpes de pilotages => décentralisation de la décision (ie forme de retrait de échelon national au profit du local).
 \end{enumerate}
 
 \item Parmi effets sociaux de autonomisation des étab, voit accroissement de ségrégation scolaire. Ecole typées sur plan social. \\
 
 \item Waux de Marie DURU-BELLAT et Agnès VAN ZANTEN (\textit{Sociologie de l'école}, 2006) : 44 \% des élèves ont 1 parent qui appartient à 1 profession classée défavorisée en 2002. 10\% collèges les - populaires en accueillent  22\% tandis 10 \% des collèges les + populaire : 68 \% \\
 
 \item assouplissement carte scolaire : donne + facilités aux familles pr choisir l'étab. collèges encore plus typés.
 
 \section{L'organisation formelle et informelle de l'étab.}
 
 \item Organisation formelle : instances (conseil classe, conseil discipline, CA...), fonctions (chef d'étab, CPE, PP, gestionnaires...), systèmes de com, dispositifs de régulation... \\
 
 \item org informelle : jeu des stratégies individuelles, rapports interpersonnels, affinités, circuit réel de l'info.\\
 \begin{enumerate}
 \item Etab st très différents ds org informelles. \\
 \item qq carac pour identifier type d'étab où on exerce : \\
 \begin{enumerate}
 \item relation maître-élève (unité de base) \\
 \item contraintes de temps (rythme cyclique de année scol) \\
 \item structure hiérarchique (déconnecté de activité pédagogique des profs ?) \\
 \item activité péda déconnectée des effets ? Profs travaillent ensemble ? Avec le CPE sur indicateurs de réussite ? Ont-ils des infos en retour sur qualité de activité ? \\
 \end{enumerate}
 \end{enumerate}
 
 \section{L'établissement est un << espace laïque de savoirs et de citoyenneté >>}
 
 \item civisme pas règle froide et abstraite ms apprentissage collectif permanent. Doit dvper pratiques de citoyenneté, initiatives citoyennes, créer espace de médiation, d'écoute, de dialogue ac jeunes et familles. \\
 
 \item cohérence action adultes de étab nécessité. 
 \begin{enumerate}
 \item doit s'appuyer sur ressource existant ds étab et ts les acteurs de communauté éduc. Chq acteur doit avoir 1 réflexion sur ce qui constitue coeur de son métier. Chacun doit connaître ses missions, rôles, tâches et compétences. \\
 \item tt adultes doivent avoir 1 discours cohérent. chacun, du surveillant à enseignant, du personnel adm à équipe de direction, doit être conscient des nécessités suivantes : 
 \begin{enumerate}
 \item faut une cohérence ds les discours et le travail quotidien. \\
 \item nécessaire que chacun applique même règles, ait même seuil de tolérance pr ce qui est inacceptable, non négociable. \\
 \end{enumerate}
 
 \item  fondamental que ds tt étab, même règles. Ces règles sont pas les même ds la rue, de la cité. Difficile aux élèves de se les approprier. surtt s'ils voient que pas les même en maths, français...
 \end{enumerate}
 
 \section{L'étab scolaire : un lieu de vie comportant des métiers divers}


\item 18 à 19 métiers (au-delà de celui de prof). Du surveillant au personnel de cuisine, du personnel d'entretien au secrétariat. Ts st des éducateurs au service des jeunes. \\
\item Diversité pas suffisamment  soulignée auprès des  élèves et familles. \\
\item idée : en 3e et 4e : travail sur les gestes professionnels, compétences requises pr exercer la 20aine de métiers hors éducation, rencontré dans étab. plus utile pr découverte des professions que actuels stages en 3e.


\section{Redonner sens au collectif.}

\item Appartenance à un collectif : élément importt du vivre ensemble. Etab scolaire doit se concevoir comme porteur d'un projet collectif approprié / tous. \\

\item dvpt d'1 véritable communauté éducative regroupant ts les acteurs d'1 étab, qq soit origines, philosophie, croyances = meilleur antidote contre replit communautaire. \\


\chapter{Secteur public - secteur privé}

\textbf{Mots-clés : } 

\begin{itemize}
\item 
\item 
\item 
\item  
\end{itemize}

%\begin{itemize}
\item Objectifs : 
\begin{enumerate}
\item connaître structures de enseignement privé en France aujourd'hui. \\
\item compde évolution institutionnelles dep loi Debré (59)
\item identifier les enjeux sociaux et culturels des différentes formes d'enseignement privé.
\end{enumerate}

\item 2012 : 2 millions d'élèves ds privé. \\
\item 3 types d'étab : 
\begin{enumerate}
\item associés sous contrat ac Etat : mission de service public contre aide financière substantielle de Etat et collectivité Tales.\\
\item Sous contrat simple. liaison moins contraignante. Certaine liberté péda. aide financière plus faible. \\
\item sans lien ac Etat. soumis aux contraintes légales de sécurité et d'hygiène, moralité et compétence. liberté péda et pilotage totale. \\
\end{enumerate}

\section{L'enseignement privé sous contrat}




%\end{itemize}
\chapter{Des palmarès aux zones prioritaires : les inégalités géographiques, 147}


\textbf{Mots-clés : } 

\begin{itemize}
\item 
\item 
\item 
\item  
\end{itemize}

%\begin{itemize}
\item Objectifs : 
\begin{enumerate}
\item compde finalité initiale des politiques compensatoires \\
\item connaître effets de ségrégation scolaire
\item identifier les marges de progression de l'éducation prioritaire.
\end{enumerate}

\section{Réseaux prioritaires et réseaux d'excellence}


\item 1981 : mise en place des ZEP. Idée d'égalité des chances pr ts les enfants. Donner plus de moyen aux étab dt pop scolaire - favorisée (d'où nom << politique compensatoire >>).

\item ds années 80's, ZEP attribuée selon 3 critères : 
\begin{enumerate}
\item approche globale des problèmes. Action éducative 1 des facteur de ensemble des paramètres de zone géo concernée.  \\
\item notion de projet. suppose analyse approfondie des besoins, choix d'objectifs, programmation d'actions et évaluation. requiert des enseignants qu'ils consacrent temps à concertation. \\
\item constitution d'1 équipe d'animation qui doit coordonner projets.

\end{enumerate}


\item aujourd'hui, ap 30 ans, atteint que partiellement objectifs initiaux d'égalités des chances (selon Agnès Van Zanten). Cause : ségrégation scolaire.
\begin{enumerate}
\item ont du revoir objectifs à la baisse : de égalité des chances, passe à lutte contre exclusion sociale et prévention des risques de décrochage et de violence en milieu scolaire.
\end{enumerate}

\fbox{
\begin{minipage}{19cm}
\textbf{Les différentes politiques compensatoires en France (81-2008)}

\begin{enumerate}
\item rentrée 82 : création 363 zones d'éduc prioritaires (ZEP). 8\% des écoliers et 10\% des collégiens.\\
\item 92 : 500 ZEP et 150 étab sensibles \\
\item 99 : 663 ZEP \\
\item 2004 : 800 réseaux d'éducation prioritaire (REP) et 707 ZEP : 8386 étab, 21 \% des élèves scolarisés en collège.\\
\item 2005 : coût REP-ZEP :  600 millions d'euros / an. \\
\item circulaire 30 mars 2006 : << plan de relance de éduc prioritaire >>. Crée 254 réseaux << ambition réussite >> (RAR). Recrute 1000 enseignants sup et 3000 assistants d'éduc. Mise en place des PPRE (prog personnalisés de réussite éducative).\\
\item 2007 : réseaux de réussite scolaire RRS arrêtés. \\
\item 2010 : mise en place dispositif CLAIR devient ECLAIR (Ecole, Collèges et Lycée pour l'Ambition, l'Innovation et la Réussite). Concentration de moyens sur nb ++ réduit de secteurs expérimentaux. difficultés scolaires moins ciblées que difficultés en matière de climat scolaire et violence. \\
\begin{flushright}
D'après GARNIER Bruno, l'Egalité en éducation, 2012, p. 127
\end{flushright}


\end{enumerate}

\end{minipage}
}

\section{Deux types extrêmes d'étab scolaire}

cf. COUSIN Olivier et FELOUZIS Georges, \textit{Devenir collégien : l'entrée en classe de 6e}, 2002

\subsection{L'étab d'excellence.}

\item collège centre-ville, ds 1 métropole régionale, inclus ds 1 lycée prestigieux, sélectionné. Palettes d'options rares et corps professoral âgé. \\

\item Pt + important : relation maître-élève sous oeil attentif des parents. \\

\item Prob : recherche du rendement social des formation dvpe 1 fort individualisme, 1 esprit de compétition qui favorise ambiance fermée et très scolaire. Incite pas au travail en équipe péda. Animations périscol et activités socio-culturelle pas jugées prioritaires. Accorde peut d'importance aux règles collectives. \\

\item niveau lycée, retrouve cette faible capacité d'intégration. Personnel adm se contente de jouer rôle tampon entre enseignants et enseignés.

\item hébergent classes européennes. St en quête de clientèle sociale la plus prometteuse. se livre à concurrence pr occuper meilleurs places ds palmarès.


\subsection{L'étab << ghetto populaire >>}

\item élèves en difficultés dominent (parfois jusqu'à 60 \% d'élèves en retard à entrée de 6e). Familles absentes de vie de étab. Vie de élèves vt avec tensions. \\

\item contenus d'enseignement et comportements scolaires n'ont guère d'intérêt pr nbeux élèves. Scolarité svt vu comme suite de jugements négatifs qui les déprécient. \\

\item Forte rotation des personnels. Svt débutants. \\


\item ces 2 types d'étab st très dépendants de leur aire géographique de recrutement.
\begin{enumerate}
\item étab ghettos très svt situés ds ghettos urbains. Ecole peut pas résoudre tte seule ts prob liés inégalités sociales et éco. \\
\end{enumerate}

\item types extrême. Avec plein de situations intermédiaires. Exemple aussi d'étab d'1 de ces types qui parvient à mobiliser acteurs, dynamiser projet fédérateur et créer 1 cercle vertueux de socialisation et réussite scolaire. \\

\item ms pas se voiler la face : pilotage / aval (école devient 1 produit que on consomme en fonction de ses moyens), situation de concurrence entre étab proche, créent situation de non-mixité sociale. \\

\section{Les stratégies des << consommateurs d'écoles >>}

\item Enquête du CREDOC (Centre de recherche pour l'étude et l'observation des conditions de vie), avril 2005. Qd parent ont choix entre plusieurs étab scol, mettent en avant << réputation >> comme 1er critère. \\
\item Or contenu de << réputation >> varie bcp : qualité de encadrement, niveau scolaire, public reçu, ... 1er critère : bouche à oreille. \\
\item A une incompréhension, 1 absence des familles populaires : s'expliquent /  ruptures ds monde du travail. Démarches qui montrent aux parents tt ce qui a changé depuis leur passage très importte. Doivent être menée / qq'1 d'indépendant par rapport aux écoles fréquentées (pr que parents se sentent pas pris / discours). \\

\section{Avenir de l'éducation prioritaire}

\item Manière dt éduc prioritaire pd en charge enfants pop : critiques depuis qq années.
\begin{enumerate}
\item Mise en cause du pcpe de << politique compensatoire >>. En voulant << compenser >> << handicap >> enfants vivant dans cette zone, met en oeuvre pédagogie aux objectifs bornés.\\
\item Enfants des milieux pop s'en sortiraient mieux ds étab ordinaires que ds étab relevant de éduc prioritaire. \\
\end{enumerate}

\fbox{
\begin{minipage}{19cm}
\textbf{Un manque d'ambition de l'éducation prioritaire ?} \\

LEGER Alain, << Une école inégalitaire >>, Les Cahiers de l'IFOREP, 1er trimestre 2003, n°105, p. 7

\begin{enumerate}
\item Manque d'ambition et de confiance ds possibilité des élèves. \\
\item Enseignants de ZEP st pessimistes sur possibilités de leurs élèves. Or sait que attentes du Me (positives ou négatives) ont un effet importt sur performances des élèves.\\
\end{enumerate}
\end{minipage}
}

\vspace{0.5cm}

\item  Entre 2007 et 2012, politiques de éduc proritaires se st plus ciblées sur certains publics ou certains prob (incivilités, violence).
\begin{enumerate}
\item dvpt de nbeux dispositifs : accompagnement éducatif en collège, << malette des parents >>, plan Sciences, opération << cours le matin, sport l'aprem >>... \\
\item situation inquiète OZP (Observatoire des Zones prioritaires). Redoute que élèves et parents soient considérés comme cobayes passifs, supposés dociles d'expérimentation hasardeuses. \\

\end{enumerate}

\item  Concentration moyens sur plus petit nb d'étab (dénonçait saupoudrage fond publics sur gde quantité de ZEP). Renforce sentiment d'abandon des objectifs initiaux de éduc prioritaire. \\
\begin{enumerate}
\item prob :faute de viser l'égalité des chances, essaierait de sortir de leurs difficultés qq élèves exemplaires (ex : quotas d'élèves ZEP pr entrer à science po.) \\
\end{enumerate}

\item  A idée de << démocratisation de la réussite >> succède idée de << démocratisation de la sélection >>. \\

\item pdt période gaulliste (création des CEG et CES) : vise à démocratisation de la sélection : élargissement  de la base du recrutement de élite. Ds période de plein emploi. Ensuite : crises éco mettent en évidence échec scolaire comme échec de école. Ac politiques de quotas, retrouve idée de << démocratisation de la sélection >> : vise à << exfiltrer >> qq bons éléments des quartiers en difficultés plutôt qu'à traiter globalement question des inégalités sociales dvt réussite scolaire et professionnelle. \\

\item Tableau critique de réformes des ECLAIR / Luc Chatel. Pr inspecteurs, ECLAIR est 1 réponse insuffisante aux problématiques de éduc prioritaire. Inadaptée pr impulser 1 dynamique de chgt ds système éduc. \\

\item Dvt commissions de Ass Nat, 23 juillet 2013, Jean-Paul Delahaye, Directeur général de l'enseignement scolaire au ministère, évoque mauvais résultats. << Milliards d'euros dépensé pr diminuer nb d'élèves / classes et payer enseignants de éduc prioritaire donne pas résultat. >> \\

\item  Hollande annonce 9 octobre 2012 refonte de politique de éduc prioritaire.
\begin{enumerate}
\item idée de sortir de logique des labels. Superposition de dispositifs. prévu que politiques de ville et de éducation travaillent ensemble pr définir étab réellement prioritaires. Ajuster aides en temps réel et ne  pas installer de vastes ensembles d'étab sous 1 appellation dépréciative. \\
\end{enumerate}






%\end{itemize}

\part{Usagers : de l'enfant à l'élève}

\chapter{La construction de l'identité personnelle du jeune vers l'âge adulte, garçons et filles}

\textbf{Mots clés : }
\begin{itemize}
\item les différents collectifs (les << nous >>)
\item construction de l'adolescence en tant que personnalité unique
\item Rôle de école ds construction
\item relations parents-écoles
\end{itemize}

\vspace{0.5cm}

\textbf{Objectif :}


\begin{enumerate}
\item comprendre spécificités de enfance et de ado
\item connaître déterminants sociaux du passage d'1 âge à l'autre
\item identifier invariants de construction de autonomie de la personne
\item comprendre rôle de école ds dvpt individuel \\
\end{enumerate}

\section{La difficile sortie de l'enfance}

%\begin{itemize}
\item pdt enfance, individu se perçoit comme membre de sa famille. Famille = 1er collectif au sein duquel enfant se pense exister.
\begin{enumerate}
\item Ado modifie ce schéma. Apparition d'autres collectifs. << Nous >> familial destabilisé / << nous >> générationel. Conflit des générations : << nous avec copains, on fait pas pareil que nous avec parents >>. \\
\end{enumerate}

\item conquête d'autonomie fonctionnera pas si c'est dans conflits permanent avec parents, entre jeune et monde des adultes, qui doit devenir son monde. \\
\begin{enumerate}
\item ici que école (et surtout collège) a rôle important. Fait vivre << communauté éducative >> autour du jeune. \\
\item c'est à entrée au collège que jeune s'aperçoit que appartenance familiale relative.\\
\item \textbf{collège = lieu où ado est écouté / d'autres que ses parents}. Lieu d'expérimentation, sous regard de ses pairs et d'autres adultes. \\
\item CPE au coeur de articulation de ts ces regards. Doit écouter, corriger appréciations erronées, aider jeune à devenir autonome, tt en respectant autres. Doit comprendre le malaise qd existe et recourir à famille s'il le faut.\\
\end{enumerate}

\fbox{
\begin{minipage}{18cm}
\textbf{La sortie de l'enfance passe par la conquête de l'autonomie.}\\
DE VOS Bernard, Actes de la table ronde bruxelloise, 5 novembre 2008\\

"Il est important d'identifier le malaise que les jeunes peuvent ressentir. [...] Il faut favoriser la participation des jeunes, reconnaître leurs compétences, poser un regard positif sur eux et les prendre au sérieux plutôt que les prendre au mot. Ce qu'il manque aux jeunes in fine, c'est l'attachement, ie le sentiment d'être important dans le regard de quelqu'un. >>
\end{minipage}
}

\section{Individualisation et socialisation}


\item recul des repères qui fondaient identité collective de nos société, lié à montée de individualisation.\\
\begin{enumerate}
\item au nom du principe qui veut qu'à tout âge on soit une personne pourvue de droit et devoirs individuels, société tend à atténuer frontières entre divers âges de la vie. << flou des âges >>. \\
\end{enumerate}

\fbox{
\begin{minipage}{19cm}
\textbf{Le flou des âges et l'individualisation} \\
SINGLY, François de, \textit{Les Adonaissants}, 2006, p. 12-13. \\

\item adultes auraient perdus leurs repères, cherchant à s'aligner sur les ados. Peur de inversion (ado adultes, adultes ado) = traduit résistance à individualisation.
\item individualisation = droit pour tout individu de ne pas être défini seulement / une place (dans ordre des générations, des sexes, des institutions). 1 garçon ou 1 fille n'est pas seulement << fille de >> avec ses parents, tt comme une femme n'est pas juste << épouse de >> avec son mari.\\
\begin{enumerate}
\item chacun peut être considéré comme individu à part entière, en tant que personne. \\

\item Nlle manière de définir les individus bouleverse les barrières traditionnelles entre les âges, genres, orientations sexuelles. \\

\item Engendre un flou : enfants ont droits sans attendre l'âge adulte. Mais signifie pas que enfant devient un adulte.
\end{enumerate}
\end{minipage}
}

\vspace{0.5cm}

\item PROBLEME : cette recherche de collectif auxquels s'assimiler est \textbf{processus nécessaire au dvpt de ado vers autonomie}, vers construction de individualité = résultat d'une succession d'adhésion de soir à différents collectifs.\\

\item Conception de processus d'individualisation à partir de collectifs trouve origines ds philosophie des Lumières. Ts individus doivent être autonomes, penser / eux-mêmes.

\vspace{0.5cm}

\fbox{
\begin{minipage}{18cm}
\textbf{Les lumières à l'origine de l'individualisation} \\

KANT Emmanuel, Qu'est-ce que les Lumières ?, 1784. \\

<< Sapere aude ! [Ose penser ! ] Aie le courage de te servir de ton propre entendement. Voici la devise des Lumières ! >> Homme doit sortir de sa minorité / sa capacité à pvr penser sans la direction d'autrui.\\
\end{minipage}
}

\vspace{0.5cm}

\item Pr le faire, jeune doit disposer des conditions nécessaires pr échapper aux tutelles. 1 des missions de école. Philosophie de éducation pense que école sert à émanciper le jeune : 
\begin{enumerate}
\item Mission d'émancipation : faire de individu quelqu'un de pensant par lui-même grâce à l'instruction. Mission qui entre en conflit ac mission de socialisation. \\
\item permettre à individu de devenir 1 être social (Emile Durkheim) comme membre d'1 ou plusieurs collectifs (famille, milieu de vie, catégorie sociale, culture...)
\end{enumerate}

\item époque actuelle : mise en tension des deux missions car affaiblissement des repères d'identifications collectives portée / école + montée de individualisation ds mondialisation.\\

\item pr bcp de jeunes, école peine à être lieu où acquiert sa liberté et autonomie au sein de la Cité. perçue comme celle qui condamne à devenir individu qui réussit ou qui échoue. \\

\section{Le rôle de l'école}


\item ts membres de communauté éduc doivent se mobiliser pr inverser tendance et venir en aide aux enfants les + exposés.\\

\fbox{
\begin{minipage}{19cm}
\textbf{Crise de l'adolescence et rôle de l'école.} \\

JEAMMET Philippe (dir), \textit{Adolescences}, 2002, p. 176-177 \\
 
<< société a changé, et avec elle relations qu'entretenaient enseignants et élèves, ens et parents, parents et enfants. Ces relations st affectée / modification de nos valeurs, en particulier rapport à la << liberté >>, à << autorité >> dans un contexte éco qui fait de la scolarité secondaire la mise en orbite soit de l' <<échec>>, soit de la <<réussite>>. [...] \\
Faire fonctionner le << groupe classe >>, pour que les jeunes se soutiennent entre eux face aux angoisses à propos de l'avenir, et être vigilant à ces groupes << SOS >> de fond de classe, dt agitation témoigne du mal-être.>> \\
Dans étab où équipe enseignante soudée, infos s'échangent sur élèves. Cela permet de mieux appeler l'attention des parents ou du médecin scolaire sur évntuels troubles de personnalité.\\
\end{minipage}
}

\item faut faire comprendre aux élèves qu'ils ne peuvent pas se construire sans la concurrence aux autres.\textbf{ Ecole est lieu où on apprend à connaître les inégalités de soi aux autre pr se reconnaître soi-même}, différent et unique, ms pas seul au monde. \\

\item Comment école peut atteindre ce double objectif ?
\begin{enumerate}
\item en s'opposant, / dispositifs éducatifs, à confusion des âges. \\
\begin{enumerate}
\item A âge ado, important de marquer fin de enfance. Ms dep 20 ans, disparitions des rituels d'intégration sociaux. flou règne entre 12 et 25 ans.\\
\item Moment sans limite précise de sortie de enfance : 10-12 ans. Mutation de société engendre difficultés pr certains jeunes : 
\begin{enumerate}
\item passage de celui qui gère la cour d'école à celui qui subit la cours du collège. Possède pas ttes les clés : peut générer certaine angoisse.\\
\item passage de reconnaissance comme << grand de primaire >> au vécu du << petit du collège >>. Jugent svt mal accueillis / élèves plus grands. Crise d'identité générée / ce changt de perspective peut être d'autant plus grave qu'elle se situe au début de ado.\\
\end{enumerate}

\item pose question des rituels d'intégration sociale
\begin{enumerate}
\item pr marquer sortie de enfance et entrée ds ère de responsabilisation. 13 ans = juridiquement en France, âge de responsabilité pénale.
\item pr marquer entrée ds âge adulte. étab scolaire, mairies doivent organiser des cérémonies pr marquer ce moment décisif de rupture.
\end{enumerate}
\end{enumerate}
\end{enumerate}

\item svt familles ne se rendent pas compte qu'une travail bien encadre sur limites, dangers, risques, peut permettre d'éviter que jeune n'aillent rechercher sensations extrêmes.\\
\item \textbf{transgresser : pr ado, moyen de prospecter les limites, tester, mesurer les interdits.} Important que adulte ne se laisse pas prendre au jeu de transgression qu'expérimente ado. S'agit pas d'être laxiste ms travailler sur limites et régulation possible.\\

\item qd interdit, au nom du principe de << précaution >> ds cours de récréation jeux de balles, rallyes d'orientation nocture... qui amène à travailler avec le jeune les peurs et dangers, on ne s'étonne pas du résultat.
\begin{enumerate}
\item pr qq cas médiatisés, regrettable, on empêche l'ado de se préparer à gérer son passage à la maturité.\\
\end{enumerate}


\section{L'importance des relations école-familles pour construire l'identité de l'élève}

\subsection{L'ado entre identité de l'élève et identité de personne}

\fbox{
\begin{minipage}{19cm}
\textbf{Favoriser de bonnes relations parents-enseignants} \\

\textit{Circulaire du 25 août 2006 sur le rôle et la place des parents à l'école.}

\begin{enumerate}
\item Tt mettre en oeuvre pr que parents puissent pde connaissance des résultats scolaires de enfant.\\

\item parents doivent être prévenus rapidement de tte difficultés rencontrée / élève (scolaire ou comportementale). Question de assiduité scolaire (élément fondamental de la réussite) attention particulière. Utilisation des SMS et autres moyens d'Internet doivent permettre des échanges + rapides avec les parents (absences, réunions...)
\begin{enumerate}
\item \textit{ce que circulaire dit pas : si sanctions prononcées sans dialogue suffisant au préalable, svt effet contraire : radicalisation, marginalisation, exclusion}.
\end{enumerate}

\item Organisation de rencontres collectives : pr ensemble des parents (info de rentrée, parents d'élèves nllement inscrits), pr 1 groupe de parents (par classe ou en sous-groupe).\\
\begin{enumerate}
\item \textit{Nécessité d'adapter taille du groupe de parents au prob à traiter, pr éviter de stigmatiser familles dt enfants rencontre difficultés.}
\end{enumerate}

\item Rencontre individuelle ac enseignants et autres personnels. Cadre adaptés à demande, respect de confidentialité des propos échangés. \\
\begin{enumerate}
\item \textit{certaines familles ont pas habitude des rencontres ou maîtrise pas bien langue française. Pierre Bourdieu : ont << habitus primaire >> éloigné de << habitude secondaire >> imposé / école. Doit prendre en compte cette distance, faire effort d'adaptation nécessaire}  \\
\end{enumerate}

\item dialogue avec parents fondée sur reconnaissance mutuelle des compétences et missions des uns et des autres (professionnalisme des enseignants ds fonction, responsabilités éducatives des parents) + souci commun respect de personnalité de élève.\\
\begin{enumerate}
\item \textit{ATTENTION : ne veut pas dire confondre les rôles. ce n'est pas pr prof ou CPE abdiquer ses prérogatives. Faut que chacun connaissent et reconnaisse clairement fonction et prérogative de autre.} \\
\end{enumerate}

\end{enumerate}
\end{minipage}
}

\vspace{0.5cm}

\fbox{
\begin{minipage}{19cm}
\textbf{Trois concepts fondamentaux caractérisent la théorie de la reproduction. \\ }

BOURDIEU Pierre, PASSERON Jean-Claude, \textit{La Reproduction, éléments pr une théorie du système d'enseignement}, 1970.\\

\textit{distance habitus primaire et secondaire responsable de la \textbf{violence symbolique} exercée / école à son insu envers enfants des milieux moins favorisés. Rend compte à elle seule du phénomène de \textbf{reproduction des inégalités sociales} au coeur du fonctionnement de école.} \\

\begin{enumerate}
\item \textbf{capital culturel} : tte ressources culturelles d'un individu (bien, diplômes, rapport au savoir et école). dépend du milieu social. corrélé au capital éco (revenus, patrimoines) et social (relations sociales). \\
 \item \textbf{habitus} : système de représentations de individus, oriente son comportement, ambition, projets. façon de s'habiller, d'évoluer, de penser le monde.  Construite pdt phase de socialisation ds famille puis école.\\
 \item \textbf{Violence symbolique} : façon dt s'exerce fonction de reproduction de école. Fonction de reproduction des inégalités de école passe inaperçu à cause d'1 pensée : si je ne réussit pas, c'est que je ne suis pas doué. Selon Bourdieu, échec scolaire dû à trop grande distance chez enfant de classe dominée, entre habitus primaire (famille) et secondaire (école). Distance fait obstacle à intériorisation de habitus secondaire.\\
\end{enumerate}
\end{minipage}
}

\vspace{0.5cm}

\item Sur plan dvpt psycho de ado, pcpal enjeu de bonnes relations entre collège-parent : pas s'installer \textbf{dichotomie entre personne de enfant et personne de élève.} Famille : fil de  qq'1, à école : élève de classe.\\

\item relation parent-école ont fonction de \textbf{prévenir césure psycho} ou réagréger personnalité du jeune en lui montrant que parents et enseignants st acteurs conscients et partenaires de son dvpt.\\

\item Apprendre à ado qu'on est 1 tt en évoluant dans monde différent (école, famille, club de sport...) sans rompre identité perso. \\

\fbox{
\begin{minipage}{19cm}
\textbf{Ambivalence du concept de communauté éducative.} \\

DE SINGLY François, <<Communauté éducative ou société scolaire démocratique ? >>, MADIOT Pierre (dir), \textit{Enseignants, parents, réussite des élèves, quel partenariat ?}, 2010 \\

\begin{enumerate}
\item Pas 1 seule communauté éduc :1 scolaire, 1 familiale. parents 2nd ds la scolaire et 1er ds la familiale.\\
\item << Tant que nous raisonnerons avec une catégorie inexacte << communauté éducative >> pour ne désigner que la << communauté éducative scolaire >>, tensions resteront. >>
\end{enumerate}
\end{minipage}
}

\vspace{0.5cm}

\subsection{Les parents connaissent mal l'école de leurs enfants}


\item Indispensable que parents et enseignants partagent même valeurs constitutives du << contrat social >> français : refus de violence, refus des préjugés et des discrimination, dvpt du sens des responsabilités, de la solidarité entre les personnes. \\
\begin{enumerate}
\item qd pas le cas, étab scolaire peut pde initiatives citoyennes (organisation de débats, d'info impliquant parents travaillant ds secteurs de vie en société : justice, santé, aide sociale...).
\end{enumerate}

\item pr certaines familles menacées de rupture sociale, compréhension des missions de école difficile. \textbf{Faut donner à voir aux élèves et parents la légitimité de école}. But : aider personnel d'éducation à déminer tt ce qui peut faire obstacle aux bonnes relations entre parents et école.\\
\begin{enumerate}
\item niveau CPE : qualité du dialogue avec parents pt central de sa mission. Si veut aider ado à construire leur identité personnelle pr construire 1 projet professionnel.\\
\end{enumerate}

\item construction d'1 école de réussite pr ts implique que : au lieu de se regarder en chiens de faïence, parents et personnel s'épaule ds respect de complémentarité de leurs rôles respectifs.\\ 
\begin{enumerate}
\item chacun veut réussite des enfants. ts font de leur mieux et quand ils se rencontrent : svt dialogue de sourds. Svt, CPE parle et parents écoutent, inquiets ou affectants de l'être, face à ce qu'ils entendent.\\
 \end{enumerate}

\item meilleur présence des parents d'élèves ds école : permet soutien réciproque. Peut améliorer : 
\begin{enumerate}
\item condition d'accomplissement missions de étab scolaire \\
\item mise en situation d'apprentissage des jeunes \\
\item choix d'orientation désirés des jeunes pdt cursus. \\
\end{enumerate}

\item démocratiser école, c'est partager savoirs concernant son organisation, son fonctionnement, ses programmes.\\

\item \textbf{Familles comprennent - - une école devenue svt opaque.} Fonctionnement, paliers d'orientation, contenus, méthodes péda : rien à voir avec génération précédente.\\
\begin{enumerate}
\item médias communiquent pas sur école. Souffre qu'aucune émision tv ne lui consacre 1 rubrique sur système scolaire.\\
\item Ds d'autres pays européens, émissions régulières sur enseignement de physique, choix d'1 LV1 en primaire, orientation professionnelle.\\
\item Trop svt, documents relatifs à école primaire et programme st présentés comme s'ils avaient tjs existés. Explique pas aux parents pourquoi et en quoi ils ont changés.\\
\item Action << La malette des parents >>. mise en place en 2009 à Créteil. élargie à tte France. But = pas de se susbstituer aux réunions parents-profs ms expliquer aux familles organisation d'un collège.\\
\begin{enumerate}
\item bilan de cette action : plus d'implication des parents, plus de rdv individuels ac profs, plus d'investissements ds organisations de parents, plus gd contrôle des enfants à la maison.\\
\end{enumerate}
\end{enumerate}

\item Enjeu = construire 1 réseau de connivences autour de école comme il existait il y a 40 ans. Pr permettre aux parents qui ne comprennent pas, de pvr s'adresser à 1 autre parent résidant à proximité pr avoir réponses.\\
\begin{enumerate}
\item 2012 : âge moyen d'1 mère d'1 enfant scolarisé en 6e : 42 ans (source : INSEE). Elle a été scolarisée au tt début du collège unique. époque où barrière du collège était fin 5e. classe de 4e atteinte que par minorité.\\
\item si enfant atteignent cette classe, bcp de parents pensent que c'est la même école. croiront que enfant a passer pcpaux obstacles alors qu'aujourd'hui tt reste à faire.\\
\item Repères de école des parents et aujourd'hui pas les mêmes.
\begin{enumerate}
\item en 75, pour être parmi les 50 \% les + instruits de la génération 16-22 ans, il fallait obtenir le BEPC. 2014 : Bac + 2. \\
\item en 75, diplôme pr être embauché : CAP (diplôme de niveau V). Apparaissait comme sésame important. En 2014, aux yeux des familles ouvrières, est dévalorisé. Sésame minimum pr entrée ds monde du travail : Bac +2 (jugé accessible pr bcp d'enfants).\\
\end{enumerate}
\end{enumerate}

\fbox{
\begin{minipage}{19cm}
\textbf{Les dominés aux études longues}  \\

SCHWARTZ Olivier, \textit{Le Monde des débats}, janvier 2000 \\

<< jeunes familles pop st socialisés à école et ds jeunesse lycéenne. reproduiront pas culture de leurs parents. Ms pr autant, viendront pas s'intégrer aux classes moyennes. Ont svt 1 BEP, bac technique, 1 année de DEUG. avenir sera celui d'1 ouvrier, d'1 employé. seront soumis à domination, pression, précarité éco croissante. seront 1 groupe de dominés : trop scolarisés pr ressembler à des classes pop, trop précarisés pr qu'on puisse parler de classe moyennes.\\
\end{minipage}
}

\vspace{0.5cm}

\item Pr améliorer circulation de info entre école et famille : création d'associations d'ancien élèves montrerait que école lieu de construction de avenir. Atout pr donner espoir aux familles et aider ado à construire image d'1 collectif auquel il puisse s'identifier.\\

\item Messages contradictoires enseignants-parents peuvent que perturber représentations du jeune. Ds malentendus, trouve aussi problème de tte la société.
\begin{enumerate}
\item détresse et misère de certaines familles. Enseignants démunis.\\
\item multiplication lieux d'apprentissages qui fragilisent école : tv, jeux vidéos, internet. Flox d'info : élèves ne savent pas trier vrai ou faux. bcp adhèrent à groupes de pressions ss s'en rendre compte : ont besoin de se construire des << nous >> autre que le << nous >> familial, pr en extraire le << je >> . \\
\item incompréhension de certains professionnels face à situation de certaines familles. Lié au fait que CPE et famille vivent ds lieux différents, issus de milieux sociaux éloignés.\\
\end{enumerate}

\item pr construire bonnes relations parents-école, faut vaincre peurs réciproques, difficultés, tensions, incompréhension qui bloquent situation.
\begin{enumerate}
\item peur des familles bloquent discussion : trop svt voient prof que quand prob, peur du jugement de enseignant sur rôle de parent, peur face au pvr des enseignants et de institution, peur de avenir pr enfant.\\
\end{enumerate}

\subsection{Les conditions d'un dialogue véritable}


\item Dialogue entre adulte veut pas dire consensus permanent. Peut vouloir dire \textbf{confrontation exigeante de points de vue contradictoires}. Faut passer d'une situation de défiance à une situation de confiance.

\item quels obstacles faut-il éviter ?
\begin{enumerate}
\item Du côté personnel de étab : 
\begin{enumerate}
\item pas accepter discussion, débat. Critique tjs vécu comme mise en cause \\
\item considérer tt désaccord, tt pt de vue différent comme conflit impossible à résoudre \\
\item avoir des difficultés à prévoir, dès début de année, ds son emploi du temps, plages possible pr réunion avec parents.\\
\item considérer parents comme << utiles >> pr voyages scolaire, certaines activités, fête de école.
\end{enumerate}

\item Côté parents : 
\begin{enumerate}
\item parent pas tjs sur que jeune donne bonne info : d'où gêne qd CEPE leur apprend évènements qui a eu lieu à école.\\
\item réticence à entrer en conflit ac enseignant, surtt s'il apparaît que jeune << joue >> enseignant contre parents \\
\item inquiétude des parents à être en désaccord avec l'enseignant.\\
\end{enumerate}
\end{enumerate}

\item Faut préférer la négociation d'1 compromis où chacun sait ce qu'il accepte à 1 consensus non réellement négocié qui peut engendrer 1 sentiment de frustration. 2 logiques à oeuvre : 
\begin{enumerate}
\item Faut pas s'occuper de intérêt privé ms fonder règles au service du bien général. que chacun accepte ces deux positionnements : soutien affectifs, accompagnement du côté famille; apprentissages scolaire du côté école.\\
\item ce qui permet aux 2 pôles de se rejoindre : reconnaître l'autre, ds respect de son rôle, droit à une parole différente. Pr rencontrer l'autre ds  1 respect mutuel qui construit le dialogue, tjs bien identifier clairement objet de rencontre. \\
\end{enumerate}

\item pr communiquer avec famille, être conscient que jeune tjs être au coeur de la rencontre, et qu'il faut tjs, au préalable préciser règles de échanges en termes de temps, contenus et objectifs.


%\end{itemize}


\chapter{L'élève au centre de l'éducation : enseignement et individualisation}

\section{Le rapport au savoir, }

\textbf{Mots-clés : } \\

\begin{itemize}
\item Apprentissage
\item Savoir cognitif
\item Elèves en difficulté.
\item Accompagnement personnalisé. \\
\end{itemize}

\textbf{Manuel, 170}

%\begin{itemize}
\item Objectifs : 
\begin{enumerate}
\item Connaître les différents types de rapport au savoir pour dépisté la difficulté scolaire à ses débuts. \\
\item Mettre en œuvre des modalités d'accompagnement adaptées aux profils cognitifs des élèves. \\
\end{enumerate}

\item Approche psychanalytique et psychologique
\begin{enumerate}
\item Rapport aux savoirs lié au désir de savoir. Désir lié au sujet. Rapport au savoir d'un sujet se modifie et se reconstruit tout au long de sa vie.\\
\item avt entrée à école, état du rapport au savoir déterminé / construction de personnalité psychoaffective ds famille.\\
\item Sigmund FREUD, théorie du complexe d'Oedipe. Enfant attiré / parent du sexe opposé. Doit être refoulé. Au bout d'1 moment, peut remplacer appétence sexuelle / appétence non sexuelle (apprentissages culturels). \\
\item Nuancé / autres psy (pas aller jusqu'à pulsion sexuelle)
\begin{enumerate}
\item Henri WALLON. 1 des fondateurs de la psychogénétique. Montre rôle des débuts de acquisition du langage comme déclencheur de ts autres apprentissages. \\
\end{enumerate}
\item ttes ces théories ont points similaires : 
\begin{enumerate}
\item entrée ds apprentissage de type scolaire provient du franchissement de plusieurs stades successifs du dvpt affectif et psy de enfant.\\
\item qualité du rapport au savoir dépend en partie de celui de environnement familial.\\
\item maîtrise du langage : facteur décisif dans dvpt de capacité d'apprendre.\\
\end{enumerate}
\end{enumerate}

\item Approche qualitative.
\begin{enumerate}
\item ALAIN, \textit{Propos sur l'éducation}, 1932. 
\begin{enumerate}
\item but de école : donner du sens aux savoirs.\\
\item ds tt enseignement, il y a 1 obstacle qui rebute les élèves. Elève doit faire un \textbf{effort}. C'est ce qui heurte la compréhension immédiate qui suscite le désir d'apprendre.
\item Pédagogue doit jamais désespéré de ces élèves. Prof doit accorder tte attention à ceux qui n'y arrivent pas plutôt qu'à ceux qui y arrivent. \\
\end{enumerate}
\end{enumerate}

\item Approche sociologique.
\begin{enumerate}
\item Sociologues st ds sillage des travaux de BOURDIEU et PASSERON. Travaux aujourd'hui en partie validés. Établissent \textbf{lien entre origine sociale des élèves et difficultés d'apprentissage}. \\
\begin{enumerate}
\item Ms explications sur manque, déficit culturel, handicap socioculturel permettent pas de penser rôle de étab et des agents ds production des difficultés scol et les inégalités entre élèves. Travaux macrosocio ne disent pas gd-chose de nature du rapport des élèves au savoir.\\
\end{enumerate}

\item Approche qualitative fait objet attention nlle (cf Bautier, Charlot, Rochex). Distingue 2 profils d'élèves : 
\begin{enumerate}
\item ceux qui respecte la logique institutionnelle de cheminement \\
 
 \begin{enumerate}
 \item + svt en difficulté scol. Savoirs leur permettent juste de << faire face >> aux situations de leur vie quotidienne. Les savoirs ne st pas transférables ds autres situations, aucuns peut les aider ds expériences à suivre. \\
 \item travail requis se limite au respect des règles : se comporter correctement, faire ses devoirs, venir en classe, y avoir le matériel demandé. Prob avec savoirs abstraits (grammaire, mathématiques, physique). \\
 \item \textbf{Métier d'élèves : assumer succession d'exercice et moment qui font quotidien de vie de classe}. \\
 \item travail intellectuel, activités d'apprentissage, contenus et compétences par perçu pr ce qu'ils sont : disparaîssent derrière faisage des tâches et exercices scolaires.
 \end{enumerate}
 
\item ceux qui mettent en œuvre 1 logique d'apprentissage et de dvpt.

\begin{enumerate}
\item Voient ces exercices comme support d'1 activité cognitive. Comprennent pourquoi ont fait. Accompagne ces tâches.\\
\end{enumerate}
\end{enumerate}
\end{enumerate}

\item Travailler le rapport au savoir chez les élèves en difficulté.

\begin{enumerate}
\item Important que CPE (qd reçoit élèves), soit attentif à nature du rapport que élèves ont au savoir. Difficulté d'apprendre n'est ni une fatalité, ni une prédestination (Alain, 1930). \\
\item du CP en Term, 3 niveaux ds interprétation des activités (si veut repérer rapport au savoir des élèves) : \\
\begin{enumerate}
\item les tâches ponctuelles (exercices qui font quotidien de classe, << travail à la maison >>) \\
\item savoirs spécifiques à 1 discipline donnée (ayant contenus, normes et contraintes spécifiques)\\
\item le savoir en général. l'apprendre et le réinvestir dans des tâches complexes comme des outils. \\
\end{enumerate}

\item Elèves en difficultés doivent faire les 3 niveaux. Lors des études dirigées, doivent être accompagnés afin d'acquérir niveaux 2 et 3. On va plus se concentrer sur ce genre de gamins (centrés sur la tâche au détriment de la dimension disciplinaire).\\
\item A court terme, élève peut réussir les tâches ponctuelles. Ms ne savent pas mobiliser ces capacités dans d'autres situations, ni dirent ce que visent de tels exercices.\\
\item pr bcp de gamins, savoir = vérité. On sait ou non. Si on ne sait pas, on ne peut pas apprendre.\\
\item Ce rapport au savoir comme vérité à apprendre et restituer a csq nég. sur possibilité de s'engager ds 1 véritable activité d'apprentissage. Le savoir appris est 1 proportion admise provisoirement, qu'il faut mettre en l'épreuve pr construire des connaissances nlles.\\

\end{enumerate}

\item Exemple d'un moyen pr raccrocher les élèves au processus d'apprentissage : \textbf{L'accompagnement personnalisé (voir Accompagnement personnalisé)} \\

\textbf{Bibliographie} : 

\begin{enumerate}
\item BAUTIER Elisabeth, RAYOU P, \textit{Les inégalités d'apprentissage. Programme, pratiques et malentendus scolaire}, 2009
\item HIL, \textit{Le mérite et la république. Essai sur la société des émules}, 2007
\end{enumerate}

%\end{itemize}


\section{Accompagnement personnalisé, 174}

\textbf{Dans programme : }
\begin{itemize}
\item Partie 2 - Pédagogie
\item Section 1 : aide à l'élève dans son travail personnel. \\
\end{itemize}

\textbf{Mots-clés : } 

\begin{itemize}
\item difficultés d'apprentissage
\item 
\item 
\item  
\end{itemize}

\textbf{Manuel, 174}

%\begin{itemize}
\item Objectifs : 
\begin{enumerate}
\item Mettre en œuvre des modalités d'accompagnement adaptées aux profils cognitifs des élèves. \\
\end{enumerate}

\item dep 10 ans, neau voc pr nommé aide aux élèves : aide, accompagnement, soutien, modules, remise à niveau, consolidation, tutorat, étude dirigée... Termes svt qualifiés par << personnalisé >>, <<individualisé >> ou << différencié >>.\\
\item + récemment, notion a 2 termes : 
\begin{enumerate}
 \item accompagnement = être aux côtés de ts les élèves (pas juste ceux qui st en difficultés). Au collège, parle de accompagnement éducatif.\\
 \item personnalisé = démarche individuelle ou en petit groupe. A école primaire, ex: les PPRE (programme personnalisé de réussite éducative).\\
\end{enumerate}

\subsection{extension de accompagnement à ts les niveaux de la scolarité dep années 2000.}
\begin{enumerate}
\item Ecole maternelle et élémentaire :  élève qui a prob ds apprentissage peut bénéficier d'1 AP de 2 heures / sem, de travail en petit groupe + stage de remise à niveau à le fin du cycle 2.\\
\item collège, plusieurs dispositifs pr la personnalisation des enseignements et des parcours : PPRE, AP en 6e, PPRE passerelle et AE (Accompagnement éducatif). \\

\item qu'est-ce que l'AP ? \\

\begin{enumerate}
\item dispositif mis en place à rentrée 2007 ds les collèges de éduc prioritaire puis généralisé à rentrée 2008 \\

\item 2h/jr. En collège, annualisé, de préférence en fin de journiée, 4 jrs / sem. \\

\item 4 domaines privilégiés : 
\begin{enumerate}
\item aide aux devoirs
\item  pratique sportive
\item pratique artistique et culturelle
\item  pratique orale des langues vivantes.
\end{enumerate}
\end{enumerate}
\end{enumerate}
%\end{itemize}

\item Notion d'aide et d'accompagnement devenue familière pr élèves. Mise en place ds tt 2nd cycle depuis réforme de 2009. Apporte dimension nlle à exercice des métiers éducatifs. \\

\item au lycée, recherche de réponses diversifiée relève de même philosophie que socle commun au collège. Plus effectuée en dehors du tps de classe ms ds tps des élèves. Concerne ensemble des lycées. 2h / sem. équipes péda propose modalités d'organisation au conseil péda. Chef d'étab le soumet à approbation du CA. 

\begin{enumerate}


\item Plusieurs activités : 

\begin{enumerate}
\item soutien aux élèves en difficultés
\item approfondissement connaissances
\item aide à orientation (s'appuie sur parcours de découverte des métiers et formations).
\end{enumerate}
\item ATTENTION : BUT : travail transversal sur des compétences ancrées dans des disciplines. \\

\item différents niveaux : 
\begin{enumerate}
\item seconde : dep rentrée 2010. aide élèves à s'adapter aux exigence du lycée, acquérir méthodes de travail et construire projet d'orientation
\item 1ère, dep rentrée 2011. acquisition compétences propres à chq voie de formation, prépare élèves à commencer à se projeter ap bac. \\
\item  Term : rentrée 2012. enseignements spécifique à chq série, pr aider élèves à se préparer aux méthodes de enseignement sup. St en mesure de finaliser choix d'orientation. \\
\end{enumerate}
\end{enumerate}

\subsection{Une marge d'initiative au sein de la politique d'établissement}

\item Modalités laissées à initiative de équipe péda. But : répondre aux besoins des élèves de manière étroite et ac souplesse nécessaire. ttes les disciplines y participent.\\

\item accompagner : autre posture que celle de surplomb svt attribués à enseignant. Demande 1 travail important sur soi, une formation, des moyens, un travail d'équipe. Ne peut se faire qu'en réinterrogeant notre posture d'enseignant. \\

\item mise en place se repose sur politique d'étab dynamique et volontaire, aux objectifs définis.\\

\item Condition de utilité de AP : identifier besoin des élèves, mettre en place organisation acceptée de tous et élaborer dispositifs au contenu préparé et réfléchi.\\

\subsection{La place du CPE ds le dispositif d'AP}

\item Réforme du lycée (2009) offre nlles perspectives. Devient 1 intervenant pédagogique ds ateliers d'accompagnement. \\
\item CPE comme autres est force de proposition et membre du conseil péda et du CA.

\item Le conseil péda : 
\begin{enumerate}
\item présidé / chef d'étab. réunit au moins 1 PP de chq niveau, au moins 1 prof / champs disciplinaire, 1 CPE. \\
\item mission : favoriser concertation entre profs, pr coordonner enseignements, notation, évaluation des activités scolaires. \\
\item prépare la partie péda du projet d'étab. \\


\end{enumerate}


\item svt, CPE est référent de vie lycéenne. informe, forme et accompagne les élus du Conseil de vie lycéenne qui seront consultés sur orientation pr mise en oeuvre de AP.
\begin{enumerate}
\item au collège, travaille en partenariat avec enseignants sur mis en place d'ateliers péda. Peut organiser aide aux devoirs ds collège. \\
\item lycée : peut s'impliquer ds projet d'accueil. Ac PP et COP, pourra participer aux atelier << travail sur le projet personnel et l'orientation de l'élève >>. \\
\item a possibilité de suivre 1 élève sur 1 partie ou totalité de sa scolarité : tutorat
\begin{enumerate}
\item en relation ac PP et COP \\
 \item aide le lycée ds élaboration de son parcours de formation. \\
 \item guide élèves vers ressources disponibles (internes ou externes à étab)
 \item aide élève à s'informer sur poursuites d'études dans enseignement sup. \\
\end{enumerate}
\end{enumerate}


\subsection{Bilan d'étape}

\item rapport des Inspections générales << Mise en oeuvre de la réforme des lycées d'enseignement général et technique >>. Bilan en 2011.\\
\begin{enumerate}
\item critique défaillances ds diagnostic : question des besoins des élèves mal résolue. \\
\item certains nb de dérives : études surveillées, aide aux devoirs, récupération d'un temps d'enseignement disciplinaire durant moment attribués à accompagnement personnalisé. \\
\item selon élèves, répond pas à leurs attentes. \\
\end{enumerate}

\part{Professionnalité du CPE}

\chapter{L'histoire du métier de CPE}

\textbf{Mots clés : }
\begin{itemize}
\item 
\item 
\item 
\item 
\end{itemize}

\vspace{0.5cm}

\textbf{Objectif :}


\begin{enumerate}
\item connaître histoire de profession. \\
\item identifier invariant de fonction \\
\item compde mutations professionnelles au regards des mutations sociales \\
\item analyser conditions d'exercice du métier aux vues des évolutions récentes. \\
\end{enumerate}


\item Métier qui a le plus changé ds Educ Nat. Transformation tardive ms totale. Le Surgé et le CPE ont peu de points communs. \\

\section{La création}

\item 1er mai 1802 : Loi FOURCROY. Création du statut au moment de formation de l'ordre secondaire. Mentionne fonction de surgé ms pas vrai statut. \\

\item 1819 : création de la fonction. \\

\item 1847 (décret du 17 nov et circulaire du 20 dec) : définit fonction ds lycées. ds enseignement catho, appelation préfet des études conservée. \\
\begin{enumerate}
\item doit habiter au lycée ac famille. pas droit d'en sortir sans autorisation proviseur. garde-chiourme ac équipe de surveillants. \textbf{doit faire appliquer règlement et sanctions} pr bonne tenu de étab. Contrôle absences, retards, tenue, propreté, politesse. \\
\item fonction ingrate, répressive, destinée à rendre travail des autres possible ds bonnes conditions, en dépit des cancres. Objet de crainte, mépris, dérision pr élèves.
\end{enumerate}


\section{La triste mission du surgé}
\item veiller application peine, tenir à jour cahier punition, récompenses. Associé à tableau d'honneur (donne note de conduite), siège conseil discipline.\\

\item pratique professionnelle pas très élaborée. doit maintenir élèves ds plus grande soumission en étouffant esprit critique. \\

\item Antoine PROST << Le lycée impérial n'est pas l'école d'instruction des citoyens éclairés par la raison >> qu'avait imaginé Condorcet. Soumission, contrôle idéologique, inculcation de connaissances destinées à former officier de armée et économie docile. \\

\item pdt 1 siècle, fonction évolue lentement. Ni Guizot (qui a fait loi sur primaire), ni Ferry ne vt changer les choses. Vrai changement: après 2GM. \\


\item évolution d'abord factuelles : sur les sanctions. usage du clairon et tambour remplacé / << cloche >> pr rythmer journée lycéen. Châtiment corporel traditionnels (fouets, férules, prison, salle de pénitence, piquet de punition, arrêts au pain sec et eau, privation d'uniforme) remplacés / lignes, << pensums >>, retenus, privation de sorties. \\
\begin{enumerate}
\item Evolution TRÈS IMPORTANTE : début d'une évolution de fct qui devient le subsidiaire de l'enseignement. 1ère tentative de rapprochement des enseignants. \\
\end{enumerate}

\item régime républicain (s'installe durablement dans les années 1880), change rapport à autorité. relation maître-élève plus la même. \\
\begin{enumerate}
\item structures du lycée impérial reste. Ms 1ère tentative de réforme des lycées (après révolte), car refus d'autorité arbitraire et brutale / la jeunesse dorée des lycées. \\
\item Nécessité de redéfinir relation élèves- autorité, pose question des moyens de punir et d'éduquer. Qui doit maintenir l'ordre si conçu de façon éducative (lignes, conjugaisons à copier) ? Le prof ou le surgé ?
\end{enumerate}

\item profs on tjs refusés de s'y coller. 
\item Monitorats, tutoras ... France a jamais réussi à le mettre en place. Tt ce que doivent faire l'élève en classe, c'est entendre la parole magistrale. Le travail personnel, l'apprentissage, l'éducation, ce n'est pas l'affaire du professeur.\\

\section{Les années de l'après-guerre}

\item reconstruction de France et croissance éco (Trente Glorieuses). Période de massification des études secondaires et techniques. Bcp de catégories sciales nlles poursuivent leurs études.

\item construction neaux étab, dotés d'internats.
\begin{enumerate}
\item CET connaissent croissance importtes. Internat accueille pop nbeuse. Ont poste de SG de CET : office d'adjoint de direction. Nlle fonction aui oère une scission ds les SG : 
\begin{enumerate}
\item Les SG de CET, recrutés sur listes d'aptitude (pr bacheliers de 28 ans ayant 5 ans d'ancienneté de service) \\
\item les SG de lycées, recrurtés  sur liste d'aptitude (pr licenciés de 25 ans ayant 1 ancienneté de 3 ans de service d'enseignement) \\
\end{enumerate}
\end{enumerate}

\item 1ers textes sur SG de CET marquent reconnaissance timide de fct° éducatives.
\begin{enumerate}
\item est surtout 1 chef de service adjoint au directeur. Responsabilité en matière de gestion du personnel (mission d'encadrer travail des adjoints d'enseignement qd chargés service des études surveillées). Doivent organiser emploi du temps de tt ce personnel. \\
\item réalité du terrain, nature du public vt conduire à évolution rapide des tâches des SG de CET.\\
\begin{enumerate}
\item pas les mêmes publics d'élèves ni de SG (SG moins diplômé en CET, plus proche des élèves), élèves moins tournés vers le travail intellectuels.\\
\item  Elèves du CET moins mobiles (pas de voitures) -> majoritairement inscrits ds internat. Faut les encadrer et les occuper + pas de devoir le soir. En plus restent ouvert pdt petites vacances. \\
\end{enumerate}
\item Fonction du SG devient rapidement celle d'animateurs ds secteurs des loisirs. Dvpt des associations socio-éducatives (voir REMY, Les conseilleurs principaux d'éducation).

\end{enumerate}

\section{La remise en cause de mai 68 et ses conséquences}

\item 2 phénomènes : désengagement de institution scolaire ds socio-éducatif + malaise croissant de jeunesse. Cause commune = emballement de société de consommation. \\

\item fin 60's, dynamique des foyers socio-éduc marque le pas. Peinent s'implanter ds ts types d'étab. Proportion élèves internes diminue (amélioration conditions de vie des classe moyenne, démocratisation de voiture, augmentation transports collectifs).\\

\item Les collectivités locales prennent la main sur les activités périscolaires. Echappent aux CET et aux SG. Dvpt du tourisme et loisirs familiaux favorise pas vie collective de jeunesse et conquête de autonomie. \\

\item jeunes se sentent désoeuvrés ds ces << bahuts >> construits à hâte. \\

\item évènements de mai 68 prennent autorités / surprise. Lycéens se précipite à suite d'étudiants. Période de protestation, de gde utopie et de projets généreux pr rendre école plus fraternelle, plus ouvertes aux droits des jeunes... \\
\begin{enumerate}
\item importte remise en cause de fonction traditionnelle des SG, soudain placé dvt demande de dialogue entre élèves, enseignants et parents. \\
\item Edgar FAURE a su donner impulsion à qq réformes importantes pr la vie scolaire + modalité d'évaluation des élèves.

\fbox{
\begin{minipage}{19cm}
\textbf{Mai 68 : une aspiration à la réforme plutôt qu'à la Révolution} \\

ROBERT, André, \textit{L'école en France de 1945 à nos jours}, 2010, p. 78 \\

Réformes de Faure ont donné satisfaction parce que mvt de mai 68 pas mvt anti-scolaire ms demande de dialogue.
\end{minipage}
}

\vspace{0.5cm}

\item 8 nov 68. décret modifie composition CA lycées et collèges : ouverture aux usages (1/6e pr parents, 1/6e pr élèves) et personnalités extérieures. Impose participation de représentants de parents et élèves aux conseils de classe. \\
\end{enumerate}


\item Assouplissement de ensemble des règles de vie scol. nb d'interdictions ds règlements intérieurs diminuent, supprime certaines sanctions, reconnaît statut d'étudiant aux élèves de classe prépa. \\

\item sur plan péda : nlles pratiques (organiser tables en cercle ds salle de classe, changer mode de notation) ont poids symbolique même si pas adoptée partout. \\

\item 26 janv 68 : foyers socio-éduc réactivés pr ts étab du 2nd degré. But : organiser activités culturelles périscolaire sous responsabilité conjointe d'élèves et prof volontaire, de adm. \\


\section{Les 1ers statuts de CPE (70-72)}

\item 70 : nouveau statut transforme le SG en personnel d'éducation (décret n°70-738 du 12 août).  Crée 2 corps recrutés sur concours : 
\begin{enumerate}
\item corps de CPE ds lycées \\
\item corps des CE ds collèges \\
\end{enumerate}

 \item syndicats différents. \\
 
 \item 31 mai 72 : circulaire n°72-222 fixe leurs missions.
 \begin{enumerate}
 \item Continuité tâche de maintien de ordre.
 \begin{enumerate}
 \item application règlement intérieur
 \item respect installation où circulent et vivent élèves
 \item application consignes de sécurité
 \item contrôle présence élève aux différents moments de vie scol et à étab.
 \item organisation activité de substitution en cas d'absence d'1 membre du personnel. 
 \item mise en oeuvre discipline indispensable. auto-discipline ou plus contraignante = moyen d'éducation.
 \item organisation service des personnels auxquels st confiés tâches de surveillance, sécurité, animation... \\
 \end{enumerate}
 \item Affirmation tâche éducatives et pédagogiques 
 \begin{enumerate}
 \item faire comprendre aux élèves qu'on doit respecter règlement
 \item s'entretenir ac élève pr difficulté dans travail
 \item s'entretenir avec les parents
 \item assister aux réunions clubs et assoc socio-éduc
 \item faciliter rencontre d'élèves avec entreprises.
 \end{enumerate}
 \end{enumerate}

\item après ca, équilibre entre sécurité, éducation et pédagogie jamais remise en cause.

\section{Crises éco récurrentes et malaise des CPE (73-81)}

\item 73 : choc pétrolier. Récession éco. Chômage de masse. doute et crainte de avenir s'installe. En même temps, double perte pr système éduc : perte de crédibilité de son efficacité sociale et éco + perte de légitimité de ses référents idéologiques.\\
\begin{enumerate}
\item peut sortir diplômé et sans emploi. \\
\item familles pop ont investi ds démocratisation scolaire : ont enfant bcp plus titré que parents, mais en situation de quasi-exclusion sociale, alors que pères relativement intégrés malgré qualification moindres.\\
\end{enumerate}

\item accroissement décalage \textbf{exigences scolaires et références des élèves}.
\begin{enumerate}
\item rend discours de école incompréhensible pr jeunes des quartiers. Résultat : violence scolaire entre élèves ou entre élèves et personnel. \\
\end{enumerate}

\item Retournement / rapport esprit de mai 68 (demandant meilleure école). fin des idéologies utopiques et généralisation comportements désabusés. \\

\subsection{Conséquences sur le métier de CPE}

\item compliqué car confronté à double malaise : csq du malaise de jeunesse + difficulté de se situer dans étab ac fonction réelle en décalage ac textes qui régissent. \\

\item période de revendications, protestations syndicales et grèves fin 70's. Réclament neaux statut, assimilation complète aux catégories enseignantes, davantage d'autonomie professionnelle + réelle identité professionnelle. \\

\item mvt revendicatif qui signale profond malaise ds profession. Refus du pvr ministériel, détérioration conditions d'exercice -> multiplient manif et radicalisent actions : refus d'assurer rentrée des élèves internes.

\section{Les avances des années 80}

\item élection Mitterand : augmentation 300\% de postes aux concours.

\item 1982 : circulaire 82-482 du 28 octobre. donne satisfaction à plusieurs revendications, définition de vie scolaire, fixe maximum horaire hebdomadaire à 39 heures (rupture CPE/personnel de direction). Responsabilités en 3 domaines : \\
\begin{enumerate}
\item fonctionnement de étab : contrôle effectifs, assiduité élèves, organisation service des personnels de surveillance, mvt des élèves.\\
\item collaboration ac personnel enseignant : échanges d'info sur comportement et activité élèves (résultats, condition travail, recherche des difficultés et intervention pr les empêcher), collaboration ds mise en oeuvre de projet. \\
\item animation éducative : relation et contact ac élèves (classe, groupe) et individuels (comportement, travail, prob perso); foyer socio-éduc et organisation des tps de loisirs (clubs, activités culturelles); organisation de concertation et participation (formation, élection et réunion élèves délégués, participation conseils d'étab...) \\
\end{enumerate}

\item inscrit CPE au cœur du projet d'étab. Fait de lui membre de équipe éducative et 1 des principaux artisans du pcpe central de loi d'Orientation de 89 : \textbf{mettre élève au cœur du système éduc.} \\
\begin{enumerate}
\item importance de la notion de travail en équipe \\
\item dvpe fonction de suivi et d'aide personnalisé ds pratique quotidienne. gde collabo ac enseignants.
\item devient réellement interface entre famille, enseignants et autres partenaires (dans et hors des murs).\\
\end{enumerate}


\section{Vers la fin de la division du travail professeur / éducateur}

\item fusion des CE et des CPE. Ds but de créer une culture commune, CPE formé en ESPE ac enseignants de ts niveaux.\\

\item formation initiale analogue à enseignant. fonctionnaires de catégorie A. Même déroulement de carrière que enseignant.\\

\item Ms depuis qq années, sollicités pr situations nlles de décrochage scolaire, rattrapage, inégalités plus flagrante, lutte vs exclusion. Objectif de égalité des chances ne cesse de s'éloigner. Objectif du vivre-ensemble, du lien social, du respect d'autrui et de société devient prégnant -> Nl objectif pr << vie scol >> : favoriser accession à citoyenneté. \\

\item Elèves devenus consommateurs d'école. Absentéisme inquiétant. CPE 1 des pcpaux acteurs pr prob. Ms fort difficile à régler.\\

\item Violence à porte de école. \\

\item ds textes récents, essait de trouver réponses : partage des tâches éducatives et pédagogique entre ts membre de équipes éducative.

\fbox{
\begin{minipage}{19cm}
\textbf{IGEN Groupe EVS, Le métier de CPE aujourd'hui : qq repères. Revue Conseiller d'éducation, mars 2006.} \\

d'après DELAHAYE JP, Le Conseiller principal d'éducation, 2009 \\

\begin{enumerate}
\item importance de certaines missions : prévention violence et traitement conduites à risques, luttes vs inégalités, éducation à citoyenneté, introduction du droit à l'école.\\
\item CPE doit procéder à des régulations pr dépasser certains clivages vie scol - vie de classe.
\end{enumerate}
\end{minipage}

}


\vspace{0.5cm}

\item mais attention danger.

\fbox{
\begin{minipage}{19cm}
\textbf{Les dangers de la spécialisation des fonctions éducatives} \\

MEIRIEU Philippe, << Réflexion sur la profession >>, \textit{Les Valeurs du métier de CPE}, 2003 \\


\begin{enumerate}
\item Attention au fait de créer des postes qd on a des besoins. \\
\item Il fallait créer des profs-docs car profs ne pouvait plus gérer tt le côté documentaire. Ms leur existence de peut décharger enseignants du souci documentaire.\\
\item Il fallait créer CPE : à partir du moment où règles scolaire plus construites mentalement / élèves avt école. Fallait créer poste qui crée loi avec élève. CPE acteur essentiel ms ne peux pas exempter autres acteurs et faire de citoyenneté sa chasse gardée. Il doit partager avec tte l'équipe pédagogique. Personnage clé à condition que son interlocution soit reconnu, et que, sans empiéter sur attributions des enseignants, il garantisse cohérence de construction de cité scolaire.
\end{enumerate}
\end{minipage}

}

\chapter{L'identité professionnelle du CPE}


\textbf{Mots clés : }
\begin{itemize}
\item 
\item 
\item 
\item 
\end{itemize}

\vspace{0.5cm}

\textbf{Objectif :}


\begin{enumerate}
\item Comprendre sens de aboutissement des évolutions historiques des fonctions du CPE
\item Situer la profession au coeur de autonomie des EPLE
\item Distinguer intervention de expertise
\item articuler le rôle pédagogique du CPE ac équipe des enseignants \\
\end{enumerate}

%\begin{itemize}

\item Identité professionnelle du CPE : fruit de histoire. Fruit d'une séparation qui diminue entre transmission des savoirs et encadrement éducatif.\\

\item Peut pas affirmer que anciennes fonctions du surgé ont pas complètement disparu du cahier des charges. Ms \textbf{nlle compétences se st ajoutées aux traditionnelles.} Métier a évolué.
\begin{enumerate}
\item  Mutation irréversible : CPE doit dépasser accomplissement tâches répétitives. \\
\end{enumerate}

\item tenir compte du contexte d'évolution de la profession. 
\begin{enumerate}
\item attentes extérieures => nlles normes professionnelles. \\
\item contexte de crise et affaiblissement valeurs républicaines. Démocratisation de enseignement et massification st accompagnées de \textbf{neaux phénomènes inquiétants} : échec et décrochage scolaire, apparition nlles conduites à risques, montée de violence. \\
\end{enumerate}

\item afin de lutter, école se transforme.
\begin{enumerate}
\item autonomie, décentralisation éducative, décloisonnements des responsabilités des acteurs des EPLE : \textbf{politiques éducatives d'étab st mises en place. CPE y interviennent} \\

\item \textbf{CPE doit compde évolution du monde éducatif et connaître sa place} \\

\item doit faire recherche (système éduc fr et étrangers) pr notions indispensables à exercice de son travail.
\end{enumerate}

\section{Le CPE doit dépasser l'assignation réductrice du maintien de l'ordre et de la discipline}

\item Gestion de assiduité, surveillance élèves rt organisation de équipe vie scol ont changé de sens. Analyse de absentéisme : \textbf{clignotant majeur ds lutte vs échec et décrochage scolaire}, repère essentiel ds conduites à risques \\

\item CPE au centre. Entouré de partenaires pr gérer hétérogénéité, exercer autorité en réglant conflits perturbateurs, situations de tension. Conserve capacité à compde, former et intéresser élèves. Doit établir climat favorable aux apprentissages et réussite des élèves, conforter autorité formatrice des adultes tt en donnant paroles aux élèves. \\

\item doit veiller, avec AED,  à prévenir et suivre évolution de phénomènes multiples de violences et incivilité.

\section{Le CPE est partie prenante de politique éduc de EPLE}

\item animateur pas exécutant. Nécessaire qu'il sache \textbf{conceptualiser ses interventions quotidiennes}, observer dispositifs existants pr établir un projet éducatif partagé.\\

\item Doit s'impliquer ds collectif éducatif. \textbf{On ne se répartit plus les tâches, on les partage}. = chaque acteurs doit connaître le projet d'étab, poursuit les mêmes objectifs et identifie son rôle.\\

\item CPE est \textbf{conseiller du chef d'étab}. Appartient pas statutairement  à équipe de direction. Ms peux pas faire son travail sans y être associé. doit aider à orienter choix de politique éduc effectué / équipe de direction ds cadre du projet d'étab. \\

 \item CPE \textbf{exerce 1 expertise en matière éducative.} Dirige équipe qui doit participer pleinement à politique éduc de étab, fortement intégrée à politique pédagogique. Idée de combattre cloisonnements ds apprentissages sociaux, comportementaux et disciplinaires. \\


\item Tableau montrant la contribution du CPE à politique éducative de EPLE (selon protocole d'accord relatif aux personnels de direction, 16 novembre 2000) \\
\begin{tabular}{|p{8cm}|p{8cm}|}
\hline Responsabilité chef d'étab & contribution CPE \\ 
\hline Contrat d'objectifs et projet d'étab & connaissance public // pts forts et faibles de étab // élaboration axe vie scol du projet étab // connaissance indicateurs << vie de élèves >> de LOLF (tx d'absentéisme...) \\ 
\hline préside CA & membre de droit du CA \\ 
\hline préside réunion CESC, impulse politique d'éducation à santé (prévention conduites à risques) & membre CESC // participe actions du CESC \\ 
\hline préside conseil discipline & Membre conseil discipline \\ 
\hline préside comité hygiène et sécurité & Membre \\ 
\hline constitue classe & connaît élèves \\ 
\hline dvpt pédagogie soutien et aides individualisées pr élèves en difficultés (PPRE), fait accompagnement éduc & aide repérage élèves // aide organisation action éducative // AED aide aux PPRE, accompagnement éduc \\ 
\hline régule modalité d'évaluation élèves & participe conseils de classe \\ 
\hline conduit politique éduc & suit assiduité // favorise modalité expression élèves (conseils de vie lycéenne, maisons des lycéens...) \\ 
\hline crée conditions accueils élèves en dehors des heures de classe & suivi animation, restauration, clubs, internat \\ 
\hline conduit politique orientation : favorise émergence projet perso de élève & suivi individuel // organisation mini-stages, portes ouvertes, réunion ancien élèves... \\ 
\hline 
\end{tabular} 

\begin{flushright}
Source : DELAHAYE JP, \textit{Le Conseiller principal d'éduc}, 2009, p. 53-54
\end{flushright}

%\end{itemize}

\section{Le CPE membre de l'équipe pédagogique}

\item Transmet savoirs et connaissances (aspect quotidien de socialisation +  intervention heure de vie de classe). Passe de animation de clubs du FSE à formation de classes ou groupes d'élèves sur des objectifs du socle.\\

\item Ac enseignants, prof doc, COP ou infirmière, participe à éducation à santé, orientation, citoyenneté. Présent dans conseil péda = pas que symbolique. Force de propositions. \\

\item Note de vie scolaire (évaluation compétences sociales et civiques).\\

\item But : favoriser autonomie et initiative (compétence du socle) : / formation représentants élèves, utilisation pédagogiques des heures de vie de classe, animation vie des élèves ds étab, accompagnements projets lycéens.\\

\section{Conseils aux CPE pr les entretiens ac les familles}

\subsection{Tenir 1 historique des rencontre avec les familles}

\item but : meilleur réussite du jeune. doit tenir 1 cahier des rencontres (pr savoir ce qu'on a évoqué aux rencontres précédentes). Doit indiquer pr chq rdv : 
\begin{enumerate}
\item qui a provoqué rdv \\
\item quelles info données / CPE, / famille \\
\item quelle question laissée en suspens, devant faire objet entretien ultérieur \\
\item quelle décisions prises \\
\item potentialités de travail en commun ac famille ? prob posés / cet entretien ? \\
\end{enumerate}

\subsection{Ce qu'il ne faut surtout pas faire lors d'une rencontre parent-CPE}

\begin{enumerate}
\item se laisser accaparer, déborder et ne pas pvr dvper les éléments essentiels à communiquer\\
\item adopter attitude fataliste : << Il n'y a plus rien à faire... >>, << son frère était déjà comme ça >>. \\
\item se mettre en situation de se justifier systématiquement pr prouver que sur tt les points, on a raison et les parents tort. \\
\item blamer en permanence attitude des parents\\
\item penser qu'on pourra résoudre prob seuls. Pas hésiter à dire aux parents de consulter autres professionnels \\
\item pas faire appel à médiateurs possibles : délégués des parents, responsable parents d'élèves étad \\
\item pas être agressif si parents sont. Maîtriser son comportement pr éviter tt débordement\\
\item ne pas répondre à demande de rencontre individuelle sous forme de convocation, mais bien être présentée comme un rdv.\\
\item être prudent dans utilisation du mot écrit.
\end{enumerate}

\section{Conclusion : le sens des évolutions récentes de la fonction de CPE}

\item en 40 ans passe du centralisme à autonomie. Modifie professionnalité de ensemble des acteurs éducs : 
\begin{enumerate}
\item obligation d'analyser particularités et besoins éducs de environnement direct \\
\item nécessité d'élaborer projets d'action articulés autour d'1 axe défini (projet acad, d'étab, vie scol) \\
\item besoin de dépasser cadre de ses responsabilités propre, partager, échanger, travailler ac partenaires. \\

\end{enumerate}

\item Sortie du cadre de ses 3 responsabilités traditionnelles (surveillance, gestion des absences et équipe vie scol) + entrée ds collectif éducatif de EPLE. Doit tjs assumer responsabilités trad pr mieux partager ac autres acteurs responsabilités de formation et éducation de ensemble des individus élèves. \\

\item Points importt : 
\begin{enumerate}
\item participation au \textbf{collectif de scolarité} de chq élève / évaluation permanente \\
\item entrée officielle et effective ds \textbf{équipe péda} \\
\item \textbf{ élargissement de la fonction pédagogique} d'animation  : heure de vie de classe, org et représentativité des élèves (apprentissage de démocratie), formation représentants élèves \\
\item \textbf{participation à élaboration projet d'étab}. Quitte son bureau, participe à élaboration politique éducative de étab, force de proposition \\
\item doit pas perdre de vue \textbf{déontologie du CPE face aux usagers} (parents, élèves, ensemble membre de communautés éduc) \\
\item \textbf{conceptualiser ensemble de ses interventions quotidiennes}, être capable de dire ce qu'il fait et pourquoi il le fait. \\

\end{enumerate}

\part{La loi du 8 juillet 2013 d'orientation et de programmation pour la refondation de l'école de la République}

\chapter{Éléments de contexte, 207}

Objectifs : \\
\item Compde gds objectifs de cette loi. \\
\item Mettre en perspective nlles mesures au regards des évaluation du système éduc française. \\
\item Identifier enjeux de réforme pr missions du CPE. \\

\section{Publication au JO et communiqué du ministre.}

\item passée au JO le mardi 9 juillet 2013. \\

\item Communiqué de presse de Peillon, 2013.
\begin{enumerate}
\item confirme création de 60.000 postes.\\
\item 1er degré : efforts + importts. Redéfinitions missions de école maternelle. renforcement liens collège.\\
 \item mise en place ESPE. \\
 \item création service public du numérique éducatif : mise en ligne ressources pédagogiques et logiciels au service prof, élèves, parents. véritable éducation aux médias, clé de citoyenneté.\\
 \item  PEDT (projet éducatifs territoriaux) : mise de la concertation locale au coeur de question éducative. Ds ce cadre, va élaborer projet qui pd en compte globalité du temps de enfant (scolaire, périscol, extrascol).\\
 \item valeurs rappelées / affichage de DDHC + apposition drapeau et devise de Rep sur façade.
\end{enumerate}

\section{Les textes règlementaires d'accompagnement.}
\item 15aine de décrets d'application en attente : élèves en situation de handicap, SCCC ...
\begin{enumerate}
\item  Handicap (art. 7) : loi précise que coopération entre étab de Educ nat / étab sociaux ou médico-sociaux organisé / conventions. But : assurer continuité du parcours de scolarisation des élèves en situation de handicap. Décret qui précisera modalité d'application. \\
\item Socle commun de connaissance, de compétence et de culture (art. 13) : voir décret.\\
\item Droit à une formation qualifiante (art 14) : << Tt jeun sortant du système éducatif sans diplôme bénéficie d'1 durée complémentaire de formation qualifiante >>.  condition fixée / décret. \\
\item Conseil supérieur des programme (art. 36). voir décret. \\
\item Conseil national d'évaluation du système scolaire (art. 33). voit décret. \\
\item Cycles (art. 34). voir décret. \\
\item Condition d'attribution du brevet (art. 54) : << atteste la maîtrise de socle commun de connaissances, compétences et cultures.>> voir décret. \\
\item Bac (art. 55). << examen sanctionne formation équilibrée qui ouvre la voie à poursuite d'études sup ou insertion professionnelle. comporte vérification d'1 niveau de connaissances, compétences et cultures définies / prog. >> voir décret. \\
\item Conseils école-collège (art. 57). voir décret \\
\item Conseil d'école (art. 59). voir décret. \\
\item GRETA (art 62). \\
\item fonds d'amorçage (art. 67). crée 1 fond d'aide aux communes pr mise en place des rythmes scolaires. \\
\item ESPE (art. 70) : décret. \\
\item Maîtres du privés (art. 80 et 81) : décret.\\
\item Comité de suivi (art. 88) : chargé d'évaluer application de la loi.\\
\end{enumerate}

\section{Les nouveaux organismes créés}

\subsection{Le conseil sup des programmes}

\item Rétabli conseil national des programmes qui avait existé entre 89 et 2005. composé à parité de 18 membres désignés pr 5 ans : 3 sénateurs, 3 députés, 2 membres du Conseil éco, social et environnemental (CESE) et 10 personnalités qualifiée nommées / ministre. \\

\item proposition sur : 
\begin{enumerate}
\item conception gale des enseignements + intro du numériques ds méthodes péda et construction du savoir. \\
\item contenu du SCCC et des prog. scolaires. \\
\item nature et contenu des épreuves. \\
\item nature et contenue des épreuves des concours de recrutement, objectifs et conception de formation initiale et continue des enseignants. \\
\end{enumerate}

\subsection{Le conseil national d'évaluation du système scolaire.}

\item suite du Haut conseil de éducation. Parité de 2 sénateurs, 2 députés, 2 membre du CESE, 8 personnalités.\\

\item But : évaluer organisation et résultat de enseignement scolaire. Réalise évaluation, se prononce sur méthodo et outils des évaluations et sur leurs résultats. \\

\item remet rapport. rendus publics. \\

\subsection{Le Conseil national de l'innovation pour la réussite scolaire}

\item représentants services académiques (recteurs, DASEN) et acteurs de terrain. Mission : impulser esprit d'innovation en matière de réussite scolaire et éducative, faire recenser et diffuser pratiques innovantes de terrains pertinantes.

\subsection{Le Conseil national éducation économie}

\item 5 représentants des syndicats employeurs, 5 chefs d'entreprises, 5 représentants salariés, 5 représentants profs, 2 conseillers régionaux et représentants ministres.\\

\item but : animer réflexion prospective sur articulation système éducatif / besoin monde éco.\\

\subsection{Le Haut conseil éducation artistique et culturelle}
\item présidé / 2 ministres de Educ Nat et Culture + 24 autres (8 représentants de Etat, 8 des coll Tales, 6 personnalités désignées pr compétences et issu monde de éducation, 2 représentants parents d'élèves).


\chapter{Vue d'ensemble, 212}

\section{4 objectifs}

\begin{enumerate}
\item Elèves doivent maîtriser compétences de base en français (écriture, lecture, compréhension, vocabulaire) + math (nb, calcul, géométrie) en fin de CE1 + amîtrise instruments fondamentaux de connaissance en fin d'école élémentaire. \\

\item Réduire de 10\% écart de maîtrise des compétences en fin CM2 entre élèves de éducation prioritaire et autres. \\

\item Réduire \ 2 nb d'élèves qui sortent du système sans qualification ° amener ts élèves à maîtriser socle commun. \\

\item réaffirmer objectif : 80\% d'1 classe d'âge au bac, 50\% ac 1 diplôme de enseignement sup.
\end{enumerate}

\item Extraits du rapport annexé
\begin{enumerate}
\item << La refondation a pour objet de faire de l'école un lieu de réussite, d'autonomie et d'épanouissement pour tous>>.
\end{enumerate}

\section{Les missions de l'école}

\item Loi réaffirme missions du système éducatif. Change article 1 du Code de l'éducation : << L'éducation est la première piorité nationale. Le service public de l'éducation est conçu et organisé en fonction des élèves et des étudiants. Il contribue à l'égalité des chances et à lutter contre les inégalités sociales et territoriales en matière de réussite scolaire et éducative. >> \\

\item Reconnait que 
\begin{enumerate}
\item ts les enfants ont capacité d'apprendre et de progresser. Ecole doit veiller à inclusion scolaire. \\
\item mixité sociale des publics. recommande participation des parents qq soit leur origine. \\
\item transmission des connaissances doit pas faire oublier importance des valeurs de République. : respect de dignité des êtres humains, liberté de conscience, laïcité, sens de coopération entre élèves. Personnel doivent mettre en oeuvre ces valeurs. \\
\end{enumerate}

\item droit à éducation garanti à chacun pr lui permettre de dvper sa personnalité, d'élever son niveau de formation initiale et continue, de s'insérer ds vie sociale et professionnelle + exercer sa citoyenneté. \\
\begin{enumerate}
\item aides attribués aux élèves selon ressources et mérites. Renforcer encadrement des élèves ds écoles et étab ds zone d'environnement social défavorisé et zone d'habitat dispersé. But : permettre au élèves en difficulté de bénéficier d'actions de soutien individualisé. \\
\end{enumerate}

\item précise lutte contre illétrisme. Priorité nationale. \\

\item Obligation scolaire. Rend possible la prolongation de la scolarité obligatoire. << Présente disposition ne fait pas obstacle à l'application des prescriptions particulières imposant une scolarité plus longue. >> Pvr permettre l'acquisition d'une formation professionnelle qualifiante : tt jeune doit se voir offrir, avt sa sortie du système éducatif et quel que soit niveau d'enseignement qu'il a atteint, une formation professionnelle. \\
\begin{enumerate}
\item But : réduire nb d'élèves qui sortent sans qualification. Tt élève qui n'a pas atteint 1 niveau de formation sanctionné / 1 diplôme national ou 1 titre professionnel doit pouvoir poursuivre des études pr l'acquérir.\\
\item tt jeune sortant sans diplôme bénéficie d'1 durée complémentaire de formation qualifiante. Peut être 1 droit au retour en formation initiale sous statut scolaire. \\
\item ATTENTION : diplôme minimum n'est plus le brevet ms 1 diplôme professionnel obtenu ds cadre d'1 2nd cycle de l'enseignement secondaire. \\
\end{enumerate}


\chapter{Les domaines prioritaires de la refondation}

\section{Evaluation et orientation des élèves.}

\subsection{Pour une évaluation constructive}

\item Rapport annexé à la loi du 8 juillet 2013. 
\begin{enumerate}
\item Doit faire évaluer notation pr éviter << notation-sanction >>. Privilégié notation positive, simple, lisible, valorisant progrès, encourageant initiatives et compréhensible / familles. \\
\item But : mesurer degré d'acquisition des connaissances et compétences + progression de élève. \\
\item réforme du livret personnel des compétence actuel. Trop complexe.
\end{enumerate}

\subsection{L'orientation et l'insertion professionnelle contre le décrochage scolaire}
\item mise en place d'un parcours individuel d'information, d'orientation et de découverte du monde économique et professionnel. \\
\begin{enumerate}
\item aider élève à élaborer son projet d'orientation scolaire et professionnel + éclairer ses choix d'orientation. Si on veut qu'orienter ne signifie pas être relégué, faut placer orientation au coeur de la scolarité de chq élèves, comme processus dynamique que chacun doit poursuivre ac concours de ensemble de communauté éducative. \\
\end{enumerate}

\item Après concertation du conseil de classe, idée d'imposer choix des parents. Rend élève acteur de son orientation au lieu qu'il subisse décision du conseil de classe. Va avec limitation du nb de redoublement : pratique coûteuse dt efficacité péda pas probante. \\
\begin{enumerate}
\item ms pr faire d'une orientation subie, une véritablement choisie, députés ont précu 1 couperet de fin de 1er cycle du second degré.\\
\end{enumerate}

\item Prob : poursuite vers lycée professionnel ou apprentissage plus svt décidée en fonction des carences des élèves ds disciplines générales qu'en raison de choix d'orientation positif de élève. Or ce st voies d'études positives pr élèves motivés. \\

\item très importt d'accompagner élèves dans construction d'1 projet d'orientation scolaire et professionnel et d'éclairer choix d'orientation.  Jeune doit se familiariser ac monde économique et professionnel (1ère connaissance du marché du travail, des professions, métiers, rôle et fonctionnement de entreprise + modalités d'insertion professionnelle.)
\begin{enumerate}
\item loi recommande large ouverture de école sur son environnement : témoignages de professionnels aux parcours éclairants + initiatives organisées ac régions (visites, stages, découvertes des métiers et entreprise, projets pr dvper esprit d'initiative et compétence à entreprendre) \\
\item Ne se limitera plus à une option de découverte professionnelle proposée uniquement aux élèves destinés à enseignement professionnel. S'adresse à tous. Doit trouver sa place dans tronc commun de formation de la 6e à la 3e. Se prolonge au lycée. \\
\end{enumerate}

\item Autre dimension : formation tt au long de vie. Collaboration renforcée entre Etats et régions. mission : << rendre effectif droit de tte personne d'accéder à 1 service gratuit et d'améliorer qualité d'information sur les formations >>. \\
\begin{enumerate}
\item but :  diviser / 2 nb des sortants sans qualification. \\
\item projets d'étab doivent mobiliser équipes éduc autour d'objectifs précis de réduction de absentéisme, 1er signe de décrochage.
\begin{enumerate}
\item ds  collèges et lycée pro (où tx décrochage élevé), 1 référent aura charge de prévention, suivi élèves décrocheurs, relation ac parents, aide au retour en formation des jeunes pr obtention d'1 diplome national ou d'1 titre professionnel de niveau V. \\
\item Tt jeune sorti du système éducatif sans diplome doit pvr disposer d'1 durée complémentaire de formation qualifiante. Pleine responsabilité aux régions ds domaine des formations professionnelles (apprentissage ou formation initiale)
\end{enumerate}
\end{enumerate}

\subsection{Le contenu du socle commun de connaissances, compétences et culture}

\item ds ancienne bouture : élément essentiel ds processus de démocratisation de réussite des élèves. Supposé permettre de suivre, / évaluations périodiques, acquisition / chq élèves des compétences nécessaire à poursuite d'étude.\\
\item va être repensé ds décret à venir. Mise en oeuvre actuellement trop imparfaite, mal comprise / enseignant. \\
\item Ajout mention << de culture >> : but : qu'il ne se réduise pas au français et math

\section{Scolarisation précoce en maternelle}

\item Amélioration de ttes les charnières entre école et vie (vie familiale, sociale, professionnelle).
\item études montre que école maternelle permet de réduire inégalités. Favorise éveil de personnalité des enfants, stimule dvpt sensoriel, moteur, cognitif, social, dvpe estime de soi.

\section{Articulation école-collège}

\item charnière la plus problématique. Malgré unification ancienne du système éducatif en niveaux successif, continue de poser problème.
\begin{enumerate}
\item socle commun pourrait y remédier si une école du socle entrait véritablement ds pratiques de ts acteurs.\\
\item mettre contenu des enseignements et progression des apprentissages au coeur de la Refondation : instituant conseil école-collège. peut proposer enseignements ou projets péda communs à élèves du collège et école. idée de liaison marche pas si pas implication des acteurs. \\
\begin{enumerate}
\item conseil propose actions de coopération, enseignements et projets pédago communs visant acquisition du SCCCC. Exemple : échanges de pratiques en d'enseignants. \\
\item conseil chargé de maîtrise du socle -> collège unique doit sortir de son isolement et renforcer liens.

\end{enumerate}
\end{enumerate}

\item Evaluation compétences des élève : élément systématique de refondation. Doit tisser lien, favoriser échanges d'avis et de pratiques. \\

\item condition d'attribution du DNB inchangé. \\

\item mise en place du conseil école-college pdt année 2014. 1er programme d'action : rentrée 2014. \\

\item Se réunit au moins 2 fois par an. conseil comprend : 
\begin{enumerate}
\item pcpal collège ou son adjoint
\item inspecteur chergé de la circonscription du 1er degré
\item personnels désignés / pcpal sur proposition du conseil péda du collège.
\item membre du conseil des Maîtres de chacun des écoles. \\
\end{enumerate}

\item décret 24 juillet 2013. organise 4 cycles de 3 ans mis en place à partir de rentrée 2014 : 
\begin{enumerate}
\item cycle école maternelle
\item cycle CP/CE2
\item cycle CM1/6e
\item cycle 5e/3e \\
\end{enumerate}

\item rentrée 2014 pr maternelle, 2015 pr CP, CM1, 5e. divisent syndicats enseignants : râle sur difficulté du corps enseignants à renoncer à logique des structrues qui accompagne division des corps enseignants à chq niveau. \\

\section{La scolarisation des élèves en situation de handicap}

\item notion d'école inclusive. Traduit évolution historique de école durant 2nde moitié du 20e siècle pr individualisation de enseignement. \\
\item fait social : participation et citoyenneté des personnes handicapées favorisent dvpt rapide de scolarisation en milieu ordinaire des enfants et ado en situation de handicap.\\

\section{Faire entrer l'école dans l'ère du numérique.}

\item But du service public du numérique éducatif et de l'enseignement à distance : 
\begin{enumerate}
\item mettre à disposition des élèves 1 offre de services numériques pr prolonger offre des enseignements qui y sont dispensés. \\
\item proposer aux enseignants ressources péda, contenus, services contribuant à formation. \\
\item assurer instruction des enfants qui peuvent pas être scolarisés ds étab scolaires.\\
\item contribuer au dvpt de projets innovants et expérimentations péda. \\
\end{enumerate}

\section{Dialogue de école avec partenaires}

\item meilleur dialogue école-famille : intégrant plus largement à ouverture de école sur son environnement. \\

\item promotion de la coéducation : 1 des pcpaux leviers de refondation. Encourage participation accrue des parents à action éducative. Accorder 1 attention particulière aux parents les + éloignés de l'institution scolaire / dispositifs innovants et adaptés. \\

\item  loi prévoit que étab, lors de sa construction ou reconstruction, doit comporter 1 local pr les parents d'élèves au même titre qu'une salle pr les profs. Lieux dédiés aux rencontres individuelles et collectives. \\

\item loi a pr but de s'ouvrir aux parents, pr favoriser réussite des élèves, ms aussi aider parents en situation d’illettrisme à entrer eux-même dans 1 démarche d'apprentissage, pr sortir de la gde précarité. \\
\begin{enumerate}
\item pas encore de décret qui dit comment. Ms aucun doute que CPE sera concerné au 1er chef. Intervention se situant à interface de étab et de son environnement / contact parents, élève, étab.\\
\end{enumerate}

\item partenariat aussi ds secteur associatif.

\section{L'organisation du temps scolaire}

\item concerne pr l'instant que école primaire. dvpée ds rapport annexé et circulaire de rentrée de 2013. \\

\item  écoliers, collégiens et lycéens ont journée plus chargée que celle de plupart des autres élèves dans le monde. Conséquence défavorables, surtt pr enfants rencontrant difficultés. \\

\item But de semaine à 4 jours ds primaire : permettre à école d'assurer aide au travail personnel dans temps scolaire et offrir à petits groupes d'élèves, après tps classe, activités pédagogiques complémentaires.  Interdiction devoirs écrits à maison, source d'inégalité des élèves, pourra devenir efficace. \\

\item aucune mesure annoncée pr 2nd degré.

\item CPE doivent intervenir ac équipe de direction des EPLE pr conception des emplois du temps pr répartition équilibrée et régulière des temps scolaires et périscolaires.

\section{La laïcité}

\subsection{L'enseignement << moral et civique >>}

\item idée de ne pas faire  de éduc civique 1 enseignement technocratique portant uniquement sur institutions de la Rep.\\
\begin{enumerate}
\item indissociable de montée des incivilités et insécurité ms aussi intégrismes et extrémismes qui menace citoyenneté et lien social. \\
\item reliée à problématique d'1 société pluriculturelle et multiethnique.
\begin{enumerate}
\item doit faire compde aux élèves : respect de la personne, de ses origines et ses différences, égalité entre femmes et hommes, valeurs de laïcité.\\
\item but : faire des citoyens responsables et libres, se forger sens critique et adopter comportement réfléchi (art. 41) \\
\item faire partager valeurs de République dt fondements de laïcité \\
\item participent à  construction d'1 vivre ensemble ds notre société. \\
\item sur façade : devise Rep, drapeau tricolore et drapeau européen. Sur mur : DDHC \\
\item affichage de charte de laïcité. \\
\end{enumerate}

\item modalité d'évaluation et de formation des enseignants pr rentrée 2015. \\
\end{enumerate}

\item CPE (ont charge de vie scol et entretiennent relations privilégiés avec parents) auront 1 rôle à jouer pr apporter 1 concours pratique, en actes, à enseignement de ces principes. \\


\subsection{La déontologie dans le statut général des fonctionnaires} 

\item But : 

\begin{enumerate}
\item reconnaître devoir d'exercer fonctions ac impartialités, probité et dignité. Fonde confiance des citoyens envers ceux qui ont fait le choix de servir intérêt général. \\
\item consacre obligation de neutralité et de réserve. But : garantir égal traitement et respect des libertés de conscience de ttes les personnes. \\
\item Garantir respect pcpes de laïcité \\
\end{enumerate}

\section{Formation initiale des personnels}

\item Mise en place des ESPE. Délivrent 1 master.
\begin{enumerate}
\item formation initiale des étudiants \\
\item formation continue des personnels enseignants et d'éducation \\
\item participe à recherche disciplinaire et pédagogique \\
\item assure dvpt et promotion méthodes pédagogiques innovantes \\
\item préparent aux enjeux du socle.
\end{enumerate}

\item Ts les personnels doivent pde en charge la personne de l'élève. Sentiment chez ts les acteurs qu'ils doivent viser ensemble l'insertion professionnelle et l'intégration sociale et citoyenne de la personne des élèves. \\

\chapter{Les contenus d'enseignement}

\item Art 35 : programmes définissent connaissances et compétences acquises à chq cycles. \\

\section{L'éducation artistique et culturelle}

\item connaissance patrimoine culturel + création contemporaine. Dvpt de créativité et pratiques artistiques \\

\section{L'éducation physique et sportive}

\item Valorisation du sport scolaire ds ses dimensions éducatives. \\

\item CPE et ancêtres ont facilité transfert de compétences et de savoir-être de école vers son environnement à travers activités sportives périscolaires (foyers socio-éduc). Loi propose activités sportives en complément des heures d'EPS à ts élèves volontaires. \\

\item Recherche de complémentarité de ts les acteurs et celle du sens pédagogique autour des valeurs transmises / sport relèvent de responsabilité du service public d'éducation. \\

\section{L'enseignement d'une langue vivante étrangère et d'une langue régionale}

\item langues et cultures régionales appartiennent au patrimoine de la France.

\section{L'éducation à la santé}

\item les << éducations non formelles >> ne se laissent pas enfermer ds 1 champs disciplinaire unique. Autt de savoir que de savoir-faire, savoir-être. \\

\item CPE concerné en 1er chef : doit les rendre possible ds vie scol + \ impulsion à actions de préventions et objets éducatifs pluridisciplinaire.\\

\item Ne saurait incombé qu'aux seuls personnels médicaux et infirmiers de Educ nat. \\

\item Réflexion consciente des élèves sur environnement + 1 attention de part des personnels d'éducation, à qualité d'1 environnement scolaire favorable à santé. \\

\item dvper connaissance élèves vis-à-vis de leur santé. \\

\section{L'éducation à l'environnement et au dvpt durable}

\item débute à école primaire. Sensibilisation à la nature, compréhension et évaluation de impact des activités humaines sur ressources naturelles. gds enjeux environnementaux : qualité de l'air, chgts climatiques, gestion des ressources, préservation de biodiversité + sensibilisation comportement écoresponsables et savoir-faire pr préserver planète. \\

\section{La promotion de la culture scientifique et technologique}


\item savoirs scientifiques porteurs de valeurs citoyennes qu'il faut faire compde afin que connaissance scientifique ne soit pas détournée de son véritable but : contribuer au bien de humanité. \\

\item Soutien aussi pr place de France ds compétition internationale des élites.\\

\item CPE doit mettre en synergie les différents acteurs afin d'engager élèves vers voies profesionnelles.\\

\chapter{Dossier en chantier durant l'année 2013-2014}

\section{La réforme de l'éducation prioritaire}

\subsection{Un diagnostic sévère}

\item rentrée 2013 : réforme de éducation prioritaire. Diagnostic associant acteurs. Question des moyens et modalités d'exercice des personnels. \\

\item éduc prioritaire concerne 1 collégien sur 4, 297 collèges. En réseau ECLAIR, 73 \% des élèves ont des parents ouvriers ou inactifs (35 \% ds hors-prioritaire), 22 \% st en retard à entrée 6e.




\end{itemize}



\end{document}