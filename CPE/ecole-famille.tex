\documentclass[12pt]{article}
\usepackage{fontspec}
\usepackage{xltxtra}
\setmainfont[Mapping=tex-text]{Century Schoolbook L}
 \usepackage[francais]{babel}
 
 \usepackage{geometry}
 \geometry{ hmargin=0.5cm, vmargin=0.5cm } 

\makeatletter
\renewcommand\section{\@startsection
{section}{1}{0mm}    
{\baselineskip}
{0.5\baselineskip}
{\normalfont\normalsize\textbf}}
\makeatother



\begin{document}



\textbf{Ecole et familles, 82} \\

\textbf{Mots clés : }
\begin{itemize}
\item concept de communauté éducative
\item crise de état et de ville / question de mixité sociale
\item lien social
\item 
\item 
\item 
\end{itemize}

\vspace{0.5cm}

\textbf{Objectif :}


\begin{enumerate}
\item Forger sentiment d'appartenance à étab de ts ses acteurs\\
\item Donner du sens au collectif, au bien commun\\
\item Construire étab scol comme espace laïque de savoirs et citoyenneté.\\
\end{enumerate}

\textbf{1. Forger le sentiment d'appartenance à un collectif large en respectant la diversité culturelle} \\

\begin{itemize}
\item But : étudier conditions favorables et défavorables à constitution du sentiment d'appartenance des familles, élèves et de ts les personnels à communauté éducative (respectueux des différences).\\


\fbox{
\begin{minipage}{19cm}
\textbf{Concept de communauté éducative}
\begin{enumerate}
\item hérité des écoles du 19e. Remis au jours deb 80's avec concept de projet d'étab. But : mieux associer à sa scolarité ts les acteurs de l'éducation d'un jeune.\\
\item Art. L111-3, Code de éducation <<Ds chq école, collège, lycée, la communauté éducative rassemble les élèves et ts ceux qui, dans l'etab ou en relation avec lui, participent à l'accomplissement de ses mission. Réunit personnels des écoles et étab, parents d'élèves, coll Tales + acteurs institutionnels, éco, sociaux, associés au service public d'éducation. >> \\
\item Mobilisation de ts apparâit comme gage de meilleur qualité et efficacité du système éduc.\\
\end{enumerate}
\end{minipage}
}

\vspace{0.5cm}

\item mutations actuelles société ont csq sur vie, fonctionnement des étab. Interroge rôle de la communauté éducative.\\

\item ds la ville, enfants ne se reconnaîssent pas. svt juxtaposition de quartiers où ségrégation sociale et spatiales. cherchent une culture d'identification en se repliant sur groupes ethniques, religieux, voir sectes.\\

\fbox{
\begin{minipage}{19cm}
\textbf{Accepter ou rejeter l'autre : le phénomène << nimby >>} \\
Le Monde, 29-30 décembre 2002.\\
\begin{enumerate}
\item Implantation équipements pr accueillir pop déshérités (pauvres, toxico, tsiganes) se heurte à riverains. << Pas dans mon jardin >>. Attitude en constante augmentation.
\end{enumerate}
\end{minipage}
}

\vspace{0.5cm}

\item ghettoïsation de certains quartiers : refus de autre. produit violence.\\
\begin{enumerate}
\item Y compris pr quartiers aisés qui refusent centre pr handicapés, structure pr toxico à côté.\\
\end{enumerate}

\item crise de nation, des composantes de l'identité nationale. Jeunes ne se sentent ni d'ici ni d'ailleurs. Prob car France régie / droit du sol. Acculturation défi car pr savoir où on va, il faut savoir d'où on vient et ce que l'on veut. \\

\textbf{2. Les mutations des espaces urbains : crise des espace de citoyenneté.} \\

\item Espace pas divisé entre ville et campagne ms entre multitude d'archipel dans le péri-urbain.\\

\item crise de Etat : interrogation sur son rôle, place des pvrs locaux. Etat semble se délité :  transfert compétences au niveau des régions, dep, communes + abandon de souverainaté au profit d'organisme internationaux (Europe, OMC).\\

\item s'y ajoute \textbf{crise de la ville comme espace de mixité sociale}. ++ archipels de quartiers où ségrégation spatiale = ségrégation sociale.
\begin{enumerate}
\item Durkeim et le concept de lien social. Ensemble des droits et devoirs découle de solidarité entre ts. ca doit régler relations sociales. Pr lui et solidariste, école a pr mission de créer ce lien social et montrant que inégalités sociales st justes si st fondées sur différences de produits du travail et si articulée de façon solidaires.\\
\item Or, déplacement de pop, mutation de habitat, dislocation des familles rendent ++ prob notion de << terroirs >> d'origine.\\
\item ex: 1 jeune personne malienne née en France. 10 ans ap, n'est pas totalement Fr (même si nationalité), ni africaine. svt ds situation instable (rejette, oublie ou fantasme communauté de naissance dt elle maîtrise pas ts les codes). Est << Fr >> en Afrique noire et << Afr >> en France. Ne se sent pas insérée (surtt terme de logements et emploi) ds 1 société fr qui respecte pas tjs règles qu'elle édicte pr acceuil des immigrés. \\
\end{enumerate}

\item \textbf{Risque du repli identitaire.}\\
\begin{enumerate}
\item devant ce << flou >>, jeune peut rechercher 1 culture d'identification qui soit ni celle de famille, ni tradition, ni pays d'accueil. peut se forger ds quartier, ds démarche de retour au religieux...\\
\end{enumerate}

\item Pdt longues années, nation (ds fondements familiaux, idéologiques, religieux) a été stable. Reposait sur certains nb de mythes fondateurs et intégrait les nouveaux arrivés (à travers institutions comme Eglise, syndicalisme, partis politique ou assoc sportive : ex avec Italiens ou Polonais).
\begin{enumerate}
\item Sous 3e Rep, école a renforcé la fonction unifiante de la culture scolaire.\\
\item culture scolaire est devenue 1 construction ++ autonome / rapport aux savoirs historiques : qq fois pure invention destinée à répondre au projet d'unification nationale. \\
\end{enumerate}

\item \textbf{Aujourd'hui, refondation d'1 projet national / école est-elle encore possible ?}
\begin{enumerate}
\item pr certains sociologues (dt Mohamed Charkaoui), partout où T nationale pas habité / sentiment d'appartenir à communauté, Etat-Nation risque de devenir 1 coquille vide qui ne peut plus fondée lien social.
\item affaiblissement du repère national comme terroir de vie et espace de valeurs commune complique fabrication du lien social / << école de la République >>.
\end{enumerate}

\item pr identités, superposition et imbrication des T de référence des élèves à différentes échelles (quartier, région, pays d'origine, pays d'accueil, Europe) requiert de école construction d'une culture commune qui ne soit pas une culture unique.\\


\textbf{3. Malaise social, malaise adolescent}\\

\item Jusque 70's, rythmes journaliers et hebdomadaires de vie était très svt les même pr ttes les professions. Différent aujourd'hui.
\begin{enumerate}
\item émiettement des rythmes de vie car aug. durée de déplacement domicile-travail.\\
\item de - en - d'horaires communs.\\
\end{enumerate}

\fbox{
\begin{minipage}{19cm}
\begin{enumerate}
\item 65 \% parents << zones sensibles >> travaillent matin avt 8h30 et 47 \% ap 19h30.
\begin{flushright}
Source : Libé, 1er août 2009
\end{flushright}
\end{enumerate}
\end{minipage}
}

\vspace{0.5cm}

\item pas surprenant que ado se plaignent que parents soient pas là. Demandes des parents : que institutions publiques puissent accueillir leurs enfants. Csq de ces situations sur les enfants ?
\begin{enumerate}
\item questions doivent mobiliser communauté éducative. Consultation des familles et évaluation des ressources. Transmission ensuite des propositions à équipe de direction.\\
\item \textbf{Rôle du CPE ds équipe de direction pr rythme scolaires}
\begin{enumerate}
\item doit se renseigner sur rythme de la majorité des parents pr décider des horaires de réunion. \\
\item Réponse doit être locale : pas la même dans étab d'1 village à 30 km de Toulouse où plupart des parents travaille et ds étab de banlieue proche de T. à proximité d'une station de métro. \\
\end{enumerate}
\end{enumerate}

\item tps des différents acteurs d'1 territoire n'est pas le même.
\begin{enumerate}
\item tps des familles, tps des élèves, tps de institution scolaire et ses rythmes annuels, tps des élus Taux ou nationaux.\\
\end{enumerate}

\textbf{4. Redéfinir le << vivre-ensemble >>} \\

\item  Au niveau philosophique, si école enseigne que ttes les cultures se valent, alors aucuns dialogues possibles entre elles. Danger d'accepter des systèmes de valeurs contraires. \\

\fbox{
\begin{minipage}{19cm}
WALZER Michael, \textit{Pluralisme et démocratie}, 1997, p. 80-81
\item \textbf{Quels sont les conditions du vivre ensemble ?} Selon Michael WALZER, école devrait favoriser chez élève 3 prises de cse : 
\begin{enumerate}
\item \textbf{Pde au sérieux identités, attentes de chq communauté.} Ne st pas nécessairement ascension sociale ou acquisition de ts savoirs scolaires. Ecole doit admettre existe autres sphères sociales où individus égaux entre eux selon d'autres critères\\
\item \textbf{faire compde que chq individu évolue ds plusieurs sphères autonomes} : une région, famille, religion, cat sociale. Propre du citoyen ds socété démoc : conserver faculté d'exercer libre volonté (lui permet de renoncer à 1 telle affiliation). \\
\item \textbf{construire le << soi postsocial >>, condition du vivre-ensemble}. Pr contrer fragmentation de société libérale, faut que individus puisse faire des choix selon conscience du bien commun ou de son propre bien.
\end{enumerate}
\end{minipage}
}

\item construire 1 projet commun pr vivre ensemble : enjeu pr société. Communauté éducative doit se donner moyen de faire coexister ds même espace individus ne partageant pas même convictions, au lieu de les juxtaposer en 1 mosaïque de communautés fermées sur elles-même et exclusives. Ecole doit être 1 moyen de faire coexister individus qui partagent pas mêmes convictions.\\

\textbf{5. Communauté éducative et vie scolaire}\\

\item communauté éducative : label ? voeux pieux ? Vrai concept ?
\begin{enumerate}
\item complexité relation école-famille, mutation que connaissent familles, soupçons sur missions lgtps indiscutables de école -> rend difficile définition de communauté éducative.\\
\end{enumerate}

\item Apparaît pas comme un concept, ms cmme 1 espace partenarial. But : être ce que les acteurs veulent ou peuvent en faire; pr accomplir, chacun dans son rôle mais en complémentarité, la tâche d'éduquer, de dvper qq'1 en devenir.\\

\fbox{
\begin{minipage}{19cm}
PFANDER-MENY Lydie, <<Ecole-famille : vers une nouvelle professionnalité des CPE ? >>, in PICQUENOT Alain, VITALI Christian (dir), \textit{De la vie scolaire à la vie d'élève}, CRDP Bourgogne, p. 112\\

\begin{enumerate}
\item CPE est un professionnel de éducation. Relation constante avec l'élève et sa famille ou ses familles. Fait tenir compte des nouveaux modes de parentalités.\\
<< Le CPE est, dans l'établissement du second degré, un professionnel de l'éducation. Il est à ce titre en relation constante avec l'élève et sa famille ou ses familles. >> \\
\item <<La question centrale posée aux établissements est de savoir comment donner ou redonner sens à la place et au statut des parents ? >> \\
\item 
\end{enumerate}
\end{minipage}
}


\item CPE joue 1 rôle très important.\\
\begin{enumerate}
\item interface entre monde enseignant et famille, intersection entre public et privé. \textbf{Peut animer cette communauté et donner parole à personne concernée} (dt élève.)\\

\end{enumerate}


\item ATTENTION : ne doit pas accaparer rôle éducatif, devenir une sorte de représentants des parents auprès des profs et vice versa.\\

\item n'est pas le spécialiste ms favorise dialogue et coopération.


\end{itemize}
\end{document}