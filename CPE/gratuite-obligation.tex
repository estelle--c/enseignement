\documentclass[12pt]{article}
\usepackage{fontspec}
\usepackage{xltxtra}
\setmainfont[Mapping=tex-text]{Century Schoolbook L}
 \usepackage[francais]{babel}
 
 \usepackage{geometry}
 \geometry{ hmargin=0.5cm, vmargin=0.5cm } 

\makeatletter
\renewcommand\section{\@startsection
{section}{1}{0mm}    
{\baselineskip}
{0.5\baselineskip}
{\normalfont\normalsize\textbf}}
\makeatother



\begin{document}



\textbf{Gratuité - Obligation, 18} \\

\textbf{Mots clés : }
\begin{itemize}
\item gratuité scolaire
\item Obligation
\item lutte vs absentéisme scolaire
\item décrochage scolaire
\end{itemize}

\vspace{0.5cm}

\textbf{Objectif :}


\begin{enumerate}
\item compde origine et sens des concepts de gratuité et d'obligation
\item identifier valeurs civiques et sociale qu'impliquent ces deux termes
\item connaître institutions, dispositifs, instructions relatifs à gratuité et à obligation.
\item analyser rôle du CPE ds étab ds prévention des situations de rupture de l'obligation (absentéisme et décrochage)\\
\end{enumerate}

\begin{itemize}
\item Obligation d'instruction en débat dès RF. D'abord pensée comme mesure nécessaire à formation du citoyen.\\

\item Projet de CONDORCET : chq enfant émancipé  des castes héréditaires de AR. doit faire de lui 1 citoyen capable d'exercer librement ces droits -> faut donc éclairer son esprit. Pd soin de distinguer éducation (relève des familles) et instruction (école). \\

\item Sous la Terreur : LEPELETIER de SAINT-FARGEAU. prévoit monopole de Etat. Obligation de enseignement et de internat ds des maisons d'éducation commune. \\

\item malentendu datant de cette époque : obligation d'enseignement est-elle compatible avec liberté d'éducation ds famille ? A dû sans cesse réaffirmer que obligation portait sur instruction et pas sur éducation. \\

\item Distinction instruction-éducation va pas tjs de soi aujourd'hui. Débat relancé sur dvpt des << éducations >> qui incluent des savoirs-êtres : éducation à environnement, à santé, à citoyenneté, au dvpt durable... \\
\begin{enumerate}
\item éducation citoyenne d'aujourd'hui ne paraît pas compatible ac obligation limitée à instruction (ie savoirs). pensent ++ que valeurs font partie de obligation d'enseignement.\\
\end{enumerate}
 
 \item Gratuité liée à obligation. Comment imposer instruction à ceux qui n'ont pas moyens de la donner à leurs enfants ? Déjà discuté sous loi Guizot (obligation aux communes d'entretenir une école primaire de garçon). Posée pdt EDG à propos de école unique.
 \begin{enumerate}
 \item Aujourd'hui, gratuité se pose encore : classes prépa. couteuses. fréquenté essentiellement / enfants de milieux socialement favorisées, doivent-elle resté gratuites ? La gratuité est-elle le meilleur instrument de l'équité ?\\
 \end{enumerate}

\textbf{1. La gratuité scolaire} \\

\textit{1.1 Les textes.} \\

\fbox{
\begin{minipage}{19cm}
\textbf{La gratuité dans les textes en vigeur en 2012}\\

\textit{Code de l'éducation, Art. L132-1 et 132-2.} \\

<< L'enseignement public dispensé dans les écoles maternelles et les classes enfantines et pendant la période d'obligation scolaire est gratuit. \\

L'enseignement est gratuit pr les élèves de collège et lycée publics qui donnent l'enseignement du 2nd degré ainsi que pr les élèves des classes prépa aux gdes écoles et à l'enseignement supérieur des étab d'enseignement public du second degré. >>
\end{minipage}
}

\vspace{0.5cm}

\item pcpe de gratuité de enseignement primaire public : loi Jules Ferry du 16 juin 1881. Etendue au secondaire / loi du 31 mai 1933.\\

\item manuels scolaires fournis aux élèves pdt scolarité obligatoire. Pris en charges / communes pr école primaire et / Etat pr collège. Etab (ac dotation de Etat) les achète et les prête chq année aux élèves. \\

\item ds lycées, depuis élections aux conseils régionaux de 2004, quasi-totalité des régions a pris décisions pr assurer gratuité pr élèves.\\

\fbox{
\begin{minipage}{19cm}
\textbf{La mise en oeuvre du pcpe de gratuité de enseignement scolaire public.} \\

\textit{Circulaire du 30 mars 2001} \\

<< pcpe de gratuité, doit être considéré de manière absolue. Concerne matériel d'enseignement à usage collectif, fournitures à caractères administratifs et dépense de fonctionnement (dont production de photocopies à destination des élèves et de leurs familles, frais de correspondance adressée aux familles, frais de téléphone). En revanche, dépenses afférentes aux activités facultatives (dt voyages scolaires) relèvent pas de ce principe. Peuvent être laissées à charge des familles. >>
\end{minipage}
}

\vspace{0.5cm}

\item si prob (frais de cantine ou participation financière à un voyage), chq étab dotée d'une somme mise à disposition des élèves : fonds social collégien et lycéen. Alloué /  chef d'étab.\\

\item Fond de vie lycéenne : destinée à financer actions en matière de formation des élus lycéens, info des élèves, communication, prévention des conduites à risques, d'éducation à la santé et citoyenneté, animations culturelles ou éducatives.\\

\item CAF verse à chq rentrée, 1 alloc dt somme fixée / Etat. \\

\item Conseil général pr apporter aide pr cas particuliers \\

\item Bourses versées aux familles des élèves de collèges et lycées. sur base des revenus familiaux, barème comportant points de charge (nb d'enfant, situation familiale ...) \\

\textbf{2. La dépense intérieure d'éducation} \\

\item 2008 : 129,4 milliards d'euros. 6,6 \% du PIB. Part de Etat baisse dep 85 et lois de décentralisation. \\

\item parmi dépenses éducatives : soutien scolaire. 75 \% des parents (selon sondage de 2005) sont prêt à recourir au soutien scolaire si enfant à difficulté. nbeuses collectivités Tales subventionnent assoc qui font de << accompagnement à la scolarité >>. \\

\textbf{3. L'obligation scolaire} \\

\textit{3.1. Définition} \\

\fbox{
\begin{minipage}{19cm}
\textbf{L'obligation scolaire ds les textes en vigueur, 2012} \\

\textit{Code de l'éducation \\}

Art. L131-1. << L'instruction est obligatoire pr les enfants des deux sexes, français et étrangers, entre 6 et 16 ans. La présente disposition ne fait pas obstacle à l'application des prescriptions particulière imposant une scolarité + longue. >> \\

Art. L122-3. << Tout jeune doit se voir offrir, avant sa sortie du système éducatif et quelque soit le niveau d'enseignement qu'il a atteint, une formation professionnelle. >> \\

Art. L122-2. << Tt élève qui, à l'issue de la scolarité obligatoire, n'a pas atteint 1 niveau de formation reconnu doit pvr poursuivre des études afin d'atteindre 1 tel niveau. L'Etat prévoit les moyens nécessaires, ds l'exercice de ses compétences, à la prolongation de scolarité qui en découle.\\
Tt mineur non émancpé dispose du droit de poursuivre sa scolarité au-delà de l'âge de 16 ans. \\
Lorsque les personnes responsables d'1 mineur non émancipé s'opposent à la poursuite de sa scolarité au-delà de 16 ans, 1 mesure d'assistance éducative peut être ordonnée dans les conditions prévues aux articles 375 et suivants du code civil afin de garantir le droit de l'enfant à l'éducation. >> \\
\end{minipage}
}

\vspace{0.5cm}

\item Depuis lois de Jules Ferry (1881-1882), enseignement obligatoire. à partir de 6 ans. Dep janvier 59, jusqu'à 16 ans révolus. \\

\item Loi du 18 décembre 1998 : renforce dispositions concernant obligation scolaire pr lutte vs dérives. \\

\item décret du 23 mars 99 : précise contenu des connaissances requis pr enfants instruits ds famille ou dans étab privés hors contrat. Actualisé en 2007 (obligation de posséder connaissance et compétence du socle.) \\

\textit{3.2. La lutte contre l'absentéisme scolaire.} \\

\item Touche en moyenne 5\% des collèges et lycées professionnels. 10 \% des lycées pro ont tx d'absentéisme atteint 10 à 16 \% des élèves. Fléau à combattre. \\

\item Ordonnance de janv 59 : sanction parents fautifs de laisser enfants manquer les cours.
\begin{enumerate}
\item pr année scol 2001-2002 : 2900 familles touchées.
\end{enumerate}


\fbox{
\begin{minipage}{19cm}
\textbf{La mise en oeuvre du pcpe de l'obligation d'assiduité pr les élèves inscrits ds étab scol.} \\

\textit{Loi du 2 janvier 2004 et le décret du 19 février 2004, BO << obligation scolaire : contrôle de l'assiduité scolaire} \\

<< Le chef d'étab pd contact ac parents de élève qui pas régulièrement présent. [...] Si dialogue inefficace, dossier transmis à inspecteur d'académie qui peut invité famille à suivre 1 module de soutien à la responsabilité parentale; si, en dépit de ensemb des mesures, assiduité de l'élève pas restaurée, proc de Rep pourra être saisi. >> \\
\end{minipage}
}

\vspace{0.5cm}

\item amende de 750 euros prévue pr gros cas. Peut aller (selon juge) jusqu'à 30.000 euros et 2 ans de prison. \\

\item 2 outils complètent : 
\begin{enumerate}
\item 1 commission de suivi de assiduité scolaire. Ds chaque département, sous autorité du préfet. Ts partenaires présents. \\
\item 1 dispositif de veille éducative ds sites prioritaires de la politique de la ville, sous responsabilité du maire. Mobiliser intervenants éducatifs et sociaux pr repérer jeunes en rupture. \\
\end{enumerate}

\item Loi en septembre 2010, loi Eric CIOTTI : si total d'absences au moins 4 demi-journée sur 1 mois constaté, directeur de CAF peut suspendre versement de alloc dû pr enfant. Après au moins 1 mois de présence de enfant, versement peut être rétabli.
\begin{enumerate}
\item texte dénoncé / observateurs de éducation. Exemple anglais (répression encore + forte) montre que pas utile, change pas phénomène. Pas forcément << démission parentale >> qui entraîne absentéisme. C'est pas en affaiblissant famille qu'on y remédie.\\ 
\item janv 2011 : Conseil sup de l'éducation a voté à unanimité contre circulation d'application de loi Ciotti = ts personnels de éduc : enseignants, parents, partenaires sociaux, coll Tale. \\
\end{enumerate}

\item Ap élection présidentielle 2012, débat relancé. Chiffres pas convaincants. 17 janvier 2013 : abrogée / Parlement. Nlle approche de lutte vs absentéisme avancée.

\item Nlle loi d'orientation en juillet 2013 : prévoit rencontre avec la famille + nomination d'1 personnel d'éducation référent pr suivre l'élève absentéiste. A suite des préconisation du Conseil européen de 2011 : améliorer offre pédagogique et prévoir intervention individuelle pr lutter vs abandon scolaire. \\

\textbf{4. Le décrochage scolaire} \\

\textit{4.1. Définition} \\

\item \textbf{Accumulation des difficultés d'apprentissage d'un élève qui aboutit à 1 point de rupture pvt conduire à absentéisme, à violence ou à déscolarisation.} \\

\item processus long, commencé en primaire. Retard d'acquisition. Difficultés s'accumulent. Décrochage survient pas d'un coup ms résulte d'un processus qui se construit au cours de scolarité.\\

\fbox{
\begin{minipage}{19cm}
\textbf{Les difficultés cognitives précèdent le décrochage scolaire.} \\

BONNERY Stéphane, << Décrochage cognitif et décrochage scolaire >>, GLASMAN Dominique et OEUVRARD Françoise [dir], \textit{La Descolarisation}, 2011, p. 149-150. \\

\begin{enumerate}
\item d'abord difficulté d'acquisition des savoirs à école primaire. Ces élèves pensent pourtt l'inverse d'eux-même : persuadés que aucun prob en primaire, que ça provient du collège. comprendre la construction du décrochage scolaire et cognitif nécessite de dépasser ces conceptions.
\end{enumerate}
\end{minipage}
}

\vspace{0.5cm}

\item Pense qu'il est possible de prévenir le décrochage. Rôle du CPE =  interface entre élève, enseignant et parents. crucial. \\

\textit{4.2. La lutte contre décrochage scolaire.}

\item 4 piliers : 
\begin{enumerate}
\item individualisation. Importt = comprendre pourquoi élève décroche. Critères sociaux. \\ 
\item Importance du lien avec famille. Restaurer ou maintenir ce lien. favoriser nouveau mode de coopération avec famille. \\
\item effort pédagogique. Individualisation des parcours pr décrocheurs : se demande si dispositif d'enseignement de droit commun est suffisant. \\
\item partenariat entre différents acteurs.
 
\end{enumerate}


\item 60.000 jeunes sortent sans qualification + 60.000 jeunes échouent ds système. \\
\begin{enumerate}
\item pr bcp en lycée pro et ont pas obtenu la filière qu'ils veulent. \\
\item selon études socio, appartiennent majoritairement aux classes sociales défavorisées. celles dt habitus primaire trop éloigné du secondaire.\\
\end{enumerate}

\underline{4.2.1. L'action éducative et pédagogique au sein de l'établissement scolaire} \\

\item CPE à écoute des élèves : voit ts les jours.  reçoit élèves en retard le matin, observe comportement d'isolement ou d'agression pdt interclasses, siège aux conseils de classe. 
\begin{enumerate}
\item 1ère mission : \textbf{dépistage des élèves décrocheurs}. \\
\item 2nd tps : doit \textbf{alerter les enseignants}, recueillir constats + \textbf{impulser projets éducatifs} au bénéfice des élèves en situation de décrochage. \\
\end{enumerate}

\item \textbf{doit tenter de réconcilier élève décrocheur avec école.} : redonnant du sens à son projet personnel. Entretien avec famille. \\

\item  dernier ressort : CPE peut mobiliser partenaires internes (COP, médecin scol) et extérieur (services sociaux). \\

\underline{4.2.2 Les dispositifs-relais} \\

\item But : remotiver jeune pr qu'il reprenne sa scolarité, pr obtenir diplôme professionnel. Atelier-relais organisés / mvts d'éducation pop pr combattre des tentations d'abandon scolaire / activités sportives et culturelles. \\

\item Création de certains étab pr lutter vs décrochage scolaire : micro-lycée de Sénart, collège << élitaire pour tous >> de Gre, << lycée innovant >> Jean Lurçat de Paris. \\

\underline{4.2.3. Les écoles de la dernière chance.} \\

\item 10aine en France. Jeunes entre 18 et 22 ans ayant pas de diplôme




























\end{itemize}
\end{document}